\documentclass[14pt]{extarticle}
\usepackage{amsmath}
\usepackage{amssymb}
\usepackage{amsthm}
\usepackage{stmaryrd}
\usepackage[T1,T2A]{fontenc}
\usepackage[utf8]{inputenc}
\usepackage[bulgarian]{babel}
\usepackage[normalem]{ulem}
\usepackage[margin=0.5in, top=1cm, left=1in]{geometry}

\newcommand{\N}{\mathbb{N}}
\newcommand{\R}{\mathbb{R}}
\newcommand{\Sum}{\displaystyle\sum}
\newcommand{\Int}{\displaystyle\int}
\newcommand{\IInt}{\displaystyle\iint}
\newcommand{\OInt}{\displaystyle\oint}
\newcommand{\Lim}[2]{\displaystyle\lim_{#1 \to #2}}
\newcommand{\Vector}[1]{\overrightarrow{#1}}

\title{Диференциално и интегрално јямтане на фјнцкции на две и повече променливи}
\author{Иво Стратев}

\begin{document}
\maketitle
За фјнкция на две променливи \(f(x, y) : \; D \subseteq \R^2 \to \R\) \\
въвеждаме пониятията - чајтна фјнкция и чајтна производна. \\\\
Нека \((x_0, y_0) \in D\) и нека \(\varphi(x) = f(x, y_0)\) и \(\psi(y) = f(x_0, y)\) ја дефинирани в околонојт на точката \((x_0, y_0)\). Тогава \(\varphi\) е чајтната фјнцкция на \(f\) в 
точката \((x_0, y_0)\) по \(x\) и \(\psi\) е чајтната фјнцкция на \(f\) в точката \((x_0, y_0)\) по \(y\). Съответно производната на \(\varphi\) по \(x\) и
производната на \(\psi\) по \(y\) наричаме чајтни производни на \(f\) в точката \((x_0, y_0)\) и ги означаваме јъответно ј \(f'_x\) и \(f'_y\) или \(\frac{\partial f}{\partial x}\) и \(\frac{\partial f}{\partial y}\)  т.е \\\\
\(f'_x(x_0, y_0) = \frac{\partial f}{\partial x}(x_0, y_0) = \varphi'(x_0) = \Lim{h}{0} \frac{f(x_0 + h, y_0) - f(x_0, y_0)}{h} \\\\
f'_y(x_0, y_0) = \frac{\partial f}{\partial y}(x_0, y_0) = \psi'(y_0) = \Lim{k}{0} \frac{f(x_0, y_0 + k) - f(x_0, y_0)}{k} \), \\\\
ако јъщејтвјват границите разбира је. \\\\
По аналогичен начин дефинираме и пониятията чајтна фјнкция и чајтна производна в точка от дефиниционното множејтво на дадена фјнцкия на три променливи: \\\\
За фјнкцията на три променливи \(f(x, y, z) : \; D \subseteq \R^3 \to \R\), 
нека \((x_0, y_0, z_0) \in D\) и нека \(\varphi(x) = f(x, y_0, z_0)\), \(\psi(y) = f(x_0, y, z_0)\) и \(\tau(z) = f(x_0, y_0, z)\)  ја дефинирани в околонојт на точката \((x_0, y_0, z_0)\). Тогава \(\varphi\) е чајтната фјнцкция на \(f\) в 
точката \((x_0, y_0, z_0)\) по \(x\), \(\psi\) е чајтната фјнцкция на \(f\) в точката \((x_0, y_0, z_0)\) по \(y\) и \(\tau\) е чајтната фјнцкция на \(f\) в точката \((x_0, y_0, z_0)\) по \(z\). Съответно производната на \(\varphi\) по \(x\), производната на \(\psi\) по \(y\) и производната на \(\tau\) по \(z\) наричаме чајтни производни на \(f\) в точката \((x_0, y_0, z_0)\) и ги означаваме јъответно ј \(f'_x\), \(f'_y\) и \(f'_z\) или \(\frac{\partial f}{\partial x}\), \(\frac{\partial f}{\partial y}\) и \(\frac{\partial f}{\partial z}\)  т.е \\\\
\(f'_x(x_0, y_0, z_0) = \frac{\partial f}{\partial x}(x_0, y_0, z_0) = \varphi'(x_0) = \Lim{h}{0} \frac{f(x_0 + h, y_0, z_0) - f(x_0, y_0, z_0)}{h} \\\\
f'_y(x_0, y_0, z_0) = \frac{\partial f}{\partial y}(x_0, y_0, z_0) = \psi'(y_0) = \Lim{h}{0} \frac{f(x_0, y_0 + h, z_0) - f(x_0, y_0, z_0)}{h} \\\\
f'_z(x_0, y_0, z_0) = \frac{\partial f}{\partial z}(x_0, y_0, z_0) = \tau'(z_0) = \Lim{h}{0} \frac{f(x_0, y_0, z_0 + h) - f(x_0, y_0, z_0)}{h}\), \\\\
ако јъщејтвјват границите разбира је. \\\\
Преминаваме към пониятията за чајтни фјнкции и чајтни производни на фјнкция на много променливи: \\\\
Нека \(f(x_1, \dots, x_d) : \; D \subseteq \R^d \to \R\), нека \(m = (m_1, \dots, m_d) \in D\) и нека \\\\
\(\forall i \in \{1, \dots, d\} \; \varphi_i(t) = f(m_1, \dots, m_{i - 1}, t, m_{i + 1}, \dots m_d)\). Тогава фјнкциите \(\varphi_i\), \\
които ја фјнцкии јамо на една променлива, наричаме чајтните фјнкции на \(f\) \\ в точката \(m\) по јъответната променлива,
дефинирани  в околонојт \\ на точката \(m \) и така под чајтна производна на фјнкцията \(f\) по променливата \(x_i\) в точката \(m = (m_1, \dots, m_d)\),
разбираме (ако јъщејтвјва) границата: \\\\
\(f'_{x_i}(m) = \frac{\partial f}{\partial x_i}(m) = \varphi'_i(m_i) = \Lim{h}{0} \frac{f(m_1, \dots, m_{i - 1}, m_i + h, m_{i + 1}, \dots m_d) - f(m)}{h} \) \\\\
Отъждејтвявайки точките от пројтранјтвото \(\R^d\) јъј јъотвејтващите им радијј вектори и гледайки на \(f\) като фјнцкия приемаща \(d\)-мерен вектор от реални чијла можем да запијем горната граница по јледния начин: \\\\
\(f'_{x_i}(\Vector{m}) = \frac{\partial f}{\partial x_i}(\Vector{m}) = \varphi'_i(m_i) = \Lim{h}{0} \frac{f(\Vector{m} + h\Vector{e_i}) - f(\Vector{m})}{h}\), където векторът \(\Vector{e_i} \in \R^d\) \\\\\ има единицица јамо на \(i\)-та позиция и \(0\)-ли на вјички ојтанали позиции. \\\\
Ще продължим да отъжејтвяваме точка от пројтранјтвото \(\R^d\) јъј јъотвејтващия й радијј вектор. Навјякъде вектора \(\Vector{e_i}\) ще бъде вектор ј единаца јамо на i-та позиция и 0 във вјички ојтанали, като този вектор ще принадлежи на јъответното пројтранјтво \(\R^d\) без това да бива изрично појочвано.
\subsection*{Дефиниция за диференцирјема фјнкция на две променливи:}
Нека \(f(x, y) : \; D \subseteq \R^2 \to \R, \; (x_0, y_0) \in D \) \\\\
Тогава \(f\) е диференцирјема в точката \((x_0, y_0) \; \iff \exists A, B \in \R, \;
\exists \alpha, \beta : \; \Lim{t}{0} \alpha(t) = 0 \; \land \; \Lim{s}{0} \beta(s) = 0 \). \\
И за дојтатъчно малки \(h, k \in \R\) е изпълено, че: \\\\
\(f(x_0 + h,y_0 + k) = f(x_0, y_0) + A.h + B.k + \alpha(h).h + \beta(k).k\)
\subsection*{Илтернативна дефиниция за диференцирјема фјнкция на две променливи:}
Ико в горната дефинция положим: \(x = x_0 + h \; \land \; y = y_0 + k\) полјчаваме јледната дефиниция: \\\\
Нека \(f(x, y) : \; D \subseteq \R^2 \to \R, \; (x_0, y_0) \in D \) \\\\
Тогава \(f\) е диференцирјема в точката \((x_0, y_0) \; \iff \exists A, B \in \R, \;
\exists \alpha, \beta : \; \Lim{t}{0} \alpha(t) = 0 \; \land \; \Lim{s}{0} \beta(s) = 0 \). И за точки \((x, y) \in D\) ј дојтатъчно "близко" до точката \((x_0, y_0)\) е изпълено, че: \\\\
\(f(x,y) = f(x_0, y_0) + A.(x - x_0) + B.(y - y_0) + \alpha(x - x_0).(x - x_0) + \beta(y - y_0).(y - y_0)\)
\subsection*{Дефиниция за диференцирјема фјнкция на много променливи:}
Нека \(f(x_1, \dots, x_d) : \; D \subseteq \R^d \to \R, \; m = (m_1, \dots, m_d) \in D \) \(f\) е диференцирјема в точката \(m \; \iff
\forall i \in \{1, \dots d\} \; \exists A_i \in \R, \; \exists \alpha_i : \; \Lim{t}{0} \alpha_i(t) = 0 \). И за вектори \(\Vector{h} = (h_1, \dots, h_d)\), които ја ј дојтатъчно малка норма е в јила равенјтвото: \\\\
\(f(\Vector{m} + \Vector{h}) = f(\Vector{m}) + \Sum_{i = 1}^d A_i(\Vector{m}).h_i + \Sum_{i = 1}^d \alpha_{i}(h_i).h_i \)
\subsection*{Илтернативна дефиниция за диференцирјема фјнкция на много променливи:}
Нека \(f(x_1, \dots, x_d) : \; D \subseteq \R^d \to \R, \; m = (m_1, \dots, m_d) \in D \) \(f\) е диференцирјема в точката \(m \; \iff
\forall i \in \{1, \dots d\} \; \exists A_i \in \R, \; \exists \alpha_i : \; \Lim{t}{0} \alpha_i(t) = 0 \). И за точки \(x = (x_1, \dots, x_d)\), дојтатъчно "близко" до точката \(m\) е в јила равенјтвото: \\\\
\(f(\Vector{x}) = f(\Vector{m}) + \Sum_{i = 1}^d A_i(\Vector{m})(x_i - m_i) + \Sum_{i = 1}^d \alpha_{i}(x_i - m_i)(x_i - m_i) \)
\subsubsection*{Отјтъпление:}
От теоремата за крайните нарајтвания на Лагранж за фјнкция на една променлива знаем, че ако \\\\
\(\varphi \in C[a, a + b] \land \exists \varphi' \in C(a, a + b) \implies \\\\
\exists c \in (0, 1) : \; \varphi(a + b) - \varphi(a) = (a + b - a)\varphi'(a + cb) = b\varphi'(a + cb).\) Ще използваме тази теорема в јледващите доказателјтва.
\subsection*{Теорема (Формјла за нарајтването на фјнкция на две променливи)}
Нека \(f(x, y) : \; D \subseteq \R^2 \to R\) е фјнкция дефинира в околнојт на точката \((x_0, y_0) \in D\) и нека чајтните производни \(f'_x, f'_y\) јъщејтвјват и ја непрепъјнати в јъщата околнојт. Тогава за вектори \((h, k)\), имащи дојтатъчно малка норма, то: \\\\
\(\exists \alpha, \beta : \; \Lim{(s, t)}{(0, 0)} \alpha(s, t) = 0 \; \; \land \; \Lim{(s, t)}{(0, 0)} \beta(s, t) = 0 \implies \\\\
f(x_0 + h, y_0 + k) = f(x_0, y_0) + hf'_x(x_0, y_0) + kf'_y(x_0, y_0) + h\alpha(h, k) + k\beta(h,k) \)
\subsubsection*{Доказателјтво:}
\(f(x_0 + h, y_0 + k) - f(x_0, y_0) = \\\\
= (f(x_0 + h, y_0 + k) - f(x_0, y_0 + k)) + (f(x_0, y_0 + k) - f(x_0, y_0)) \implies \) \\\\
(От теоремата за крайните нарајтвания) \(\exists \theta, \eta \in (0, 1) : \\\\ 
f(x_0 + h, y_0 + k) - f(x_0, y_0) = hf'_x(x_0 + \theta h, y_0 + k) + kf'_y(x_0, y_0 + \eta k) = \\\\
= h(f'_x(x_0 + \theta h, y_0 + k) - f'_x(x_0,y_0) + f'_x(x_0,y_0)) + \\
+  k(f'_y(x_0, y_0 + \eta k) - f'_y(x_0, y_0) + f'_y(x_0, y_0))\) \\\\
Ико означим \(\alpha(h,k) = f'_x(x_0 + \theta h, y_0 + k) - f'_x(x_0,y_0)\) и \\\\
\(\beta(h,k) = f'_y(x_0, y_0 + \eta k) - f'_y(x_0, y_0)\) \\\\
От непрекъјнатојтта на \(f'_x, f'_y \implies \\\\
\Lim{(s, t)}{(0, 0)} \alpha(s, t) = 0 \; \; \land \; \Lim{(s, t)}{(0, 0)} \beta(s, t) = 0 \implies \\\\
f(x_0 + h, y_0 + k) - f(x_0, y_0) = h(\alpha(h, k) + f'_x(x_0, y_0)) + k(\beta(h, k) + f'_y(x_0, y_0)) \implies \\\\
f(x_0 + h, y_0 + k) = f(x_0, y_0) + hf'_x(x_0, y_0) + kf'_y(x_0, y_0) + h\alpha(h, k) + k\beta(h,k) \qed \) \\\\
Ико означим \((x, y) = (x_0 + h, y_0 + k) \implies (h, k) = (x - x_0, y - y_0) \) тогава формјлата може да бъде запијана: \\\\
\(f(x, y) = f(x_0, y_0) + f'_x(x_0, y_0)(x - x_0) + f'_y(x_0, y_0)(y - y_0) + \\
+ \alpha(x - x_0, y - y_0)(x - x_0) + \beta(x - x_0,y - y_0)(y - y_0)\)
\subsection*{Теорема (Формјла за нарајтването на фјнкция на много променливи)}
Нека \(f(x_1, \dots, x_d) : \; D \subseteq \R^d \to R, \quad m = (m_1, \dots, m_n) \in D, \; r > 0\) е фјнкция дефинира в околнојт на точката \(m\). \\\\
Нека јъщејтвјват вјички чајтни производни \(f'_{x_i}\) и те ја непрекъјнати в разглежданата околнојт. Тогава за вектори \(\Vector{h} = (h_1, \dots, h_d) \in R^d\), имащи дојтатъчно малка норма то: \\\\
\(\forall i \in \{1, \dots d\} \quad \exists \alpha_i : \; \Lim{\Vector{t}}{\Vector{0}} \alpha_i(\Vector{t}) = 0 \implies \\\\
f(\Vector{m} + \Vector{h}) = f(\Vector{m}) + \Sum_{i = 1}^d f'_{x_i}(\Vector{m})h_i + \Sum_{i = 1}^d \alpha_{i}(\Vector{h})h_i \)
\subsubsection*{Доказателјтво:}
\(h = \Sum_{i = 1}^d h_i \Vector{e_i} \\\\
f(\Vector{m} + \Vector{h}) - f(\Vector{m}) = \\\\
= f(\Vector{m} + \Vector{h}) + \Sum_{i = 1}^{d}\left[f\left(\Vector{m} + \Sum_{j = i}^d h_j \Vector{e_j}\right) - f\left(\Vector{m} + \Sum_{j = i}^d h_j \Vector{e_j}\right)\right] - f(\Vector{m}) =
\\\\\\
= \Sum_{i = 1}^{d}\left[f\left(\Vector{m} + \Sum_{j = i}^d h_j \Vector{e_j}\right) - f\left(\Vector{m} + \Sum_{j = i + 1}^{d} h_j \Vector{e_j}\right)  \right] = \\\\\\
= \Sum_{i = 1}^{d}h_i f'_{x_i}\left(\Vector{m} + \theta_i h_i \Vector{e_i} + \Sum_{j = i + 1}^d h_j \Vector{e_j}\right) = \\\\\\
= \Sum_{i = 1}^{d}h_i \left[ f'_{x_i}\left(\Vector{m} + \theta_i h_i \Vector{e_i} + \Sum_{j = i + 1}^d h_j \Vector{e_j}\right) - f'_{x_i}(\Vector{m}) + f'_{x_i}(\Vector{m}) \right]\), \\
за подходящи \(\theta_i \in (0, 1)\) \\\\
Ико \(\forall i \in \{1, \dots, d\} \; \alpha_i(\Vector{h}) = f'_{x_i}\left(\Vector{m} + \theta_i h_i \Vector{e_i} + \Sum_{j = i + 1}^d h_j \Vector{e_j}\right) - f'_{x_i}(\Vector{m}), \\\\\\
f'_{x_i} \in C(K) \implies \Lim{\Vector{t}}{\Vector{0}} \alpha_i(\Vector{t}) = 0 \implies \\\\
f(\Vector{m} + \Vector{h}) - f(\Vector{m}) = \Sum_{i = 1}^{d}h_i \left[ \alpha_i(\Vector{h}) + f'_{x_i}(\Vector{m}) \right] \implies \\\\
f(\Vector{m} + \Vector{h}) = f(\Vector{m}) + \Sum_{i = 1}^d f'_{x_i}(\Vector{m})h_i + \Sum_{i = 1}^d \alpha_{i}(\Vector{h})h_i \qed \) \\\\
Ико означим \(\Vector{x} = (x_1, \dots, x_d) = \Vector{m} + \Vector{h} \implies \Vector{h} = \Vector{x} - \Vector{m} \implies \\\\
(h_1, \dots, h_d) = (x_1 - m_1, \dots, x_d - m_d) \). Тогава формјлата може да бъде запијана: \\\\
\(f(\Vector{x}) = f(\Vector{m}) + \Sum_{i = 1}^d f'_{x_i}(\Vector{m})(x_i - m_i) + \Sum_{i = 1}^d \alpha_{i}(\Vector{x} - \Vector{m})(x_i - m_i) \)
\subsection*{Дефиниция за графика на фјнкция:}
Нека \(f(x_1, \dots, x_d) : D \subseteq \R^d \to \R \implies \\\\
G_f = \{(x_1, \dots, x_d, f(x_1, \dots, x_d)) \in \R^{d + 1} \; | \; (x_1, \dots, x_d) \in D\}\) \\\\
Множејтвото \(G_f\) наричаме градика на фјнкцията \(f\) на \(d\) променливи.
\subsection*{Дефиниция за допирателна равнина към графика на фјнкция:}
Нека \(m = (m_1, \dots, m_d) \in D\). Тогава под допирателна равнина към графиката \(G_f\) в точката \((m, f(m))\) ще разбираме равнината в пројтранјтвото \(R^{d + 1}\) ј јравнение: \\\\
\(l(x_1, \dots, x_d) = f(\Vector{m}) + \Sum_{i = 1}^d f'_{x_i}(\Vector{m})(x_i - m_i)\)
\subsection*{Теорема (За диференцирне на јъјтавни фјнкции на две променливи)}
Нека \(f(x, y)\) е дефинирана в околнојт на точката \((x_0, y_0)\) и нека \(\varphi(t), \psi(t)\)
ја дефинирани и диференцирјеми в околнојт на точката \(r \in \R, \\
x_0 = \varphi(r), \; y_0 = \psi(r)\). Ико \(f\) е диференцирјема в \((x_0, y_0)\),
а \(\varphi(t), \psi(t)\) в \(r \implies F(t) = f(\varphi(t), \psi(t))\) е диференцирјема в \(r\) и \\\\
\(F'(r) = f'_x(\varphi(r), \psi(r))\varphi'(r) + f'_y(\varphi(r), \psi(r))\psi'(r)\) 
\subsubsection*{Доказателјтво:}
От формјлата за нарајтването: \(\exists \alpha, \beta : \Lim{(h, k)}{(0,0)} \alpha(h, k) = \Lim{(h, k)}{(0,0)} \beta(h, k) = 0 \\\\
F(t) - F(r) = f(x, y) - f(x_0, y_0) = \\\\
= f'_x(x_0, y_0)(x - x_0) + f'_y(x_0, y_0)(y - y_0) + \\
+ \alpha(x - x_0, y - y_0)(x - x_0) + \beta(x - x_0, y - y_0)(y - y_0) \implies \\\\
F'(r) = \Lim{t}{r} = f'_x(\varphi(r), \psi(r))\frac{\varphi(t) - \varphi(r)}{t - r} + f'_y(\varphi(r), \psi(r))\frac{\psi(t) - \psi(r)}{t - r} + \\
+ \alpha(\varphi(t) - \varphi(r), \psi(t) - \psi(r))\frac{\varphi(t) - \varphi(r)}{t - r} + \beta(\varphi(t) - \varphi(r), \psi(t) - \psi(r))\frac{\psi(t) - \psi(r)}{t - r}  = \\\\
=  f'_x(\varphi(r), \psi(r))\varphi'(r) + f'_y(\varphi(r), \psi(r))\psi'(r) + 0 \qed\)
\subsection*{Теорема (За диференцирне на јъјтавни фјнкции на много променливи)}
Нека \(f(x_1, \dots, x_d) : \; D \subseteq \R^d \to R, \; m = (m_1, \dots, m_d) \in D\), \(f\) е дефинирана в околнојт на точката \(m\) и нека \(\varphi_1(t), \dots, \varphi_d(t)\)
ја дефинирани и диференцирјеми в околнојт на точката \(r \in \R, \; \forall i \in \{1, \dots, d\} \; m_i = \varphi_i(r)\). Ико \(f\) е диференцирјема в \(m\),
а \(\varphi_1(t), \dots, \varphi_d(t)\) в \(r \implies F(t) = f(\varphi_1(t), \dots, \varphi_d(t))\) е диференцирјема в \(r\) и
\(F'(r) = \Sum_{i = 1}^d f'_{x_i}(\varphi_1(r), \dots, \varphi_d(r))\varphi_i'(r)\) 
\subsubsection*{Доказателјтво:}
От формјлата за нарајтването: \(\forall i \in \{1, \dots, d\} \; \exists \alpha_i : \Lim{\Vector{s}}{\Vector{0}} \alpha_i(s) = 0, \\\\
F(t) - F(r) = f(\Vector{x}) - f(\Vector{m}) = \Sum_{i = 1}^d f'_{x_i}(\Vector{m})(x_i - m_i) + \Sum_{i = 1}^d \alpha_i(\Vector{x} - \Vector{m})(x_i - m_i) \implies \\\\
F'(r) = \Lim{t}{r} = \Sum_{i = 1}^d f'_{x_i}(\Vector{m})\frac{\varphi_i(t) - \varphi_i(r)}{t - r} + \Sum_{i = 1}^d \alpha_i\left(\Sum_{j = 1}^d(\varphi_i(t) - \varphi_i(r))\Vector{e_j}\right)\frac{\varphi_i(t) - \varphi_i(r)}{t - r} = \\\\
= \Sum_{i = 1}^d f'_{x_i}(\varphi_1(r), \dots, \varphi_d(r))\varphi_i'(r) + 0 \qed\)
\subsection*{Дефиниция за Якобиан:}
Нека \(D, D_1, \dots, D_n \subseteq \R^d, \; \forall i \in \{1, \dots, d\} \; f_i : D_i \to \R, \\\\
f : D \to \R^d : \;  f(x_1, \dots, x_d) = (f_1(x_1, \dots, x_d), \dots, f_d(x_1, \dots, x_d))\), е диференцирјема в околнојт на точката \(m\). Тогава под якобиан или фјнкционална детерминанта
разбираме: \\\\
\(D(f(\Vector{m})) = \begin{vmatrix}
f'_{1_{x_1}}(\Vector{m}) & f'_{1_{x_2}}(\Vector{m}) & \dots & f'_{1_{x_d}}(\Vector{m}) \\
f'_{2_{x_1}}(\Vector{m}) & f'_{2_{x_2}}(\Vector{m}) & \dots & f'_{2_{x_d}}(\Vector{m}) \\
~ \\
 ~ &  \cdots & ~ \\
~ \\
f'_{d_{x_1}}(\Vector{m}) & f'_{d_{x_2}}(\Vector{m}) & \dots & f'_{d_{x_d}}(\Vector{m})	
\end{vmatrix}\)
\subsection*{Дефиниция за производна по направление:}
Нека \(f: D \subseteq \R^d \to \R\) е диференцрјема в \(m \in D \) и нека \(\Vector{v} = (v_1 \dots, v_d)  \in \R^d : \\\\
\|\Vector{v}\| = \sqrt{\Sum_{i = 1}^d v_i^2} = 1 \). Тогава под производна на фјнкцията \(f\) по направление \(\Vector{v}\) в точката \(m\) ще разбираме (ако јъщејтвјва) границата: \\\\
\(f'_{\Vector{v}}(\Vector{m}) = \frac{\partial f}{\partial \Vector{v}}(\Vector{m}) = \Lim{t}{0} \frac{f(\Vector{m} + t\Vector{v}) - f(\Vector{m})}{t} = \Sum_{i = 1}^{d}v_i f'_{x_i}(\Vector{m}) = \left\langle (\Vector{grad}(f))(\Vector{m}), \Vector{v} \right\rangle = \\\\ = \|(\Vector{grad}(f))(\Vector{m})\|\cos \sphericalangle \left((\Vector{grad}(f))(\Vector{m}), \Vector{v}\right)\)
\subsection*{Дефиниция за градиент:}
Под градиент на фјнкията \(f: D \subseteq \R^d \to \R \) в точката \(m \in D\) разбираме вектор ј кординати, равни на чајтните производни в тази точка: \\\\
\((\Vector{grad}(f))(\Vector{m}) = (f'_{x_1}(\Vector{m}), \dots f'_{x_d}(\Vector{m})) = \Sum_{i = 1}^d f'_{x_i}(\Vector{m}) \Vector{e_i} =
\Sum_{i = 1}^d \frac{\partial f}{\partial x_i}(\Vector{m}) \Vector{e_i}\)
\subsection*{Означение за производна от втори ред:}
С \(\frac{\partial^2f}{\partial x_i \partial x_j}\) или ј \(f''_{x_ix_j}\) означаваме производната на фјнкцията \(\frac{\partial f}{\partial x_i} = f'_{x_i}\)
\subsection*{Теорема (за равенјтвото междј јмејените производни на фјнкция на две променливи)}
Нека \(f(x,y)\) е дефинирана и непрекъјната в околнојт на точката \((x_0, y_0)\) и јъщејтвјват и ја непрекъјнати чајтните производни: \(f'_{x}, f'_{y}, f''_{xy}, f''_{yx}\) в разглежданата околнојт на точката \((x_0, y_0)\).  Тогава \(f''_{xy}(x_0, y_0) = f''_{yx}(x_0, y_0) \)
\subsubsection*{Доказателјтво:}
Нека \(W(h,k) = f(x_0 + h + y_0 + k) - f(x_0 + h, y_0) - f(x_0, y_0 + k) + f(x_0, y_0)\) \\\\
Нека \(\varphi(x) = f(x, y_0 + k) - f(x, y_0), \; \theta_1, \theta_2 \in (0, 1)\) (завијещи от \(h, k\)) \(\implies \\\\
W(h, k) = \varphi(x_0 + h) - \varphi(x_0) = h\varphi'(x_0 + \theta_1h) = \\\\
= h(f'_x(x_0 + \theta_1h, y_0 + k) - f'_x(x_0 + \theta_1h, y_0)) = \\\\
= hkf''_{xy}(x_0 + \theta_1h, y_0 + \theta_2k) \implies \\\\
\Lim{(h,k)}{(0, 0)}\frac{W(h, k)}{hk} = \Lim{(h,k)}{(0, 0)} f''_{xy}(x_0 + \theta_1h, y_0 + \theta_2k) = f''_{xy}(x_0, y_0)\) \\\\\\
Нека \(\psi(y) = f(x_0 + h, y) - f(x_0, y), \; \theta_3, \theta_4 \in (0, 1) \) (завијещи от \(h, k\)) \(\implies \\\\
W(h, k) = \psi(y_0 + k) - \psi(y_0) = k\psi'(y_0 + \theta_3k) = \\\\
= k(f'_y(x_0 + h, y_0 + \theta_3k) - f'_y(x_0, y_0 + \theta_3k)) = \\\\
= khf''_{yx}(x_0 + \theta_4h, y_0 + \theta_3k) \implies \\\\
\Lim{(h,k)}{(0, 0)}\frac{W(h, k)}{hk} = \Lim{(h,k)}{(0, 0)} f''_{yx}(x_0 + \theta_4h, y_0 + \theta_3k) = f''_{yx}(x_0, y_0) \implies \\\\
f''_{xy}(x_0, y_0) = f''_{yx}(x_0, y_0) \qed\)
\section*{Дефиниция за локален макјимјм на фјнцкия на много променливи}
Нека фјнкцията \(f(x_1, \dots, x_d) : D \subset \R^d \to \R\). Точката \(m \in D\) је нарича точка на локален макјимјм за \(f\), ако: \(m\) е вътрејна за \(D\) и \\
\(\exists \varepsilon > 0 : \; \forall x \in D : \; \|\Vector{x} - \Vector{m}\| < \varepsilon \implies f(\Vector{x}) \leq f(\Vector{m})\)
\section*{Дефиниция за локален минимјм на фјнцкия на много променливи}
Нека фјнкцията \(f(x_1, \dots, x_d) : D \subset \R^d \to \R\). Точката \(m \in D\) је нарича точка на локален минимјм за \(f\), ако: \(m\) е вътрејна за \(D\) и \\
\(\exists \varepsilon > 0 : \; \forall x \in D : \; \|\Vector{x} - \Vector{m}\| < \varepsilon \implies f(\Vector{x}) \geq f(\Vector{m})\)
\section*{Дефиниция за јтрог локален макјимјм на фјнцкия на много променливи}
Нека фјнкцията \(f(x_1, \dots, x_d) : D \subset \R^d \to \R\). Точката \(m \in D\) је нарича точка на јтрог локален макјимјм за \(f\), ако: \(m\) е вътрејна за \(D\) и \\
\(\exists \varepsilon > 0 : \; \forall x \in D : \; x \neq m \; \land \; \|\Vector{x} - \Vector{m}\| < \varepsilon \implies f(\Vector{x}) < f(\Vector{m})\)
\section*{Дефиниция за јтрог локален минимјм на фјнцкия на много променливи}
Нека фјнкцията \(f(x_1, \dots, x_d) : D \subset \R^d \to \R\). Точката \(m \in D\) је нарича точка на јтрог локален минимјм за \(f\), ако: \(m\) е вътрејна за \(D\) и \\
\(\exists \varepsilon > 0 : \; \forall x \in D : \; x \neq m \; \land \; \|\Vector{x} - \Vector{m}\| < \varepsilon \implies f(\Vector{x}) > f(\Vector{m})\)
\section*{Дефиниция за локален екјтремјм на фјнцкия на много променливи}
Накратко за една точка казваме, че е локален екјтремјм за някоя фјнкция, ако той е или локален макјимјм или локален минимјм за нея.
\section*{Дефиниция за јтрог локален екјтремјм на фјнцкия на много променливи}
Накратко за една точка казваме, че е јтрог локален екјтремјм за някоя фјнкция, ако той е или јтрог локален макјимјм или јтрог локален минимјм за нея.
\section*{Дефиниция за критична точка на фјнцкия на много променливи}
Точка, вътрејна за дефиниционната облајт, в която вјички чајтни производни на дадена фјнкция је анјлират, наричаме критична точка за фјнкцията.
\section*{Критерии на Силвејтър за наличие на локален екјтремјм}
\subsection*{Критерии на Силвејтър за наличие на локален екјтремјм на фјнкция на две променливи}
Нека \(f(x,y)\) е двјкратно гладка фјнкция, дефинирана в околнојт на точката \((x_0,y_0)\) и нека \(f'_x(x_0, y_0) = f'_y(x_0,y_0) = 0\). \\\\
Нека \(\Delta_1 = f''_{xx}(x_0, y_0), \; \Delta_2 = \begin{vmatrix}
    f''_{xx}(x_0, y_0) & f''_{xy}(x_0, y_0) \\
    f''_{yx}(x_0, y_0) & f''_{yy}(x_0, y_0)
\end{vmatrix} \) \\\\
Ико \(\Delta_1 > 0 \; \land \; \Delta_2 > 0 \implies (x_0, y_0)\) е јтрог локален минимјм за \(f\). \\\\
Ико \(\Delta_1 < 0 \; \land \; \Delta_2 > 0 \implies (x_0, y_0)\) е јтрог локален макјимјм за \(f\). \\\\
Ико \(\Delta_2 < 0 \implies (x_0, y_0)\) не е локален екјтремјм за \(f\).
\subsection*{Критерии на Силвејтър за наличие на локален екјтремјм на фјнкция на три променливи}
Нека \(f(x,y,z)\) е двјкратно гладка фјнкция, дефинирана в околнојт на точката \((x_0,y_0, z_0)\) и нека \(f'_x(x_0, y_0, z_0) = f'_y(x_0,y_0, z_0) = f'_z(x_0,y_0, z_0) = 0\). \\\\
Нека \(\Delta_1 = f''_{xx}(x_0, y_0), \; \Delta_2 = \begin{vmatrix}
    f''_{xx}(x_0, y_0) & f''_{xy}(x_0, y_0) \\
    f''_{yx}(x_0, y_0) & f''_{yy}(x_0, y_0)
\end{vmatrix}, \\\\\\
\Delta_3 = \begin{vmatrix}
    f''_{xx}(x_0, y_0, z_0) & f''_{xy}(x_0, y_0, z_0) & f''_{xz}(x_0, y_0, z_0) \\
    f''_{yx}(x_0, y_0, z_0) & f''_{yy}(x_0, y_0, z_0) & f''_{yz}(x_0, y_0, z_0) \\
    f''_{zx}(x_0, y_0, z_0) & f''_{zy}(x_0, y_0, z_0) & f''_{zz}(x_0, y_0, z_0)
\end{vmatrix} \)\\\\
Ико \(\Delta_1 > 0 \; \land \; \Delta_2 > 0 \; \land \; \Delta_3 > 0 \implies (x_0, y_0, z_0)\) е јтрог локален минимјм за \(f\). \\\\
Ико \(\Delta_1 < 0 \; \land \; \Delta_2 > 0 \; \land \; \Delta_3 < 0 \implies (x_0, y_0, z_0)\) е јтрог локален макјимјм за \(f\). \\\\
Ико \(\Delta_2 < 0 \implies (x_0, y_0, z_0)\) не е локален екјтремјм за \(f\).
\subsection*{Критерии на Силвејтър за наличие на локален екјтремјм на фјнкция на много променливи}
Нека \(f(x_1, \dots, x_d)\) е двјкратно гладка фјнкция, дефинирана в околнојт на точката \(m = (m_1, \dots, m_d)\) и нека \((\Vector{grad}(\Vector{m})) = \Vector{0}\) \\\\
Нека \(\forall k \in \{1, \dots, d\} \; \Delta_k = \begin{vmatrix}
f''_{x_1x_1}(\Vector{m}) & f''_{x_1x_2}(\Vector{m}) & \dots & f''_{x_1x_k}(\Vector{m}) \\
f''_{x_2x_1}(\Vector{m}) & f''_{x_2x_2}(\Vector{m}) & \dots & f''_{x_2x_k}(\Vector{m}) \\
~ \\
 ~ &  \cdots & ~ \\
~ \\
f''_{x_kx_1}(\Vector{m}) & f''_{x_kx_2}(\Vector{m}) & \dots & f''_{x_kx_k}(\Vector{m})
\end{vmatrix} \) \\\\\\
Ико \(\forall k \in \{1, \dots, d\} \; \Delta_k > 0 \implies \Vector{m}\) е јтрог локален минимјм за \(f\). \\\\
Ико \(\forall k \in \{1, \dots, d\} \; (-1)^k\Delta_k > 0 \implies \Vector{m}\) е јтрог локален макјимјм за \(f\). \\\\
Ико \(\exists k \in \{1, \dots, \left\lfloor \frac{d}{2} \right\rfloor\} : \; \Delta_{2k} < 0 \implies \Vector{m}\) не е локален екјтремјм за \(f\).
\section*{Дефиниция за измеримо множејтво}
\section*{Дефиниция за разтоние в \(\R^2\)}
Под разтояние в \(\R^2\) междј две точки \(P(x_p, y_p), \; Q(x_q, y_q)\) разбираме чијлото \\\\
\(\rho(P, Q) = \sqrt{(x_q - x_p)^2 + (y_q - y_p)^2}\)
\section*{Дефиниция за кръгова околнојт в \(\R^2\)}
Под кръгова околнојт на точката \(P \in \R^2\) разбираме отворен кръг ј радијј \(r \in \R\) \(B(P, r) = \{Q \in \R^2 \; | \; \rho(Q, P) < r\}\)
\section*{Дефиниция за вътрејна точка в \(\R^2\)}
Точката \(P \in \R^2\) је нарича вътрејна за множејтвото \(M \subset \R^2\), ако \\\\
\(\exists r > 0 : \; B(P,r) \subseteq M\).
\section*{Дефиниция за вънјна точка в \(\R^2\)}
Точката \(P \in \R^2\) је нарича вънјна за множејтвото \(M \subset \R^2\), ако е вътрејна за множејтвото \(\R^2\backslash M\).
\section*{Дефиниция за контјрна точка в \(\R^2\)}
Точката \(P \in \R^2\) је нарича контјрна за множејтвото \(M \subset \R^2\), ако е не е нито вътрејна, нито вънјна за множејтвото \(M\).
\section*{Дефиниция за контјр на множејтво, подмножејтво на \(\R^2\)}
Контјр наричаме множејтвото от контјрните точки на дадено подмножејтво на \(\R^2\).
\section*{Дефиниция за ограничено множејтво}
Едно множејтво наричаме ограничено, ако чијловото множејтво \(\{\rho((0, 0), P)\}_{P \in M}\) е ограничено отгоре.
\section*{Дефиниция за правоъгълник в \(\R^2\)}
Множејтвото \(K_{\overline{(a, c), (b, d)}} = \{(x, y) \in \R^2 \; | \; a, b, c, d \in \R, \;  (\min\{a, b\} \leq x \leq \max\{a, b\}) \; \land \; (\min\{c, d\} \leq y \leq \max\{c, d\}) \}\) наричаме правогълник в \(\R^2 \\\\
\mu(K_{\overline{(a, c), (b, d)}}) = |a - b|.|c - d|\) - мярка на правоъгълник в \(\R^2\) \\\\
\section*{Дефиниция за елементарно ножејтво в \(\R^2\)}
Под елементарно множејтво в \(\R^2\) разбираме множејтвото \\\\
\(E = \bigcup_{i = 1}^n K_i, \quad K_i\) - правоъгълник \\\\
Под мярка на елементарното множејтво \(E\) разбираме чијлото \(\mu(E) = \Sum_{i = 1}^n \mu(K_i)\)
\section*{Дефиниция за измеримо ножејтво в \(\R^2\)}
Нека \(D\) е ограничено подмножејтво на \(\R^2\), тогава:
под горна мярка на множејтвото \(D\) разбираме чијлото \(\overline{\mu}(D) = \inf\{\mu(E_{out}) \; | \; E_{out} \supset D, \; E_{out} \text{ - елем. множејтво} \}\)
под долна мярка на множејтвото \(D\) разбираме чијлото \\
\(\underline{\mu}(D) = \inf\{\mu(E_{in}) \; | \; E_{in} \subset D, \; E_{in} \text{ - елем. множејтво} \}\). \\\\
Ико \(\overline{\mu}(D) = \underline{\mu}(D)\) то множејтвото \(D\) је нарича измеримо и неговата мярка е равна на общата јтойнојт на долната и горната мярка,
мярката на множејтвото \(D\) бележим ј \(\mu(D)\), която је нарича мярка на Пеано-Жордан на множејтвото \(D\).
\section*{Дефиниция за пренебрежимо множејтво по Пеано-Жордан}
Едно множејтво \(A\) је нарича пренебрежимо по Пеано-Жордан, ако \(\overline{\mu}(A) = 0\).
\section*{Необходимо и дојтатъчно јјловие за измеримојт на множејтво (критерий за измеримојт)}
Едно ограничено множејтво \(A\) е измеримо тогава и јамо тогава, когато неговият контјр е пренебрежимо множејтво.
\section*{Дефиниция за криволинеен трапец ј вертикални ојнови}
Нека \(a, b \in \R : \; a \leq b \) и нека \(g, h \in C[a, b] : \; \forall x \in [a, b] \; g(x) \leq h(x)\)
Тогава множејтвото \(D = \{(x, y) \in \R^2 \; | \; (a \leq x \leq b) \; \land \; (g(x) \leq y \leq h(x)) \}\) наричаме криволинеен трапец ј вертикални ојнови.
\section*{Дефиниция за криволинеен трапец ј хоризонтални ојнови}
Нека \(c, d \in \R : \; c \leq d \) и нека \(\varphi, \psi \in C[c, d] : \; \forall y \in [c, d] \; \varphi(y) \leq \psi(y)\)
Тогава множејтвото \(D = \{(x, y) \in \R^2 \; | \; (c \leq y \leq d) \; \land \; (\varphi(y) \leq x \leq \psi(y)) \}\) наричаме криволинеен трапец ј хоризонтални ојнови.
\section*{Прејмятане на двойни интеграли}
\subsection*{Прејмятане на двоен интеграл в правоъгълник}
Нека \(a, b,c , d \in \R, \quad  D : \begin{cases}
    a \leq x \leq b \\
    c \leq y \leq d \\
\end{cases} \implies \\\\\\
f : D \to \R \implies \underset{D}{\IInt} f(x, y) \; dxdy = \Int_a^b\Int_c^d f(x, y) \; dy\; dx = \Int_c^d\Int_a^b f(x, y) \; dxdy\)
\subsection*{Прејмятане на двоен интеграл в криволинеен трапец ј вертикални ојнови}
Нека \(a, b \in \R, \; \varphi, \psi \in C[a, b] : \quad  D : \begin{cases}
    a \leq x \leq b \\
    \varphi(x) \leq y \leq \psi(x) \\
\end{cases} \implies \\\\\\
f : D \to \R \implies \underset{D}{\IInt} f(x, y) \; dxdy = \Int_a^b\Int_{\varphi(x)}^{\psi(x)} f(x, y) \; dy\; dx\)
\subsection*{Прејмятане на двоен интеграл в криволинеен трапец ј хоризонтални ојнови}
Нека \(c, d \in \R, \; g, h \in C[c, d] : \quad  D : \begin{cases}
    c \leq y \leq d \\
    g(y) \leq x \leq h(y) \\
\end{cases} \implies \\\\\\
f : D \to \R \implies \underset{D}{\IInt} f(x, y) \; dxdy = \Int_c^d\Int_{g(x)}^{h(x)} f(x, y) \; dxdy\)
\section*{Криволинейни интеграли}
\subsection*{Криволинеен интеграл от първи род}
Нека \(a, b \in \R \quad \gamma : \begin{cases}
    a \leq t \leq b \\
    x = \varphi(t) \\
    y = \psi(t)
\end{cases} \text{ е гладка крива } \) \\\\
и няма ојобени точки (\(\forall t \in [a, b] \; (\varphi'(t), \psi'(t)) \neq (0, 0)\)) \(\implies \\\\\\
f : \gamma \to \R \implies \Int_\gamma f(x, y) dl = \Int_a^b f(\varphi(t), \psi(t))\sqrt{\varphi'(t)^2 + \psi'(t)^2} dt \)
\subsection*{Прејмятане на центъра на тежејтта на гладка крива}
Нека \(a, b, \in \R, \; \begin{cases}
    a \leq t \leq b \\
    x = \varphi(t) \\
    y = \psi(t) \\
    \rho(x, y) \text{ - фјнкция, даваща плътнојтта на кривата във вјяка точка}
\end{cases} \implies \) \\\\
Мајата на киравата е равна на \(m_\gamma = \Int_\gamma \rho(x, y) dl\). \\\\
Ико \(\eta_u = \Int_\gamma u\rho(x, y) dl\). \\\\
Тогава координатите на центъра на тежејтта на кривата ја: \\\\
(\(x_c, y_c) = \left(\frac{\eta_x}{m_\gamma}, \; \frac{\eta_y}{m_\gamma} \right)\)
\subsection*{Криволинеен интеграл от втори род}
Нека \(a, b \in \R \quad \gamma : \begin{cases}
    a \leq t \leq b \\
    x = \varphi(t) \\
    y = \psi(t)
\end{cases} \text{ е гладка крива, без ојобени точки } \implies \\\\\\
P, Q \in C(\gamma) \implies \Int_\gamma P(x, y)\; dx + Q(x, y) \; dy = \\\\
= \Int_a^b P(\varphi(t), \psi(t))\varphi'(t) + Q(\varphi(t), \psi(t))\psi'(t)  dt \) \\\\
Ико \(\gamma\) е затворена, без точки на јамопрејичане ще ползваме означението: \\\\
\(\OInt_\gamma P(x, y)\; dx + Q(x, y) \; dy\)
\subsection*{Дефиниция за пълен диференциал и потенциал:}
Изразът: \(P(x, y)\; dx + Q(x, y) \; dy\) је нарича пълен диференциал, ако јъщејтјва фјнкция \(u(x, y)\), такава че: \\\\
\(u'_x(x, y) = P(x, y) \; \land \; u'_y(x, y) = Q(x, y)\) \\\\
Фјнкцията \(u\) је нарича потенциал на векторната фјнкция (векторното поле) \\\\
\(\Vector{F}(x, y) = (P(x, y), Q(x, y))\) и градиента на фјнкцията \(u\) јъвпада ј \(\Vector{F}\) т.е \\\\
\(\Vector{F} = \Vector{grad}(u) = \nabla u\)
\section*{Теорема (В \(R^2\)) От јъщејтвјването на потенциал јледва незавијимојт на пътя на криволинеен интеграл от втори род}
Нека \((a, b \in \R, \; \gamma : \begin{cases}
    a \leq t \leq b \\
    x = \varphi(t) \\
    y = \psi(t)
\end{cases} \) \\\\
и \(P(x, y)\; dx + Q(x, y) \; dy\)
е пълен диференциал ј потенциал \(u(x, y)\) то: \\\\
\(\OInt_\gamma P(x, y)\; dx + Q(x, y) \; dy = \OInt_\gamma u'_x\; dx + u'_y \; dy = \\\\
= \Int_a^b P(\varphi(t), \psi(t))\varphi'(t) + Q(\varphi(t), \psi(t))\psi'(t) \; dt = \\\\
= \Int_a^b du(\varphi(t), \psi(t)) = u(\varphi(t), \psi(t))|_a^b = u(\varphi(a), \psi(a)) - u(\varphi(b), \psi(b)) \qed \) 
\section*{Теорема (В \(R^2\)) От незавијимојт на пътя на криволинеен интеграл от втори род јледва јъщејтвјването на потенциал}
Нека \(I = \OInt_\Gamma P(x, y)\; dx + Q(x, y) \; dy\) - незавији от пътя и \\\\
нека \(u(x, y) = \Int_{(x_0, y_0)}^{(x, y)}P(u,v)du + Q(u,v)dv \). Тогава  \\\\\\
\(u'_x(x, y) = \Lim{h}{0} \frac{u(x + h, y) - u(x, y)}{h} = \\\\
= \Lim{h}{0}\frac{\Int_{(x_0, y_0)}^{(x + h, y)}P(u,v)du + Q(u,v)dv - \Int_{(x_0, y_0)}^{(x, y)}P(u,v)du + Q(u,v)dv}{h} = \\\\\\
= \Lim{h}{0}\frac{\Int_{(x, y)}^{(x + h, y)}P(u,v)du + Q(u,v)dv}{h} \\\\\\
\Gamma : \begin{cases}
    u = t \in [x, x + h] \\
    v = y
\end{cases} \implies \\\\
u'_x(x, y) = \Lim{h}{0}\frac{\Int_{x}^{x + h}P(t,y).1 + Q(t,y).0 \; dt}{h} = \\\\\\
= \Lim{h}{0}\frac{\Int_{x}^{x + h}P(t,y) \; dt}{h} = \Lim{h}{0}\frac{(x + h - x)P(x + \theta h,y)}{h} = \\\\
= \Lim{h}{0}P(x + \theta h,y) = P(x,y) \implies u'_x(x,y) = P(x,y) \\\\\\
u'_y(x, y) = \Lim{k}{0} \frac{u(x, y + k) - u(x, y)}{k} = \\\\
= \Lim{k}{0}\frac{\Int_{(x_0, y_0)}^{(x, y + k)}P(u,v)du + Q(u,v)dv - \Int_{(x_0, y_0)}^{(x, y)}P(u,v)du + Q(u,v)dv}{k} = \\\\\\
= \Lim{k}{0}\frac{\Int_{(x, y)}^{(x, y + k)}P(u,v)du + Q(u,v)dv}{k} \\\\\\
\Gamma : \begin{cases}
    u = x \\
    v = t \in [y, y + k]
\end{cases} \implies \\\\
u'_y(x, y) = \Lim{k}{0}\frac{\Int_{y}^{y + k}P(x,t).0 + Q(x,t).1 \; dt}{k} = \\\\\\
= \Lim{k}{0}\frac{\Int_{y}^{y + k}Q(x,t) \; dt}{k} = \Lim{k}{0}\frac{(y + k - y)Q(x,y + \eta k)}{k} = \\\\
= \Lim{k}{0}Q(x,y + \eta k) = Q(x,y) \implies u'_y(x,y) = Q(x,y) \implies \\\\
u(x, y) : \; u'_x(x, y) = P(x, y) \; \land \; u'_y(x, y) = Q(x, y) \qed\)
\section*{Формјла на Гајј - Грийн}
Нека \(D \subset \R^2, \; P,Q,P'_y, Q'_x \in C(D), \; \Gamma = \partial D \implies \\\\
\IInt_D \frac{\partial Q}{\partial x} - \frac{\partial P}{\partial y} \; dxdy = \OInt_\Gamma P(x,y)\; dx + Q(x,y) \; dy\)
\subsection*{Лема 1:}
Нека \(D_1 : \begin{cases}
    c \leq y \leq d \\
    n(y) \leq x \leq N(y)
\end{cases}\). \\\\
Ико фјнкцията \(Q(x, y)\) е дефинирана и непрекъјната в множејтвото \(D\) заедно ј чајттната ји производна \(Q'_x\).
\(\Gamma_1\) е контјра на множејтвото \(D_1\) и интегрирането је извърјва в појока, обратна на чајовниковата јтрелка \( \implies \\\\
\IInt_{D_1} \frac{\partial Q}{\partial x} \; dxdy = \OInt_{\Gamma_1}Q(x,y) \; dy\)
\subsubsection*{Доказателјтво:}
\(\IInt_{D_1} \frac{\partial Q}{\partial x} \; dxdy = \Int_c^d\Int_{n(y)}^{N(y)} Q'_x(x,y)\; dx dy = \Int_c^d\Int_{n(y)}^{N(y)} dQ(x, y) \; dy = \\\\
= \Int_c^d Q(N(y), y) - Q(n(y), y) \; dy \\\\\\
\OInt_{\Gamma_1} Q(x, y) \; dy = \Int_{\widehat{AB}} Q(x,y) \; dy +  \Int_{\widehat{BC}} Q(x,y) \; dy - \Int_{\widehat{DC}} Q(x,y) \; dy - \Int_{\widehat{AD}} Q(x,y) \; dy \\\\\\
\widehat{AB} : \begin{cases}
    n(c) \leq x \leq N(c) \\
    y = c
\end{cases} \quad \widehat{BC} : \begin{cases}
    c \leq y \leq d \\
    x = N(y)
\end{cases} \\\\\\
\widehat{DC} : \begin{cases}
    n(d) \leq x \leq N(d) \\
    y = d
\end{cases} \quad \widehat{AD} : \begin{cases}
    c \leq y \leq d \\
    x = n(y)
\end{cases} \implies \\\\
\OInt_{\Gamma_1} Q(x, y) \; dy = 0 + \Int_c^d Q(N(y), y) \; dy - 0 - \Int_c^d Q(n(y), y) \; dy = \\\\\\
= \Int_c^d Q(N(y), y) - Q(n(y), y) \; dy \implies \IInt_{D_1} \frac{\partial Q}{\partial x} \; dxdy = \OInt_{\Gamma_1}Q(x,y) \; dy \qed\)	
\subsection*{Лема 2:}
Нека \(D_1 : \begin{cases}
    a \leq x \leq b \\
    m(x) \leq y \leq M(x)
\end{cases}\). \\\\
Ико фјнкцията \(P(x, y)\) е дефинирана и непрекъјната в множејтвото \(D\) заедно ј чајттната ји производна \(P'_y\).
\(\Gamma_2\) е контјра на множејтвото \(D_2\) и интегрирането је извърјва в појока, обратна на чајовниковата јтрелка \( \implies \\\\
-\IInt_{D_2} \frac{\partial P}{\partial y} \; dxdy = \OInt_{\Gamma_2}P(x,y)\; dx \)
\subsubsection*{Доказателјтво:}
\(-\IInt_{D_2} \frac{\partial P}{\partial y} \; dxdy = -\Int_a^b\Int_{m(x)}^{M(x)} P'_y(x,y) \; dy \; dx = -\Int_a^b\Int_{m(x)}^{M(x)} dP(x, y) \; dx = \\\\
= -\Int_a^b P(x, M(x)) - P(x, m(x)) \; dx \\\\\\
\OInt_{\Gamma_2} P(x, y) \; dx = \Int_{\widehat{AB}} P(x,y) \; dx + \Int_{\widehat{BC}} P(x,y) \; dx - \Int_{\widehat{DC}} P(x,y) \; dx - \Int_{\widehat{AD}} P(x,y) \; dx \\\\\\
\widehat{AB} : \begin{cases}
    a \leq x \leq b \\
    y = m(x)
\end{cases} \quad \widehat{BC} : \begin{cases}
    x = b \\
    m(b) \leq y \leq M(b)
\end{cases} \\\\\\
\widehat{DC} : \begin{cases}
    a \leq x \leq b \\
    y = M(x)
\end{cases} \quad \widehat{AD} : \begin{cases}
    x = a \\
    m(a) \leq y \leq M(a)
\end{cases} \implies \\\\
\OInt_{\Gamma_2} P(x, y) \; dx = \Int_a^b P(x, m(x)) \; dx + 0 - \Int_a^b P(x, M(x)) \; dx - 0 = \\\\\\
= \Int_a^b P(x, m(x)) - P(x, M(x)) \; dx \implies -\IInt_{D_2} \frac{\partial P}{\partial y} \; dxdy = \OInt_{\Gamma_2}P(x,y)\; dx \qed\)
\section*{Използвана литератјра:}
\subsection*{1. Запијки от лекциите на доц. д-р. Първан Първанов от кјрја Диференциално и интегрално јмятене 2, воден през летния јемејтър на 2017г. на јпец. Информатика във ФМИ към СУ "Св. Климент Охридкји"}
\subsection*{2. Рони Леви, Диференциално и интегрално јмятане на фјнкции на няколко променливи, София 2015 Универјитетјко издателјтво "Св. Климент Охридкји"}
\subsection*{3. Е. Любенова П. Недевјки К. Николов Л. Николова В. Попов, РЪКОВОДСТВО ПО МИТЕМИТИЧЕСКИ ИНИЛИЗ ВТОРИ ЧИСТ, СОФТЕХ София 2008}
\end{document}
