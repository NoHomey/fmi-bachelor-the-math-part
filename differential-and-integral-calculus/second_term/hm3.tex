\documentclass[14pt]{extarticle}
\usepackage{amsmath}
\usepackage{amssymb}
\usepackage{amsthm}
\usepackage{stmaryrd}
\usepackage{tikz}
\usepackage[T1,T2A]{fontenc}
\usepackage[utf8]{inputenc}
\usepackage[bulgarian]{babel}
\usepackage[normalem]{ulem}
\usepackage[margin=0.5in, top=1cm, left=1in]{geometry}

\newcommand{\N}{\mathbb{N}}
\newcommand{\R}{\mathbb{R}}
\newcommand{\Sum}{\displaystyle\sum_{n = 1}^\infty}
\newcommand{\Lim}[1]{\displaystyle\lim_{y \to #1}}

\title{Домашна работа 3. Вариант 2, № 45342, Група 3}
\author{Иво Стратев}

\begin{document}
\maketitle
\section{Задача 1.}
Дадена е фунцкията \(f(x, y) = (x^2 -xy)e^{-y}\)
\subsection{а) Да се изследва \(f(x, y)\) за локални екстремуми.}
Решение: \\\\
\(f(x, y) = (x^2 -xy)e^{-y} = xe^{-y}(x - y) \\\\
f_x'(x, y) = e^{-y}(x - y) + xe^{-y} = e^{-y}(2x - y) \\\\
f_y'(x, y) = -(x^2 -xy)e^{-y} -xe^{-y} = xe^{-y}(y - x - 1) \\\\
\begin{cases}
    f_x'(x, y) = 0 \\\\
    f_y'(x, y) = 0
\end{cases} \implies \begin{cases}
    e^{-y}(2x - y) = 0 \\\\
    xe^{-y}(y - x - 1) = 0
\end{cases} \implies \begin{cases}
    2x - y = 0 \\\\
    x(y - x - 1) = 0
\end{cases} \implies \\\\\\\\
\begin{cases}
    y = 2x  \\\\
    2x(2x - x - 1) = 0
\end{cases} \implies \begin{cases}
    y = 2x  \\\\
    x(x - 1) = 0
\end{cases} \implies \{(0, 0), \; (1, 2)\} \\\\\\
f_{xx}''(x, y) = 2e^{-y} \\\\ 
f_{xy}''(x, y) = -e^{-y}(2x - y) - e^{-y} = e^{-y}(y - 2x - 1) \\\\
f_{yx}''(x, y) = e^{-y}(y - x - 1) - xe^{-y} =  e^{-y}(y - 2x - 1) \\\\
f_{yy}''(x, y) = -xe^{-y}(y - x - 1) + xe^{-y} = xe^{-y}(x - y + 2) \\\\
\Delta(x,y) = \begin{vmatrix}
    f_{xx}''(x, y) & f_{xy}''(x, y) \\\\
    f_{yx}''(x, y) & f_{yy}''(x, y)
 \end{vmatrix} = 
 \begin{vmatrix}
    2e^{-y} & e^{-y}(y - 2x - 1) \\\\
    e^{-y}(y - 2x - 1) & xe^{-y}(x - y + 2)
 \end{vmatrix} = \\\\\\
 = 2xe^{-2y}(x - y + 2) - e^{-2y}(y - 2x - 1)^2 = \\\\
 = e^{-2y}(2x^2 - 2xy + 4x) + e^{-2y}(-y^2 + 4xy + 2y - 4x^2 - 4x - 1) = \\\\
 = e^{-2y}(2xy - y^2 - 2x^2 + 2y - 1) \\\\
f_{xx}''(1, 2) = \frac{2}{e^2} > 0 \\\\
\Delta(1, 2) = e^{-4}(4 - 4 - 2 + 4 - 1) = \frac{1}{e^4} > 0 \implies (1, 2) \) е локален минимум \\\\
\(\Delta(0, 0) = -1 < 0 \implies (0, 0) \) е седлова точка
\subsection{б) Да се изследва \(f(x, y)\) за глобални екстремуми в множеството \(D = \{(x, y) \in \R^2 \; | \; x^2 -xy + y^2 \leq 3\}\) }
Решение: \\\\
\((x^2 - xy)\) е непрекъсната фунцкия, като полином на две променливи е дефинирана навсякъде в \(\R^2\) и
\(\frac{1}{e^y}\) е дефинирана и непрекъсната навсякъде в \(\R\). Тоест \(f(x, y)\) е непрекъсната и дефинирана навсякъде в \(\R^2\) и множеството \(D\)
е компакт тогава от теоремата на Вайерщрас за \(f(x,y) \implies \\\\
\exists \; \min\{f(x, y) \; | \; (x, y) \in D\} \; \wedge \; \exists \; \max\{f(x, y) \; | \; (x, y) \in D\}\) \\\\
От изследването ни досега знаем, че \(f(x, y)\) има локален минимум в \((1, 2) \\\\
1^2 - 2 + 4 = 3 \leq 3 \implies (1, 2) \in D\).  \\\\
От теоремата на Вайерщрас знаем, че глобалния минимум и максимум в компакт се реализират или във вътрешна точка или по контура на компактното множество.
Тоест остава ни да разледаме \(f(x, y)\) в множеството: \\\\
\(\overline{D} = \{(x, y) \in \R^2 \; | \; x^2 -xy + y^2 = 3\} = \{(x, y) \in \R^2 \; | \; x^2 -xy = 3 - y^2\}\). \\\\
За целта разглеждаме функцията: \(g(y) = f(x, y)|_{\overline{D}} = (3 - y^2)e^{-y} \\\\
g'(y) = -e^{-y}(3 - y^2) - 2ye^{-y} = e^{-y}(y^2 - 2y - 3) \implies \\\\
g'(y) = 0 \iff y^2 - 2y - 3 = 0 \iff (y + 1) (y - 3) = 0 \iff y = - 1 \; \vee \; y = 3 \) \\\\
\(g'(y)\): \begin{tikzpicture}
        \draw(0,0) -- (6,0);
        \draw (2 cm,3pt) -- (2 cm,-3pt);
        \draw (4 cm,3pt) -- (4 cm,-3pt);
        \draw (1,0) node[above=-3pt] {$ + $};
        \draw (2,0) node[below=3pt] {$ -1 $};
        \draw (3,0) node[above=-3pt] {$ - $};
        \draw (4,0) node[below=3pt] {$ 3 $};
        \draw (5,0) node[above=-3pt] {$ + $};
\end{tikzpicture} \(\implies\) \(g(y)\): \begin{tikzpicture}
        \draw(0,0) -- (6,0);
        \draw (2 cm,3pt) -- (2 cm,-3pt);
        \draw (4 cm,3pt) -- (4 cm,-3pt);
        \draw (1,0) node[above=-3pt] {$ \nearrow $};
        \draw (2,0) node[below=3pt] {$ -1 $};
        \draw (3,0) node[above=-3pt] {$ \searrow $};
        \draw (4,0) node[below=3pt] {$ 3 $};
        \draw (5,0) node[above=-3pt] {$ \nearrow $};
\end{tikzpicture} \\\\\\
\(\implies g(-1)\) е локален максимум, а \(g(3)\) е локален минимум. \\\\
\(\Lim{\infty} g(y) = \Lim{\infty} (3 - y^2)e^{-y} = \Lim{\infty} \frac{3 - y^2}{e^y} = 0 \\\\
\Lim{-\infty} g(y) = \Lim{-\infty} (3 - y^2)e^{-y} = \Lim{-\infty} (3 - y^2)e^y = -\infty \implies \\\\
g(y) \) е не ограничена при \(y \to -\infty\) \\\\
За \(g(y)\) като произведение на непрекъснати фунцкии, дефинирани навсякъде в \(R\) можем да търсим глобалния
минимум и максимум само в подходящ краен и затворен интервал, в който да приложим теоремата на Вайерщрас. Тогава: \\\\
\(\forall (x, y) \in D \; x^2 -xy + y^2 \leq 3 \implies x^2 -2x\frac{y}{2} + 4\left(\frac{y}{2}\right)^2 \leq 3 \implies \\\\
0 \leq \left(x - \frac{y}{2}\right)^2 + \frac{3}{4}y^2 \leq 3 \implies 0 \leq \left(x - \frac{y}{2}\right)^2 \leq 3 - \frac{3}{4}y^2 \implies 0 \leq 3 \left(1 - \frac{1}{4}y^2\right) \implies \\\\
0 \leq 1 - \frac{1}{4}y^2 \implies y^2 \leq 4 \implies |y| \leq 2 \implies y \in [-2, 2] \) \\\\
Следователно търсим глобален минимум и максимум на \(g(y)\) в интервала \([-2, 2]: \\
3 \notin [-2, 2] \\\\
g(-1) = (3 - 1)e^1 = 2e \\\\
g(2) = (3 - 4)e^{-2} = - \frac{1}{e^2} \\\\
g(-2) = (3 - 4)e^2 = -e^2 \\\\
f(1, 2) = 1e^{-2}(1 - 2) = - \frac{1}{e^2} \implies \\\\
\min\{f(x, y) \; | \; (x, y) \in D\} = g(-2) = -e^2 \\\\
\max\{f(x, y) \; | \; (x, y) \in D\} = g(-1) = 2e \\\\
3 - y^2 = x^2 -xy \)\\\\
При \(y = -1 \implies 2 = x^2 + x \implies \\\\
x^2 + x -2 = 0 \implies x = 1 \; \vee \; x = -2\) \\\\
При \(y = -2 \implies -1 = x^2 + 2x \implies \\\\
x^2 + 2x + 1 = 0 \implies x = -1\)
\subsection*{Отговор:}
Глобалният минимум на \(f(x, y) \) за \((x, y) \in D\) е: \(-e^2\) и той се достига в точката с координати \((-1, -2)\). \\\\
Глобалният максимум на \(f(x, y) \) за \((x, y) \in D\) е: \(2e\) и той се достига в точките с координати \((1, -1), \; (-2, -1)\)
\subsection*{Проверка чрез използването на множители на Лагранж}
Нека \(h(x, y) = x^2 -xy + y^2 - 3\) \\\\
Тогава разглеждаме Лагранжиана: \\\\
\(\Phi(x, y, \lambda) =  (\Lambda(f, h, \lambda))(x, y) = f(x, y) + \lambda h(x, y) = \\\\
= (x^2 -xy)e^{-y} + \lambda(x^2 -xy + y^2 - 3) = (x^2 -xy)(e^{-y} + \lambda) + \lambda y^2 = 3\lambda \\\\
\Phi'_x(x, y, \lambda) = (e^{-y} + \lambda)(x, y) + x(e^{-y} + \lambda) = (e^{-y} + \lambda)(2x - y) \\\\
\Phi'_y(x, y, \lambda) = f'_y(x, y) + (\lambda(x^2 -xy + y^2 - 3))'_y = xe^{-y}(y - x - 1) + \lambda (2y - x) \\\\
\Phi'_\lambda(x, y, \lambda) = x^2 -xy + y^2 - 3 \\\\
\begin{cases}
\Phi'_x(x, y, \lambda) = 0 \\
\Phi'_y(x, y, \lambda) = 0 \\
\Phi'_\lambda(x, y, \lambda) = 0
\end{cases} \implies
\begin{cases}
(e^{-y} + \lambda)(2x - y) = 0 \\
xe^{-y}(y - x - 1) + \lambda (2y - x) = 0 \\
x^2 -xy + y^2 - 3 = 0
\end{cases} \implies \\\\\\
(e^{-y} + \lambda) = 0 \; \vee \; (2x - y) = 0\) \\\\
Ако \(e^{-y} + \lambda = 0 \implies 
\begin{cases}
\lambda = -e^{-y} \\
xe^{-y}(y - x - 1) - e^{-y}(2y - x) = 0 \\
x^2 -xy + y^2 - 3 = 0
\end{cases} \implies \\\\
\begin{cases}
\lambda = -e^{-y} \\
e^{-y}(yx - x^2 - x) = e^{-y}(2y - x) \\
x^2 -xy + y^2 - 3 = 0
\end{cases} \implies
\begin{cases}
\lambda = -e^{-y} \\
yx - x^2 = 2y \\
x^2 -xy + y^2 - 3 = 0
\end{cases} \implies \\\\
\begin{cases}
\lambda = -e^{-y} \\
x^2 = xy - 2y \\
xy - 2y -xy + y^2 - 3 = 0
\end{cases} \implies
\begin{cases}
\lambda = -e^{-y} \\
x^2 = xy - 2y \\
y^2 -2y - 3 = 0
\end{cases} \implies 
\begin{cases}
\lambda = -e^{-y} \\
x^2 = xy - 2y \\
(y - 3)(y + 1) = 0
\end{cases} \implies \\\\
\begin{cases}
\lambda = -e^{-3} \\
x^2 -3x + 6 = 0\\
y = 3
\end{cases} \; \bigvee \quad \begin{cases}
\lambda = -e \\
x^2 + x - 2 = 0\\
y = -1
\end{cases} \implies \\\\\\
\begin{cases}
\lambda = -e^{-3} \\
x \notin \R \\
y = 3
\end{cases} \; \bigvee \quad \begin{cases}
\lambda = -e \\
(x - 1)(x + 2) = 0 \\
y = -1
\end{cases} \implies
\begin{cases}
\lambda = -e \\
x = -2 \\
y = -1
\end{cases} \; \bigvee \quad \begin{cases}
\lambda = -e \\
x = 1 \\
y = -1
\end{cases}\) \\\\\\
Ако \(2x - y = 0 \implies
\begin{cases}
y = 2x \\
xe^{-2x}(2x - x - 1) + \lambda (4x - x) = 0 \\
x^2 -2x^2 + 4x^2 - 3 = 0
\end{cases} \implies \\\\\\
\begin{cases}
y = 2x \\
xe^{-2x}(x - 1) + 3\lambda x = 0 \\
3x^2 = 3
\end{cases} \implies \\\\\\
\begin{cases}
y = 2x \\
xe^{-2x}(x - 1) + 3\lambda x = 0 \\
|x| = 1
\end{cases}  \implies \\\\\\
\begin{cases}
y = -2 \\
2e^{2} - 3\lambda = 0 \\
x = -1
\end{cases}  \; \bigvee \quad \begin{cases}
y = 2 \\
0 - 3\lambda = 0 \\
x = 1
\end{cases} \implies \begin{cases}
y = -2 \\
\frac{2}{3}e^{2} = \lambda \\
x = -1
\end{cases}  \; \bigvee \quad \begin{cases}
y = 2 \\
\lambda = 0 \\
x = 1
\end{cases} \implies \\\\\\
\{(-2, -1), \; (1, -1), \; (-1, - 2), \; (1, 2)\}\)
\end{document}