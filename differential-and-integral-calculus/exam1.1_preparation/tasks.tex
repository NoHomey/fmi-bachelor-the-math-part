\documentclass{article}
\usepackage{amsmath}
\usepackage{amssymb}
\usepackage{stmaryrd}
\usepackage{tikz}
\usepackage[T1,T2A]{fontenc}
\usepackage[utf8]{inputenc}
\usepackage[bulgarian]{babel}
\usepackage[normalem]{ulem}
\newcommand{\pto}[2]{\xrightarrow[#1 \to #2]{}}
\newcommand{\xto}[1]{\pto{x}{#1}}
\newcommand{\R}[0]{\mathbb{R}}
\newcommand{\spc}[0]{\quad}
\newcommand{\e}[0]{\varepsilon}
\newcommand{\seq}[1]{\{#1_n\}_{n=1}^{\infty}}
\newcommand{\cntrdcn}[0]{\lightning}
\newcommand{\limk}[0]{\displaystyle\lim_{k \to \infty}}

\title{Първата задача доц. Първанов я показа вчера на дъската, 2-рата е от мен :) Последното нещо е уравнение на допирателна към графика на фунцкия и връзката й с полинома на Тейлор}
\author{Иво Стратев}

\begin{document}
    \pagenumbering{gobble}
    \maketitle
    \section{}
    \(f: [0, 6) \to \R\\
    f(0) = 2\\
    f(1) = 1\\
    \displaystyle\lim_{x \to 6 - 0} f(x) = 3\\
    1. \exists \min f(x), \; x \in [0, 6)\\
    2. \text{Графика на фунцкия, която няма максимум}\)\\\\
    \begin{tikzpicture}
        \draw(-1,0) -- (7,0);
        \draw(0,-1) -- (0,4);
        \draw(0,2) to[out=20,in=-90] (1,1);
        \draw(1,1) to[out=20,in=-100] (6,3);
        \node [circle,draw,inner sep=0pt,minimum width=0.3cm] at (6,3) {}; 
        \draw [dashed] (-1,2.5) -- (7,2.5);
        \draw [dashed] (-1,3) -- (7,3);
        \draw [dashed] (-1,3.5) -- (7,3.5);
        \draw [dashed] (5.5,-1) -- (5.5,4);
        \draw [dashed] (6,-1) -- (6,4);
        \foreach \x in {0,1,2,3,4,5,6}
            \draw (\x cm,3pt) -- (\x cm,-3pt) (\x, 0) node[below=3pt] {$ \x $};
        \foreach \x in {1,2,3}
            \draw (-3pt, \x cm) -- (3pt, \x cm) (0, \x) node[left=3pt] {$ \x $};
    \end{tikzpicture}\\
    \(\exists \delta < 1\; \forall x \in (6 - \delta, 6) \; f(x) \in (2,5,3,5)\\
    x > 6 - \delta \implies f(x) > 2,5\\
    \implies x \in [0, 6 - \delta] \; \exists \min f(x) \leq 1\)
    \section{}
    \(f: [0, 6) \to \R\\
    f(0) = 4\\
    f(4) = 2\\
    \displaystyle\lim_{x \to 6 - 0} f(x) = 1\\
    1. \exists \max f(x), \; x \in [1, 6)\\
    2. \text{Графика на фунцкия, която няма минимум}\)\\\\
    \begin{tikzpicture}
        \draw(-1,0) -- (7,0);
        \draw(0,-1) -- (0,5);
        \draw(0,4) to[out=20,in=-100] (4,2);
        \draw(4,2) to[out=20,in=100] (6,1);
        \node [circle,draw,inner sep=0pt,minimum width=0.3cm] at (6,1) {}; 
        \draw [dashed] (-1,1.5) -- (7,1.5);
        \draw [dashed] (-1,1) -- (7,1);
        \draw [dashed] (-1,0.5) -- (7,0.5);
        \draw [dashed] (5.5,-1) -- (5.5,5);
        \draw [dashed] (6,-1) -- (6,5);
        \foreach \x in {0,1,2,3,4,5,6}
            \draw (\x cm,3pt) -- (\x cm,-3pt) (\x, 0) node[below=3pt] {$ \x $};
        \foreach \x in {1,2,3,4}
            \draw (-3pt, \x cm) -- (3pt, \x cm) (0, \x) node[left=3pt] {$ \x $};
    \end{tikzpicture}\\
    \(\exists \delta < 1\; \forall x \in (6 - \delta, 6) \; f(x) \in (0,5,1,5)\\
    x >  6 - \delta \implies f(x) < 1,5\\
    \implies x \in [0, 6 - \delta] \; \exists \max f(x) \geq 4\)
    \section{Уравнение на допирателна към графика на фунцкия}
    \subsection{Полином на Тейлор}
    \((T_n(a))(x) = \displaystyle\sum_{k = 0}^n \frac{f^{(k)}(a)}{k!}(x - a)^k\)
    \subsection{Уравнение на допирателна към графика на фунцкия}
    \(y = T_1(x_0) = f(x_0) + f'(x_0)(x - x_0)\)
\end{document}