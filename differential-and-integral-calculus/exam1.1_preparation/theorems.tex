\documentclass{article}
\usepackage{amsmath}
\usepackage{amssymb}
\usepackage{stmaryrd}
\usepackage[T1,T2A]{fontenc}
\usepackage[utf8]{inputenc}
\usepackage[bulgarian]{babel}
\usepackage[normalem]{ulem}
\newcommand{\pto}[2]{\xrightarrow[#1 \to #2]{}}
\newcommand{\xto}[1]{\pto{x}{#1}}
\newcommand{\nto}[0]{\pto{n}{\infty}}
\newcommand{\R}[0]{\mathbb{R}}
\newcommand{\spc}[0]{\quad}
\newcommand{\e}[0]{\varepsilon}
\newcommand{\seq}[1]{\{#1_n\}_{n=1}^{\infty}}
\newcommand{\cntrdcn}[0]{\lightning}

\title{Теоремки за средни стойности. Връзка между знак на първа производна и монотоност на функция. + Теоремка за константа функция}
\author{Иво Стратев}

\begin{document}
    \pagenumbering{gobble}
    \maketitle
    \section{Th(Ферма)}
    Нека \(f(x)\) е дефинирна в \(x_0\) - лок. екст. и \(f(x)\) е диференцируема в \(x_0\), то \(f'(x_0) = 0\)
    \bigbreak
    Д-во:
    \subsection{\(x_0\) е т. лок. макс.}
    \(\exists \delta > 0; \; \forall x \in (x_0 - \delta, x_0 + \delta) \spc f(x) \leq f(x_0)\\
    \implies f(x) - f(x_0) \leq 0 \spc \forall x \in (x_0 - \delta, x_0 + \delta)\)
    \subsubsection{\(x > x_0\)}
    \(x > x_0 \implies x - x_0 > 0 \spc \forall x \in (x_0 - \delta, x_0 + \delta)\\
    \implies \displaystyle\lim\limits_{\substack{x \to x_0 \\ x > x_0}} \frac{f(x) - f(x_0)}{x - x_0} \leq 0\)
    \subsubsection{\(x < x_0\)}
    \(x < x_0 \implies x - x_0 < 0 \spc \forall x \in (x_0 - \delta, x_0 + \delta)\\
    \implies \displaystyle\lim\limits_{\substack{x \to x_0 \\ x < x_0}} \frac{f(x) - f(x_0)}{x - x_0} \geq 0\)\\
    От 1.1.1 и 1.1.2 \(\implies \displaystyle\lim_{x \to x_0} \frac{f(x) - f(x_0)}{x - x_0} = f'(x_0) = 0\)
    \subsection{\(x_0\) е т. лок. мин.}
    \(\exists \delta > 0; \; \forall x \in (x_0 - \delta, x_0 + \delta) \spc f(x) \geq f(x_0)\\
    \implies f(x) - f(x_0) \geq 0 \spc \forall x \in (x_0 - \delta, x_0 + \delta)\)
    \subsubsection{\(x > x_0\)}
    \(x > x_0 \implies x - x_0 > 0 \spc \forall x \in (x_0 - \delta, x_0 + \delta)\\
    \implies \displaystyle\lim\limits_{\substack{x \to x_0 \\ x > x_0}} \frac{f(x) - f(x_0)}{x - x_0} \geq 0\)
    \subsubsection{\(x < x_0\)}
    \(x < x_0 \implies x - x_0 < 0 \spc \forall x \in (x_0 - \delta, x_0 + \delta)\\
    \implies \displaystyle\lim\limits_{\substack{x \to x_0 \\ x < x_0}} \frac{f(x) - f(x_0)}{x - x_0} \leq 0\)\\
    От 1.2.1 и 1.2.2 \(\implies \displaystyle\lim_{x \to x_0} \frac{f(x) - f(x_0)}{x - x_0} = f'(x_0) = 0\)
    \section{Th(Рол)}
    Нека \(f(x) \in C[a, b]\) и \(f(x)\) е диференцируема в \((a, b)\) и \(f(a) = f(b)\\
    \exists c \in (a, b); \; f'(c) = 0\)
    \bigbreak
    Д-во:
    \subsection{\(f(x) \equiv const\)}
    \(f'(x) = 0\)
    \subsection{\(f(x) \not\equiv const\)}
    От Th(Вайерщрас) \(\implies\)\\
    \(\exists x_{min}, x_{max} \in [a, b];\\
    f(x_{max}) = max\{f(x) : \; x \in [a, b]\}\\
    f(x_{min}) = min\{f(x) : \; x \in [a, b]\}\)
    \subsubsection{\(x_{min}, x_{max} \notin (a, b)\)}
    \(x_{min} \neq x_{max}, f(a) = f(b) \implies minf(x) = maxf(x)\\
    \implies f(x) \equiv const \implies \cntrdcn (f(x) \not\equiv const)\)
    \subsubsection{Ако поне една от \(x_{min}\) или \(x_{max} \in (a, b)\)}
    Ако \(x_{max} \in (a, b), c = x_{max}\) в противен случай \(c = x_{min}\\
    \implies c \in (a, b)\) - лок. екст. от Th(Ферма)\\
    \(\implies f'(c) = 0\)
    \section{Th(За крайните нараствания на Лагранж)}
    Нека \(f(x) \in C[a, b]\) и \(f(x)\) е диференцируема в \((a, b)\\
    \exists c \in (a, b); \; f(b) - (a) = f'(c)(b - a)\)
    \bigbreak
    Д-во:
    \bigbreak
    \(h(x) = f(x) - kx\\
    h(a) = f(a) - ka = f(b) - kb = h(b)\\
    k; \; f(a) - ka = f(b) - kb\\
    kb - ka = f(b) - f(a)\\
    k(b - a) = f(b) - f(a)\\
    k = \frac{f(b) - f(a)}{b - a}\\
    \text{От Th(Рол) за} \; h(x) \implies \exists c \in (a, b); \; h'(c) = 0\\
    h'(c) = f'(c) - k = 0\\
    \implies f'(c) = k = \frac{f(b) - f(a)}{b - a}\\
    \implies f'(c)(b - a) = f(b) - f(a)\)
    \section{Обобщена Th(За крайните нараствания на Коши)}
    Нека \(f(x), g(x) \in C[a, b]\) и \(f(x), g(x)\) са диференцируеми в \((a, b),\\
    \text{като } g'(x) \neq 0 \; \forall x \in (a, b)\\
    \exists c \in (a, b); \; \frac{f(b) - (a)}{g(b) - g(a)} = \frac{f'(c)}{g'(c)}\)
    \bigbreak
    Д-во:
    \bigbreak
    Коректност:
    \bigbreak
    Допс., че \(g(b) - g(a) = 0\) то от Th(Рол)\\
    \(\implies \exists c_2 \in (a, b); \; g'(c_2) = 0\\
    \implies \cntrdcn (g'(x) \neq 0 \; \forall x \in (a, b))\)
    \smallbreak
    \(h(x) = f(x) - kg(x)\\
    k; \; h(a) = f(a) - kg(a) = f(b) - kg(b) = h(b)\\
    f(a) - kg(a) = f(b) - kg(b)\\
    kg(b) - kg(a) = f(b) - f(a)\\
    k(g(b) - g(a)) = f(b) - f(a)\\
    k = \frac{f(b) - f(a)}{g(b) - g(a)}\\
    \text{От Th(Рол) за} \; h(x) \implies \exists c \in (a, b); \; h'(c) = 0\\
    h'(c) = f'(c) - kg'(x) = 0\\
    \implies f'(c) = kg'(c) = \frac{f(b) - f(a)}{g(b) - g(a)}g'(c)\\
    \implies \frac{f'(c)}{g'(c)} = \frac{f(b) - f(a)}{g(b) - g(a)}\)
    \section{Th(\(f(x)\uparrow \; \in C(a, b) \iff f'(x) \geq 0 \; \forall x \in (a, b)\))}
    \subsection{Th(\(f(x)\uparrow \; \in C(a, b) \implies f'(x) \geq 0 \; \forall x \in (a, b)\))}
    \(t \in (a, b)\)
    \subsubsection{\(x > t\)}
    \(x > t, \; f(x)\uparrow \implies f(x) \geq f(t)\\
    \implies f(x) - f(t) \geq 0, \; x - t > 0\\
    \implies \displaystyle\lim_{x \to t} \frac{f(x) - f(t)}{x - t} = f'(t) \geq 0\)
    \subsubsection{\(x < t\)}
    \(x < t, \; f(x)\uparrow \implies f(x) \leq f(t)\\
    \implies f(x) - f(t) \leq 0, \; x - t < 0\\
    \implies \displaystyle\lim_{x \to t} \frac{f(x) - f(t)}{x - t} = f'(t) \geq 0\)
    \smallbreak
    От 5.1.1 и 5.1.2 \(\implies \forall x \in (a, b) f'(x) \geq 0\)
    \subsection{Th(\(f(x) \in C(a, b); \; f'(x) \geq 0 \; \forall x \in (a, b) \implies f(x)\uparrow\))}
    \(f(x)\uparrow \implies \forall x_1, x_2 \in (a, b); \; x_1 < x_2 \implies f(x_1) \leq f(x_2)\)\\
    Доп., че \(\exists t_1, t_2 \in (a, b); \; t_1 < t_2; \; f(t_1) > f(t_2)\)\\
    От Th(Лагранж за крайните нараствания) \(\implies \exists c \in (t_1, t_2);\\
    f(t_2) - f(t_1) = f'(c)(t_2 - t_1)\\
    f'(x) \geq 0 \; \forall x \in (a, b) \implies f'(c) \geq 0\\
    t_1 < t_2 \implies t_2 - t_1 > 0\\
    \implies f(t_2) - f(t_1) \geq 0 \implies f(t_2) \geq f(t_1) \implies \cntrdcn (f(t_1) > f(t_2))\\
    \implies f(x) \in C(a, b); \; f'(x) \geq 0 \; \forall x \in (a, b) \implies f(x)\uparrow\)
    \section{Th(\(f(x)\downarrow \; \in C(a, b) \iff f'(x) \leq 0 \; \forall x \in (a, b)\))}
    \subsection{Th(\(f(x)\downarrow \; \in C(a, b) \implies f'(x) \leq 0 \; \forall x \in (a, b)\))}
    \(t \in (a, b)\)
    \subsubsection{\(x > t\)}
    \(x > t, \; f(x)\downarrow \implies f(x) \leq f(t)\\
    \implies f(x) - f(t) \leq 0, \; x - t > 0\\
    \implies \displaystyle\lim_{x \to t} \frac{f(x) - f(t)}{x - t} = f'(t) \leq 0\)
    \subsubsection{\(x < t\)}
    \(x < t, \; f(x)\downarrow \implies f(x) \geq f(t)\\
    \implies f(x) - f(t) \geq 0, \; x - t < 0\\
    \implies \displaystyle\lim_{x \to t} \frac{f(x) - f(t)}{x - t} = f'(t) \leq 0\)
    \smallbreak
    От 6.1.1 и 6.1.2 \(\implies \forall x \in (a, b) f'(x) \leq 0\)
    \subsection{Th(\(f(x) \in C(a, b); \; f'(x) \leq 0 \; \forall x \in (a, b) \implies f(x)\downarrow\))}
    \(f(x)\downarrow \implies \forall x_1, x_2 \in (a, b); \; x_1 < x_2 \implies f(x_1) \geq f(x_2)\)\\
    Доп., че \(\exists t_1, t_2 \in (a, b); \; t_1 < t_2; \; f(t_1) < f(t_2)\)\\
    От Th(Лагранж за крайните нараствания) \(\implies \exists c \in (t_1, t_2);\\
    f(t_2) - f(t_1) = f'(c)(t_2 - t_1)\\
    f'(x) \leq 0 \; \forall x \in (a, b) \implies f'(c) \leq 0\\
    t_1 < t_2 \implies t_2 - t_1 > 0\\
    \implies f(t_2) - f(t_1) \leq 0 \implies f(t_2) \leq f(t_1) \implies \cntrdcn (f(t_1) < f(t_2))\\
    \implies f(x) \in C(a, b); \; f'(x) \leq 0 \; \forall x \in (a, b) \implies f(x)\downarrow\)
    \section{\(f'(x) = 0 \; \forall x \in (a, b) \implies f(x) \equiv const \; \forall x \in (a, b)\)}
    Нека \(f(x)\) е дефинирна и диференцируема в \((a, b)\) и\\
    \(f'(x) = 0 \; \forall x \in (a, b) \implies f(x) \equiv const \; \forall x \in (a, b)\)
    \bigbreak
    Д-во:\\
    \(x_0 \in (a, b)\)\\
    От Th(Лагранж за крайните нараствания)\\
    \(\implies \exists c \in (a, b); f(x) - f(x_0) = f'(c)(x - x_0) \; \forall x \in (a, b)\\
    f'(x) = 0 \; \forall x \in (a, b) \implies f'(c) = 0 \implies f(x) - f(x_0) = 0\\
    \implies f(x) = f(x_0) \; \forall x \in (a, b) \implies f(x) \equiv const \; \forall x \in (a, b)\) 
\end{document}