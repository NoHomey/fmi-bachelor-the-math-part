\documentclass[12pt]{article}
\usepackage{amsmath}
\usepackage{amssymb}
\usepackage{amsthm}
\usepackage{cancel}
\usepackage{stmaryrd}
\usepackage[T1,T2A]{fontenc}
\usepackage[utf8]{inputenc}
\usepackage[bulgarian]{babel}
\usepackage[normalem]{ulem}
\usepackage[margin=0.5in, top=1cm, left=0.5in]{geometry}
\newcommand{\stkout}[1]{\ifmmode\text{\sout{\ensuremath{#1}}}\else\sout{#1}\fi}
\newcommand{\Z}{\mathbb{Z}}
\newcommand{\N}{\mathbb{N}}
\newcommand{\Q}{\mathbb{Q}}
\newcommand{\R}{\mathbb{R}}
\newcommand{\Rev}{\mathbb{U}}

\newcommand\tab[1][1cm]{\hspace*{#1}}

\DeclareMathSizes{12}{14}{12}{8}

\title{Домашна работа 2, №45342, група 3, Информатика}
\author{Иво Стратев}

\begin{document}
    \pagenumbering{gobble}
    \maketitle
    \section*{Задача 1.}
    Нека \(R = \{a, b,c, d\}\) е пръстен със следните таблици за събиране и умножение: \\\\
    \begin{tabular}{ l | c c c c  }
	  + & a & b & c & d \\ \hline
	  a & d & c & b & a \\
	  b & c & d & a & b \\
	  c & b & a & d & c \\
	  d & a & b & c & d
	\end{tabular} \quad \quad \quad
	\begin{tabular}{ l | c c c c  }
	  * & a & b & c & d \\ \hline
	  a & ~ & d & d & d \\
	  b & ~ & ~ & ~ & d \\
	  c & a & b & ~ & d \\
	  d & d & d & d & d
	\end{tabular} \\\\
	Да се попълнят празните места в таблицата за умножение и да се опишат всички идеали на \(R\)
	\subsection*{Решение:}
	\(cc = c(a + b) = ca + cb = a + b = c\) \\\\
	\begin{tabular}{ l | c c c c  }
	  + & a & b & c & d \\ \hline
	  a & d & c & b & a \\
	  b & c & d & a & b \\
	  c & b & a & d & c \\
	  d & a & b & c & d
	\end{tabular} \quad \quad \quad
	\begin{tabular}{ l | c c c c  }
	  * & a & b & c & d \\ \hline
	  a & ~ & d & d & d \\
	  b & ~ & ~ & ~ & d \\
	  c & a & b & c & d \\
	  d & d & d & d & d
	\end{tabular} \\\\
	\(aa = a(c + b) = ac + ab = d + d = d\) \\\\
	\begin{tabular}{ l | c c c c  }
	  + & a & b & c & d \\ \hline
	  a & d & c & b & a \\
	  b & c & d & a & b \\
	  c & b & a & d & c \\
	  d & a & b & c & d
	\end{tabular} \quad \quad \quad
	\begin{tabular}{ l | c c c c  }
	  * & a & b & c & d \\ \hline
	  a & d & d & d & d \\
	  b & ~ & ~ & ~ & d \\
	  c & a & b & c & d \\
	  d & d & d & d & d
	\end{tabular} \\\\
	\(ba = (c + a)a = ca + aa = a + d = a\) \\\\
	\begin{tabular}{ l | c c c c  }
	  + & a & b & c & d \\ \hline
	  a & d & c & b & a \\
	  b & c & d & a & b \\
	  c & b & a & d & c \\
	  d & a & b & c & d
	\end{tabular} \quad \quad \quad
	\begin{tabular}{ l | c c c c  }
	  * & a & b & c & d \\ \hline
	  a & d & d & d & d \\
	  b & a & ~ & ~ & d \\
	  c & a & b & c & d \\
	  d & d & d & d & d
	\end{tabular} \\\\
	\(bb = (c + a)b = cb + ab = b + d = b\) \\\\
	\begin{tabular}{ l | c c c c  }
	  + & a & b & c & d \\ \hline
	  a & d & c & b & a \\
	  b & c & d & a & b \\
	  c & b & a & d & c \\
	  d & a & b & c & d
	\end{tabular} \quad \quad \quad
	\begin{tabular}{ l | c c c c  }
	  * & a & b & c & d \\ \hline
	  a & d & d & d & d \\
	  b & a & b & ~ & d \\
	  c & a & b & c & d \\
	  d & d & d & d & d
	\end{tabular} \\\\
	\(bc = (c + a)c = cc + ac = c + d = c\) \\\\
	\begin{tabular}{ l | c c c c  }
	  + & a & b & c & d \\ \hline
	  a & d & c & b & a \\
	  b & c & d & a & b \\
	  c & b & a & d & c \\
	  d & a & b & c & d
	\end{tabular} \quad \quad \quad
	\begin{tabular}{ l | c c c c  }
	  * & a & b & c & d \\ \hline
	  a & d & d & d & d \\
	  b & a & b & c	 & d \\
	  c & a & b & c & d \\
	  d & d & d & d & d
	\end{tabular}
	\subsection*{Идеали на \(R\):}
	Очевидно \(d\) е нулевия елемент на \(R\)
	\subsubsection*{Тривиалните идеали на \(R: \{d\}, \; R\)}
	\subsubsection*{Нетривиални идеали на \(R\):}
	\(\begin{cases}
	a + d, \; d + a \in \{d, a\} \\
	\forall r \in R \; ra, \; ar \in \{d, a\}  \\
	\forall r \in R \; rd = dr = d \in \{d, a\} \end{cases} \implies \{d, a\} \triangleleft R \\\\\\
	ba = a \notin \{d, b\} \implies \{d, b\} \; \cancel\triangleleft \; R \\\\
	ca = a \notin \{d, c\} \implies \{d, c\} \; \cancel\triangleleft \; R \\\\
	b + a = c \notin \{d, a, b\} \implies \{d, a, b\} \; \cancel\triangleleft \; R \\\\
	c + a = b \notin \{d, a, c\} \implies \{d, a, c\} \; \cancel\triangleleft \; R \\\\
	b + c = a \notin \{d, b, c\} \implies \{d, b, c\} \; \cancel\triangleleft \; R \\\\
	a + a = d \notin \{a, b, c\} \implies \{a, b, c\} \; \cancel\triangleleft \; R \)
	\subsubsection*{Отговор: Идеалите на \(R\) са: \(\{d\}, \; \{d, a\}, \; R\)}
	\section*{Задача 2.}
    Нека \(I = (387	) \triangleleft \Z\) и \(J = \{x \in \Z \; | \; \exists n = n(x) \in \N \; : \; x^n \in I\}\)
    \subsection*{Тв. \(J \triangleleft \Z\)}
    \subsubsection*{Док-во:}
    Нека \(x, y \in J \; (x - y) \in J \iff \exists k = k(x - y) \in \N : \; (x - y)^k \in I \\\\
    \Z \text{ е комутативен пръстен с 1 } \implies \\\\
    (x - y)^k = \displaystyle{\sum_{i = 0}^k} \binom{k}{i} x^i(-y)^{k - i} = \displaystyle{\sum_{i = 0}^k} (-1)^{k - i}\binom{k}{i} x^iy^{k - i} \\\\
    x, y \in J \implies \begin{cases}
    	\exists n = n(x) \in \N : \; x^n \in I \\
    	\exists m = m(y) \in \N : \; y^m \in I \\
    \end{cases} \\\\
    \text{При } k = n + m \; (x - y)^{n + m} = \displaystyle{\sum_{i = 0}^{n + m}} (-1)^{n + m - i}\binom{n + m}{i} x^iy^{n + m - i} = \\\\
    = \displaystyle{\sum_{i = 0}^{n - 1}} (-1)^{n + m - i}\binom{n + m}{i} x^iy^{n + m - i} + \displaystyle{\sum_{l = n}^{n + m}} (-1)^{n + m - l}\binom{n + m}{l} x^ly^{n + m - l} = \\\\
    = \displaystyle{\sum_{i = 0}^{n - 1}} (-1)^{n + m - i}\binom{n + m}{i} x^iy^{n - i}y^m + \displaystyle{\sum_{l = n}^{n + m}} (-1)^{n + m - l}\binom{n + m}{l} x^{l - n}y^{n + m - l}x^n \\\\
    \forall t \in 0, \dots, (n + m) \; \; (-1)^{n + m - t}\binom{n + m}{t} \in \Z \\\\
    \forall t \in 0, \dots, (n - 1) \; \; n - t \geq 1 \implies x^ty^{n - t} \in \Z \quad \underset{\implies}{y^m \in I} \\\\
    \forall t \in 0, \dots, (n - 1) \; \; (-1)^{n + m - t}\binom{n + m}{t} x^ty^{n - t}y^m \in I \implies \\\\
    \displaystyle{\sum_{i = 0}^{n - 1}} (-1)^{n + m - i}\binom{n + m}{i} x^iy^{n - i}y^m \in I \\\\
    \forall t \in n, \dots, (n + m) \; \; t - n \geq 0 \implies x^{t - n}y^{n + m - t} \in \Z \quad \underset{\implies}{x^n \in I} \\\\ 
    \forall t \in n, \dots, (n + m) \; \; (-1)^{n + m - t}\binom{n + m}{t} x^{t - n}y^{n + m - t}x^n \in I \implies \\\\
    \displaystyle{\sum_{l = n}^{n + m}} (-1)^{n + m - l}\binom{n + m}{l} x^{l - n}y^{n + m - l}x^n \in I \implies \\\\
    (x - y)^{n + m} \in I \implies (x - y) \in J \implies \forall a, b \in J \; a - b \in J \) \\\\
    Нека \(z \in \Z, \; j \in J \implies \exists h = h(j) \in \N : \; j^h \in I \\\\
    \Z \text{ е комутативен пръстен с 1 } \implies (zj)^h = z^hj^h \quad \underset{\implies}{z^h \in \Z} \; \in I \implies \\\\
    zj \in J \implies \forall r \in \Z, \; \forall g \in J \; rg \in J \implies J \triangleleft \Z \qed \)
    \subsection*{Тв. \(I \subset J \)}
    \subsubsection*{Док-во:}
    Нека \(k \in I \implies k^1 = k \in I \subset \Z \implies k \in J \implies \\\\
    \forall j \in I \; j \in J \implies I \subset J \qed \) \\\\
    \(387 = 3^2.43\)
    \subsection*{Тв. \(J = (3.43) \)}
    \subsubsection*{Док-во:}
    Нека \(K = (3.43) = \{3.43z \; | \; z \in \Z\} \\\\
    I = (387) = (3^2.43) = \{3^2.43z \; | \; z \in \Z\} \) \\\\
    Нека \(j \in J \implies \exists n = n(j) \in \N : \; j^n \in I \implies \exists k \in \Z : \; j^n = 3^2.43.k \implies \\\\
    3.43 | j^n \implies 3 | j \; \wedge \; 43 | j \implies 3.43 | j \implies j \in K \implies \forall a \in J \; a \in K \implies J \subseteq K\) \\\\
    Нека \(k \in K \implies \exists b \in \Z : \; k = 3.43.b \implies \\\\
    \begin{cases}
    k^1 \in I, & 3|b \\
    k^2 \in I, & 3\not|b
    \end{cases} \implies k \in J \implies \forall h \in K \; h \in J \implies K \subseteq J \implies J = K = (3.43) \qed \)
    \section*{Задача 3.}
    Нека \(I = (2 + \sqrt{-11}) \triangleleft \Z\left[\sqrt{-11}\right] = \{a + b\sqrt{-11} \; | \; a, \; b \in \Z \} \\\\
    J = \{a + b\sqrt{-11} \; | \; a, \; b \in \Z : \; 15 \; | \; b + 7a \} \)  \\\\
    Да се докаже, че: \(I = J \; \wedge \; \Z\left[\sqrt{-11}\right]/I \cong \Z_{15} \)
    \subsection*{Решение:}
    Нека \(z \in I \implies \exists a, \; b \in \Z: \; z = (a + b\sqrt{-11})(2 + \sqrt{-11}) = \\\\
    = 2a + 2b\sqrt{-11} + a\sqrt{-11} + i^211b = (2a - 11b) + (a + 2b)\sqrt{-11} \) \\\\
    Нека \(A = 2a - 11b, \; B = a + 2b \implies B + 7A = a + 2b + 14a - 77b = \\\\
    = 15a - 75b = 15(a - 5b) \implies 15 \; | \; B + 7A \implies z \in J \implies \forall r \in I \; r \in J \implies I \subseteq J \) \\\\
    Нека \(j \in J \implies \exists C, \; D \in \Z : \; j = C + D\sqrt{-11} \; \wedge \; 15 \; | \; D + 7C \implies \\\\
    \exists k \in \Z : \; D + 7C = 15k \implies D = 15k - 7C \) \\\\
    Нека \(C = 2c - 11d, \; D = c + 2d \implies C - 2D = C - 30k + 14C = \\\\
    = 15(C - 2k) = -15d \implies d = 2k - C \in \Z \implies c = D - 2d \in \Z \implies \\\\
    j = C + D\sqrt{-11} = (c + d\sqrt{-11})(2 + \sqrt{-11}) \implies j \in I \implies \forall t \in J \; t \in I \implies J \subseteq I \\\\
    \implies I = J \) \\\\
    Нека \(c, \; d \in \Z\left[\sqrt{-11}\right]: \; c = a_1 + b_1\sqrt{-11}, \; d = a_2 + b_2\sqrt{-11} \implies \\\\
    c + I = d + I \iff c - d = 0 + I = I \iff c - d \in I \iff (a_1 - a_2) + (b_1 - b_2)\sqrt{-11} \in I \\\\
    \iff 15 \; | \; (b_1 - b_2) + 7(a_1 - a_2) \iff 15 \; | \; (b_1 + 7a_1) - (b_2 + 7a_2) \\\\
    \iff (b_1 + 7a_1) \equiv (b_2 + 7a_2) \pmod {15} \iff 13(b_1 + 7a_1) \equiv 13(b_2 + 7a_2) \pmod {15} \\\\
    \iff (13b_1 + 91a_1) \equiv (13b_2 + 91a_2) \pmod {15} \iff (a_1 + 13b) \equiv (a_2 + 13b) \pmod {15} \\\\
    15 = 2.7 + 1 \implies 1 = 15 -2.7 \implies \overline{1} = \overline{15 -2.7} = \overline{15} + \overline{-2.7} = \overline{13.7} = \overline{13}.\overline{7} \\\\
    \begin{array}{cccc}
    \varphi : & \Z\left[\sqrt{-11}\right] & \to & \Z_{15} \\\\
     ~ &  a + b\sqrt{-11} & \mapsto & \overline{a + 13b}
    \end{array} \\\\\\
    \varphi(c + d) = \varphi(a_1 + b_1\sqrt{-11} + a_2 + b_2\sqrt{-11}) = \varphi((a_1 + a_2) + (b_1 + b_2)\sqrt{-11}) = \\\\
    = \overline{(a_1 + a_2) + 13(b_1 + b_2)} = \overline{a_1 + 13b_1} + \overline{a_2 + 13b_2} = \varphi(a_1 + b_1\sqrt{-11}) + \varphi(a_2 + b_2\sqrt{-11}) = \\\\
    = \varphi(c) + \varphi(d) \implies \forall u, \; v \in \Z\left[\sqrt{-11}\right] \; \varphi(u + v) = \varphi(u) + \varphi(v) \\\\
    \varphi(cd) = \varphi((a_1 + b_1\sqrt{-11})(a_2 + b_2\sqrt{-11})) = \varphi((a_1a_2 - 11b_1b2) + (a_1b_2 + b_1a_2)\sqrt{-11}) = \\\\
    = \overline{(a_1a_2 - 11b_1b2) + 13(a_1b_2 + b_1a_2)} = \overline{(a_1a_2 + 4b_1b2) + 13(a_1b_2 + b_1a_2)} = \\\\
    = \overline{(a_1a_2 + 169b_1b2) + 13(a_1b_2 + b_1a_2)} = \overline{a_1(a_2 + 13b_2) + 13b_1(a_2 + 13b_2)} = \\\\
    = \overline{(a_1 + 13b_1)(a_2 + 13b_2)} = \overline{(a_1 + 13b_1)} \; \overline{(a_2 + 13b_2)} = \varphi(a_1 + 13b_1)\varphi(a_2 + 13b_2) = \\\\
    = \varphi(c)\varphi(d) \implies \forall x, \; y \in \Z\left[\sqrt{-11}\right] \; \varphi(xy) = \varphi(x)\varphi(y) \implies \varphi \text{ е ХММ на пръстени}\\\\
    Im\varphi = \{\tau \in \Z_{15} \; | \; \exists \delta \in \Z\left[\sqrt{-11}\right] : \; \tau = \varphi(\delta)\}
    =\{\varphi(\delta) \; | \; \delta \in \Z\left[\sqrt{-11}\right]\} \implies \\\\
    Im\varphi \subseteq \Z_{15} \implies Im\varphi = \Z_{15} \iff \Z_{15} \subseteq Im\varphi \\\\
    \forall \mu \in \{0, \; \dots, \; 14\} \; \mu + 0\sqrt{-11} \in \Z\left[\sqrt{-11}\right] \; \wedge \; \varphi(\mu + 0\sqrt{-11}) = \overline{\mu + 13.0} = \overline{\mu} \implies \\\\
    \forall \alpha \in \Z_{15} \;  \exists \beta \in \Z\left[\sqrt{-11}\right]: \; \varphi(\beta) = \alpha \implies \alpha \in Im\varphi \implies \Z_{15} \subseteq Im\varphi \implies Im\varphi = \Z_{15} \\\\ Ker\varphi = \{\delta \in \Z\left[\sqrt{-11}\right] \; | \; \varphi(\delta) = \overline{0} \} = \{s + m\sqrt{-11} \in \Z\left[\sqrt{-11}\right] \; | \; \overline{s + 13m} = \overline{0} \} = \\\\
    = \{s + m\sqrt{-11} \in \Z\left[\sqrt{-11}\right] \; | \; s + 13m \equiv 0 \pmod {15} \} = \\\\
    = \{s + m\sqrt{-11} \in \Z\left[\sqrt{-11}\right] \; | \; 7s + 91m \equiv 0 \pmod {15} \} = \\\\
    = \{s + m\sqrt{-11} \in \Z\left[\sqrt{-11}\right] \; | \; m + 7s \equiv 0 \pmod {15} \} = \\\\
    \{s + m\sqrt{-11} \in \Z\left[\sqrt{-11}\right] \; | \; 15 \; | \; m + 7s \} = J = I \implies \) \\\\
    От теорема за ХММ-ите на пръстени \(\implies \Z\left[\sqrt{-11}\right]/I \cong \Z_{15} \qed \)
    \section*{Задача 4.}
    Нека \(f(x) = x^3 + \overline{2}x^2 + \overline{1} \in \Z_3[x] \\\\
    \Z_3 = \{\overline{0}, \; \overline{1}, \; \overline{2}\} \\\\
    f(\overline{0}) = \overline{1} \neq \overline{0} \\\\
    f(\overline{1}) = \overline{1} + \overline{2}.\overline{1} + \overline{1} = \overline{4} = \overline{1}	 \neq \overline{0} \\\\
    f(\overline{2}) = \overline{8} + \overline{2}.\overline{4} + \overline{1} = \overline{17} = \overline{2} \neq \overline{0} \\\\
    \implies f \text{ е неразложим над } \Z_3 \implies \Z_3[x]	/(f) \text{ е поле} \) \\\\
    Нека \(g(x) = x^2 + x + \overline{1} \in \Z_3[x] \\\\
    \tab x^3 + \overline{2}x^2 + \overline{1} \; : \; x^2 + x + \overline{1} = x + \overline{1} \\
    - \\
    \tab x^3 + x^2 + x \\
    \tab \noindent\rule{3cm}{0.4pt} \\\\
    \tab \tab x^2 + \overline{2} + 1 \\
    \tab - \\
    \tab \tab x^2 + x + \overline{1} \\
    \tab \tab \noindent\rule{2.3cm}{0.4pt} \\
    \tab \tab \tab x \) \\\\
    Нека \(q(x) = x + \overline{1}, \; r = x \in \Z_3[x] \\\\
    f = gq + r \implies r = f - gq \\\\
    \tab x^2 + x + \overline{1} \; : \; x = x + \overline{1} \\
    - \\
    \tab x^2 \\
    \tab \noindent\rule{2.2cm}{0.4pt} \\\\
    \tab \tab x + \overline{1} \\
    \tab - \\
    \tab \tab x \\
    \tab \tab \noindent\rule{1.2cm}{0.4pt} \\\\
    \tab \tab \tab \overline{1} \\\\
    \implies (f,g) = \overline{1} \\\\
    g = rq + (f,g) \implies \\\\
    (f,g) = \overline{1} = g - rq = g - (f - gq)q = g - fq + q^2g = (1 + q^2)g - fq \implies \\\\
    ((1 + q^2)g + (-q)f) + (f) = \overline{1} + (f) \implies \\\\
    ((1 + q^2)g + (f)) + ((-q)f) + (f)) = \overline{1} + (f) \implies  \\\\
    (1 + q^2)g + (f) = \overline{1} + (f) \implies \\\\
    1 + (x + \overline{1})^2 = x^2 + \overline{2}x + \overline{2} \) е обратният елемент на \(x^2 + x + \overline{1}\) в \(\Z_3[x]/(f) \)
    \subsection*{Проверка:}
    \(gg^{-1} = (x^2 + x + \overline{1})(x^2 + \overline{2}x + \overline{2}) = x^4 + \overline{2}x^3 + \overline{2}x^2 + x^3 + \overline{2}x^2 + \overline{2}x + x^2 + \overline{2}x + \overline{2} = 
    \\\\ = x^4 + \overline{2}x^2 + x + \overline{2} \\\\
    \tab x^4 + \overline{2}x^2 + x + \overline{2} \; : \;  x^3 + \overline{2}x^2 + \overline{1} = x + \overline{1} \\
    - \\
    \tab x^4 + \overline{2}x^3 + x \\
    \tab \noindent\rule{3.8cm}{0.4pt} \\\\
    \tab \tab x^3 + \overline{2}x^2 + \overline{2} \\
    \tab - \\
    \tab \tab  x^3 + \overline{2}x^2 + \overline{1} \\
    \tab \tab \noindent\rule{2.8cm}{0.4pt} \\\\
    \tab \tab \tab \tab \quad \; \overline{1} \)
    \section*{Задача 5.}
    Нека \(\Q(i) = \{a + bi \; | \; a, b \in \Q\}, \; \Z[i] = \{a + bi \; | \; a, b \in \Z\}, \\\\
    \Rev = \Z[i]^*, \quad \forall \alpha = a + bi \in \Q(i) \; N(\alpha) = a^2 + b^2\)
    \section*{а)}
    \subsection*{Тв. \(\forall \alpha \in \Q(i) \; N(\alpha) = |\alpha|^2 = \alpha\overline{\alpha} \)}
    \subsubsection*{Док-во:}
    Нека \(\alpha = a + bi \in \Q(i)\). Гледайки на \(\Q(i)\) като подпространство на \(\mathbb{C}\), на което гледаме като двумерно Евклидово пространство и следвайки дефиницията за дължина на вектор получаваме \(|\alpha| = \sqrt{<\alpha, \alpha>} = \sqrt{a^2 + b^2} \; | \; ()^2 \implies \\\\ 
 	|\alpha|^2 = (\sqrt{a^2 + b^2})^2 = a^2 + b^2 = N(\alpha) \) \\\\
    По дефиниция \(\overline{\alpha} = a - bi \implies \alpha\overline{\alpha} = (a + bi)(a - bi) = a^2 - abi + abi - b^2i^2 = \\\\
    = a^2 + b^2 = N(\alpha) \implies N(\alpha) = |\alpha|^2 = \alpha\overline{\alpha} \implies \forall \beta \in \Q(i) \; N(\beta) = |\beta|^2 = \beta\overline{\beta} \qed \)
    \subsection*{Тв. \(\forall \alpha, \; \beta \in \Q(i) \; N(\alpha\beta) = N(\alpha)N(\beta) \)}
    \subsubsection*{Док-во:}
    Нека \(\alpha = a + bi, \; \beta = c + di \in \Q(i) \; N(\alpha\beta) = N((a + bi)(c + di)) = N(ac - bd + (ad + bc)i) = \\\\
    = (ac - bd)^2 + (ad + bc)^2 = (ac)^2 - 2acbd + (bd)^2 + (ad)^2 + 2adbc + (bc)^2 = \\\\
    = a^2c^2 + b^2d^2 + a^2d^2 + b^2c^2 = a^2(c^2 + d^2) + b^2(c^2 + d^2) = (c^2 + d^2)(a^2 + b^2) = \\\\
    = N(\beta)N(\alpha) = N(\alpha)N(\beta) \implies \forall q, t \in \Q(i) \; N(qt) = N(q)N(t) \qed\)
    \subsection*{Тв. \(\forall \alpha \in \Q(i)\backslash\{0\} \; N\left(\frac{1}{\alpha}\right) = \frac{1}{N(\alpha)} \)}
    \subsubsection*{Док-во:}
    Нека \(\alpha = a + bi \in \Q(i)\backslash\{0\} \;  \frac{1}{\alpha} = \frac{\overline{\alpha}}{\alpha\overline{\alpha}} = \frac{a - bi}{N(\alpha)} = \frac{a}{N(\alpha)} - \frac{b}{N(\alpha)}i \implies \\\\
    N\left(\frac{1}{\alpha}\right) = \frac{a}{N(\alpha)}^2 + \left(- \frac{b}{N(\alpha)}\right)^2 = \frac{a^2}{N(\alpha)^2} + \frac{b^2}{N(\alpha)^2}
    = \frac{a^2 + b^2}{N(\alpha)^2} = \frac{N(\alpha)}{N(\alpha)^2} = \frac{1}{N(\alpha)} \implies \\\\
    \forall \beta \in \Q(i)\backslash\{0\} \; N\left(\frac{1}{\beta}\right) = \frac{1}{N(\beta)} \qed \)
    \subsection*{Тв. \(\forall \alpha \in \Q(i), \; \beta \in \Q(i)\backslash\{0\} \; N\left(\frac{\alpha}{\beta}\right) = \frac{N(\alpha)}{N(\beta)} \)}
    \subsubsection*{Док-во:}
    \(\forall \alpha \in \Q(i), \; \beta \in \Q(i)\backslash\{0\} \; N\left(\frac{\alpha}{\beta}\right) = N\left(\alpha\frac{1}{\beta}\right) =
    N(\alpha)N\left(\frac{1}{\beta}\right) = N(\alpha)\frac{1}{N(\beta)} = \frac{N(\alpha)}{N(\beta)} \qed \)
    \section*{б)}
    \subsection*{Тв. \(\Rev = \{\alpha \in \Z[i] \; | \; N(\alpha) = 1 \} = \{1, \; -1, \; i, \; -i\}\)}
    \subsubsection*{Док-во: \(\Rev = \{\alpha \in \Z[i] \; | \; N(\alpha) = 1 \} \)}
    Нека \(\alpha, \; \beta \in \Rev : \; \alpha\beta = 1 \implies N(\alpha\beta) = N(1) \implies \\\\
    N(\alpha)N(\beta) = 1 \iff N(\alpha) = 1 \; \wedge \; N(\beta) = 1 \implies \alpha, \; \beta \in \{z \in \Z[i] \; | \; N(z) = 1 \} \implies \\\\
    \forall u \in \Rev \; u \in \{u \in \Z[i] \; | \; N(u) = 1 \} \implies \Rev \subseteq \{z \in \Z[i] \; | \; N(z) = 1 \} \) \\\\
    Нека \(\gamma \in \{z \in \Z[i] \; | \; N(z) = 1 \} \implies N(\gamma) = \gamma\overline{\gamma} = 1 \implies \gamma, \; \overline{\gamma} \in \Rev \implies \\\\
    \forall t \in \{z \in \Z[i] \; | \; N(z) = 1 \} \; t, \; \overline{t} \in \Rev \implies \{z \in \Z[i] \; | \; N(z) = 1 \} \subseteq \Rev \implies \\\\
    \Rev = \{z \in \Z[i] \; | \; N(z) = 1 \} \qed \)
    \subsubsection*{Док-во: \(\Rev = \{1, \; -1, \; i, \; -i\} \)}
    Нека \(\alpha = a + bi \in \Z[i] \implies N(\alpha) = a^2 + b^2 = 1 \implies a^2 = 1 - b^2 = (1 - b)(1 + b) \implies \\\\
    1 - b = 1 + b \implies b = 0 \implies a^2 = 1 \implies a = \pm 1 \implies \\\\
    (a, b) = (\pm 1, 0 ) \text{ е решение } \implies (a, b) = (0, \pm 1) \text{ също е решение, защото }  \\\\
    N(\alpha) \text{ е симетричен относно } a \text{ и } b \implies \\\\
    \Rev = \{a + bi \in \Z[i] \; | \; (a, b) = (\pm 1, 0) \vee (a, b) = (0, \pm 1)\} = \\\\
    = \{\pm 1 + 0i, 0 \pm 1 i \} = \{1, \; -1, \; i, \; -i\} \qed \)
    \section*{в)}
    Казваме, че две цели гаусови числа за асоциирани и пишем: \\\\
    \(\beta \sim \alpha \iff \exists \varepsilon \in \Rev : \; \beta = \varepsilon\alpha \)
    \subsection*{Тв. \(\forall \alpha \in \Z[i] \; \beta \sim \alpha \iff N(\beta) = N(\alpha) \)}
    \subsubsection*{Док-во:}
    Нека \(\alpha, \; \beta \in \Z[i] : \; \beta \sim \alpha \iff \exists \varepsilon \in \Rev : \; \beta = \varepsilon\alpha \iff N(\beta) = N(\varepsilon\alpha) \iff  \\\\
    \iff N(\beta) = N(\varepsilon)N(\alpha) \iff N(\beta) = 1N(\alpha) \iff N(\beta) = N(\alpha) \qed \)
    \subsection*{Тв асоциираността на цели гаусови числа е релация на еквивалентност}
    \subsubsection*{Док-во:}
    \(\forall \alpha \in \Z[i] \implies \alpha = 1\alpha \implies N(\alpha) = N(1)N(\alpha) = N(\alpha) \implies \\\\
    \alpha \sim \alpha \implies \sim\) е рефлексивна \\\\
    \(\forall \alpha, \beta \in \Z[i] : \; \alpha \neq \beta, \alpha \sim \beta \implies \exists \varepsilon \in \Rev : \; \alpha = \varepsilon\beta \implies \varepsilon^{-1}\alpha = \beta \implies \\\\
    \beta \sim \alpha \implies \sim\) е симетрична \\\\
    \(\forall \alpha, \; \beta, \; \gamma \in \Z[i] : \; \alpha \sim \beta \wedge \beta \sim \gamma \implies \exists \varepsilon, \; \delta \in \Rev : \; \alpha = \varepsilon\beta \wedge \beta = \delta\gamma \\\\
    N(\varepsilon\delta) = N(\varepsilon)\N(\delta) = 1.1 = 1 \implies \varepsilon\delta \in \Rev \implies \\\\
    \alpha = \varepsilon\delta\gamma \implies \alpha \sim \gamma \implies \sim\) е транзитивна \(\implies \sim\) е релация на еквивалентност
    \subsubsection*{Следствие:}
    \(\forall \alpha \in \Z[i] \quad [\alpha] = \{\beta \in \Z[i] \; | \; \alpha \sim \beta \} = \{\varepsilon\alpha \; | \; \varepsilon \in \Rev \}\)
    \section*{г)}
    \subsection*{Тв \(\forall c, d \in \Z, \; d \neq 0 \; \exists! \; q, \;  r \in \Z : \; c = qd + r \; \wedge \; |r| \leq \frac{1}{2}d \)}
    \subsubsection*{Док-во:}
    Нека \(c, d \in \Z\) делим ги с частно и остатък и получаваме: \\\\
    \(\exists! \; q, \;  r \in \Z : \; c = qd + r \; \wedge \; 0 \leq r < d\) Сега ако \(r \leq \frac{1}{2}d \text{ полагаме } q' = q \; \wedge \; r' = r \). \\\\
    Ако \(r > \frac{1}{2}d \text{ полагаме } q' = q + 1 \; \wedge \; r' = r - d \implies |r'| = |r - d| = |d - r| < \frac{1}{2}d \implies \\\\
    c = q'd + r' \; \wedge \; |r'| \leq \frac{1}{2}d \implies \forall z, \; t \in \Z \; \exists! \; u, \; v \in \Z : \; z = ut + v \; \wedge \; |v| \leq \frac{1}{2}t \qed \)
    \subsection*{Тв. \(\forall \alpha, \; \beta \in \Z[i], \; \beta \neq 0 \; \exists q, \; r \in \Z[i] : \alpha = \beta q + r \; \wedge \; N(r) < N(\beta) \)}
    \subsubsection*{Док-во:}
    Нека \(\alpha, \; \beta \in \Z[i], \; \beta \neq 0 \\\\
    \frac{\alpha}{\beta} =  \frac{\alpha\overline{\beta}}{\beta\overline{\beta}} = \frac{\alpha\overline{\beta}}{N(\beta)}\), полагаме \(\alpha\overline{\beta} = a + bi, \; a, b \in \Z \) \\\\
    Делим \(a\) и \(b\) с частно и остатък на \(N(\beta) ползвайки горното твърдение : \; \begin{cases}
    	a = q_1N(\beta) + r_1, \; q_1, \; r_1 \in \Z : \; |r_1| \leq \frac{1}{2}N(\beta) \\\\
    	b = q_2N(\beta) + r_2, \; q_2, \; r_2 \in \Z : \; |r_2| \leq \frac{1}{2}N(\beta)
    \end{cases} \\\\
    \implies \frac{\alpha}{\beta} = \frac{q_1N(\beta) + r_1 + (q_2N(\beta) + r_2)i}{N(\beta)} = q_1 + q_2i + \frac{r_1 + r_2i}{N(\beta)} \implies \alpha =
    (q_1 + q_2i)\beta + \frac{r_1 + r_2i}{\overline{\beta}} \) \\\\
    Полагаме \(q = q_1 + q_2i \in \Z[i] \; \wedge \; r = \alpha - q\beta = \frac{r_1 + r_2i}{\overline{\beta}} \in \Z[i] \\\\
    N(r) = N\left(\frac{r_1 + r_2i}{\overline{\beta}}\right) = \frac{N(r_1 + r_2i)}{N(\overline{\beta})} = \frac{r_1^2 + r_2^2}{\overline{\beta}.\overline{\overline{\beta}}}
    =  \frac{r_1^2 + r_2^2}{\overline{\beta}\beta} = \\\\
    = \frac{r_1^2 + r_2^2}{N(\beta)} \leq \frac{ \frac{1}{4} N(\beta)^2 + \frac{1}{4} N(\beta)^2 }{N(\beta)} = \frac{1}{2}N(\beta) < N(\beta) \implies N(r) < N(\beta) \implies \\\\
    \forall \gamma, \; \delta \in \Z[i], \; \delta \neq 0 \; \exists u, \; t \in \Z[i] : \gamma = \delta u + t \; \wedge \; N(t) < N(\delta) \qed \)
    \section*{д)}
    \subsection*{Тв. \(\forall I \trianglelefteq \Z[i] \; \exists z \in I : \; I = (z)\)}
    \subsubsection*{Док-во:}
    Нека \(I \neq \{0\} \trianglelefteq \Z[i] \implies \exists z \neq 0 \in I : \forall a \in I \; N(z) \leq N(a) \implies \\\\
    \forall r \in \Z[i] \; rz \in (z) \; \wedge \; rz \in I \implies (z) \subseteq I\) \\\\
    Нека \(x \in I \; \wedge \; \exists q, \; r \in \Z[i] : \; x = zq + r \; \wedge \; N(r) \leq N(z) \implies r = x - zq \implies r \in I \) \\\\
    Ако \(r \neq 0 \implies \lightning (\forall b \in I \; N(z) \leq N(b)) \implies r = 0 \implies x = zq \implies x \in (z) \implies \\\\
    \forall h \in I \; h \in (z) \implies I \subseteq (z) \implies I = (z) \implies \forall J \trianglelefteq \Z[i] \; \exists j \in J : \; J = (j) \qed \)
    \section*{е)}
    \subsection*{Тв. Нека \(\rho, \alpha, \beta \in \Z[i] : \; \rho \notin \Rev, \; \rho = \alpha\beta \\
    N(\beta) = N(\rho) \vee N(\alpha) = N(\rho) \iff \alpha \in \Rev \vee \beta \in \Rev \) \\
    Число \(\rho\) с тези свойства ще наричаме просто в \(\Z[i]\)}
    \subsubsection*{Док-во:}
    \(N(\beta) = N(\rho) \iff \beta \sim \rho \iff \alpha \in \Rev \\\\
    N(\alpha) = N(\rho) \iff \alpha \sim \rho \iff \beta \in \Rev \qed \)
    \section*{ж)}
    \subsection*{Тв. \(\forall \alpha, \; \beta \in \Z[i] : \; \beta \; | \; \alpha \implies N(\beta) \; | \; N(\alpha) \)}
    \subsubsection*{Док-во:}
    Нека \(\alpha, \; \beta \in \Z[i] : \; \beta \; | \; \alpha \implies \exists \gamma \in \Z[i]: \; \alpha = \gamma\beta \implies \\\\
    N(\alpha) = N(\beta\gamma) = N(\beta)N(\gamma) \implies N(\beta) \; | \; N(\alpha) \implies \\\\
    \forall a, \; b \in \Z[i]: \; b \; | \; a \implies N(b) \; | \; N(a) \qed \)
    \subsection*{Тв. Нека \(\rho\) е просто в \(\Z\). Тоагава \(\rho\) е просто в \(\Z[i]\) или съществува просто \(\pi \in \Z[i] : \; \rho = \pi\overline{\pi}, \; N(\pi) = N(\overline{\pi}) = \rho\) } 
    \subsubsection*{Док-во:}
    Нека \(\rho\) не е просто в \(\Z[i] \implies \exists \alpha, \; \beta \in \Z[i]\backslash\Rev: \; \rho = \alpha\beta \implies \\\\
    N(\rho) = N(\rho + 0i) = \rho^2 = N(\alpha\beta) = N(\alpha)N(\beta) \implies N(\alpha) = N(\beta) = \rho\ \implies \\\\
    \alpha\beta = \rho = N(\alpha) \implies \beta = \frac{N(\alpha)}{\alpha} = \frac{\alpha\overline{\alpha}}{\alpha} = \overline{\alpha} \implies \rho = \alpha\overline{\alpha} \) \\\\
    Остава да докажем, че \(\alpha\) е просто в \(\Z[i]\) \\\\
    Нека \(\gamma \in \Z[i]: \; \gamma \; | \; \alpha \implies N(\gamma) \; | \; N(\alpha) \implies N(\gamma) \; | \; \rho \implies \\\\
    N(\gamma) = 1 \; \vee \;  N(\gamma) = \rho \implies \gamma \in \Rev \; \vee \; \gamma \sim \alpha \implies \alpha \) е просто в \(\Z[i] \qed \)
    \section*{з)}
    \subsection*{Тв. \(\forall z \in \Z \; z^2 \equiv 0 \pmod 4 \; \vee \; z^2 \equiv 1 \pmod 4\) } 
    \subsubsection*{Док-во:}
    Нека \(z \in \Z\) \\\\
    Ако \(z \equiv 0 \pmod 4 \implies z^2 \equiv 0 \pmod 4\) \\\\
    Ако \(z \equiv 1 \pmod 4 \implies z^2 \equiv 1 \pmod 4\) \\\\
    Ако \(z \equiv 2 \pmod 4 \implies z^2 \equiv 2^2 \equiv 0 \pmod 4\) \\\\
    Ако \(z \equiv 3 \pmod 4 \implies z^2 \equiv 3^2 \equiv 1 \pmod 4 \qed \)
    \subsection*{Тв. Нека \(p\) е просто в \(\Z: \; p \equiv 3 \pmod 4 \implies \not\exists \; x, \; y \in \Z: \; x^2 + y^2 = p\) } 
    \subsubsection*{Док-во:}
    \(\forall x, \; y \in \Z, \; z = x^2 + y^2 \implies z \equiv 0 \pmod 4 \; \vee \; z \equiv 1 \pmod 4 \; \vee \; z \equiv 2 \pmod 4 \implies \\\\
    z \not\equiv 3 \pmod 4 \implies z \not\equiv p \pmod 4 \implies \not\exists \; a, \; b \in \Z: \; a^2 + b^2 = p \qed \)
    \section*{и)}
    \subsection*{Тв. Нека \(p\) е просто в \(\Z: \; p \equiv 3 \pmod 4 \implies p\) е просто и в \(\Z[i]\) } 
    \subsubsection*{Док-во:}
    Нека \(\beta = a + bi \in \Z[i]\) е прост делител на \(p \implies \exists \alpha \in \Z[i]: \; p = \alpha\beta \implies \\\\
    N(p) = N(p + 0i) = p^2 = N(\alpha\beta) = N(\alpha)N(\beta) \implies N(\beta) = p \; \vee \; N(\beta) = p^2 \implies \\\\
    N(\beta) = p^2 \; (a^2 + b^2 \neq p) \implies N(\alpha) = 1 \implies \alpha \in \Rev \implies \\\\
    \beta \sim p \implies \beta \) е просто в \(\Z[i] \qed \)
    \section*{к)}
    \subsection*{Тв. Нека \(p\) е просто в \(\Z: \; p \equiv 1 \pmod 4 \implies \\\\
    \left(\left(\frac{p - 1}{2}\right)!\right)^2 \equiv - 1 \pmod p \; \wedge \; \exists m \in \N : \; p \; | \; (m^2 + 1) \) } 
    \subsubsection*{Док-во:}
    \(p \equiv 1 \pmod 4 \implies \exists k \in \N : \; p = 4k + 1 \implies \frac{p - 1}{2} = 2k \in \N\) \\\\
    От теоремата на Уилсън получаваме: \((p - 1)! \equiv (4k)! \equiv (2k)! \displaystyle\prod_{n = 2k + 1}^{4k}n \equiv \displaystyle\prod_{n = 1}^{2k}n\displaystyle\prod_{n = 1}^{2k}(p - n) \equiv\\\\
    \equiv \displaystyle\prod_{n = 1}^{2k}n\displaystyle\prod_{n = 1}^{2k} (-n) \equiv (-1)^{2k} \left(\displaystyle\prod_{n = 1}^{2k}n\right)^2
    \equiv \left((2k)!\right)^2 \equiv -1 \pmod p \implies \\\\
    \left((2k)!\right)^2 + 1 \equiv 0 \pmod p \implies p \; \Big| \; \left(\left((2k)!\right)^2 + 1\right) \qed \)
    \section*{л)}
    \subsection*{Тв. Нека \(p\) е просто число \(\implies \sqrt{p} \in \R\backslash\Q \) } 
    \subsubsection*{Док-во:}
    Допс. противното: нека \(\sqrt{p} = \frac{a}{b} \in \Q: \; a, \; b \in \N \implies p = \frac{a^2}{b^2} \implies pb^2 = a^2 \) \\\\
    Нека \(p^{k_1'}\displaystyle\prod_{i = 2}^m p_i'^{k_i'}\) е каноничното представяне на \(a\) и нека \\\\
    \(p^{k_1''}\displaystyle\prod_{i = 2}^n p_i''^{k_i''}\) е каноничното представяне на \(b\) тогава \\\\
    \(pp^{2k_1''}\displaystyle\prod_{i = 2}^n p_i''^{2k_i''} = p^{2k_1'}\displaystyle\prod_{i = 2}^m p_i'^{2k_i'} \implies p^{2k_1'' + 1} = p^{2k_1'} \implies 2k_1'' + 1 = 2k_1' \\\\
    \implies 2(k_1' - k_1'') = 1 \implies k_1' - k_1'' = \frac{1}{2}, \; k_1', \; k_1'' \in \N \cup \{0\} \implies \\\\
    k_1' - k_1'' \in \Z \implies \frac{1}{2} \in \Z  \implies \lightning \implies \sqrt{p} \in \R\backslash\Q \qed \)
    \subsection*{Тв. Нека \(p\) е просто в \(\Z: \; p \equiv 1 \pmod 4, \; m \in \N: \; p \; | \; (m^2 + 1) \) \\
    Нека \(P = \{a \in \N \; | \; 0 \leq a < \sqrt{p} \}, \;  M = \{x + my \; | \; x, \; y \in P \} \implies \\
    \exists x_1 + my_1, \; x_2 + my_2 \in M: x_1 + my_1 \neq  x_2 + my_2, \; x_1 + my_1 \equiv x_2 + my_2 \pmod p \) \\
    Нека \(x = x_1 - x_2, \; y = y_1 - y_2 \implies p \; | \; (x + my) \; \wedge \; p \; | \; (x^2 + y^2) \; \wedge \; p = x^2 + y^2 \implies \\
    \exists \pi \text{ - просто } \in \Z[i] : \; p = \pi\overline{\pi}, \; N(\pi) = N(\overline{\pi}) = p, \; \not\exists \; \varepsilon \in \Rev : \; \overline{\pi} = \varepsilon\pi \) }
    \subsubsection*{Док-во:}
    \(\forall r \in \R \; \lfloor r \rfloor := \max\{m \in \Z \; | \; m \leq r \}, \; [[r]] := r - \lfloor r \rfloor \implies r = \lfloor r \rfloor + [[r]] \) \\\\
    \(p\) е просто число \(\implies \sqrt{p} \in \R\backslash\Q \implies [[\sqrt{p}]] > 0 \implies \N \ni \lfloor \sqrt{p} \rfloor < \sqrt{p} \\\\
    |P| = 1 + \lfloor \sqrt{p} \rfloor, \; |\Z_p| = p, \; |M| = |P|^2 \implies |M| - (|\Z_p| + 1) =  \\\\
    = (1 + \lfloor \sqrt{p} \rfloor)^2 - p - 1 = 2\lfloor \sqrt{p} \rfloor + \lfloor \sqrt{p} \rfloor^2 - \sqrt{p}^2 = \\\\
    = 2\lfloor \sqrt{p} \rfloor + \lfloor \sqrt{p} \rfloor^2 - (\lfloor \sqrt{p} \rfloor + [[\sqrt{p}]])^2 = \\\\
    = 2\lfloor \sqrt{p} \rfloor - 2\lfloor \sqrt{p} \rfloor[[\sqrt{p}]] - [[\sqrt{p}]]^2 = 2\lfloor \sqrt{p} \rfloor(1 - [[\sqrt{p}]]) - [[\sqrt{p}]]^2 \) \\\\
    Допс. \quad \(2\lfloor \sqrt{p} \rfloor(1 - [[\sqrt{p}]]) - [[\sqrt{p}]]^2 \leq 0 \implies \\\\
    1 < 2\lfloor \sqrt{p} \rfloor(1 - [[\sqrt{p}]]) \leq [[\sqrt{p}]]^2 < 1 \; (0 < [[\sqrt{p}]] < 1 \; \wedge \; [[\sqrt{p}]] \ll \lfloor \sqrt{p} \rfloor) \implies \\\\
    \lightning \implies 2\lfloor \sqrt{p} \rfloor(1 - [[\sqrt{p}]]) - [[\sqrt{p}]]^2 > 0 \implies |M| > (|\Z_p| + 1) \implies \) \\\\
    от принципа на Дирихле \(\implies \\\\
    \exists x_1 + my_1, \; x_2 + my_2 \in M: x_1 + my_1 \neq  x_2 + my_2, \; x_1 + my_1 \equiv x_2 + my_2 \pmod p \implies \\\\
    p \; | \; (x_1 + my_1 - (x_2 + my_2)) \implies p \; | \; (x_1 - x_2) + m(y_1 - y_2)) \implies \\\\
    p \; | \; x + my \implies \exists k \in \N : \; x + my = kp \implies x = kp - my \implies \\\\
    x^2 + y^2 = (kp - my)^2 + y^2 = k^2p^2 - 2kpmy + m^2y^2 + y^2 = \\\\
    = k^2p^2 - 2kpmy + (m^2 + 1)y^2 \implies p \; | \; (x^2 + y^2) \\\\
    \text{Ако } x = 0 \implies x_1 = x_2 \implies my_1 \equiv my_2 \pmod p \implies (y_1, \; y_2 \in P) \\\\
    y_1 = y_2 \implies x_1 + my_1 = x_2 + my_2 \implies \lightning \implies |x| > 0 \\\\
    \text{Ако } y = 0 \implies y_1 = y_2 \implies x_1 \equiv x_2 \pmod p \implies (x_1, \; x_2 \in P) \\\\
    x_1 = x_2 \implies x_1 + my_1 = x_2 + my_2 \implies \lightning \implies |y| > 0 \\\\
    x_1, x_2 \in P \implies 0 < |x| = |x_1 - x_2| \leq \lfloor \sqrt{p} \rfloor \implies 0 < x^2 \leq \lfloor \sqrt{p} \rfloor^2 < p \\\\
    y_1, y_2 \in P \implies 0 < |y| = |y_1 - y_2| \leq \lfloor \sqrt{p} \rfloor \implies 0 < y^2 \leq \lfloor \sqrt{p} \rfloor^2 < p \implies \\\\
    0 < x^2 + y^2 < 2p \; \wedge \; p \; | \; (x^2 + y^2) \implies x^2 + y^2 = p \implies p = (x - iy)(x + iy) \implies \\\\
    \exists \pi \in \Z[i]: \; \; \pi = x + y i, \; \overline{\pi} = x - y i \implies N(\pi) = N(\overline{\pi}) = p \implies \\\\
    p \text{  не е просто в } \Z[i]  \implies \pi \text{ е просто в  } \Z[i]\\\\
    1\pi = \pi = x + y i \neq  x - y i \\\\
    -1\pi = - x - y i \neq x - y i \\\\
    i\pi = xi - y = -y + xi \neq x - y i \\\\
    -i\pi = y - xi \neq x - y i \implies \\\\ 
    \implies \pi \not\sim \overline{\pi} \implies \not\exists \; \varepsilon \in \Rev : \; \overline{\pi} = \varepsilon\pi \qed\)
    \section*{м)}
    \subsection*{Тв. Нека \(\pi \in \Z[i], \; \pi\) е просто в \(\Z[i] \iff \exists \varepsilon \in \Rev, \; \rho \in \Z[i]: \; \pi = \varepsilon\rho \implies \\
    (\rho = 1 + i \; \vee \; \rho \equiv 3 \pmod 4 \text{ е просто в } \Z \; \vee N(\rho) \equiv 1 \pmod 4 \text{ е просто в } \Z) \) } 
    \subsubsection*{Док-во:}
    Нека \(\exists \varepsilon \in \Rev, \; \rho \in \Z[i]: \; \pi = \varepsilon\rho \implies \pi \sim \rho\) такива съществуват, защото \\\\
    \(\forall z \in \Z[i] \; 1.z = z, \; 1 \in \Rev\) и нека \(\rho\) е просто в \(\Z[i] \implies \rho \; | \; N(\rho) = \rho\overline{\rho} \\\\
    \forall z \in \Z[i] \; N(z) \in \N \implies \) от основната теорема на аритметиката, получаваме, че \(\rho\) дели някое просто число.
    Нека означим това просто число с \(r\) тогава нека \(\exists \tau \in \Z[i]: \; r = \tau\rho\). \\\\
    \(\implies N(r) = N(\tau\rho) = N(\tau)N(\rho) \implies N(\rho) = r \; \vee \; N(\rho) = r^2\) \\\\
    Aко \(r = 2 \implies N(\rho) = 2 \; \vee \; N(\rho) = 4 \) \\\\
    Ако \(N(r) = 4 \implies \rho \sim 2 = (1 + i)(1 - i) = (-1 + i)(-1 -i) \implies \) \\\\
    2 не е просто \(\implies N(\rho) = 2 \implies \rho \in [1 + i] \implies \rho \) е просто в \(\Z[i]\) \\\\
    Ако \(r \equiv 3 \pmod 4 \overset{\text{от з)}}{\implies} r \) е просто в \(\Z[i] \; \wedge \; \rho \; | \; r \implies \\\\
    \rho \sim r \implies \rho \) е просто в \(\Z[i]\) \\\\
    Ако \(r \equiv 1 \pmod 3 \) Ако \(N(\rho) = r^2 \) от доказаното в л) ще следва к) от където ще следва, че \(\rho\) не е просто в \(
    \Z[i] \implies N(\rho) = r \implies \) от ж) получаваме, че \(\rho\) е просто в \(\Z[i] \qed\)
\end{document}
