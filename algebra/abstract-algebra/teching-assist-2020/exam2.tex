\documentclass[17pt]{extarticle}
\usepackage[utf8]{inputenc}
\usepackage[T2A]{fontenc}
\usepackage[english,bulgarian]{babel}
\usepackage{amsmath}
\usepackage{amssymb}
\usepackage{amsthm}
\usepackage{mathabx}
\usepackage{relsize}
\usepackage{euler}
\usepackage[a4paper, portrait, margin=1.8cm]{geometry}

\begin{document}
\section*{Задача 1.}
Намерете всички двойки от естествено число \(n\) и просто число \(p\), такива че \(\varphi(\varphi(n)) = p^2\).
\section*{Задача 2.}
\begin{itemize}
    \item Нека \(G\) образува група. Нека \(a, b \in G\). Докажете, че редовете на елементите \(a\) и \(b^{-1}ab\) съвпадат;
    \item Нека \(G\) образува група. Нека \(a \in G\) и нека \(a\) е единственият елемент от ред 2. Докажете, че \(a\) комутира с всеки елемент на групата.
\end{itemize}
\section*{Задача 3.}
Нека \(R = \left\{ \begin{pmatrix}
    a & b \\
    -10b & a
\end{pmatrix} \mid a,b \in \mathbb Z \right\}\).
Нека \(I\) е главният идеал, породен от \(\begin{pmatrix}
    3 & 1 \\
    -10 & 3
\end{pmatrix}\).
Докажете
\begin{itemize}
    \item \(R\) образува комутативен пръстен с единица;
    \item \(I = \left\{ \begin{pmatrix}
    a & b \\
    -10b & a
\end{pmatrix} \mid a,b \in \mathbb Z \;\&\; 19 \mid b - 6a\right\}\);
    \item \(R / I \simeq \mathbb{Z}_{19}\).
\end{itemize}
\section*{Задача 4.}
Нека \(f = x^4 + ax^3 + bx^2 + cx + d \in \mathbb{R}[x]\)
има корени \(x_1, x_2, x_3, x_4\), които не са реални числа и
\(x_1 + x_2 = 1 + i\) и \(x_3x_4 = 1 - i\).
Намерете \(a, b, c\).
\end{document}
