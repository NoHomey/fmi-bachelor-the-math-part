\documentclass[12pt]{article}
\usepackage[utf8]{inputenc}

\usepackage[T2A]{fontenc}
\usepackage[english,bulgarian]{babel}
\usepackage{amsmath}
\usepackage{amssymb}
\usepackage{amsthm}
\usepackage{relsize}
\usepackage{tensor}

\title{Задачи за упражнение върху теория на числата за Информатика 2019}
\author{Иво Стратев}

\begin{document}
\maketitle

\section*{Стандартни задачи}

\subsection*{Евлид и Безу}
Да се намери Н.О.Д. на числата \(a\) и \(b\), както и числа отговарящи на тъждеството на Безу, където:

\begin{itemize}
    \item \(a = 10\) и \(b = 15\)
    \item \(a = 120\) и \(b = 70\)
    \item \(a = -100\) и \(b = 150\)
    \item \(a = -10\) и \(b = -30\)
\end{itemize}

\subsection*{Диофант 2 за 1}
Да се реши диофантовото уравнение \(ax + by = c\), където:

\begin{itemize}
    \item \(a = 10\), \(b = 15\) и \(c = 5\)
    \item \(a = 10\), \(b = 15\) и \(c = 13\)
    \item \(a = 120\), \(b = 70\) и \(c = 30\)
    \item \(a = -100\), \(b = 150\) и \(c = 50\)
    \item \(a = -10\), \(b = -30\) и \(c = 7\)
\end{itemize}

\subsection*{Сравнения}

Да се намери остатъка на \(7^7\) при \(2\), \(5\) и \(50\).

\subsubsection*{Линейни сравнения}
Да се реши сравнението \(ax \equiv b \pmod{n}\), където:


\begin{itemize}
    \item \(a = 10\), \(b = 15\) и \(n = 7\)
    \item \(a = 64\), \(b = 78\) и \(n = 5\)
    \item \(a = 10\), \(b = 15\) и \(n = 5\)
    \item \(a = 9\), \(b = 21\) и \(n = 6\)
    \item \(a = 100\), \(b = 150\) и \(n = 50\)
    \item \(a = 10\), \(b = 15\) и \(n = -7\)
\end{itemize}

\subsubsection*{Системи линейни сравнения}

Да се решат системите:
\begin{align*}
    \begin{cases}
        7x \equiv 320 \pmod{3} \\
        49x \equiv 101 \pmod{5}
    \end{cases}
\end{align*}

\begin{align*}
    \begin{cases}
        16x \equiv 120 \pmod{6} \\
        49x \equiv 78 \pmod{5}
    \end{cases}
\end{align*}

\begin{align*}
    \begin{cases}
        7x \equiv 320 \pmod{3} \\
        49x \equiv 101 \pmod{5} \\
        16x \equiv 28 \pmod{13}
    \end{cases}
\end{align*}

\begin{align*}
    \begin{cases}
        7x \equiv 20 \pmod{4} \\
        4x \equiv 8 \pmod{6} \\
        10x \equiv 30 \pmod{5}
    \end{cases}
\end{align*}

\section*{По-занимателни задачи}

\begin{enumerate}
    \item Да се определят цифрите \(a\) и \(b\), така че 105 да дели числото \(25ab5\)
    \item Да се пресметне \(\varphi(\varphi(\varphi(9!)))\)
    \item Да се реши уравнениeто \(\varphi(n) = 4\)
    \item Да се реши уравнениeто \(\varphi(n) = \displaystyle\frac{3}{2}n\)
    \item Да се намерят последните две цифри на \(33^{101} + 177^{202}\)
    \item Да се намери остатъка на числото \((5^{1000} + 13^{200})^{64}\) спрямо \(9\)
    \item Да се определи за кои прости числа \(p\) е изпълнено \(9^p \equiv 29 \pmod{p}\)
    \item Да се определи за кои прости числа \(p\): \(p^p\) дели \(20^p + 1\)
    \item Да се докаже \((\forall n \in \mathbb{N})(15 \mid 11n^8 + 34n^{16} )\)
    \item Да се определи за кои естествени числа \(n\) \(36\) дели \(5^n + 10^n\)
\end{enumerate}

\end{document}
