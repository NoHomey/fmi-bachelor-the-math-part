\documentclass[12pt]{article}
\usepackage[left=3cm,right=3cm,top=1cm,bottom=2cm]{geometry}
\usepackage{amsmath}
\usepackage{amssymb}
\usepackage[T1,T2A]{fontenc}
\usepackage[utf8]{inputenc}
\usepackage[bulgarian]{babel}
\usepackage[normalem]{ulem}

\setlength{\parindent}{0mm}

\title{Теоритично контролно №2, I, Информатика}
\author{Иван Йочев}

\begin{document}

\maketitle

\section{Действие на група върху множество}

\subsection{Определение}
Нека G - група и M - множество\\
$f : G \times M \rightarrow M$ \\
G действа върху M, ако:\\
0) $\exists f : G \times M \rightarrow M$ \\
1) $\forall g \in G\ \forall m \in M \ f(g,m) \in M$ \\
2) $\forall g_1,g_2 \in G \; \forall m \in M \; f(g_1g_2,m) = f(g_1,f(g_2,m))$\\
3) $\forall m \in M \; f(e_H,m) = m$

\subsection{Стабилизатор на елемент}
Нека G - група, M - множество и G действа върху M\\
$f : G \times M \rightarrow M$ е действие на група върху множество\\
Нека $x \in M$ \\ 
$St_G(x) = \{g \in G|f(g,x)=x\} \subseteq G $  

\subsection{Орбита на елемент}
Нека G - група и M - множество и G действа върху M\\
$f : G \times M \rightarrow M$ е действие на група върху множество\\
Нека $x \in M$ \\
$O_G(x) = \{f(g,x)|g \in G\} \subseteq M$

\subsection{Изразяване на дължина на орбита}
Нека G - група и M - множество и G действа върху M\\
Нека $x \in M$ \\
$|O_G(x)| = |G\ :\ St_G(x)|$

\subsection{Действие на група върху себе си чрез спрягане}
Нека G - група \\
$\forall g \in G$ дефинираме $\phi_g : G \rightarrow G   \\
 \phi_g(x) = gxg^{-1}$ \\
Тогава $gxg^{-1}$ наричаме спрегнат на $x$ по $g$

\subsection{Клас спрегнати елементи}
Нега G е група \\
Тогава за $x \in G \\
C_x = O_G(x) = \{gxg^{-1} \ | \ g \in G \}$\\
наричаме клас спрегнати с x елементи

\subsection{Централизатор}
Нега G е група и G действа върху себе си, чрез спрягане \\
Тогава за $x \in G \\
C_G(x) = St_G(x) = \{g \in G \ | \ gxg^{-1}=x \}$\\
наричаме централизатор на x в G

\subsection{Център на група}
Нека G e група \\
Множеството $Z(G) = \{z \in G | \ \forall g \in G \ zg = gz \}$\\
наричаме център на групата G
$Z(G) = G$, когато G e абелева

\subsection{Формула за класовете}
Нека G е група и G действа върху M - множество. \\
$|M| = n < \infty$ \\
$M = \{x_1, x_2, ..., x_n\}$ \\
$ \forall i \in \mathbb{N} \ \forall j \in \mathbb{N} \ O(x_i) \cap O(x_j) =\emptyset $ \\
$ M = \bigcup\limits_{i=1}^{n}O(x_i) $ \\
$Z(G) = \{x_{i_1},...,x_{i_t}\}, t \leq n$ \\
$|M| = \sum\limits_{i=1}^{n}|O(x_i)|$ 
$= \sum\limits_{i=1}^{n}|G : St_G(x_i)|$ \\
Ако $G \equiv M$ и G действа върху себе си, чрез спрягане\\
$|M| = |Z(G)| + \sum\limits_{j=1}^{t}|C_{x_{i_j}}|$

\subsection{Теорема на Кейли}
Нека G - група
$|G| = n < \infty$ \\
$\Rightarrow \ \exists \ G' \leq S_n : G \cong G'$

\section{Пръстени}

\subsection{Определение}
$(R, +, \cdot)$ е пръстен, ако: \\
0) $(R, +)$ е абелева група \\\\
1) $(\forall a \in R \ \forall b \in R) ab \in R$ \\ затвореност относно умножение \\ \\
2) $(\forall a \in R \ \forall b \in R \ \forall c \in R) \\ a(bc)=(ab)c=abc$ \\ асоциативност за умножение \\ \\
3) $(\forall a \in R \ \forall b \in R \ \forall c \in R) \\ (a+b)c = ac+bc \ \land a(b+c)=ab+ac$ \\ дистрибутивност за умножение

\subsection{Пръстен с единица}
$(R, +, \cdot)$ е пръстен с единица, ако: \\
0) $(R, +, \cdot)$ е пръстен \\
1) $\exists e \in R \ \forall a \in R \ ae = ea = a$

\subsection{Комутативен пръстен}
$(R, +, \cdot)$ е комутативен пръстен, ако: \\
0) $(R, +, \cdot)$ е пръстен \\
1) $\forall a \in R \ \forall b \in R : ab = ba$

\subsection{Област на цялост(област)}
Нека $(R, +, \cdot)$ - комутативен пръстен \\
R e област, ако \\
0) $R \neq \{0\}$ \\
1) Нека $a \in R$, $0_R \neq b \in R$ \\
 $(ab=0_R) \rightarrow (a=0_R) $ \\
(в R има единствен делител на нулата и това е $0_R$)

\subsection{Делител на нулата}
Нека $(R, +, \cdot)$ - комутативен пръстен, и $a \in R$\\
a е делител на нулата, ако \\
$\exists b \in R, b \neq 0_R : ab = ba = 0_R$

\subsection{Поле}
Нека $(R, +, \cdot)$ е пръстен \\
R e поле ако: \\
1) R e комутативен \\
2) R е тяло 

\subsection{Тяло}
$(R, +, \cdot)$ е тяло, ако: \\
0) $(R, +, \cdot)$ е пръстен \\
1) $\exists 0 \in R \ \land \ \exists 1 \in R \ \land \ 0_R \neq 1_R$ \\
2) $\forall a \in R \ \exists a' \in R : aa' = a'a = 1_R$

\subsection{Подпръстен}
Нека $(R, +, \cdot)$ е пръстен
и нека $S \subseteq R, S \neq \emptyset$ \\
S е подпръстен, ако
$(\forall a \in S \ \forall b \in S) \\ (a \pm b \in S) \land (ab \in S)$

\subsection{Мултипликативна група на пръстен}
Нека $(R, +, \cdot)$ е пръстен с единица \\
$R^* = \{a \in R \ \vert \ \exists a' \in R : aa' = a'a = 1_R \}$ \\
$R^*$ се нарича мултипликативна група на пръстен R

\clearpage

\section{Полета}

\subsection{Характеристика на поле}
Нека F e поле \\\\
Нека $m \in F n \in F, m \neq n$\\
Ако $m.1_R \neq n.1_R$, то $char F = 0$ \\
Ако $m.1_R = n.1_R$, то $char F = |m-n|$ \\\\
Алтернативна дефиниция:\\
$char F = |1|_{F^{+}}$

\subsection{Общ вид на характеристика на поле}
Нека F e поле \\
$char F = 0$ или $char F = p$ - просто

\subsection{Подполе}
Нека F е поле \\
Нека $K \subsetеq, F, |K| \geq 2$ \\
K e подполе ($K \leq F $), ако: \\
1) $\forall a \in K \forall b \in K \ a \pm b \in K$ \\
2) $\forall a \in K \forall b \in K \ ab \in K$ \\
3) $\forall a \in K \ a^{-1} \in K$

\subsection{Разширение на поле}
Нека F e поле и $K \leq F$ \\
F наричаме разширение на K

\subsection{Просто поле}
Нека F е поле \\
F е просто, ако\\
$(\forall K \leq F) \ K = F$

\subsection{Възможни прости подполета}
Нека F е поле \\
$P = \bigcap\limits_{K \leq F}K$ \\
P е единствено просто подполе на F

\pagebreak
\section{Хомоморфизъм на пръстени}

\subsection{Определение}
Нека R и R' са пръстени \\
Нека $\phi : R \rightarrow R'$ \\
$\phi$ е хомоморфизъм, ако \\
1) $\forall a \in R \ \forall b \in R \ \phi (a+b)=\phi (a)+ \phi (b)$\\
2) $\forall a \in R \ \forall b \in R \ \phi (ab)=\phi (a) \phi (b)$

\subsection{Изоморфизъм}
Нека R и R' са пръстени \\
Нека $\phi : R \rightarrow R'$\\
$\phi$ е изоморфизъм, ако\\
1) $\phi$ e хомоморфизъм \\
2) $\phi$ е биекция \\

\subsection{Ядро на хомоморфизъм}
Нека R и R' са пръстени \\
Нека $\phi : R \rightarrow R'$ - хомоморфизъм\\
$Ker \phi = \{a \in R \ \vert \ \phi (a) = 0_{R'}\}$

\subsection{Образ на хомоморфизъм}
Нека R и R' са пръстени \\
Нека $\phi : R \rightarrow R'$ - хомоморфизъм\\
$Im \phi = \{ \phi (a) \in R' \ \vert \ a \in R \}$

\clearpage
\section{Идеали}

\subsection{Определение за ляв(десен) идеал}
Нека R - пръстен \\
$I \trianglelefteq R$, ако \\
0) $\emptyset \neq I \subseteq R$ \\
1) $\forall a \in I \ \forall b \in I \ a-b \in I$ \\
2) $\forall a \in I \ \forall r \in R$ \\
$ra \in I $(ляв идеал) \\
$ar \in I $(десен идеал) \\

\subsection{Определение за двустранен идеал}
Нека R - пръстен \\
$I \trianglelefteq R$, ако \\
0) $\emptyset \neq I \subseteq R$ \\
1) $\forall a \in I \ \forall b \in I \ a-b \in I$ \\
2) $\forall a \in I \ \forall r \in R \ ra \in I \ \land \ ar \in I$

\subsection{Сума на идеали}
Нека R - пръстен \\
$I \trianglelefteq R, J \trianglelefteq R$ \\
$I+J = \{i+j \vert i \in I, j \in J\} \trianglelefteq R$\\
I+J се нарича сума на идеали

\subsection{Главен идеал, породен от елемент}
Нека R e комутативен пръстен с единица \\
Нека $a \in R$ \\
$(a) = \{ar \vert r \in R\}$ \\
$(a)$ се нарича главен идеал породен от елемента a

\subsection{Идеали в пръстена $\mathbb{Z}$}
Нека $I \trianglelefteq \mathbb{Z}$ \\
$\Rightarrow I = (n) = n\mathbb{Z}, n \in \mathbb{N} \cup {0_Z}$

\subsection{Събиране в факторпръстен}
Нека R - пръстен\\
$I \trianglelefteq R$ \\
Нека $a \in R$ и $b \in R$ \\
$\bar{a} = a + I$\\
$\bar{b} = b + I$\\
$\bar{a} + \bar{b} = (a+I)+(b+I)$ \\
$=(a+b)+I=\overline{a+b}$

\clearpage
\subsection{Умножение в факторпръстен}
Нека R - пръстен\\
$I \trianglelefteq R$ \\
Нека $a \in R$ и $b \in R$ \\
$\bar{a} = a + I$\\
$\bar{b} = b + I$\\
$\bar{a} \cdot \bar{b} = (a+I)(b+I)$ \\
$=(ab)+I=\overline{ab}$

\subsection{Теорема за хомоморфизмите за пръстени}
Нека R, R' - пръстени \\
Нека $\phi : R \rightarrow R'$ - хомоморфизъм на пръстени  \\
Нека Ker$\phi = I$ \\
$\Rightarrow I \trianglelefteq R$ и $R/I \cong Im\phi$

\clearpage
\section{Доказателства}

\subsection{Поле няма нетривиални идеали}
Нека R е поле \\
$ \{ 0 \} \neq I \trianglelefteq R$ \\
$a \in I, a \leq 0 \\
\Rightarrow 1 = a^{-1}a \in I$ \\
$\Rightarrow 1 \in I$ \\
$\Rightarrow R = (1) \subseteq I \ \land \ I \subseteq R$ \\
$\Rightarrow I = R$

\subsection{Ако комутативен пръстен с единица няма нетривиални идеали, той е поле}

Нека R e пръстен и R няма нетривиални идеали \\
Нека $0 \neq a \in R$ \\
$\Rightarrow (a) \neq \{ 0 \}$\\
$\Rightarrow (a) = R = (1)$ \\
$\Rightarrow \exists a' \in R : aa'=a'a=1$\\
$\Rightarrow$ a e обратим и $a^{-1} = a'$\\ 
$\Rightarrow \forall a \in R$ e обратим\\
$\Rightarrow$ R е поле

\subsection{Всяко поле съдържа едниствено просто подполе}

Нека R e поле. Ще докажем, че единственото\\
просто подполе на R e $P = \bigcap\limits_{K \leq R} K$ \\\\
Да допуснем, че $\exists X : X < P$ (P не е просто поле) \\
$\Rightarrow X < R$ \\
$(P \ = \bigcap\limits_{K \leq R} K ) \rightarrow X \supseteq P$, но по допускане $X < P$\\\\
$\Rightarrow X=P$ \\
Следователно P e единствено просто подполе

\end{document}