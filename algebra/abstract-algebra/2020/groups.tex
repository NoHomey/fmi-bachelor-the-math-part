\documentclass{article}[12pt]
\usepackage{amsmath,amsthm}
\usepackage{amssymb}
\usepackage{lipsum}
\usepackage{stmaryrd}
\usepackage[T1,T2A]{fontenc}
\usepackage[utf8]{inputenc}
\usepackage[bulgarian]{babel}
\usepackage[normalem]{ulem}
\usepackage{xcolor}

\setlength{\parindent}{0mm}

\title{Задачи за групи}
\author{Иво Стратев}

\begin{document}

\maketitle

\section*{Еквивалентна дефиниция за Група}

\((G, e, *, inv)\) е група тогава и само тогава когато:

\begin{enumerate}
\item \(G\) е непразно множество;
\item \(e \in G\);
\item \(* \; : G \times G \to G\) - бинарна операция;
\item \(inv \; : \; G \to G\) - унарна операция;
\item \((\forall a \in G)(\forall b \in G)(\forall c \in G)[\; (a * b) * c = a * (b * c) \;]\);
\item \((\forall g \in G)[\; g * e = g = e * g \;]\);
\item \((\forall g \in G)[\; g * inv(g) = e = inv(g) * g \;]\)
\end{enumerate}

\section*{Задача 1.}

Нека \((G, e_G, *, inv_G)\) и \((K, e_K, \odot, inv_K)\) са групи.

Разглеждаме декартовото произведение на \(G\) и \(K\).

Тоест \(G \times K = \{<g, k> \; | \; g \in G, \; k \in K\}\).

Дефинираме 

\begin{align*}
<a, b> \; \oplus \; <c, d> = <a * c, b \odot d> \\
e = <e_G, e_K> \\
inv(<g, k>) = <inv_G(g), inv_K(k)>
\end{align*}

Докажете, че 

\begin{itemize}
\item \((\forall a \in G)(\forall c \in G)(\forall b \in K)(\forall d \in K)[\; <a, b> \; \oplus \; <c, d> \; \in G \times K] \;\)
\item \((\forall g \in G)(\forall k \in K)[\; inv(<g, k>) \in G \times K \;]\)
\item \((G \times K, e, \oplus, inv)\) е група.
\end{itemize}

\section*{Задача 2.}

Нека \((G, e_G, *, inv_G)\) е група. Нека \(X\) е непразно множество.

Нека \(Func(X, G) = \{f \; | \; f \; : \; X \to G\}\) е множеството на фукциите от \(X\) в \(G\).

Нека \(\theta = (x \mapsto e_G) \in Func(X, G)\) е константната функция, която на всеки елемент от \(X\) съпоставя \(e_G\).

Нека \(inv(f) = (x \mapsto inv_G(f(x)))\). Тоест \(inv = (f \mapsto (x \mapsto inv_G(f(x))))\).

Проверете, че \(inv \; : \; Func(X, G) \to Func(X, G)\).

Нека \(f \in Func(X, G)\) и \(g \in Func(X, G)\) и нека означим \\
\(f \otimes g = (x \mapsto f(x) * g(x))\).

Проверете, че \((\forall f \in Func(X, G))(\forall g \in Func(X, G))[\; f \otimes g \in Func(X, G) \;]\).

Докажете, че \((Func(X, G), \theta, \otimes, inv)\) е група.

\end{document}