\documentclass[12pt]{article}
\usepackage[utf8]{inputenc}

\usepackage[T2A]{fontenc}
\usepackage[english,bulgarian]{babel}
\usepackage{amsmath}
\usepackage{amssymb}
\usepackage{amsthm}
\usepackage{relsize}
\usepackage{tensor}
\usepackage{enumitem}
\usepackage{tikz-cd}

\title{Упражнение за циклични групи, нормални групи и фактор групи. Първа теорема за хомоморфизмите}
\author{Иво Стратев}

\begin{document}
\maketitle

\section{Подгрупи на циклични групи}
Нека си припомним каква ни беше дефиницията за подгрупа на дадена група.
\\
\vspace{2mm}
\\
Нека \(<G, e, *, ()^{-1}>\) е група. Нека \(H \subseteq G\).
Казваме, че \(H\) образува подгрупа на \(<G, e, *, ()^{-1}>\), ако
\begin{enumerate}
    \item \(e \in H\)
    \item \((\forall a \in H)(\forall b \in H)(a * b \in H)\)
    \item \((\forall a \in H)(a^{-1} \in H)\)
\end{enumerate}

Сега очевидно ако \(\mathcal{A} = <A, e_\mathcal{A}, *_\mathcal{A}, inv_\mathcal{A}>\)
и \(\mathcal{B} = <B, e_\mathcal{B}, *_\mathcal{B}, inv_\mathcal{B}>\) са групи,
за които е в сила
\begin{enumerate}
    \item \(e_\mathcal{A} = e_\mathcal{B}\)
    \item \((\forall b \in B)(\forall c \in B)(b *_\mathcal{B} c = b *_\mathcal{A} c)\)
    \item \((\forall b \in B)(inv_\mathcal{B}(b) = inv_\mathcal{A}(b))\)
\end{enumerate}

То \(\mathcal{B}\) е подгрупа на \(\mathcal{A}\) тогава и само тогава когато  \(B \subseteq A\).
\\
\vspace{2mm}
\\
Така нека сега дадем дефиниция за циклична група.
Нека \(<G, e, *, ()^{-1}>\) е група.
Казваме, че тя е циклична, ако се поражда от един елемент. Тоест
\((\exists g \in G)(G = \; <g> \; = \{g^z \; | \; z \in \mathbb{Z}\})\).
\\
\vspace{2mm}
\\
От лекции знаем, че \(<\mathbb{Z}, 0, +, ->\) е циклична група,
защото \(<1> = 1\mathbb{Z} = \{z.1 \; | \; z \in \mathbb{Z}\} = \mathbb{Z}\).
Също така, знаем, че с точност до изоморфизъм това е единствената циклична група от безкраен ред.
Както и ако \(n \in \mathbb{N}^+\), то групата на \(n\)-тите корени на единицата
\(<\mathbb{C}_n, 1, . , ()^{-1}>\) е циклична група (поражда се от елемента \(\omega_n^1\)) и с точност до изоморфизъм е единствената циклична от ред \(n\).

\subsection{Подгрупи на групата на целите числа}
Така нека \(n \in \mathbb{N}\). Тогава \(<n> \; = \{z.n \; | \; z \in \mathbb{Z}\} = n\mathbb{Z}\). Това са всички цели числа кратни на \(n\).
Очевидно \(<-n> \; = \; <n>\).
Ето няколко примера
\begin{enumerate}
    \item \(<0> \; = 0\mathbb{Z} = \{0\}\)
    \item \(<1> \; = 1\mathbb{Z} = \mathbb{Z}\)
    \item \(<2> \; = \{z.2 \; | \; z \in \mathbb{Z}\} = \{z \in \mathbb{Z} \; | \; z \equiv 0 \pmod{2}\} = 2\mathbb{Z}\)
    \item \(<6> \; = \{z.6 \; | \; z \in \mathbb{Z}\} = \{z \in \mathbb{Z} \; | \; z \equiv 0 \pmod{6}\} = 6\mathbb{Z}\)
\end{enumerate}
Е оказва се, че всяка подгрупа на целите числа е от вида \(n\mathbb{Z}\) за някое естествено \(n\). Това се доказва изключително лесно. Ето едно бързо доказателство.
Първо нека \(n \in \mathbb{N}\) и да видим, че \(<n\mathbb{Z}, 0, +, ->\) е група.
\begin{enumerate}
    \item \(0.n = 0\) това е празната сума, по друг начин \(0 = n + (-n)\), тоест всеки случай \(0 \in n\mathbb{Z}\) .
    \item \(a.n + b.n = (a + b).n \in n\mathbb{Z}\) тоест имаме затвореност относно събирането понеже сума от кратни на \(n\) е кратно на \(n\).
    \item \(-(a.n) = -a.n\) (очевидно)
\end{enumerate}
Значи \(n\mathbb{Z}\) образува подгрупа на \(<\mathbb{Z}, 0, +, ->\).
\\
\vspace{2mm}
\\
Задача за любознателните:  \\
Нека \(M\) е множество такова, че \(\{0\} \subset M \subset \mathbb{Z}\),
което образува подгрупа на \(<\mathbb{Z}, 0, +, ->\). Покажете, че
\((\exists n \in \mathbb{N} \setminus \{0, 1\})(M = n\mathbb{Z})\).
\\
\vspace{2mm}
\\
Сега на въпроса каква е връзката между подгрупите на \(<\mathbb{Z}, 0, +, ->\).
Нека \(n \in \mathbb{N}\) и \(m \mathbb{N}\). Както видяхме \(n\mathbb{Z}\) и \(m\mathbb{Z}\) образуват подгрупи на \(<\mathbb{Z}, 0, +, ->\).
Очевидно \(n\mathbb{Z}\) е подгрупа на \(m\mathbb{Z}\) ТСТК \(n\mathbb{Z} \subseteq m\mathbb{Z}\) понеже и двете са подгрупи на \(<\mathbb{Z}, 0, +, ->\).
От друга страна \(n\mathbb{Z} \subseteq m\mathbb{Z} \longleftrightarrow (\forall z \in \mathbb{Z})(m \mid z.n)\). В частност при \(z = 1\) получаваме \(m \mid n\), тоест
\begin{align*}
    n\mathbb{Z} \leq m\mathbb{Z} \longleftrightarrow m \mid n
\end{align*}
\subsection{Подгрупи на групата на \(n\)-тите корени на единициата}
Нека си припомним, че \(\mathbb{C}_n = \{\omega_n^k \; | \; k \in \{0, 1, \dots, n - 1\}\}\). Сега на готово ще използваме твърдението от лекции, че всяка подгрупа на циклична група е циклична. Тогава подгрупите на \(\mathbb{C}_n\) са точно цикличните групи породени от някой елемент на \(\mathbb{C}_n\). Тоест това са групите \(<\omega_n^k>\), за някое 
\(k \in \{0, 1, \dots, n - 1\}\). Така сега използвайки следните две твърдения
\begin{enumerate}
    \item \(ord(\omega_n^k) = \displaystyle\frac{ord(\omega_n^1)}{gcd(ord(\omega_n^1), k)} = \displaystyle\frac{n}{gcd(n, k)}\)
    \item \(|<\omega_n^k>| = ord(\omega_n^k)\)
\end{enumerate}
и имайки предвид, че \(<\omega_n^k>\) е циклична група от краен ред.
Получаваме \(<\omega_n^k> = \mathbb{C}_{\displaystyle\frac{n}{gcd(n, k)}}\).
Възможни са два случая
\begin{enumerate}
    \item \(gcd(n, k) = 1\) тогава \(<\omega_n^k> = \mathbb{C}_n\).
    \item \(d = gcd(n, k) > 1\) тогава \(d \mid n\) и значи \(<\omega_n^k> = \mathbb{C}_{\displaystyle\frac{n}{d}}\) обаче щом \(d \mid n\), то \(\displaystyle\frac{n}{d} \mid n\)
\end{enumerate}
Значи всички подгрупи на \(<\mathbb{C}_n, 1, . , ()^{-1}>\) са 
\(<\mathbb{C}_d, 1, . , ()^{-1}>\), където \(d \mid n\).
Сега въпросът е кога \(\mathbb{C}_k\) е подгрупа на \(\mathbb{C}_d\),
това очевидно е ТСТК \(\mathbb{C}_k \subseteq \mathbb{C}_d\).
Използвайки, че ако \(z \in \mathbb{C}\), то \(z \in \mathbb{C}_d \longleftrightarrow z^d = 1\). Получаваме \(\mathbb{C}_k \subseteq \mathbb{C}_d\) ТСТК
\((\forall s \in \{0, 1, \dots, k - 1\})((\omega_k^s)^d = 1)\).
В частност трябва да е изпълнено и за \(s = 1\), тоест \((\omega_k^1)^d = \omega_k^d = 1\) е това се случва ТСТК \(k \mid d\). Така \(\mathbb{C}_k \leq \mathbb{C}_d \longleftrightarrow k \mid d\).
\section{Съседни класове. Теорема на Лагранж. Нормални подгрупи. Фактор групи}
Нека \(<G, e, op, inv>\) е група.
По естествен начин операцията \\
\(op : G \times G \to G\)
може да бъде разширена до операция \\
\(OP_{left} : G \times \mathcal{P}(G) \to \mathcal{P}(G)\).
По следния начин \\
\(OP_{left}(g, S) = \{op(g, s) \; | \; s \in S\}\).
Аналогично можем да дефинираме \(OP_{right} : \mathcal{P}(G) \times G \to \mathcal{P}(G)\).
По следния начин \\
\(OP_{right}(S, g) = \{op(s, g) \; | \; s \in S\}\).
Сега ако \(H\) образува подгрупа на \(<G, e, op, inv>\).
То множеството \(OP_{left}(g, H)\) се нарича ляв съседен клас на \(H\) по \(g\),
а множеството \(OP_{right}(H, g)\) се нарича десен съседен клас на \(H\) по \(g\).
\subsection{Мултипликативен запис:}
Ако \(<G, 1, . , ()^{-1}>\) е мултипликативно записана група.
И \(H\) образува нейна подгрупа, то 
\begin{enumerate}
    \item \(gH = \{gh \; | \; h \in H\}\) е левия съседен клас на \(H\) по \(g\).
    \item \(Hg = \{hg \; | \; h \in H\}\) е десния съседен клас на \(H\) по \(g\).
\end{enumerate}
\subsection{Адитивен запис:}
Ако \(<G, 0, + , ->\) е адитивно записана група.
И \(H\) образува нейна подгрупа, то 
\begin{enumerate}
    \item \(g + H = \{g + h \; | \; h \in H\}\) е левия съседен клас на \(H\) по \(g\).
    \item \(H + g = \{h + g \; | \; h \in H\}\) е десния съседен клас на \(H\) по \(g\).
\end{enumerate}
\subsection{Важни свойства ще ги напишем в мултипликативен запис и само за левите съседни класове (за десните е аналогично)}
Нека \(<G, e, . , ()^{-1}>\) е група и \(H\) образува нейна подгрупа.
Нека \(g \in G\) е произволен.
Дефинираме релацията \(\sim\) в \(G\). Така \(a \sim b \longleftrightarrow aH = bH\). Тогава
\begin{enumerate}
    \item \(\sim\) е релация на еквивалетност
    \item \([a]_\sim = aH\)
    \item \(\{gH \; | \; g \in G\}\) е разбиване на \(G\)
    \item \(a \sim b \longleftrightarrow b \in aH\)
    \item \(a \sim b \longleftrightarrow ab^{-1} \in H\)
    \item \(gH\) образува подгрупа на \(<G, 1, . , ()^{-1}>\) ТСТК \(g = e\),
    тоест единствения съседен клас, който е група е самото \(H\) (\(H = eH\))
    \item \(gH\) и \(H\) са равномощни (\(h \mapsto gh\) е биекция)
\end{enumerate}
С \(G : H\) бележим множеството на левите съседни класове тоест \(G : H = \{gH \; | \; g \in G\}\)). Мощността на множеството \(G : H\) се нарича индекс на групата образувана от \(H\) в \(<G, e, . , ()^{-1}>\). Тоест индексът на групата образувана от \(H\) е \(|G : H|\).
\subsection{Теорема на Лагранж}
Нека \(<G, e, . , ()^{-1}>\) е крайна група и \(H\) образува нейна подгрупа.
Тогава \(|G| = |H|.|G : H|\). Следствие \(|G : H| = \displaystyle\frac{|G|}{|H|}\).
\subsection{Примери}
\subsubsection{\(\mathbb{R}^*\) и \(\mathbb{R}^+\)}
Да си припомним, че \(\mathbb{R}^* = \mathbb{R} \setminus \{0\}\)
и \(<\mathbb{R}^*, 1, . , ()^{-1}>\) е група. Също така
\(\mathbb{R}^+ = \{r \in \mathbb{R} \; | \; r > 0\}\)
и \(\mathbb{R}^- = \{r \in \mathbb{R} \; | \; r < 0\}\).
Понеже произведение на положителни е положително и обратен на положително е положително, то \(\mathbb{R}^+\) образува подгрупа на \(<\mathbb{R}^*, 1, . , ()^{-1}>\).
Нека намерим кои са съседните класове и на колко е равен индексът на \(\mathbb{R}^+\).
Нека \(a \in \mathbb{R}^*\). Тогава са възможни два случая
\begin{enumerate}
    \item \(a > 0\). Тогава \(a\mathbb{R}^+ = \{ar \; | \; a \in \mathbb{R}^+ \} = \mathbb{R}^+ \)
    \item \(a < 0\). Тогава \(a\mathbb{R}^+ = \{ar \; | \; a \in \mathbb{R}^+ \} = \mathbb{R}^-\)
\end{enumerate}
Ползвахме, че \(x \mapsto ax\) е биекция в \(\mathbb{R}\).
От друга страна \\
\(a\mathbb{R}^+ = b\mathbb{R}^+ \longleftrightarrow \displaystyle\frac{a}{b} \in \mathbb{R}^+ \longleftrightarrow \displaystyle\frac{a}{b} > 0 \longleftrightarrow sign(a) = sign(b) \longleftrightarrow \\
(a \in \mathbb{R}^+ \; \& \; b \in \mathbb{R}^+) \; \lor \; (a \in \mathbb{R}^- \; \& \; b \in \mathbb{R}^-)\).\\
Значи \(\mathbb{R}^* : \mathbb{R}^+ = \{\mathbb{R}^+, \mathbb{R}^- \}\).
Така \(|\mathbb{R}^* : \mathbb{R}^+| = 2\).
\subsubsection{\(\mathbb{Z}\) и \(6\mathbb{Z}\)}
Нека \(a\ \in \mathbb{Z}\). Да разделим \(a\) с частно и остатък на \(6\).
Тоест \(a = 6s + r\) и \(0 \leq r < 6\).
Тогава \(a + 6\mathbb{Z} = \{a + 6z \; | \; z \in \mathbb{Z}\} = \{6s + r + 6z \; | \; z \in \mathbb{Z}\} = \{r + 6(z + s) \; | \; z \in \mathbb{Z}\} = \{r + 6z \; | \; z \in \mathbb{Z}\} = r + 6\mathbb{Z}\). Ползвахме, че \(z \mapsto z + s\) е биекция в \(\mathbb{Z}\). От друга страна \\
\(a + 6\mathbb{Z} =  b + 6\mathbb{Z} \longleftrightarrow a - b \in 6\mathbb{Z} \longleftrightarrow 6 \mid a - b \longleftrightarrow a \equiv b \pmod{6}\).
Ясно е, че класовете на еквивалетност са
\begin{enumerate}
    \item \(6\mathbb{Z}\)
     \item \(1 + 6\mathbb{Z}\)
      \item \(2 + 6\mathbb{Z}\)
       \item \(3 + 6\mathbb{Z}\)
        \item \(4 + 6\mathbb{Z}\)
         \item \(5 + 6\mathbb{Z}\)
\end{enumerate}
Така \(|\mathbb{Z} : 6\mathbb{Z}| = 6\).
\subsubsection{\(\mathbb{C}_8\) и \(\mathbb{C}_4\)}
Понеже \(4 \mid 8\), то \(\mathbb{C}_4\) образува подгрупа на \(<\mathbb{C}_8, 1, . , ()^{-1}>\). Така че можем да говорим за съседни класове и индекс.
Ето един трик
\begin{align*}
    \omega_4^k = \cos\left(\displaystyle\frac{2k\pi}{4}\right) + i\sin\left(\displaystyle\frac{2k\pi}{4}\right) = \cos\left(\displaystyle\frac{2(2k)\pi}{8}\right) + i\sin\left(\displaystyle\frac{2(2k)\pi}{8}\right) = \omega_8^{2k}
\end{align*}
Нека си припомним с пример, че \(\omega_8^{21} = \omega_8^{16 + 5} = \omega_8^{2.8}.\omega_8^5 = 1^2.\omega_8^5 = \omega_8^5\).
Тоест можем да редуцираме степента до остатък на \(8\). Сега смятаме
\begin{enumerate}
    \item \(\omega_8^0\mathbb{C}_4 = 1\mathbb{C}_4 = \mathbb{C}_4 = \{1, \omega_8^2, \omega_8^4, \omega_8^6\} = \{\omega_8^0, \omega_8^2, \omega_8^4, \omega_8^6\}\)
    \item \(\omega_8^1\mathbb{C}_4 = \omega_8^1\{1, \omega_8^2, \omega_8^4, \omega_8^6\} = \{\omega_8^{1 + 0}, \omega_8^{1 + 2}, \omega_8^{1 + 4}, \omega_8^{1 + 6}\} = \{\omega_8^1, \omega_8^3, \omega_8^5, \omega_8^7\}\)
    \item \(\omega_8^2\mathbb{C}_4 = \omega_8^2\{1, \omega_8^2, \omega_8^4, \omega_8^6\} = \{\omega_8^{2 + 0}, \omega_8^{2 + 2}, \omega_8^{2 + 4}, \omega_8^{2 + 6}\} = \{\omega_8^2, \omega_8^4, \omega_8^6, \omega_8^0\}\)
    \item \(\omega_8^3\mathbb{C}_4 = \omega_8^3\{1, \omega_8^2, \omega_8^4, \omega_8^6\} = \{\omega_8^{3 + 0}, \omega_8^{3 + 2}, \omega_8^{3 + 4}, \omega_8^{3 + 6}\} = \{\omega_8^3, \omega_8^5, \omega_8^7, \omega_8^1\}\)
    \item \(\omega_8^4\mathbb{C}_4 = \omega_8^4\{1, \omega_8^2, \omega_8^4, \omega_8^6\} = \{\omega_8^{4 + 0}, \omega_8^{4 + 2}, \omega_8^{4 + 4}, \omega_8^{4 + 6}\} = \{\omega_8^4, \omega_8^6, \omega_8^0, \omega_8^2\}\)
    \item \(\omega_8^5\mathbb{C}_4 = \omega_8^5\{1, \omega_8^2, \omega_8^4, \omega_8^6\} = \{\omega_8^{5 + 0}, \omega_8^{5 + 2}, \omega_8^{5 + 4}, \omega_8^{5 + 6}\} = \{\omega_8^5, \omega_8^7, \omega_8^1, \omega_8^3\}\)
    \item \(\omega_8^6\mathbb{C}_4 = \omega_8^6\{1, \omega_8^2, \omega_8^4, \omega_8^6\} = \{\omega_8^{6 + 0}, \omega_8^{6 + 2}, \omega_8^{6 + 4}, \omega_8^{6 + 6}\} = \{\omega_8^6, \omega_8^0, \omega_8^2, \omega_8^4\}\)
    \item \(\omega_8^7\mathbb{C}_4 = \omega_8^7\{1, \omega_8^2, \omega_8^4, \omega_8^6\} = \{\omega_8^{7 + 0}, \omega_8^{7 + 2}, \omega_8^{7 + 4}, \omega_8^{7 + 6}\} = \{\omega_8^7, \omega_8^1, \omega_8^3, \omega_8^5\}\)
\end{enumerate}
Тоест \(\mathbb{C}_8 : \mathbb{C}_4 = \{\{\omega_8^0, \omega_8^2, \omega_8^4, \omega_8^6\}, \{\omega_8^1, \omega_8^3, \omega_8^5, \omega_8^7\}\}\).
Така \\
\(|\mathbb{C}_8 : \mathbb{C}_4| = 2 = \displaystyle\frac{8}{4} = \displaystyle\frac{|\mathbb{C}_8|}{|\mathbb{C}_4|}\) (Лагранж).
Това което забелязваме е, че разглеждайки съседният клас на \(\mathbb{C}_4\) по \(\omega_8^k\). Ако \(k\) е четно, то резултата е множеството от тези на четна степен,
съответно ако \(k\) е нечетно, то резултата е множеството от тези на нечетна степен.
\subsection{Нормални групи}
Нека \(<G, e, . , ()^{-1}>\) е група и \(H\) образува нейна подгрупа.
Казваме, че подгрупата образувана от \(H\) на групата \(<G, e, . , ()^{-1}>\) е нормална, ако \((\forall g \in G)(gH = Hg)\).
\subsubsection{Важни неща}
\begin{enumerate}
    \item \(H\) образува нормална подгрупа на \(<G, e, . , ()^{-1}>\) ТСТК \\
    \((\forall g \in G)(\forall h \in H)(g^{-1}hg \in H)\).
    \item Ако \(<G, e, . , ()^{-1}>\) е абелева група, то всяка нейна подгрупа е нормална.
    \item Ако \(|G : H| = 2\), то \(H\) образува нормална подгрупа на \(<G, e, . , ()^{-1}>\).
\end{enumerate}
\subsection{Фактор група}
Нека \(<G, e, . , ()^{-1}>\) е група и \(H\) образува нейна нормална подгрупа.
Тогава в множеството \(G : H = \{gH \; | \; g \in G\}\) можем да въведем групова операция.
Това, че \(H\) е нормална подгрупа ни позволява да го направим, иначе нямаше да е групова операция ... Та дефинираме следната операция
\begin{align*}
    aH * bH = (a.b)H \quad ([a]_{\sim_H} * [b]_{\sim_H} = [a.b]_{\sim_H})
\end{align*}
Така казахме, че \(*\) е групова операция.
Нека видим, че наистина получаваме група:
\begin{enumerate}
    \item \((aH * bH) * cH = (a.b)H * cH = ((a.b).c)H = (a.(b.c))H = aH * (b.c)H = aH * (bH * cH)\)
    \item \(aH * H = aH * eH = (a.e)H = aH\) и \(H * aH = eH * aH = (e.a)H = aH\)
    \item \(aH * a^{-1}H = (a.a^{-1})H = eH = H\) и \(a^{-1}H * aH = (a^{-1}.a)H = eH = H\)
\end{enumerate}
Значи можем преспокойно да твърдим, че относно новата операция \(*\) \((aH)^{-1} = a^{-1}H\).
Така групата \(<G : H, H, *, ()^{-1}>\) се нарича фактор група на \(<G, e, . , ()^{-1}>\)
по нормалната ѝ подгрупа образувана от \(H\) и се бележи с \(G / H\).
\\
\vspace{2mm}
\\
Четири вметвания:
\begin{enumerate}
    \item Ако \(<G, e, . , ()^{-1}>\) е абелева група, то всяка нейна фактор група е абелева.
    \item Ако \(<G, e, . , ()^{-1}>\) е циклична група, то всяка нейна фактор група е циклична.
    \item Ако \(<G, e, . , ()^{-1}>\) е крайна група, то всяка нейна фактор група е крайна.
    \item Редът на една фактор група \(G / H\) съвпада с индексът на \(H\) в \(G\).
\end{enumerate}
\subsection{Примери}
\subsubsection{\(\mathbb{R}^*\) и \(\mathbb{R}^+\)}
Понеже \(<\mathbb{R}^*, 1, . , ()^{-1}>\) е абелева, то и \(<\mathbb{R}^+, e, . , ()^{-1}>\) е абелева, а значи и нормална подгрупа.
Значи е коректно да разгледаме фактор групата \(\mathbb{R}^* / \mathbb{R}^+\).
Както видяхме \(\mathbb{R}^* : \mathbb{R}^+ = \{\mathbb{R}^+, \mathbb{R}^-\}\).
Да видим как се смята в тази фактор група, тя е абелева, така че
\begin{enumerate}
    \item \(\mathbb{R}^+ * \mathbb{R}^+ = 1\mathbb{R}^+ * 1\mathbb{R}^+ = (1.1)\mathbb{R}^+ = \mathbb{R}^+\)
    \item \(\mathbb{R}^+ * \mathbb{R}^- = 1\mathbb{R}^+ * (-1)\mathbb{R}^+ = (1.(-1))\mathbb{R}^+ = (-1)\mathbb{R}^+ = \mathbb{R}^-\)
    \item \(\mathbb{R}^- * \mathbb{R}^- = (-1)\mathbb{R}^+ * (-1)*\mathbb{R}^+ = (-1.-1)\mathbb{R}^+ = 1\mathbb{R}^+ = \mathbb{R}^+\)
\end{enumerate}
Тоест очевидно нещата се случват точно както се случват в подгрупата \(<\{1, -1\}, 1, . , ()^{-1}>\) на \(<\mathbb{R}^*, 1, . , ()^{-1}>\), която е циклична, защото \(\{1, -1\} = <-1>\).
\subsubsection{\(\mathbb{C}_8\) и \(\mathbb{C}_4\)}
Както видяхме \(\mathbb{C}_8 : \mathbb{C}_4 = \{\{\omega_8^0, \omega_8^2, \omega_8^4, \omega_8^6\}, \{\omega_8^1, \omega_8^3, \omega_8^5, \omega_8^7\}\}\).
Нека означим
\begin{enumerate}
    \item \(A = \{\omega_8^0, \omega_8^2, \omega_8^4, \omega_8^6\} = \mathbb{C}_4 = 1\mathbb{C}_4\)
    \item \(B = \{\omega_8^1, \omega_8^3, \omega_8^5, \omega_8^7\} = \omega_8^1\mathbb{C}_4\)
\end{enumerate}
Понеже \(<\mathbb{C}_8, 1, . , ()^{-1}>\) e циклична, в частност е и абелева.
То \(<\mathbb{C}_4, 1, . , ()^{-1}>\) е нормална и факторът \(\mathbb{C}_8 / \mathbb{C}_4\) е циклична от ред \(2\), тоест е изоморфна на \(\mathbb{C}_2\).
Като сметки
\begin{enumerate}
    \item \(A * A = 1\mathbb{C}_4 * 1\mathbb{C}_4 = (1.1)\mathbb{C}_4 = \mathbb{C}_4 = A\)
    \item \(A * B = 1\mathbb{C}_4 * \omega_8^1\mathbb{C}_4 = (1.\omega_8^1)\mathbb{C}_4 = \omega_8^1\mathbb{C}_4 = B\)
    \item \(B * B = \omega_8^1\mathbb{C}_4 * \omega_8^1\mathbb{C}_4 = \omega_8^2\mathbb{C}_4 = 1\mathbb{C}_4 = A\)
\end{enumerate}
\subsubsection{\(\mathbb{Z}\) и \(6\mathbb{Z}\)}
Понеже алгебриците са пестеливи от към запис, то \(r + 6\mathbb{Z}\) означаваме още с \(\overline{r}\). Така \(\mathbb{Z} : 6\mathbb{Z} = \{\overline{0}, \overline{1}, \overline{2}, \overline{3}, \overline{4}, \overline{5}\}\).
Също така множеството \(\mathbb{Z} : 6\mathbb{Z}\) бележим с \(\mathbb{Z}_6\).
Действието тук се случва по модул  остатък на \(6\).
Например ако групата запишем така \(<\mathbb{Z}_6, \overline{0}, \oplus, \ominus>\), то
\begin{enumerate}
    \item \(\overline{2} \oplus \overline{3} = \overline{2 + 3} = \overline{5}\)
    \item \(\overline{4} \oplus \overline{3} = \overline{4 + 3} = \overline{7} = \overline{1}\)
    \item \(\ominus \overline{4} = \overline{-4} = \overline{6 - 4} = \overline{2}\)
\end{enumerate}
\section{Една класика от Информатика}
Нека \(\alpha = \cos\left(\displaystyle\frac{\pi}{33}\right) + i\sin\left(\displaystyle\frac{\pi}{33}\right)\).
Нека \(\beta = \alpha^{28}\). Нека \(G = \; <\alpha>\).
Нека \(H = \; <\beta>\).
\begin{enumerate}[label=\alph*)]
    \item Намерете редовете на \(\alpha\), \(\beta\), \(G\) и \(H\).
    \item Да се намерят всички подгрупи на \(<G, 1, ., ()^{-1}>\) и да се направи схема на включванията между тях.
    \item Да се реши уравнението \(G / H \cong \mathbb{Z}_t\).
    \item Да се реши уравнението \(G / <\alpha^s> \; \cong H\).
\end{enumerate}
\subsection{Решение}
\begin{enumerate}[label=\alph*)]
    \item Човек трябва да е много съобразителен и да забележи, че липсва една 2-ка в числителя ... Ей това е нещото дето убива невнимателните млади Информатици ...
    Така \(\alpha = \cos\left(\displaystyle\frac{2\pi}{66}\right) + i\sin\left(\displaystyle\frac{2\pi}{66}\right) = \omega_{66}^1\).
    Тогава \(G = \; <\alpha> \; = \; <\omega_{66}^1> \; = \mathbb{C}_{66}\).
    Така \(ord(\alpha) = |G| = |\mathbb{C}_{66}| = 66\).
    За \(\beta\) имаме \(\beta = \alpha^{28} = (\omega_{66}^1)^{28} = \omega_{66}^{28} = \omega_{2.3.11}^{2.2.7} = \omega_{33}^{14}\). Използвайки, че \(gcd(33, 14) = 1\) получаваме, че \(H = <\beta> = <\omega_{33}^{14}> = \mathbb{C}_{33}\) и значи
    \(ord(\beta) = |H| = |\mathbb{C}_{33}| = 33\).
    \item От предната \(G = \mathbb{C}_{66}\). \\
    Естествените делители на \(66\) са \(\{1, 2, 3, 6, 11, 22, 33, 66\}\).
    Тогава подгрупите са тези образувани от \(\mathbb{C}_d\) за \(d \in \{1, 2, 3, 6, 11, 22, 33, 66\}\).
    \begin{tikzcd}[
      every arrow/.append style={->,thick},
      matrix of math nodes maybe/.append style={/tikz/cells={/tikz/nodes={/tikz/draw,/tikz/shape=circle,align=center,text width=\widthof{$\mathbb{C}_{66}$}}}},
      row sep=large,
      column sep=large,
      thick,
      ]
    & \mathbb{C}_{66} \drar \dar \dlar \\
    \mathbb{C}_{22} \drar \dar &  \mathbb{C}_{6}  \drar\dlar & \mathbb{C}_{33} \dar \dlar \\
    \mathbb{C}_{2} \drar  & \mathbb{C}_{11} \dar &  \mathbb{C}_{3} \dlar \\
    & \mathbb{C}_{1}
    \end{tikzcd}
    \item \(G / H = \mathbb{C}_{66} / \mathbb{C}_{33} \cong \mathbb{C}_{2} \cong \mathbb{Z}_2\) Значи \(t = 2\).
    \item Искаме да решим уравнението \(\mathbb{C}_{66} / <\alpha^s> \; \cong \mathbb{C}_{33}\), което свеждаме до \(<\alpha^s> = \mathbb{C}_{2}\).
    Тоест искаме \(|<\alpha^s>| = 2\), но \(|<\alpha^s>| = ord(\alpha^s)\).
    От друга страна \(ord(\alpha^s) = \displaystyle\frac{ord(\alpha)}{gcd(ord(\alpha), s)} = \displaystyle\frac{66}{gcd(66, s)}\). Значи търсим \(s\), такова че
    \(\displaystyle\frac{66}{gcd(66, s)} = 2\), тоест \(\displaystyle\frac{1}{gcd(66, s)} = \displaystyle\frac{2}{66} = \displaystyle\frac{1}{33}\). Значи \(gcd(66, s) = 33\).
    Очевидно това се случва когато \(s \equiv 33 \pmod{66}\). Понеже \(gcd(66, 66z + 33) = gcd(66, 33) = 33\).
\end{enumerate}
\section{Първа теорема за хомоморфизмите}
Нека \(\mathcal{A} = <A, e_\mathcal{A}, *_\mathcal{A}, inv_\mathcal{A}>\)
и \(\mathcal{B} = <B, e_\mathcal{B}, *_\mathcal{B}, inv_\mathcal{B}>\) са групи.
Нека \(\varphi : A \to B\) е хомоморфизъм от \(\mathcal{A}\) към \(\mathcal{B}\).
Тоест \begin{align*}
    (\forall x \in A)(\forall y \in A)(\varphi(x *_\mathcal{A} y) = \varphi(x) *_\mathcal{B} \varphi(y))
\end{align*}. Тогава
\begin{enumerate}
    \item \(\mathrm{Ker}(\varphi) = \{a \in A \; | \; \varphi(a) = e_\mathcal{B}\}\) образува \textbf{нормална} подгрупа на \(\mathcal{A}\).
    \item \(\mathrm{Im}(\varphi) = \varphi[A] = \{\varphi(a) \; | \; a \in A\} \) образува подгрупа на \(\mathcal{B}\).
    \item \(A / \mathrm{Ker}(\varphi) \cong \mathrm{Im}(\varphi)\)
\end{enumerate}
\section{Схема за задачите ползващи първата теорема за хомоморфизмите}
Нека \(\mathcal{A} = <A, e_\mathcal{A}, *_\mathcal{A}, inv_\mathcal{A}>\)
и \(\mathcal{B} = <B, e_\mathcal{B}, *_\mathcal{B}, inv_\mathcal{B}>\).
Нека \(C \subseteq A\).
Искаме да докажем, че
\begin{enumerate}
    \item \(C\) образува нормална подгрупа на \(\mathcal{A}\).
    \item \(A / C \cong B\).
\end{enumerate}
За тази цел трябва да измислим подходящо изображение
\(\varphi : A \to B\), което да е такова, че
\begin{enumerate}
    \item \(\varphi\) е хомоморфизъм от \(\mathcal{A}\) към \(\mathcal{B}\).
    \item \(\varphi\) е сюрективно, което ще ни подсигури \(\mathrm{Im}(\varphi) = B\).
    \item \(\mathrm{Ker}(\varphi) = C\)
\end{enumerate}
Ако имаме такова изображение по първата теорема за хомоморфизмите заключаваме, че
\(C\) образува нормална подгрупа на \(\mathcal{A}\) и \(A / C \cong B\).
И така за да решим задачата трябва кажем как точно ще разсъждаваме, че да измислим изображението, което търсим. Идеята е да тръгнем от зад на пред ...
Да предположим, че имаме изображение \(\varphi : A \to B\), което е хомоморфизъм,
ще видим как трябва да действа \(\varphi\), така че \(\mathrm{Ker}(\varphi) = C\).
Нека \(x, y \in A\) са такива, че \(\varphi(x) = \varphi(y)\). Тогава
\begin{align*}
    \varphi(x) = \varphi(y) \longleftrightarrow \\
    \varphi(x) *_{\mathcal{B}} inv_{\mathcal{B}}(\varphi(y)) = e_{\mathcal{B}} \longleftrightarrow \\
    \varphi(x) *_{\mathcal{B}} \varphi(inv_{\mathcal{A}}(y)) = e_{\mathcal{B}} \longleftrightarrow \\
    \varphi(x *_{\mathcal{A}} inv_{\mathcal{A}}(y)) = e_{\mathcal{B}} \longleftrightarrow \\
    x *_{\mathcal{A}} inv_{\mathcal{A}}(y) \in \mathrm{Ker}(\varphi) \longleftrightarrow \\
    x *_{\mathcal{A}} inv_{\mathcal{A}}(y) \in C
\end{align*}
Значи искаме \(x\) и \(y\) да имат един и същ образ ТСТК \(x *_{\mathcal{A}} inv_{\mathcal{A}}(y) \in C\). Сега ако \(\mathrm{Ker}(\varphi) = C\), то \(C\) образува нормална подгрупа на \(\mathcal{A}\) и тогава \(x *_{\mathcal{A}} inv_{\mathcal{A}}(y) \in C\) е еквивалетно с \(x *_{\mathcal{A}} C = y *_{\mathcal{A}} C\). Това е просто вметка,
която цели да ни каже, че същност искаме двете релации \(\varphi(x) = \varphi(y)\) и \(x *_{\mathcal{A}} C = y *_{\mathcal{A}} C\) да съвпадат!
Та ако имаме хубав критерий (под хубав имаме предвид такъв, който включва равенство!) за принадлежност към множеството \(C\) сравнително лесно можем да се досетим какво да изображението вземайки предвид, че образите искаме да са елементите на \(B\)! Нека видим няколко примера :)

\section{Първа порция примери}
Нека \(\mathcal{C} = <\mathbb{C}^*, 1, ., ()^{-1}>\).
Да се докаже, че \(M\) образува нормална подгрупа на \(\mathcal{C}\) и 
\(\mathbb{C}^* / M \cong S\), където 
\begin{enumerate}[label=\alph*)]
    \item \(M = \mathbb{U}\) и \(S = \mathbb{R}^+\)
    \item \(M = \mathbb{R}^+\) и \(S = \mathbb{U}\)
    \item \(M = \mathbb{R}^*\) и \(S = \mathbb{U}\)
\end{enumerate}
\subsection{Решение:}
Първо това, че \(M\) образува нормална подгрупа на \(\mathcal{C}\),
можем да решим генерално в тази порция примери така:
И в трите примера знаем или лесно можем да съобразим, че \(M\) образува подгрупа на \(\mathcal{C}\). Тоест \(<M, 1, ., ()^{-1}>\) е подгрупа на \(<\mathbb{C}^*, 1, ., ()^{-1}>\), която е абелева. Следователно \(M\) образува нормална подгрупа на \(\mathcal{C}\). (Като можем лесно да се измъкнем е хубаво да го направим!)
\begin{enumerate}[label=\alph*)]
    \item Търсим изображение \(\varphi : \mathbb{C}^* \to \mathbb{R}^+\),
    което е сюрективен хомоморфизъм и \(\mathrm{Ker}(\varphi) = \mathbb{U}\).
    Така нека първо разгледаме трите множества, които имаме и да сме сигурни, че имаме хубав (такъв с равенство) критерий за принадлежност към \(\mathbb{U}\).
    \begin{enumerate}[label=\arabic*.]
        \item Понеже \(\mathbb{C}\) образува поле, то \(\mathbb{C}^* = \mathbb{C} \setminus \{0\}\). Имаме \(x \in \mathbb{C}^* \iff x \in \mathbb{C} \; \& \; x \neq 0\).
        \item \(\mathbb{R}^+ = \{r \in \mathbb{R} \; | \; r > 0\}\). Имаме \(x \in \mathbb{R}^+ \iff x \in \mathbb{R} \; \& \; x > 0\).
        \item \(\mathbb{U} = \{z \in \mathbb{C} \; | \; |z| = 1\}\). Имаме 
        \(x \in \mathbb{U} \iff x \in \mathbb{C}^* \; \& \; |x| = 1\).
    \end{enumerate}
    Искаме \begin{align*}
        \varphi(x) = \varphi(y) \longleftrightarrow 
        \displaystyle\frac{x}{y} \in \mathbb{U} \longleftrightarrow 
        \left|\displaystyle\frac{x}{y}\right| = 1 \longleftrightarrow
        \displaystyle\frac{|x|}{|y|} = 1 \longleftrightarrow
        |x| = |y|
    \end{align*} 
    От друга страна ако \(x \in \mathbb{C}^*\), то \(|x| > 0\) понеже \(|x| = 0 \iff x = 0\) и \(|x| \geq 0\). Значи ако \(x \in \mathbb{C}^*\), то \(|x| \in \mathbb{R}^+\).
    Нека тогава пробваме с изображението \(\varphi(x) = |x|\) и да проверим, че то ни върши работа.
    \begin{enumerate}[label=\arabic*.]
        \item \(\varphi(x.y) = |x.y| = |x|.|y| = \varphi(x).\varphi(y)\) Значи  \(\varphi\) е ХММ.
        \item Нека \(r \in \mathbb{R}^+\). В частност \(r \in \mathbb{C}^*\).
        Но \(r = |r| = \varphi(r)\). Значи \(\varphi\) е сюрекция.
        \item \(x \in \mathbb{U} \iff x \in \mathbb{C}^* \; \& \; |x| = 1 \iff
        x \in \mathbb{C}^* \; \& \; \varphi(x) = 1 \iff x \in \mathrm{Ker}(\varphi)\).
        Значи \(\mathrm{Ker}(\varphi) = \mathbb{U}\).
    \end{enumerate}
    От първата теорема за ХММ имаме \(\mathbb{C}^* / \mathrm{Ker}(\varphi) \cong \mathrm{Im}(\varphi)\), тоест \(\mathbb{C}^* / \mathbb{U} \cong \mathbb{R}^+\).
    \item Търсим изображение \(\varphi : \mathbb{C}^* \to \mathbb{U}\),
    което е сюрективен хомоморфизъм и \(\mathrm{Ker}(\varphi) = \mathbb{R}^+\).
    Така търсим критерий за принадлежност към \(\mathbb{R}^+\), в който да участва равенство. Ами ние вече използвахме едно свойство на елементите включващо равенство.
    Именно \(x \in \mathbb{R}^+ \iff x \in \mathbb{C}^* \; \& \; x = |x|\).
    Нека пробваме с него.
    Искаме \begin{align*}
        \varphi(x) = \varphi(y) \longleftrightarrow 
        \displaystyle\frac{x}{y} \in \mathbb{R}^+ \longleftrightarrow 
        \displaystyle\frac{x}{y} = \left|\displaystyle\frac{x}{y}\right| \longleftrightarrow
        \displaystyle\frac{x}{y} = \displaystyle\frac{|x|}{|y|} \longleftrightarrow
        \displaystyle\frac{x}{|x|} = \displaystyle\frac{y}{|y|}
    \end{align*}
    Получихме \(\varphi(x) = \varphi(y) \longleftrightarrow
        \displaystyle\frac{x}{|x|} = \displaystyle\frac{y}{|y|}\).
    Забележете, че от двете страни на равенствата \(x\) и \(y\) са разделени.
    Това ни помага да усетим каква информация искаме нашият хомоморфизъм да прехвърли от едната група към другата! След това всичко опира до правилния пренос на информацията ...
    Така ние искаме \(\varphi(x) \in \mathbb{U}\), тоест \(\varphi(x) \in \mathbb{C}^* \; \& \; |\varphi(x)| = 1\). Човек лесно съобразява, че ако  \(x \in \mathbb{C}^*\),
     то \(|x| \in \mathbb{R}^+\) и значи \(\displaystyle\frac{x}{|x|} \in \mathbb{C}^*\).
    Я дайте да сметнем модула на \(\displaystyle\frac{x}{|x|}\), на колко ли е равен ?
    \begin{align*}
        \left|\displaystyle\frac{x}{|x|}\right| = \displaystyle\frac{|x|}{||x||} = \displaystyle\frac{|x|}{|x|} = 1
    \end{align*}
    Значи излезе, че ако \(x \in \mathbb{C}^*\), то \(\displaystyle\frac{x}{|x|} \in \mathbb{U}\). Много неочаквано ... Дали пък няхме специален термин когато разделяхме един вектор на дължината му ? Да веее имаме нормираност му викахме на това с идеята, че дължината на нормиран вектор винаги е 1-ца. Спомени от 1-ви курс :)
    Така значи пробваме с изображението \(\varphi(x) = \displaystyle\frac{x}{|x|}\).
    \begin{enumerate}[label=\arabic*.]
        \item \(\varphi(x.y) = \displaystyle\frac{x.y}{|x.y|} = \displaystyle\frac{x.y}{|x|.|y|} = \displaystyle\frac{x}{|x|}.\displaystyle\frac{y}{|y|} = \varphi(x).\varphi(y)\) Значи  \(\varphi\) е ХММ.
        \item Нека \(z \in \mathbb{U}\). В частност \(z \in \mathbb{C}^*\).
        Но тогава \(\varphi(z) = \displaystyle\frac{z}{|z|} = \displaystyle\frac{z}{1} = z\). Значи \(\varphi\) е сюрекция.
        \item \(x \in \mathbb{R}^+ \iff x \in \mathbb{C}^* \; \& \; x = |x| \iff
        x \in \mathbb{C}^* \; \& \; \varphi(x) = \displaystyle\frac{x}{|x|} = \displaystyle\frac{x}{x} = 1 \iff x \in \mathrm{Ker}(\varphi)\).
        Значи \(\mathrm{Ker}(\varphi) = \mathbb{R}^+\).
    \end{enumerate}
    От първата теорема за ХММ имаме \(\mathbb{C}^* / \mathrm{Ker}(\varphi) \cong \mathrm{Im}(\varphi)\), тоест \(\mathbb{C}^* / \mathbb{R}^+ \cong \mathbb{U}\).
    \item Търсим изображение \(\varphi : \mathbb{C}^* \to \mathbb{U}\),
    което е сюрективен хомоморфизъм и \(\mathrm{Ker}(\varphi) = \mathbb{R}^*\).
    За целта първо търсим критерий с равенство за принадлежност към \(\mathbb{R}^*\).
    Понеже \(\mathbb{R}^* \subset \mathbb{C}^*\), то дайте да видим какво значи едно ненулево комплексно число да е ненулево реално. Нека \(z = a + ib\) и \((a, b) \neq (0, 0)\). Ако се случи, че \(b \neq 0\), то \(z \notin \mathbb{R}^*\).
    Значи ако, \(z \in \mathbb{R}^*\), то \(z = a + i0\). Но тогава комплексно спрегнатото на \(z\) е \(a - i0\), което е \(z\). Тоест \(z \in \mathbb{R}^*\), то \(z = \overline{z}\). Е ясно е и обратното включване. Та значи ето го критерия за принадлежност към \(\mathbb{R}^*\). \(x \in \mathbb{R}^* \iff x \in \mathbb{C}^* \; \& \; x = \overline{x}\).
    Нека пробваме с него.
    Искаме \begin{align*}
        \varphi(x) = \varphi(y) \longleftrightarrow 
        \displaystyle\frac{x}{y} \in \mathbb{R}^* \longleftrightarrow 
        \displaystyle\frac{x}{y} = \overline{\left(\displaystyle\frac{x}{y}\right)} \longleftrightarrow
        \displaystyle\frac{x}{y} = \displaystyle\frac{\overline{x}}{\overline{y}} \longleftrightarrow
        \displaystyle\frac{x}{\overline{x}} = \displaystyle\frac{y}{\overline{y}}
    \end{align*}
    Получихме \(\varphi(x) = \varphi(y) \longleftrightarrow
        \displaystyle\frac{x}{\overline{x}} = \displaystyle\frac{y}{\overline{y}}\).
    Забележете, че променливите пак се разделиха, значи това е информацията, която искаме да прехвърлим. Сега остава човек да съобрази, че ако \(x \in \mathbb{C}^*\), то
    \(\displaystyle\frac{x}{\overline{x}} \in \mathbb{U}\). Първо ако \(x \in \mathbb{C}^*\), то е ясно, че \(\displaystyle\frac{x}{\overline{x}} \in \mathbb{C}^*\). Сега ако \(x \in \mathbb{C}^*\), то \(\left|\displaystyle\frac{x}{\overline{x}}\right| = \displaystyle\frac{|x|}{|\overline{x}|} = \displaystyle\frac{|x|}{|x|} = 1\).
    Значи \(\varphi : \mathbb{C}^* \to \mathbb{U}\) и \(\varphi(x) = \displaystyle\frac{x}{\overline{x}}\) е коректно дефинирано.
    Да видим, че ни върши работа.
    \begin{enumerate}[label=\arabic*.]
        \item \(\varphi(x.y) = \displaystyle\frac{x.y}{\overline{x.y}} = \displaystyle\frac{x.y}{\overline{x}.\overline{y}} = \displaystyle\frac{x}{\overline{x}}.\displaystyle\frac{y}{\overline{y}} = \varphi(x).\varphi(y)\) Значи  \(\varphi\) е ХММ.
        \item Нека \(z \in \mathbb{U}\). В частност \(z \in \mathbb{C}^*\).
        Но тогава \(\varphi(z) = \displaystyle\frac{z}{\overline{z}} = \displaystyle\frac{z.z}{\overline{z}.z} = \displaystyle\frac{z^2}{|z|^2} = \displaystyle\frac{z^2}{1^2} = z^2\).
        Значи нека \(s = \sqrt{z}\). Тогава \(|s| = \sqrt{|z|} = \sqrt{1} = 1\)
        Значи \(s \in \mathbb{U}\) и \(\varphi(s) = s^2 = (\sqrt{z})^2 = z\).
        Значи \(\varphi\) е сюрекция.
        \item \(x \in \mathbb{R}^* \iff x \in \mathbb{C}^* \; \& \; x = \overline{x} \iff
        x \in \mathbb{C}^* \; \& \; \varphi(x) = \displaystyle\frac{x}{\overline{x}} = \displaystyle\frac{x}{x} = 1 \iff x \in \mathrm{Ker}(\varphi)\).
        Значи \(\mathrm{Ker}(\varphi) = \mathbb{R}^*\).
    \end{enumerate}
    От първата теорема за ХММ имаме \(\mathbb{C}^* / \mathrm{Ker}(\varphi) \cong \mathrm{Im}(\varphi)\), тоест \(\mathbb{C}^* / \mathbb{R}^* \cong \mathbb{U}\).
\end{enumerate}
\section{Още един пример}
Да се докаже, че \(\mathbb{U} / \mathbb{C}_n \cong \mathbb{U}\).
\subsection{Решение:}
Първо защо \(\mathbb{C}_n\) образува подгрупа на \(<\mathbb{U}, 1, . , ()^{-1}>\) ?
Ами нека \(z \in \mathbb{C}_n\). Тогава \(z = |z|\left(\cos\left(Arg(z)\right) + i\sin\left(Arg(z)\right)\right)\). Тогава от Моавър 1 имаме
\(z^n = |z|^n\left(\cos\left(n.Arg(z)\right) + i\sin\left(n.Arg(z)\right)\right) = 1\).
Значи \(|z|^n = 1\). Понеже \(|z| \geq 0\), то \(|z| = 1\) и значи \(z \in \mathbb{U}\).
Така \(\mathbb{C}_n \subseteq \mathbb{U}\).
Обаче \(<\mathbb{C}_n, 1, . , ()^{-1}>\) e група.
Значи \(<\mathbb{C}_n, 1, . , ()^{-1}>\) е подгрупа на \(<\mathbb{U}, 1, . , ()^{-1}>\).
Даже е и нормална понеже е абелева.
Имаме директен критерий за принадлежност към \(\mathbb{C}_n\), в който участва равенство.
\(x \in \mathbb{C}_n \iff x \in \mathbb{U} \; \& \; x^n = 1\).
Търсим изображение \(\varphi : \mathbb{U} \to \mathbb{U}\),
    което е сюрективен хомоморфизъм и \(\mathrm{Ker}(\varphi) = \mathbb{C}_n\).
Искаме \begin{align*}
        \varphi(x) = \varphi(y) \longleftrightarrow 
        \displaystyle\frac{x}{y} \in \mathbb{C}_n \longleftrightarrow 
        \left(\displaystyle\frac{x}{y}\right)^n = 1 \longleftrightarrow
        \displaystyle\frac{x^n}{y^n} = 1 \longleftrightarrow
        x^n = y^n
    \end{align*}
    Получихме \(\varphi(x) = \varphi(y) \longleftrightarrow x^n = y^n\).
    Така въпросът е дали ако \(x \in \mathbb{U}\), то \(x^n \in \mathbb{U}\)?
    Ами ако \(x \in \mathbb{U}\), то \(|x| = 1\) и \(|x^n| = |x|^n = 1^n = 1\).
    Значи да ако, \(x \in \mathbb{U}\), то \(x^n \in \mathbb{U}\).
    Значи \(\varphi(x) = x^n\) е коректно изображение от \(\mathbb{U}\) към \(\mathbb{U}\). Да проверим че ни върши работа.
    \begin{enumerate}
        \item \(\varphi(x.y) = (x.y)^n = x^n . y^n = \varphi(x) . \varphi(y)\). Значи \(\varphi\) е ХММ.
        \item Нека \(z \in \mathbb{U}\). Имаме, че \(\varphi(z) = z^n\).
        Нека \(s = \sqrt[n]{z}\). Тогава \(|s| = \sqrt[n]{|z|} = \sqrt[n]{1} = 1\).
        Значи \(s \in \mathbb{U}\).
        От друга страна \(\varphi(s) = s^n = \sqrt[n]{z}^n = z\). Значи \(\varphi\) е сюрекция.
        \item Имаме \(x \in \mathbb{C}_n \iff x \in \mathbb{U} \; \& \; x^n = 1 \iff x \in \mathbb{U} \; \& \; \varphi(x) = 1 \iff x \in \mathrm{Ker}(\varphi)\).
        Значи \(\mathrm{Ker}(\varphi) = \mathbb{C}_n\).
    \end{enumerate}
    От първата теорема за ХММ имаме \(\mathbb{U} / \mathrm{Ker}(\varphi) \cong \mathrm{Im}(\varphi)\), тоест \(\mathbb{U} / \mathbb{C}_n \cong \mathbb{U}\).
\section{Още един пример}
Нека \(H_n = \{z \in \mathbb{C}^* \; | \; (\exists r \in \mathbb{R}^+)(\exists s \in \mathbb{C}_n)(z = r.s)\}\). Да се докаже, че
\begin{enumerate}[label=\alph*)]
    \item \(<H_n, 1, ., ()^{-1}>\) е група.
    \item \(H_n = \{z \in \mathbb{C}^* \; | \; z^n = |z|^n\}\).
    \item \(H_n\) образува нормална подгрупа на \(<H_{n^2}, 1, ., ()^{-1}>\).
    \item \(H_{n^2} / H_n \cong \mathbb{C}_n\).
\end{enumerate}
\subsection{Решение:}
\begin{enumerate}[label=\alph*)]
    \item Очевидно \(H_n \subset \mathbb{C}^*\) и \(1 \in H_n\).
    Ще покажем, че е затворено относно умножението и относно обратен елемент.
    Нека \(a \in H_n\) и нека \(b \in H_n\). Тогава нека \(r_a \in \mathbb{R}^+\),
    \(r_b \in \mathbb{R}^+\), \(s_a \in \mathbb{C}_n\) и \(s_b \in \mathbb{C}_n\).
    Тогава \(a.b = (r_a.s_a).(r_b.s_b) = (r_a.r_b)(s_a.s_b)\). Понеже \(\mathbb{R}^+\) и \(\mathbb{C}_n\) образуват подгрупи на \(<\mathbb{C}^*, 1, ., ()^{-1}>\),
    то \(a.b \in H_n\). Също така \(a^{-1} = (r_a.s_a)^{-1} = s_a^{-1}.r_a^{-1} = r_a^{-1}.s_a^{-1}\) и значи \(a^{-1} \in H_n\). Следователно \(<H_n, 1, ., ()^{-1}>\) е подгрупа на \(<\mathbb{C}^*, 1, ., ()^{-1}>\). В частност е и група.
    \item Нека \(z \in \mathbb{C}^*\) е такова, че \(z^n = |z|^n\).
    Нека тригонометричния вид на \(z\) е \(|z|(\cos(\alpha) + i\sin(\alpha))\).
    Тогава \(z^n = |z|^n(\cos(n\alpha) + i\sin(n\alpha)) = |z|^n\).
    Значи \(\cos(n\alpha) + i\sin(n\alpha) = 1 = 1 + i0\).
    Значи \(\cos(n\alpha) = 1\) и \(\sin(n\alpha) = 0\).
    Решенията на тази система са \(n\alpha = 2k\pi\) за \(k \in \mathbb{Z}\).
    Обаче тогава \(\alpha = \displaystyle\frac{2k\pi}{n}\) и както знаем от първи курс решенията с точност до кратност на \(2\pi\) са само \(n\) за \(k \in \{0, 1, \dots, n - 1\}\). Но тогава \(z = |z|\omega_n^k\) за \(k \in \{0, 1, \dots, n - 1\}\).
    Значи \(\{z \in \mathbb{C}^* \; | \; z^n = |z|^n\} \subseteq H_n\).
    Обратното включване е директно от тригонометричния вид на всяко комплексно (получава се \(z^n = r^n\) ... )
    \item От а) имаме, че \(<H_n, 1, ., ()^{-1}>\) е абелева група. Единственото, което трябва да видим е, че \(H_n \subseteq H_{n^2}\), което е доста лесно.
    Нека \(z \in H_n\). Тогава от б) \(z^n = |z|^n\). Тогава \(z^{n^2} = z^{n.n} = (z^n)^n = (|z|^n)^n = |z|^{n.n} = |z|^{n^2}\). Значи \(z \in H_{n^2}\).
    Следователно \(H_n \subseteq H_{n^2}\). Вземайки предвид, че \(<H_n, 1, ., ()^{-1}>\) е абелева група получаваме, че е нормална подгрупа на  \(<H_{n^2}, 1, ., ()^{-1}>\).
    \item Вече имаме и критерий за принадлежност към \(H_n\) с равенство.
    Така, че търсим изображение \(\varphi : H_{n^2} \to \mathbb{C}_n\),
    което е сюрективен хомоморфизъм и \(\mathrm{Ker}(\varphi) = H_n\).
    Искаме \begin{align*}
        \varphi(x) = \varphi(y) \longleftrightarrow 
        \displaystyle\frac{x}{y} \in H_n \longleftrightarrow 
        \left(\displaystyle\frac{x}{y}\right)^n = \left|\displaystyle\frac{x}{y}\right|^n \longleftrightarrow \\
        \displaystyle\frac{x^n}{y^n} = \displaystyle\frac{|x|^n}{|y|^n} \longleftrightarrow
        \displaystyle\frac{x^n}{|x|^n} = \displaystyle\frac{y^n}{|y|^n}
    \end{align*}
    Получихме \(\varphi(x) = \varphi(y) \longleftrightarrow \displaystyle\frac{x^n}{|x|^n} = \displaystyle\frac{y^n}{|y|^n}\).
    Така въпросът е дали ако \(x \in H_{n^2}\), то \(\displaystyle\frac{x^n}{|x|^n} \in \mathbb{C}_n\)?
    Ами ако \(x \in H_{n^2}\), то \(x^{n^2} = |x|^{n^2}\).
    Сега за да видим, че \(\displaystyle\frac{x^n}{|x|^n} \in \mathbb{C}_n\)
    трябва да видим, че на повдигнат на \(n\)-та степен дава \(1\).
    Тогава ако \(x \in H_{n^2}\), то \(\left(\displaystyle\frac{x^n}{|x|^n}\right)^n =
    \displaystyle\frac{x^{n^2}}{|x|^{n^2}} =
    \displaystyle\frac{|x|^{n^2}}{|x|^{n^2}} = 1\).
    Значи да ако, \(x \in H_{n^2}\), то \(\displaystyle\frac{x^n}{|x|^n} \in \mathbb{C}_n\).
    Значи \(\varphi(x) = \displaystyle\frac{x^n}{|x|^n}\) е коректно изображение от \(H_{n^2}\) към \(\mathbb{C}_n\). Да проверим че ни върши работа.
    \begin{enumerate}
        \item \(\varphi(x.y) = \displaystyle\frac{(x.y)^n}{|x.y|^n} = \displaystyle\frac{x^n.y^n}{|x|^n.|y|^n} = \displaystyle\frac{x^n}{|x|^n}. \displaystyle\frac{y^n}{|y|^n} = \varphi(x) . \varphi(y)\). Значи \(\varphi\) е ХММ.
        \item Искаме да видим, че \(\varphi\) е сюрекция.
        Нека тогава вземем произволен елемент на \(\mathbb{C}_n\).
        Нека това \(z = \omega_n^k\) за някое \(k \in \{0, 1, \dots, n - 1\}\).
        Искаме да намерим \(x \in H_{n^2}\), такова че \(\varphi(x) = z\).
        Ами човек ако поогледа така двете множества и му дойдат някакви спомени от 1-ви курс ще се усети кой е директния кандидат. Хубаво е той да има модул 1 понеже \(|z| = |\omega_n^k| = 1\). Нека пробваме с \(\omega_{n^2}^k\).
        \(\varphi(\omega_{n^2}^k) = \displaystyle\frac{(\omega_{n^2}^k)^n}{1^n} = (\omega_{n^2}^n)^k = (\omega_n^1)^k = \omega_n^k = z\).
        Сега остава да се убедим, че \(\omega_{n^2}^k \in H_{n^2}\).
        Имаме \(|\omega_{n^2}^k| = 1\) и \((\omega_{n^2}^k)^{n^2} = 1\).
        Значи \((\omega_{n^2}^k)^{n^2} = 1 = |\omega_{n^2}^k|^{n^2}\).
        Така \(\omega_{n^2}^k \in H_{n^2}\) и \(\varphi(\omega_{n^2}^k) = \omega_n^k = z\). Следователно \(\varphi\) е сюрекция.
        \item Имаме \(x \in H_n \iff x \in H_{n^2} \; \& \; x^n = |x|^n
        \iff x \in H_{n^2} \; \& \; \displaystyle\frac{x^n}{|x|^n} = 1 \iff
        x \in H_{n^2} \; \& \; \varphi(x) = 1 \iff x \in \mathrm{Ker}(\varphi)\).
        Значи \(\mathrm{Ker}(\varphi) = H_n\).
    \end{enumerate}
    От първата теорема за ХММ имаме \(H_{n^2} / \mathrm{Ker}(\varphi) \cong \mathrm{Im}(\varphi)\), тоест \(H_{n^2}  / H_{n}  \cong \mathbb{C}_n\).
\end{enumerate}
\section{Последна задача за хомоморфизми}
Нека
\begin{enumerate}
    \item \(G = \left\{\begin{pmatrix}
    a & b \\
    0 & 1
    \end{pmatrix} \; \Big| \; a \in \mathbb{Q}^* \; \& \; b \in \mathbb{Q}\right\}\)
    \item \(M = \left\{\begin{pmatrix}
    a & 0 \\
    0 & 1
    \end{pmatrix} \; \Big| \; a \in \mathbb{Q}^*\right\}\)
    \item \(H = \left\{\begin{pmatrix}
    1 & b \\
    0 & 1
    \end{pmatrix} \; \Big| \; b \in \mathbb{Q}\right\}\)
\end{enumerate}
Да се докаже, че
\begin{enumerate}[label=\alph*)]
    \item \(G\) образува група относно операцията умножение на квадратни матрици с рационални коефициенти.
    \item \(M\) образува подгрупа на групата образувана от \(G\)
    и тя е изоморфна на \(<\mathbb{Q}^*, 1, . , ()^{-1}>\).
    \item \(H\) образува нормална подгрупа на групата образувана от \(G\) \\
    и \(G / H \cong <\mathbb{Q}^*, 1, . , ()^{-1}>\). 
\end{enumerate}
\subsection{Решение:}
\begin{enumerate}[label=\alph*)]
    \item
\end{enumerate}
\section{Задача от теория на числата!}
Нека \(a, b \in \mathbb{Z}\) са такива, че \(gcd(a, b) = 5\).
Да се намери на колко може да е равно \(gcd(13a + 36b, 2a + 5b)\) ?
\subsection{Решение:}
Нека \(d = gcd(13a + 36b, 2a + 5b)\). Тогава
\begin{align*}
    d = gcd(13a + 36b, 2a + 5b) = gcd(12a + a + 30b + 6b, 2a + 5b) = \\
    gcd(6(2a + 5b) + a + 6b, 2a + 5b) = gcd(a + 6b, 2a + 5b) = \\
    gcd(2a - a + 5b + b, 2a + 5b) = gcd(b - a, 2a + 5b).
\end{align*}
Тогава \(d \mid 5(b - a) + (-1)(2a + 5b)\) значи \(d \mid 7a\).
Но \(d \mid 2(b - a) + 1(2a + 5b)\) значи \(d \mid 7b\).
Щом \(d \mid 7a\) и \(d \mid 7b\), то по дефиниция \(d \mid gcd(7a, 7b)\).
Но \(gcd(7a, 7b) = 7.gcd(a, b) = 7.5\). Значи \(d \mid 7.5\).
Тогава \(d \in \{1, 5, 7, 35\}\).
Видяхме \(d = gcd(b - a, 2a + 5b)\).
Но \(5 \mid b - a\) и \(5 \mid 2a + 5b\) следователно \(5 \mid d\).
Значи остава \(d \in \{5, 35\}\).
Сега остава да проверим, кои от тези случай могат да се реализират.
Търсим конкретни \(a\) и \(b\) така, че \(gcd(a, b) = 5\) и \(d = 5\).
Да пробваме с \(b = 5\) и \(a = 0\). Първо ясно е, че \(gcd(5, 0) = 5\).
\(gcd(5 - 0, 2.0 + 5.5) = gcd(5, 25) = 5\). Значи \(d = 5\) при \(b = 5\) и \(a = 0\).
Търсим конкретни \(a\) и \(b\) така, че \(gcd(a, b) = 5\) и \(d = 35\).
Да пробваме с \(b = 5\) и \(a = 5\). Първо ясно е, че \(gcd(5, 5) = 5\).
\(gcd(5 - 5, 2.5 + 5.5) = gcd(0, 7.5) = 7.5 = 35\). Значи \(d = 35\) при \(b = 5\) и \(a = 5\).
\\
\vspace*{2mm}
\\
Отговор: \(5\) и \(35\)
\end{document}
