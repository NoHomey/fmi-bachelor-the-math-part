\documentclass[12pt]{article}
\usepackage[left=3cm,right=3cm,top=1cm,bottom=2cm]{geometry}
\usepackage{amsmath}
\usepackage{amssymb}
\usepackage[T1,T2A]{fontenc}
\usepackage[utf8]{inputenc}
\usepackage[bulgarian]{babel}
\usepackage[normalem]{ulem}

\setlength{\parindent}{0mm}

\title{Теоритично контролно №3, I, Информатика}
\author{Иван Йочев \and Кристиян Симов}

\begin{document}

\maketitle

\section{Полиноми на една променливва}

\subsection{Теорема за деление с частно и остатък за полиноми}

Нека F - поле\\
Нека $f,g \in F[x], g \neq 0$ \\
$\Rightarrow \exists! q,r \in F[x]\ :\ f=g.q + r,\ deg(r) < deg(g)$\\

\subsection{Схема на Хорнер}

Нека F - поле \\
Нека $f = \sum\limits_{i=0}^{n} a_ix^{n-i},\ g = \alpha - x \in F[x]$\\
$  q,r \in F[x]\ : f=g.q + r ,\ deg(r) < deg(g)$ \\
$q = b_0x^{n-1}+...+b_n$\\\\
Схема на Хорнер: \\
$b_0 = a_0$ \\
$b_1 = a_1 + \alpha b_0$ \\
$b_2 = a_2 + \alpha b_1$ \\
\vdots \\
$b_{n-1} = a_{n-1} + \alpha b_{n-2}$\\
$r = a_n + \alpha b_{n-1}$

\subsection{Идеали в пръстена от полиномите с коефиценти от дадено поле}
Ако F - поле, то всеки идеал на $F[x]$ е главен.

\subsection{Максимален брой различни корени на ненулев полином с коефиценти от дадена област и от степен n}
Нека A - област \\
Нека $f \in A[x]\ :\ f \neq 0,\ deg(f) = n$ \\
$\Rightarrow$ f има най-много n различни корена

\subsection{Принцип за сравняване на коефицентите на полиноми}
Нека A - област \\
Нека $g_1, g_2 \in A[x] : \ deg(g_1),deg(g_2)\leq n$ \\
Ако $\alpha_1 ... \alpha_{n+1} \in A \text{ - различни и } \\ \forall i \in \{1,...,n+1\}\ :\ g_1(\alpha_i) = g_2(\alpha_i)$\\
тогава $g_1 = g_2$

\section{Аритметика в пръстена на полиномите}

\subsection{Определeние за деление на полиноми}
Нека F - поле.
Нека $g,f \in F[x], g \neq 0$ \\
$g$ дели $f \ (g \ \vert \ f)$ ако:\\
$\exists h \in F[x]\ :\ f=g.h$  

\subsection{Полином дели произведението на два полинома и е взаимно прост с един от тях}
Нека F - поле
Нека $g, f_1, f_2 \in F[x]$\\
Ако $g \ \vert \ f_1f_2\ \land\ (g, f_1)=1\ \Rightarrow\  g \ \vert \ f_2$\\

\subsection{Най-голям общ делител на два полинома}
Нека F - поле \\
Нека $f,g \in F[x]$, БОО $g \neq 0$ \\
$\Rightarrow d \in F[x]$ e НОД на f и g ($(f,g)=d$), ако:\\
1) $d \ \vert \ f \ \land d \ \vert \ g \ $ \\
2) $(\exists d_1 \in F[x]\ :\ d_1 \ \vert \ f \ \land d_1 \ \vert \ g \ ) \rightarrow d_1 \ \vert \ d$

\subsection{Тъждество на Безу за два полинома}
Нека F - поле \\
Нека $f,g \in F[x]$ \\
Нека $d \in F[x]\ :\ (f,g) = d$\\
$\Rightarrow \exists u,v \in F[x]\ :\ uf + vg = d$

\subsection{Най-малко общо кратно на два полинома}
Нека F - поле \\
Нека $f,g \in F[x]$ \\
$k \in F[x]$е НОК на f и g ($[f,g] = k$), ако: \\
1) $f \ \vert \ k \land g \ \vert \ k$ \\
2) ($\exists k_1 \in F[x] : f \ \vert \ k_1 \land g \ \vert \ k_1$) $\rightarrow k \ \vert \ k_1$

\subsection{Пораждащ елемент на идеала $(f) + (g)$}
Нека F - поле \\
Нека $f,g \in F[x]$ \\
Тогава идеалът $(f) + (g)$ се поражда от елемента $(f,g) \in F[x]$ \\
т.е  $(f) + (g) = ((f,g))$\\

\subsection{Пораждащ елемент на идеала $(f) \cap (g)$}
Нека F - поле \\
Нека $f,g \in F[x]$ \\
Тогава идеалът $(f) \cap (g)$ се поражда от елемента $[f,g] \in F[x]$ \\
т.е  $(f) \cap (g) = ([f,g])$\\

\subsection{Неразложим полином над дадено поле}
Нека F - поле \\
Нека $f \in F[x],\ deg(f) > 0$ \\
f - неразложим, ако: \\
$ \nexists g,h \in F[x] : f = gh\ \land\ (0 < deg(g),deg(h) < deg(f))$

\subsection{Неразложим полином дели произведението на два други}
Нека F - поле\\
Нека $p,f_1,f_2 \in F[x]$, p - неразложим \\
Тогава $(p \ \vert \ f_1f_2) \leftrightarrow p \ \vert \ f_1 \lor p \ \vert \ f_2$

\subsection{Теорема за разлагане на полином на неразложими множители}
Нека F - поле \\
Нека $f \in F[x]\ :\ deg(f) > 0$ \\
$\Rightarrow \exists $*$! p_1,...,p_k \in F[x] \ (p_i$ - неразложим, $i = 1,...,k) : f = p_1p_2...p_k$ \\ \\
*Разлагането е единствено с точност до реда на полиномите и мултипликативни ненулеви константи, т.е: \\
Ако $f = p_1,...,p_k = q_1,...,q_s$ \\
$\Rightarrow k = s \ \land \ q_i=a_ip_i,\\
 a_i \in F (a_i \neq 0),\ i=1,...,k$ \\

\section{Корени на полиномите}

\subsection{Какъв е полином е f, ако факторпръстенът F[x]/(f) е поле}
Нека F - поле \\
Нека $f \in F[x] : deg(f) > 0$ \\
Тогава ако $F[x]/(f)$ e поле $ \Rightarrow$ f - неразложим

\subsection{Какъв е факторпръстенът F[x]/(f), ако f e неразложим}
Нека F - поле \\
Нека $f \in F[x] : deg(f) > 0$ \\
Тогава ако f - неразложим $\Rightarrow F[x]/(f)$ е поле 

\subsection{Определение за поле на разлагане}
Нека F - поле\\
Нека $f \in F[x] : deg(f) > 0$\\
$\alpha_1,...,\alpha_n$ са всички корени на f\\
Нека $L > F : \alpha_1,...,\alpha_n \in L$ \\
$\Rightarrow\ K = \bigcap\limits_{\substack{F \leq P \leq L \\ \alpha_1,...,\alpha_n \in P}} P,  $ - поле на разлагане на f над F\\
Ако $K_1,\ K_2$ - полета на разлагане на f над F $\Rightarrow\ K_1 \cong K_2$\\
т.е полето на разлагане е единствено с точност до изоморфизъм

\subsection{Формули на Виет за полином от четвърта степен}
Нека F - поле \\
Нека $f = \sum\limits_{i=0}^{4}a_ix^{4-i}$ \\
Нека $\alpha_1 , \alpha_2 , \alpha_3 , \alpha_4$ - са корени на f \\
т.е $f = a_0(x-\alpha_1)(x-\alpha_2)(x-\alpha_3)(x-\alpha_4)$ \\\\
Формули на Виет\\
1) $\alpha_1 + \alpha_2 + \alpha_3 + \alpha_4 = -\frac{a_1}{a_0}$ \\
2) $\alpha_1\alpha_2 + \alpha_1\alpha_3 + \alpha_1\alpha_4 + \alpha_2\alpha_3 + \alpha_2\alpha_4 + \alpha_3\alpha_4 = \frac{a_2}{a_0}$\\
3) $\alpha_1\alpha_2\alpha_3 + \alpha_1\alpha_2\alpha_3 + \alpha_1\alpha_2\alpha_4 + \alpha_1\alpha_3\alpha_4 + \alpha_2\alpha_3\alpha_4 = -\frac{a_3}{a_0}$\\
4) $\alpha_1\alpha_2\alpha_3\alpha_4 = \frac{a_4}{a_0}$

\subsection{Определение за k-кратен корен на полином}
Нека F - поле, K > F \\
Нека $f \in F[x],\ \alpha \in K$  \\
$\alpha$ е k-кратен корен на f, ако:\\
$f = (x-\alpha)^k.g, \ g \in K[x] : g(\alpha) \neq 0$

\subsection{НДУ полином над поле с характеристика 0 да има k-кратен корен}
Нека F - поле и char F = 0, K > F \\
Нека $f \in F[x],\ \alpha \in K$\\
$\alpha$ е k-кратен корен $\leftrightarrow $\\
1) $f(\alpha) = f'(\alpha)=...=f^{(k-1)}(\alpha) = 0$ \\
2) $f^{(k)}(\alpha) \neq 0$

\section{Симетрични полиноми}

\subsection{Лема за старшия едночлен за полиноми на много променливи}
Нека A - област \\
Нека $0 \neq f,g \in A[x_1,...x_n]$ \\
Нека $u = a\prod\limits_{k=1}^{n}x_k^{i_k} (0 \neq a \in A)$ - старши едночлен на f\\
Нека $v = b\prod\limits_{k=1}^{n}x_k^{j_k} (0 \neq b \in A)$ - старши едночлен на g\\
$\Rightarrow u.v = ab\prod\limits_{k=0}^{n}x_k^{i_k+j_k}$ - старши едночлен на f.g

\subsection{Лексикографска наредба на едночлени на n променливи}
Нека A - област \\
Нека $u = a\prod\limits_{k=1}^{n}x_k^{i_k}(0 \neq a \in A)$\\
Нека $v = b\prod\limits_{k=1}^{n}x_k^{j_k}(0 \neq a \in A)$\\
u и v са неподобни едночлени \\
$u > v$, ако $\exists k \in \mathbb{N}$ : \\
$(\forall t\in\{1,2,...,k-1\} \ i_t = j_t) \land (i_k > j_k)$

\subsection{Симетричен полином}
Нека A - област\\
Нека $f = f(x_1, f_2, ..., f_n) \in A[x_1,x_2, ..., x_n]$ \\
f е симетричен $\leftrightarrow \forall \sigma \in S_n : \\
f(x_1, x_2, ..., x_n) = f(x_{\sigma(1)}, x_{\sigma(2)}, ..., x_{\sigma(n)})$

\clearpage
\subsection{Елементарни симетрични полиноми}
$\sigma_1=\sigma_1(x_1,x_2,...,x_n)=x_1+x_2+...+x_n$\\
$\sigma_2=\sigma_2(x_1,x_2,...,x_n)=x_1x_2+x_1x_3+...+x_{n-1}x_n$\\
$\sigma_3=\sigma_3(x_1,x_2,...,x_n)=x_1x_2x_3+x_1x_2x_4+...+x_{n-2}x_{n-1}x_n$\\
$\vdots$\\
$\sigma_n=\sigma_n(x_1,x_2,...,x_n)=x_1x_2x_3...x_n$\\\\
За n = 4 \\
$\sigma_2=\sigma_2(x_1,x_2,x_3,x_4)=x_1x_2+x_1x_3+x_1x_4+x_2x_3+x_2x_4+x_3x_4$\\

\subsection{Основна теорема за симетричните полиноми}
Нека A - област\\
Нека $f = f(x_1, x_2, ..., x_n) \in A[x_1, x_2, ..., x_n]$\\
$\Rightarrow \exists! g\in A[x_1, x_2, ..., x_n] : f(x_1, ..., x_n) = g(\sigma_1, ..., \sigma_n)$\\

\subsection{Формули на Нютон}
Нека A - област\\
Нека $f = f(x_1,...x_n) \in A[x_1,...,x_n]$\\
Нека $S_k = x_1^k + x_2^k + ... + x_n^k \ 1 < k \leq n$\\\\
Формули на Нютон \\
$S_k - \sigma_1S_{k-1} + \sigma_2S_{k-2} - ... + (-1)^{k-1}\sigma_{k-1}S_1 + (-1)^k\sigma_kk = 0$

\section{Дискриминанта и резултанта}

\section{Полиноми с рационални коефиценти}

\subsection{Определение за примитивен полином}
Нека $f = a_0x^n+...+a_n \in \mathbb{Z}[x]$ \\
$f$ - примитивен $\leftrightarrow (a_0, a_1, ..., a_n) = 1$

\subsection{Лема на Гаус за полиноми с цели коефиценти}
Нека $g = a_0x^n+...+a_n \in \mathbb{Z}[x]$ \\
Нека $h = b_0x^n+...+b_n \in \mathbb{Z}[x]$ \\
f и g са примитивни полиноми \\
Тогава $f = g.h$ също е примитивен полином

\subsection{Редукционен критерий за неразложимост на полиноми с цели коефиценти}
Нека $f \in \mathbb{Z}[x]$ \\
Нека $p \in \mathbb{P}$ - произволно просто число \\
Нека $\bar{f} \in \mathbb{Z}_p[x]$ е полиномът f, редуциран по модул p \\
$\bar{f}$ е неразложим над $\mathbb{Z}_p \Rightarrow f$ е неразложим над $\mathbb{Z}$

\subsection{Критерий на Айзенщайн за неразложимост на полиноми с цели коефиценти}
Нека $f = \sum\limits_{i=0}^{n}a_ix^{n-i} \in \mathbb{Z}[x]$ \\
f е неразложим над $\mathbb{Q}$, ако $\exists p \in \mathbb{P}: $\\
1) $p \nmid a_0$ \\
2) $p \mid a_1, ..., a_n$ \\
3) $p^2 \nmid a_n$


\end{document}
