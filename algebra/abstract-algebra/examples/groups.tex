\documentclass[a4paper, 12pt, oneside]{article}
\usepackage[left=3cm,right=3cm,top=1cm,bottom=2cm]{geometry}
\usepackage{amsmath,amsthm}
\usepackage{amssymb}
\usepackage{lipsum}
\usepackage{stmaryrd}
\usepackage[T1,T2A]{fontenc}
\usepackage[utf8]{inputenc}
\usepackage[bulgarian]{babel}
\usepackage[normalem]{ulem}

\newcommand{\N}{\mathbb{N}}
\newcommand{\G}{\mathbb{G}}

\setlength{\parindent}{0mm}

\title{Предложения за домашно 2019}
\author{Иво Стратев}

\begin{document}
\maketitle
\section*{Задача 1.}
Нека \((\G, .)\) е група, която действа на множеството \(\Omega\).
\subsection*{а) Да се докаже, че ако \(n \in \N\) и \(n > 1\), то \((\G, .)\) действа на \(\Omega^n\)}
Нека \(n \in \N\) и \(n > 1\).
\((\G, .)\) действа на множеството \(\Omega\) следователно
\begin{align*}
    \exists \; \phi \in Hom((\G, .), (S_\Omega, \circ))
\end{align*}
Нека \(\phi \in Hom((\G, .), (S_\Omega, \circ))\).
Дефинираме
\begin{align*}
    \Phi \; : \; \G \to (\Omega^n \to \Omega^n) \\
    g \mapsto [(\omega_1, \dots, \omega_n) \mapsto (\phi(g)(\omega_1), \dots, \phi(g)(\omega_n))]
\end{align*}
Ще докажем, че \(\Phi \in Hom((\G, .), (S_{\Omega^n}, \circ))\) \\
Нека \(g_1, g_2 \in \G\) и \(\omega = (\omega_1, \dots, \omega_n) \in \Omega^n\) тогава
\begin{align*}
   \Phi(g_1.g_2)(\omega) = (\phi(g_1.g_2)(\omega_1), \dots, \phi(g_1.g_2)(\omega_n)) \\
   = (\phi(g_1)(\phi(g_2)(\omega_1)), \dots, \phi(g_1)(\phi(g_2)(\omega_n))) \\
   = \Phi(g_1)((\phi(g_2)(\omega_1), \dots, \phi(g_2)(\omega_n))) \\
   = \Phi(g_1)(\Phi(g_2)(\omega)) = (\Phi(g_1) \circ \Phi(g_2))(\omega)
\end{align*}
Следователно е в сила \(\Phi(g_1.g_2) = \Phi(g_1) \circ \Phi(g_2)\). \\
Тоест вярно е \(\forall g_1 \in \G \; \forall g_2 \in \G \; \Phi(g_1.g_2) = \Phi(g_1) \circ \Phi(g_2)\). \\
Следователно \(\Phi \in Hom((\G, .), (S_{\Omega^n}, \circ))\). \(\qed\)
\subsection*{б) Да се докаже, че \(St_\G((\omega_1, \dots, \omega_n)) =  St_\G(\omega_1) \cap \dots \cap St_\G(\omega_n)\)}
Нека \((\omega_1, \dots, \omega_n) \in \Omega^n\) нека \(g \in \G\). Очевидно
\((\forall i \in \{1, \dots, n\} \; \phi(g)(\omega_i) = \omega_i) \iff \Phi(g)((\omega_1, \dots, \omega_n)) = (\phi(g)(\omega_1), \dots, \phi(g)(\omega_n)) = (\omega_1, \dots, \omega_n)\) \\
Тоест \(g \in S_\G(\omega_1) \cap \dots \cap S_\G(\omega_n) \iff g \in S_\G((\omega_1, \dots, \omega_n))\). \\
Следователно \(St_\G((\omega_1, \dots, \omega_n)) =  St_\G(\omega_1) \cap \dots \cap St_\G(\omega_n)\). \(\qed\)
\end{document}
