\documentclass[12pt]{article}
\usepackage[left=3cm,right=3cm,top=1cm,bottom=2cm]{geometry}
\usepackage{amsmath}
\usepackage{amssymb}
\usepackage[T1,T2A]{fontenc}
\usepackage[utf8]{inputenc}
\usepackage[bulgarian]{babel}
\usepackage[normalem]{ulem}

\setlength{\parindent}{0mm}

\newcommand{\stkout}[1]{\ifmmode\text{\sout{\ensuremath{#1}}}\else\sout{#1}\fi}
\newcommand{\x}[2]{\text{\tiny{\(#1 \times #2\)}}}

\title{Теоритично контролно №1 1, I, Информатика}
\author{Иво Стратев}

\begin{document}
    \maketitle
    \section{Комплексни числа (\(\mathbb{C}\))}
    \(z = -5 -4\imath\)
    \subsection{\(Re\,z\)}
    \(Re\,z = -5\)
    \subsection{\(Im\,z\)}
    \(Im\,z = -4\)
    \subsection{\(|z|\)}
    \(|z| = \sqrt{(Re\,z)^2 + (Im\,z)^2} = \sqrt{25 + 16} = \sqrt{41}\)
    \subsection{\(\tg{Arg\,z}\)}
    \(\tg{Arg\,z} = \frac{Im\,z}{Re\,z} = \frac{-4}{-5} = \frac{4}{5}\)
    \subsection{\(\sin{Arg\,z}\)}
    \(\sin{Arg\,z} = \frac{Im\,z}{|z|} = \frac{-4}{\sqrt{41}}\frac{\sqrt{41}}{\sqrt{41}} = \frac{-4\sqrt{41}}{41}\)
    \subsection{\(\cos{Arg\,z}\)}
    \(\cos{Arg\,z} = \frac{Re\,z}{|z|} = \frac{-5}{\sqrt{41}}\frac{\sqrt{41}}{\sqrt{41}} = \frac{-5\sqrt{41}}{41}\)
    \section{\(z = \frac{5 - 3\imath}{4 + \imath} \quad Re\,z + Im\,z\)}
    \(z = \frac{5 - 3\imath}{4 + \imath}\\
    \\
    z = \frac{5 - 3\imath}{4 + \imath} \frac{4 - \imath}{4 - \imath}\\
    \\
    z = \frac{(5 - 3\imath)(4 - \imath)}{4^2 + 1^2}\\
    \\
    z = \frac{20 - 5\imath - 12\imath - 3}{17}\\
    \\
    z = \frac{\stkout{17} - \stkout{17}\imath}{\stkout{17}}\\
    \\
    z = 1 - \imath\\
    \\
    Re\,z + Im\,z = 1 + (-1) = 1 - 1 = 0\)
    \section{Формули на Моавър}
    \subsection{\(z^n\)}
    \(z^n = |z|^n(\cos{\left(nArg\,z\right)} + \imath\sin{\left(nArg\,z\right)})\)
    \subsection{\(\sqrt[n]z\)}
    \(\sqrt[n]z\ = \sqrt[n]{|z|}(\cos{\frac{Arg\,z + 2k\pi}{n}} + \imath\sin{\frac{Arg\,z + 2k\pi}{n}}) \quad k = 0, \, 1, \, \dots, \, n - 1\)
    \section{Системи линейни уравнения}
    \subsection{съвместима}
    Една система от линейни уравнения се нарича съвместима, когато има поне едно решение.
    \subsection{несъвместима}
    Една система от линейни уравнения се нарича несъвместима, когато няма решение.
    \subsection{определена}
    Една система от линейни уравнения се нарича определена, когато е съвместима и има точно едно решение.
    \subsection{неопределена}
    Една система от линейни уравнения се нарича неопределена, когато е съвместима и има повече от едно решение.
    \section{Релации и изображения}
    \subsection{Релации}
    \(R \, \subseteq \, A \times A;\)
    \subsubsection{симетрична релация}
    \((\forall x \in A) \; (\forall y \in A) \; [\; (x, \; y) \in R \implies (y, \; x) \in R \; ]\)
    \subsubsection{транзитивна релация}
    \((\forall x \in A) \; (\forall y \in A) \; (\forall \; z \in A) \; [\; (x, \; y), \; (y, \; z) \in R \implies (x, \; z) \in R \;]\)
    \subsubsection{рефлексивна релация}
    \((\forall x \in A) \; [\; (x, \; x) \in R \;]\)
    \subsection{Изображения}
    \(f \, : \, X \to Y\)\\
    \subsubsection{инективно изображение}
    \((\forall x_1 \in X) \; (\forall x_2 \in X) \; [\; x_1 \neq x_2 \implies f(x_1) \neq f(x_2) \;]\)
    \subsubsection{сюрективно изображение}
    \((\forall y \in Y) \; (\exists x \in X) \; [\; y = f(x) \;]\)
    \subsubsection{биекция}
    Биекция наричаме изображение, което е едновременно инкеция и сюрекция. Тоест
    \(f \, : \, X \to Y\)\\
    \((\forall x_1 \in X) \; (\forall x_2 \in X) \; [\; x_1 \neq x_2 \implies f(x_1) \neq f(x_2) \;]\)\\
    \((\forall y \in Y) \; (\exists x \in X) \; [\; y = f(x) \;]\)
    \section{Бинарни операции}
    \(*: M  \times M \to M\)
    \subsection{асоциативност}
    \((\forall a \in M) \; (\forall b \in M) \; (\forall c \in M ) \; [\; (a \, * \, b) \, * \, c = a \, * \, (b \, * \, c) = a \, * \, b \, * \, c \;]\)
    \subsection{комутативност}
    \((\forall a \in M) \; (\forall b \in M) \; [\; a \, * \, b = b \, * \, a\;]\)
    \subsection{неутрален елемент}
    \((\exists \, \theta \in M) \; (\forall x \in M) \; [\; x \, * \, \theta = \theta \, * \, x = x \;]\)
    \section{Матрици}
    \subsection{\(A ^ t\)}
    \(A = (a_{ij}) \x{m}{n} \in M\x{m}{n}(F) \; (i = 1, \, 2, \, \dots, \, m, \quad j = 1, \, 2, \, \dots, \, n) \\\\
    B = (b_{ij}) \x{n}{m} = A ^ t \in M\x{n}{m}(F) \; :  \; b_{ij} = a_{ji} \; (i = 1, \, 2, \, \dots, \, n, \quad j = 1, \, 2, \, \dots, \, m);\)
    \subsection{\(A + B\)}
    \(A = (a_{ij}) \x{m}{n}, \, B = (b_{ij}) \x{m}{n} \in M\x{m}{n}(F) \\\\
    A + B = C = (c_{ij}) \x{m}{n} \in M\x{m}{n}(F)\\\\
    c_{ij} = a_{ij} + b_{ij} \; (i = 1, \, 2, \, \dots, \, m, \quad j = 1, \, 2, \, \dots, \, n)\)
    \subsection{\(\lambda A\)}
    \(\lambda \in F, \, A = (a_{ij}) \x{m}{n} \in M\x{m}{n}(F) \\\\
    \lambda A = C  = (c_{ij}) \x{m}{n} \in M\x{m}{n}(F) \\\\
    c_{ij} = \lambda a_{ij} \; (i = 1, \, 2, \, \dots, \, m, \quad j = 1, \, 2, \, \dots, \, n)\)
    \section{Вектори в линейно пространство}
    Нека (F, +, .) е числово поле. Нека (V, +, .) е линейно пространство над (F, +, .).
    \subsection{нулевият вектор е единствен}
    Нека \(\theta' \, \text{и} \, \theta''\) са нулеви вектори от V. Тогава:\\\\
    \(\theta' + \theta'' = \theta''\) (защото \(\theta'\) е нулев вектор)\\\\
    \(\theta' + \theta'' = \theta'\) (защото \(\theta''\) е нулев вектор)\\\\
    \(\implies \theta' = \theta''\)
    \subsection{противоположният вектор е единствен}
    Нека a е вектор от V и нека \(a' \, \text{и} \, a''\) са негови противоположни вектори от V. Тогава:\\\\
    \(a' + a + a'' = (a' + a) + a'' = \theta + a'' = a''\) (защото \(a'\) е противоположен вектор на a)\\\\
    \(a' + a + a'' = a' + (a + a'') = a' + \theta = a'\) (защото \(a''\) е противоположен вектор на a)\\\\
    \(\implies a' = a''\)
    \subsection{\(0a = \theta\)}
    Нека \(a \in V\). Тогава\\\\
    \(a + -a = \theta \implies \\\\
    1.a + -a = \theta \implies \\\\
    (1 + 0).a + -a = \theta \implies \\\\
    1.a + 0.a + -a = \theta \implies \\\\
    a + 0.a + -a = \theta \implies \\\\
    \theta + 0.a = \theta \implies \\\\
    0.a = \theta\)
    \subsection{\(\lambda \theta = \theta\)}
    \(a \in V\\\\
    a + 0a = 1a + 0a = (1 + 0)a = a\\\\
    a + 0a = a \, | + (-a)\\\\
    \theta + 0a = \theta\\\\
    0a = \theta\)\\\\
    Сега в равенството: \(\lambda(\mu a) = (\lambda \mu)a \) избираме \(\mu = 0\) \\\\
    и получаваме: \(\lambda \theta = 0a = \theta\)
    \subsection{\(-1a = -a\)}
    \(a + (-1)a = 1.a + (-1)a = (1 + -1)a = 0a = \theta \implies -a = (-1)a\)
    \section{Линейно пространство, линейна комбинация и линейна зависимост/независимост}
    Нека (F, +, .) е числово поле. Нека (V, +, .) е линейно пространство над (F, +, .).
    \subsection{линейна комбинация}
    Нека \(n \in \mathbb{N}\).\\\\
    Нека \(a_1, \, a_2, \, \dots, \, a_n \in V\). \\\\
    Нека \(\lambda_1, \, \lambda_2, \, \dots, \, \lambda_n \in F\).\\\\
    Тогава \(\displaystyle\sum_{i=1}^{n} \lambda_i a_i \in V\) е линейна комбинация
    на  \(a_1, \, a_2, \, \dots, \, a_n \in V\)
    с коефициенти \(\lambda_1, \, \lambda_2, \, \dots, \, \lambda_n\).
    \subsection{линейно подпространство}
    Нека \(\emptyset \neq W \subseteq V\).\\\\
    \(W\) е подпространство на \(V\), ако
    \((\forall w_1 \in W) \; (\forall w_2 \in W) \; [\; w_1 + w_2 \in W  \;]\\\\
    (\forall \lambda \in F) \; (\forall w \in W) \; [\; \lambda . w \in W  \;]\)
    \subsection{линейна обвивка}
    Нека \(\emptyset \neq A \subseteq V\)\\\\
    Тогава \(l(A) = \displaystyle\bigcap_{A \subseteq W \leq V} W\) е линейната обвивка на множеството \(A\).
    \subsection{линейна зависимост}
    Нека \(n \in \mathbb{N}\).\\\\
    Нека \(a_1, \, a_2, \, \dots, \, a_n \in V\). Векторите \(a_1, \, a_2, \, \dots, \, a_n\)  са линейно зависими, ако \\\\
    \((\exists \lambda_1 \in F) \; (\exists \lambda_2 \in F) \; \dots \; (\exists \lambda_n \in F) \; \left[\; \displaystyle\sum_{i=1}^{n} \lambda_i a_i = \theta \; \land \; (\lambda_1, \, \lambda_2, \, \dots, \, \lambda_n) \neq (0, \, 0, \, \dots, \, 0) \; \right ]\).
    \subsection{линейна независимост}
    Нека \(n \in \mathbb{N}\).\\\\
    Нека \(a_1, \, a_2, \, \dots, \, a_n \in V\). Векторите \(a_1, \, a_2, \, \dots, \, a_n\)  са линейно независими, ако \\\\
    \((\forall \lambda_1 \in F) \; (\forall \lambda_2 \in F) \; \dots \; (\forall \lambda_n \in F) \; \left[\; \displaystyle\sum_{i=1}^{n} \lambda_i a_i = \theta  \implies (\lambda_1, \, \lambda_2, \, \dots, \, \lambda_n) = (0, \, 0, \, \dots, \, 0) \; \right ]\).
    \section{Линейна зависимост/независимост}
    \subsection{Aко един вектор е линейно независим, то той е ненулев вектор}
    Логически еквивалетно твърдение на "ако един вектор е линейно независим, то той е ненулев вектор" е "ако един вектор е линейно зависим, то той е нулев вектор"\\\\
    За това ще докажем него, от което ще следва и исканото твърдение. \\\\
    Нека \(\lambda \in F, \, a \in V\) и \(\lambda a = \theta \; , \; \lambda \neq 0\). \\\\
    Тогава \(\lambda a = \theta \, | \, \lambda^{-1}\\\\
    1.a = \theta\\\\
    a = \theta\\\\
    \implies \text{ако един вектор е линейно независим, то той е ненулев вектор}\)
    \subsection{Aко един вектор е линейно зависим то, той е нулевият вектор}
    Нека \(\lambda \in F, \, a \in V\) и \(\lambda a = \theta \; , \; \lambda \neq 0\). \\\\
    Тогава \(\lambda a = \theta \, | \, \lambda^{-1}\\\\
    1.a = \theta\\\\
    a = \theta\).
    \subsection{Всяка подсистема на линейно независима система от вектори е също линейно независима}
    Нека \(n, \, k \in \mathbb{N}\), са такива, че \(k \leq n\).\\\\
    Нека \(A = \{a_1, \, a_2, \, \dots, \, a_n\} \quad \text{е линейно независими система от вектори.}\\\\
    \text{Нека БОО} \; B = \{a_1, \, a_2, \, \dots, \, a_k\} \quad \text{допускаме, че B е линейно зависима}\\\\
    \implies (\exists \lambda_1 \in F)\; (\exists \lambda_2 \in F) \; \dots \; (\exists \lambda_k \in F)  \; [\; (\lambda_1, \dots , \lambda_k) \neq (0, \dots, 0) \; \land \; \displaystyle\sum_{i=1}^{k} \lambda_i a_i = \theta\).\\\\
    Нека тогава \(\lambda_1 \in F, \dots, \lambda_k \in F\) и \((\lambda_1, \dots , \lambda_k)  \neq (0, \dots, 0) \; \land \; \displaystyle\sum_{i=1}^{k} \lambda_i a_i = \theta\).
    Тогава \((\lambda_1, \dots , \lambda_k, 0, \; \dots \; 0) \neq (0, \dots , 0, 0, \; \dots \; 0)\)
    и \(\displaystyle\sum_{i=1}^{k} \lambda_i a_i + \displaystyle\sum_{j= k + 1}^{n} 0 a_j = \theta\). 
    Следователно \(a_1, \, a_2, \, \dots, \, a_n\) са линейно зависими и значи A е линейно независима система от вектори.
    Но това е Абсурд, понеже А е линейно независима. Следователно \(B\) е линейно зависима.
    \subsection{Ако една система от вектори съдържа линейно зависима подсистема, то тази система също е линейно зависима}
    Нека \(n, \, k \in \mathbb{N}\), са такива, че \(k \leq n\).\\\\
    Нека \(A = \{a_1, \, a_2, \, \dots, \, a_n\} \quad \text{ са система от вектори }\\\\
    \text{Нека БОО} \; B = \{a_1, \, a_2, \, \dots, \, a_k\} \quad \text{ е линейно зависима подсистема от вектори}\\\\
    \text{От B линейно зависима подсистема от вектори, следва}\\\\
    (\exists \lambda_1 \in F)\; (\exists \lambda_2 \in F) \; \dots \; (\exists \lambda_k \in F)  \; [\; (\lambda_1, \dots , \lambda_k) \neq (0, \dots, 0) \; \land \; \displaystyle\sum_{i=1}^{k} \lambda_i a_i = \theta\)\\\\
    Нека тогава \(\lambda_1 \in F, \dots, \lambda_k \in F\) и \((\lambda_1, \dots , \lambda_k)  \neq (0, \dots, 0) \; \land \; \displaystyle\sum_{i=1}^{k} \lambda_i a_i = \theta\).
    Тогава \((\lambda_1, \dots , \lambda_k, 0, \; \dots \; 0) \neq (0, \dots , 0, 0, \; \dots \; 0)\)
    и \(\displaystyle\sum_{i=1}^{k} \lambda_i a_i + \displaystyle\sum_{j= k + 1}^{n} 0 a_j = \theta\). 
    Следователно \(a_1, \, a_2, \, \dots, \, a_n\) са линейно зависими и значи A е линейно независима система от вектори.
    \subsection{Ако една система от вектори съдържа два пропорционални вектора, то тя е линейно зависима}
    Нека \(n \in \mathbb{N}\) и \(A = \{a_1, \, a_2, \, \dots, \, a_n\} \quad \text{- система от вектори.}\\\\
    \text{Нека БОО} \; a_2 = \lambda a_1 \implies \lambda a_1 - a_2 = \theta\\\\
    \implies \lambda a_1 + (-1)a_2 + \displaystyle\sum_{i=3}^{n} 0a_i = \theta\\\\
    \implies \text{A е линейно зависима система от вектори}\)
    \subsection{Ако в една система от поне два вектора един от векторите е линейна комбинация на останалите, то системата е линейно зависима}
    Нека \(n \in \mathbb{N}^+, \\\\
    A = \{a_1, \, a_2, \, \dots, \, a_n\} \quad \text{- система от вектори.}\\\\
    \text{Нека БОО} \; a_1 = \displaystyle\sum_{i=2}^{n} \lambda_i a_i\\\\
    \implies -a_1 + \displaystyle\sum_{i=2}^{n} \lambda_i a_i = \theta\\\\
    \implies (-1)a_1 + \displaystyle\sum_{i=2}^{n} \lambda_i a_i = \theta\\\\
    \implies \text{A е линейно зависима система от вектори}\)
    \subsection{В една линейно зависима система от поне два вектора поне един вектор е линейна комбинация на останалите}
    Нека \(n \in \mathbb{N}^+, \\\\
    A = \{a_1, \, a_2, \, \dots, \, a_n\} \quad \text{- линейно зависима система от поне два вектора. Тогава}\\\\
    (\exists \lambda_1 \in F)\; (\exists \lambda_2 \in F) \; \dots \; (\exists \lambda_n \in F)  \; [\; (\lambda_1, \dots , \lambda_n) \neq (0, \dots, 0) \; \land \; \displaystyle\sum_{i=1}^{n} \lambda_i a_i = \theta\)\\\\
    Нека тогава \(\lambda_1 \in F, \dots, \lambda_n \in F\) и \((\lambda_1, \dots , \lambda_n)  \neq (0, \dots, 0) \; \land \; \displaystyle\sum_{i=1}^{n} \lambda_i a_i = \theta\).\\
    Нека БОО \(\lambda_1 \neq 0\). Тогава \\
    \(\displaystyle\sum_{i=1}^{n} \lambda_i a_i = \theta \, | \lambda_1^{-1} \implies a_1 + \displaystyle\sum_{i=2}^{n} \frac{\lambda_i}{\lambda_1} a_i = \theta\\\\
    \implies a_1 = \displaystyle\sum_{i=2}^{n} -\frac{\lambda_i}{\lambda_1} a_i\)
    \section{Базис и размерност}
    Нека (V, +, .) е линейно пространство над числовото поле (F, +, .).
    \subsection{Основна лема на алгебрата}
    Нека \(n, k \in \mathbb{N}\).\\\\
    Нека \(a_1, \; a_2, \; \dots, \; a_n \in V\). \\\\
    Нека \(b_1, \; b_2, \; \dots, \; b_k \in l(a_1, \; a_2, \; \dots, \; a_n)\).
    Ако \(k > n\), то \(b_1, \; b_2, \; \dots, \; b_k \) са линейно зависими вектори.
    \subsection{Базис}
    Нека \(n \in \mathbb{N}\). \\\\
    Нека \(b_1, \, b_2, \, \dots b_n \in V\).
    \(b_1, \, b_2, \, \dots b_n\) са базис на \(V\),
    ако \(b_1, \, b_2, \, \dots b_n\) са линейно независими и \(V = l(b_1, \, b_2, \, \dots b_n)\).
    \subsection{Крайномерно линейно пространство}
    Нека \(n \in \mathbb{N}^+\). \\\\
    Нека \(b_1, \, b_2, \, \dots b_n \in V\).
    \(b_1, \, b_2, \, \dots b_n\) са базис на \(V\),
    ако \(b_1, \, b_2, \, \dots b_n\) са линейно независими и \(V = l(b_1, \, b_2, \, \dots b_n)\).
    \subsection{Крайнопородено линейно пространство}
    \(V\) е крайнопородено линейно пространство, ако \\
    \((\exists n \in \mathbb{N}^+) \; (\exists b_1 \in V) \; \dots \; (\exists b_n \in V) \; [\; V = l(b_1, \, b_2, \, \dots b_n) \;] \).
    \subsection{Размерност на линейно пространство}
    Нека \(n \in \mathbb{N} \cup \{\infty\}\).
    Нека \(b_1, \, b_2, \, \dots b_n \in V\) - са базис на \(V\).
    Тогава размерността на \(V\) е  \(n\), бележим с \(dimV\). \\\\
    С други думи казано размерността дефинираме като броя на векторите в кой да е базис на \(V\) \\\\ 
    Ако едно крайномерно линейно пространство и едно негово линейно\\
    подпространство имат една и съща размерност, то те съвпадат!
    \subsection{Координати на вектор в даден базис}
    Нека \(n = dimV\). Нека \(b_1, \, b_2, \, \dots b_n \in V\) - са базис на \(V\). \\\\
    Нека \(v \in V\). Нека още \(\lambda_1, \dots, \lambda_n \in F\) са такива,
    че \(v = \displaystyle\sum_{i=1}^{n} \lambda_i b_i\).
    Тогава \(\lambda_1, \, \lambda_2, \, \dots, \, \lambda_n\) са координатите на \(v\) в базиса \(b_1, \, b_2, \, \dots, \, b_n\).
    \section{Сума на подпространства, директна сума на подпространства и ранг на система вектори}
    \subsection{Сума на подпространства и директна сума на подпространства}
    Нека (V, +, .) е линейно пространство над числовото поле (F, +, .).\\
    Нека \(V_1, \, V_2\) са линейни подпространства на \(V\). \\
    Тогава \(V_1 + V_2 = \{ v_1 + v_2 \; | \; v_1 \in V_1, \; v_2 \in V_2 \}\).
    \subsubsection{Връзка между размерностите на сумата и сечението на две крайномерни линейни подпространства на дадено линейно пространство}
    \(\dim{(V_1 + V_2)} = \dim{V_1} + \dim{V_2} - \dim{(V_1 \cap V_2)}\).
    \subsubsection{НДУ едно линейно пространство да е директна сума на две свои подпространства}
    \(V = V_1 \oplus V_2 \iff V = V_1 + V_2, \; V_1 \cap V_2 = \{ \theta \}\).
    \subsection{Ранг на система вектори}
    \subsubsection{Максимална линейно независима подсистема вектори на дадена система вектори}
    Нека (V, +, .) е линейно пространство над числовото поле (F, +, .).\\
    Нека \(\emptyset \neq T \subseteq V\). Нека \(\emptyset \neq S \subseteq T\).
    \(S\) е максимална линейно независима подсистема вектори на дадена система вектори, ако \\
    \((\forall v \in T \backslash S) \; [\; S \cup \{v\} \text{ - е лин. зависима система} \;] \)
    \subsubsection{Ранг на система вектори}
    Нека (V, +, .) е линейно пространство над числовото поле (F, +, .).\\
    Нека \(\emptyset \neq T \subseteq V\). Ранга на \(T\) бележим с \(r(T)\) и \(r(T) = dim(l(T))\).
\end{document}
