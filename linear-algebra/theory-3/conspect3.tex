\documentclass{article}
\usepackage{amsmath}
\usepackage{amssymb}
\usepackage[T1,T2A]{fontenc}
\usepackage[utf8]{inputenc}
\usepackage[bulgarian]{babel}
\usepackage[normalem]{ulem}
\newcommand{\stkout}[1]{\ifmmode\text{\sout{\ensuremath{#1}}}\else\sout{#1}\fi}
\newcommand{\V}{\mathbb{V}}
\newcommand{\F}{\mathbb{F}}
\newcommand{\W}{\mathbb{W}}
\newcommand{\R}{\mathbb{R}}
\newcommand{\UV}{\mathbb{U}}
\newcommand{\n}[1]{#1_1, \; \dots, \; #1_n}
\newcommand{\nplusone}[1]{#1_1, \; \dots, \; #1_{n + 1}}
\newcommand{\m}[1]{#1_1, \; \dots, \; #1_m}
\newcommand{\OV}{\theta}
\newcommand{\ieqn}{i = 1, \; \dots, \; n}
\newcommand{\iandj}{i, \; j \; : \; 1 \leq i \neq j \leq n}
\newcommand{\forallij}{\forall \iandj}
\newcommand{\N}{\mathbb{N}}
\newcommand{\inv}{[\sigma]}
\newcommand{\aij}{(a_{11}, \; \dots, \; a_{1n}), \; \dots, \; (a_{n1}, \; \dots, \; a_{nn}) \in \F^n}
\newcommand{\arows}{Нека \(a_1 = (a_{11}, \; \dots, \; a_{1n}), \; \dots, \; a_n = (a_{n1}, \; \dots, \; a_{nn})\) са редовете на \(A\) \\\\ Тогава }
\newcommand{\acols}{Нека \(a_1 = (a_{11}, \; \dots, \; a_{n1})^t, \; \dots, \; a_n = (a_{1n}, \; \dots, \; a_{nn})^t\) са стълбовете на \(A\) \\\\ Тогава }
\newcommand{\leta}{\(\F\) - числово поле и \(A \in M_n(\F)\)}

\title{Теоритично контролно №3 1, I, Информатика}
\author{Иво Стратев}

\begin{document}
    \maketitle
    \section{}
    \subsection{Определение за полилинейна фунцкия}
    \(\V\) - Л.П. над числовото поле \(\F, \; dim\V = n \in \N \\\\
    f: \V \times \V \times \dots \times \V \to \F \quad (f : \V^n \to \F) \\\\
    f \text{ е полилинейна ако } \forall i \in \{1, \; \dots, \; n\} \\\\
    \forall v_1, \; \dots, \; v_{i - 1}, \; v_{i + 1}, \; \dots, \; v_n, \; v', \; v'' \in \V \\\\
    \forall \lambda, \; \mu \in \F \; : \; v_i = \lambda v' + \mu v''\\\\
    f(\n{v}) = \begin{array}{l}
        \lambda f(v_1, \; \dots, \; v_{i - 1}, \; v', \; v_{i + 1}, \; \dots, \; v_n)\\
        +\\
        \mu f(v_1, \; \dots, \; v_{i - 1}, \; v'', \; v_{i + 1}, \; \dots, \; v_n)
    \end{array}\)
    \subsection{Определение за антисиметрична фунцкия}
    \(\V\) - Л.П. над числовото поле \(\F, \; dim\V = n \in \N \\\\
    f: \V \times \V \times \dots \times \V \to \F \quad (f : \V^n \to \F) \\\\
    f \text{ е антисиметрична ако } \forallij \\\\
    \forall v_1, \; \dots, \; v_i, \; \dots, \; v_j, \; \dots, \; v_n \in \V \\\\
    f(v_1, \; \dots, \; v_i, \; \dots, \; v_j, \; \dots, \; v_n) = -f(v_1, \; \dots, \; v_j, \; \dots, \; v_i, \; \dots, \; v_n)\)
    \subsection{Необходимо и достатъчно условие една полилинейна функция да е антисиметрична}
    \(\V\) - Л.П. над числовото поле \(\F, \; dim\V = n \in \N \\\\
    f: \V \times \V \times \dots \times \V \to \F \quad (f : \V^n \to \F) \text{ е полилинейна функция} \\\\
    f \text{ е антисиметрична } \iff \forallij \\\\
    \forall v_1, \; \dots, \; v_{i - 1}, \; v_{i + 1}, \; \dots, \; v_{j - 1}, \; v_{j + 1}, \; \dots, \; v_n, \; v \in \V\\\\
    f(v_1, \; \dots, \; v_{i - 1}, \; v, \; v_{i + 1}, \; \dots, \; v_{j - 1}, \; v, \; v_{j + 1}, \; \dots, \; v_n) = 0\)
    \subsection{Определение за пермутация}
    Нека \( n \in \N, \; \Omega_n = \{1, \; 2, \; \dots, \; n\} \\\\
    S(\Omega_n) = \{\sigma \; : \; \Omega_n \to \Omega_n \; | \; \sigma \text{ е биекция}\} \\\\
    \forall \sigma \in S(\Omega_n), \; \sigma \text{ е пермутация} \\\\
    |S(\Omega_n)| = n!\)    
    \subsection{Определение за инверсия в пермутация}
    \(\sigma \in S(\Omega_n), \; \forallij \in \Omega_n \\\\
    \sigma(i), \; \sigma(j) \text{ образуват инверсия} \iff \sigma(i) > \sigma(j), \text{ но } i < j \\\\
    \inv \text{ - брой инверсии} \implies 1 \leq \inv \leq n\)
    \subsection{Определение за транспозиция}
    \(\sigma \in S(\Omega_n), \; \forallij \in \Omega_n \\\\
    \sigma \in S(\Omega_n) \implies \sigma = \{\sigma(1), \; \dots, \; \sigma(i), \; \dots, \; \sigma(j), \; \dots, \; \sigma(n)\} \\\\
    \sigma_t \in S(\Omega_n) \text{ е транспозиция} \iff \sigma_t = \{\sigma(1), \; \dots, \; \sigma(j), \; \dots, \; \sigma(i), \; \dots, \; \sigma(n)\}\)
    \subsection{Определение за четност на пермутация}
    \(\sigma \in S(\Omega_n) \\\\
    \sigma \begin{cases}
        \text{четна} & \inv \mod 2 = 0\\
        \text{нечетна} & \inv \mod 2 = 1
    \end{cases} \\\\
    sign\sigma = \begin{cases}
        1 & \text{четна}\\
        -1 & \text{нечетна}
    \end{cases} = (-1)^{\inv}\)
    \subsection{Определение за детерминантна функция}
    \(f \; : \; \F^n \times \F^n \times \dots \times \F^n \to \F \quad (f : (\F^n)^n \to \F) \\\\
    f \text{ е детерминантна функция, ако е полилинейна, антисиметрична и} \\\\
    f(\n{e}) = 1, \quad \{\n{e}\} \text{ е стандартният базис на } \F^n\)
    \section{Детерминанта и нейните свойства}
    \subsection{Определение за Детерминанта}
    \(F\) е числово поле, \(A \in M_n(\F) \; \det A\) \\\\
    е стойността на единствената детерминантна фунцкия: \\\\
    \(D : \F^n \times \F^n \times \dots \times \F^n \to \F \quad (f : (\F^n)^n \to \F)\) \\\\
    на редовете на \(A : \; \aij\)
    \subsection{Формула за пресмятане на Детерминанта}
    \(\F\) - числово поле, \(A \in M_n(\F), \; A = (a_{ij})_{n \times n} \\\\
    \det A = \displaystyle\sum_{\sigma \in S(\Omega_n)} sign\sigma \displaystyle\prod_{i = 1}^n a_{i \sigma(i)}\)
    \subsection{Връзката между детерминантите на една матрица и транспонираната й матрица}
    \(\det A = \det A^t\)
    \subsection{Промяна на детерминантата на една матрица, ако заменим един нейн ред с нулев ред}
    \leta \\\\
    \arows \(A = \begin{pmatrix} a_1\\ \vdots\\ a_n \end{pmatrix} \quad i \in \{1, \; \dots, \; n\}, \; a_i = (0. \; \dots, \; 0) \implies \det A = 0\)
    \subsection{Промяна на детерминантата на една матрица, ако заменим един нейн стълб с нулев стълб}
    \leta \\\\
    \acols \(A = (a_1 \; \dots \; a_n) \quad i \in \{1, \; \dots, \; n\}, \; a_i = (0. \; \dots, \; 0)^t \implies \det A = 0\)
    \subsection{Промяна на детерминантата на една матрица, ако разменим два нейни реда}
    \leta \\\\
    \arows \(A = \begin{pmatrix} a_1\\ \vdots\\ a_n \end{pmatrix} \; \text{ и нека } \iandj \\\\
    \det\begin{pmatrix} a_1\\ \vdots\\ a_i\\ \vdots\\ a_j\\ \vdots\\ a_n \end{pmatrix} = -\det\begin{pmatrix} a_1\\ \vdots\\ a_j\\ \vdots\\ a_i\\ \vdots\\ a_n \end{pmatrix}\)
    \subsection{Промяна на детерминантата на една матрица, ако разменим два нейни стълба}
    \leta \\\\
    \acols \(A = (a_1 \; \dots \; a_n) \;\text{ и нека } \iandj \\\\
    \det(a_1 \; \dots \; a_i \; \dots \; a_j \; \dots \; a_n) = -\det(a_1 \; \dots \; a_i \; \dots \; a_j \; \dots \; a_n)\)
    \subsection{Промяна на детерминантата на една матрица, ако умножим един нейн ред с някакво число}
    \leta\\\\
    \arows \(A = \begin{pmatrix} a_1\\ \vdots\\ a_n \end{pmatrix} \quad i \in \{1, \; \dots, \; n\}, \; \lambda \in \F \\\\
    \det\begin{pmatrix} a_1\\ \vdots\\ \lambda a_i\\  \vdots\\ a_n \end{pmatrix} = \lambda \det\begin{pmatrix} a_1\\ \vdots\\ a_i\\  \vdots\\ a_n \end{pmatrix}\)
    \subsection{Промяна на детерминантата на една матрица, ако умножим един нейн стълб с някакво число}
    \leta \\\\
    \acols \(A = \begin{pmatrix} a_1\\ \vdots\\ a_n \end{pmatrix} \quad i \in \{1, \; \dots, \; n\}, \; \lambda \in \F \\\\
    \det(a_1 \; \dots \; \lambda a_i \; \dots \; a_n) = \lambda \det(a_1 \; \dots \; a_i \; \dots \; a_n)\)
    \subsection{Промяна на детерминантата на една матрица, ако един нейн ред заменим с друг, различен от него, ред}
    \leta \\\\
    \arows \(A = \begin{pmatrix} a_1\\ \vdots\\ a_n \end{pmatrix} \; \text{ и нека } \iandj \\\\
    a_i := a_j \implies \det\begin{pmatrix} a_1\\ \vdots\\ a_j\\ \vdots\\ a_j\\ \vdots\\ a_n \end{pmatrix} = 0\)
    \subsection{Промяна на детерминантата на една матрица, ако един нейн стълб заменим с друг, различен от него, стълб}
    \leta \\\\
    \acols \(A = (a_1 \; \dots \; a_n) \; \text{ и нека } \iandj \\\\
    a_i := a_j \implies \det(a_1 \; \dots \; a_j \; \dots \; a_j \; \dots \; a_n) = 0\)
    \subsection{Промяна на детерминантата на една матрица, ако един нейн ред заменим с умножен с число друг, различен от него, ред}
    \leta\\\\
    \arows \(A = \begin{pmatrix} a_1\\ \vdots\\ a_n \end{pmatrix} \; \text{ и нека } \iandj, \; \lambda \in \F \\\\
    a_i := \lambda a_j \implies \det\begin{pmatrix} a_1\\ \vdots\\ \lambda a_j\\ \vdots\\ a_j\\ \vdots\\ a_n \end{pmatrix}
    = \lambda \det\begin{pmatrix} a_1\\ \vdots\\ a_j\\ \vdots\\ a_j\\ \vdots\\ a_n \end{pmatrix} = \lambda 0 = 0\)
    \subsection{Промяна на детерминантата на една матрица, ако един нейн стълб заменим с умножен с число друг, различен от него, стълб}
    \leta \\\\
    \acols \(A = (a_1 \; \dots \; a_n) \; \text{ и нека } \iandj, \; \lambda \in \F \\\\
    a_i := \lambda a_j \implies \det(a_1 \; \dots \; \lambda a_j \; \dots \; a_j \; \dots \; a_n) = \\\\
    = \lambda \det(a_1 \; \dots \; a_j \; \dots \; a_j \; \dots \; a_n) = \lambda 0 = 0\)
    \subsection{Промяна на детерминантата на една матрица, ако към един нейн ред прибавим друг, различен от него, ред}
    \leta \\\\
    \arows \(A = \begin{pmatrix} a_1\\ \vdots\\ a_n \end{pmatrix} \; \text{ и нека } \iandj \\\\
    a_i := a_i + a_j \implies \det\begin{pmatrix} a_1\\ \vdots\\ a_i + a_j\\ \vdots\\ a_j\\ \vdots\\ a_n \end{pmatrix}
    = \det\begin{pmatrix} a_1\\ \vdots\\ a_i\\ \vdots\\ a_j\\ \vdots\\ a_n \end{pmatrix} +  \det\begin{pmatrix} a_1\\ \vdots\\ a_j\\ \vdots\\ a_j\\ \vdots\\ a_n \end{pmatrix}
    = \det A + 0 = \det A\)
    \subsection{Промяна на детерминантата на една матрица, ако един нейн стълб прибавим друг, различен от него, стълб}
    \leta \\\\
    \acols \(A = (a_1 \; \dots \; a_n) \; \text{ и нека } \iandj \\\\
    a_i := a_i + a_j \implies \det(a_1 \; \dots \; a_i + a_j \; \dots \; a_j \; \dots \; a_n) = \\\\
    = \det(a_1 \; \dots \; a_i \; \dots \; a_j \; \dots \; a_n) + \det(a_1 \; \dots \; a_j \; \dots \; a_j \; \dots \; a_n) = \det A + 0 = \det A\)
    \subsection{Промяна на детерминантата на една матрица, ако към един нейн ред прибавим м умножен с число друг, различен от него, ред}
    \leta \\\\
    \arows \(A = \begin{pmatrix} a_1\\ \vdots\\ a_n \end{pmatrix} \; \text{ и нека } \iandj, \; \lambda \in \F \\\\
    a_i := a_i + \lambda a_j \implies \det\begin{pmatrix} a_1\\ \vdots\\ a_i + \lambda a_j\\ \vdots\\ a_j\\ \vdots\\ a_n \end{pmatrix}
    = \det\begin{pmatrix} a_1\\ \vdots\\ a_i\\ \vdots\\ a_j\\ \vdots\\ a_n \end{pmatrix} +
    \det\begin{pmatrix} a_1\\ \vdots\\ \lambda a_j\\ \vdots\\ a_j\\ \vdots\\ a_n \end{pmatrix} =\ \\\\
    = \det A + \lambda \det\begin{pmatrix} a_1\\ \vdots\\ a_j\\ \vdots\\ a_j\\ \vdots\\ a_n \end{pmatrix} = \det A + \lambda 0 = \det A + 0 = \det A\)
    \subsection{Промяна на детерминантата на една матрица, ако един нейн стълб прибавим м умножен с число друг, различен от него, стълб}
    \leta \\\\
    \acols \(A = (a_1 \; \dots \; a_n) \; \text{ и нека } \iandj, \; \lambda \in \F \\\\
    a_i := a_i + \lambda a_j \implies \det(a_1 \; \dots \; a_i + \lambda a_j \; \dots \; a_j \; \dots \; a_n) = \\\\
    = \det(a_1 \; \dots \; a_i \; \dots \; a_j \; \dots \; a_n) + \det(a_1 \; \dots \; \lambda a_j \; \dots \; a_j \; \dots \; a_n) = \\\\
    = \det A + \lambda \det(a_1 \; \dots \; a_j \; \dots \; a_j \; \dots \; a_n) = \det A + \lambda 0 = \det A + 0 = \det A\)
    \subsection{\(\det A = 0 \iff\) редовете на \(A\) са линейно зависими}
    \subsection{\(\det A = 0 \iff\) стълбовете на \(A\) са линейно зависими}
    \section{\(A \in M_n, \; \det A = 3\)}
    \subsection{\(B = A^t \implies \det B = \det A = 3\)}
    \subsection{\(B = A := a_{i,*} = (0, \; \dots, \; 0) \implies \det B = 0\)}
    \subsection{\(B = A := a_{*,i} = (0, \; \dots, \; 0)^t \implies \det B = 0\)}
    \subsection{\(B = A := a_{i,*} = a_{j,*} + a_{k,*} \implies \det B = 0\)}
    \subsection{\(B = A := a_{*,i} = a_{*,j} + a_{*,k} \implies \det B = 0\)}
    \subsection{\(B = A := a_{i,*} = 4a_{i,*} \implies \det B = 4\det A = 4.3 = 12\)}
    \subsection{\(B = A := a_{*,i} = -4a_{*,i} \implies \det B = -4\det A = -4.3 = -12\)}
    \subsection{\(B = A := a_{i,*} = a_{j,*} \implies \det B = 0\)}
    \subsection{\(B = A := a_{*,i} = a_{*,j} \implies \det B = 0\)}
    \subsection{\(B = A := a_{i,*} = 8a_{j,*} \implies \det B = 8.0 = 0\)}
    \subsection{\(B = A := a_{*,i} = 6a_{*,j} \implies \det B = 6.0 = 0\)}
    \subsection{\(B = A := a_{i,*} + 5a_{j,*} \implies \det B = \det A + 5.0 = \det A + 0 = \det A\)}
    \subsection{\(B = A := a_{*,i} + 4a_{*,j} \implies \det B = \det A + 4.0 = \det A + 0 = \det A\)}
    \subsection{\(B = A := a_{i,*} = a_{j,*}, \; a_{j,*} = a_{i,*}  \implies \det B = -\det A\)}
    \subsection{\(B = A := a_{*,i} = a_{*,j}, \; a_{*,j} = a_{*,i}  \implies \det B = -\det A\)}
    \subsection{\(B = A := a_{i,*} = \displaystyle\sum_{j = 1, \; j \neq i}^n a_{j,*} \implies \det B = 0\)}
    \subsection{\(B = A := a_{*,i} = \displaystyle\sum_{j = 1, \; j \neq i}^n a_{*,j} \implies \det B = 0\)}
    \section{}
    \subsection{}
    \(A \in M_{n \geq 8} \; \displaystyle\sum_{k = 1}^n a_{8k}A_{1k} = \delta_{81}\det A = 0.\det A = 0\)
    \subsection{}
    \(A \in M_{n \geq 8} \; \displaystyle\sum_{k = 1}^n a_{k8}A_{k1} = \delta_{81}\det A = 0.\det A = 0\)
    \subsection{}
    \(A, B \in M_n, \; \det A = 8, \; \det B = 1 \\\\ \det AB = \det A\det B = 8.1 = 8\)
    \subsection{}
    \(A, B \in M_n, \; \det A = 8, \; \det B = 1 \\\\ \det(AB)^t = \det AB = \det A\det B = 8.1 = 8\)
    \subsection{}
    \(A, B \in M_{n \geq 8}, \; \det A = 9, \; \det B = 1 \\\\ A, B \) се различават само по 8-мия си ред/стълб \\\\
    \(\det(A + B) = 2^{n - 1}(\det A + \det B) = \\\\ =2^{n - 1}(9 + 1) = 2^{n - 1}.10 = 5.2^n\)
    \section{Развитие на детерминанта по ред/стълб на}
    \(\Delta = \begin{vmatrix}
        -9 & -8 & 6 & 5\\
        7 & -2 & 6 & 1\\
        0 & -2 & 1 & 5\\
        1 & 9 & 5 & -6
    \end{vmatrix}\)
    \subsection{Развитие по първи ред}
    \(\Delta = -9.(-1)^{1 + 1} \begin{vmatrix}
        -2 & 6 & 1\\
        -2 & 1 & 5\\
         9 & 5 & -6
    \end{vmatrix} - 8.(-1)^{1 + 2} \begin{vmatrix}
        7 & 6 & 1\\
        0 & 1 & 5\\
        1 & 5 & -6
    \end{vmatrix} \\\\\\
    + 6.(-1)^{1 + 3} \begin{vmatrix}
        7 & -2 & 1\\
        0 & -2 & 5\\
        1 & 9 & -6
    \end{vmatrix} + 5.(-1)^{1 + 4} \begin{vmatrix}
        7 & -2 & 6\\
        0 & -2 & 1\\
        1 & 9 & 5
    \end{vmatrix}\)
    \subsection{Развитие по първи стълб}
    \(\Delta = -9.(-1)^{1 + 1} \begin{vmatrix}
        -2 & 6 & 1\\
        -2 & 1 & 5\\
        9 & 5 & -6
    \end{vmatrix} + 7.(-1)^{2 + 1} \begin{vmatrix}
        -8 & 6 & 5\\
        -2 & 1 & 5\\
        9 & 5 & -6
    \end{vmatrix} \\\\\\
    + 0.(-1)^{3 + 1} \begin{vmatrix}
        -8 & 6 & 5\\
        -2 & 6 & 1\\
        9 & 5 & -6
    \end{vmatrix} + 1.(-1)^{4 + 1} \begin{vmatrix}
        -8 & 6 & 5\\
        -2 & 6 & 1\\
        -2 & 1 & 5\\
    \end{vmatrix}\)
    \subsection{Развитие по втори ред}
    \(\Delta = 7.(-1)^{2 + 1} \begin{vmatrix}
        -8 & 6 & 5\\
        -2 & 1 & 5\\
        9 & 5 & -6
    \end{vmatrix} - 2.(-1)^{2 + 2} \begin{vmatrix}
        -9 & 6 & 5\\
        0 & 1 & 5\\
        1 & 5 & -6
    \end{vmatrix} \\\\\\
    + 6.(-1)^{2 + 3} \begin{vmatrix}
        -9 & -8 & 5\\
        0 & -2 & 5\\
        1 & 9 & -6
    \end{vmatrix} + 1.(-1)^{2 + 4} \begin{vmatrix}
        -9 & -8 & 6\\
        0 & -2 & 1\\
        1 & 9 & 5
    \end{vmatrix}\)
    \subsection{Развитие по втори стълб}
    \(\Delta = -8.(-1)^{1 + 2} \begin{vmatrix}
        7 & 6 & 1\\
        0 & 1 & 5\\
        1 & 5 & -6
    \end{vmatrix} - 2.(-1)^{2 + 2} \begin{vmatrix}
        -9 & 6 & 5\\
        0 & 1 & 5\\
        1 & 5 & -6
    \end{vmatrix} \\\\\\
    - 2.(-1)^{3 + 2} \begin{vmatrix}
        -9 & 6 & 5\\
        7 & 6 & 1\\
        1 & 5 & -6
    \end{vmatrix} + 9.(-1)^{4 + 2} \begin{vmatrix}
        -9 & 6 & 5\\
        7 & 6 & 1\\
        0 & 1 & 5
    \end{vmatrix}\)
    \section{Формули на Крамер}
    \(\begin{array}{|lcr}
        -5x_1 & -2x_2 = & 2\\
        8x_1 & -7x_2 = & - 5
    \end{array}\\\\
    x_i = \displaystyle\frac{\Delta_{i}}{\Delta}\\\\
    x_1 = \displaystyle\frac{\begin{vmatrix}
        2 & -2\\
        -5 & -7
    \end{vmatrix}}{\begin{vmatrix}
        -5 & -2\\
        8 & -7
    \end{vmatrix}}, \quad  x_2 = \displaystyle\frac{\begin{vmatrix}
        -5 & 2\\
        8 & -5
    \end{vmatrix}}{\begin{vmatrix}
        -5 & -2\\
        8 & -7
    \end{vmatrix}}\)
    \section{}
    \subsection{Определение за обратима матрица}
    Нека \(\F\) е поле и \(A \in M_n(\F), \; A \text{ е обратима } \\\\
    \iff \exists B \in M_n(\F) \; : \; AB = BA = E\)
    \subsection{Неособена матрица}
    Нека \(\F\) е поле и \(A \in M_n(\F), \; A \text{ е неособена } \iff \det A \neq 0\)
    \subsection{Формула за обратна матрица на неособена матрица}
    Нека \(\F\) е поле и \(A \in M_n(\F), \; A \text{ е неособена, тогава } A^{-1} = \frac{1}{\det A} (A_{ji}) = \frac{1}{\det A} ((-1)^{j + i}\Delta_{ji})\)
    \subsection{Определение за ранг на матрица}
    Нека \(\F\) е поле и \(A \in M_n(\F), \; r(A) = rr(A) = rc(A)\)
    \subsection{Втора теорема за ранг на матрица}
    Нека \(\F\) е поле и \(A = (a_{ij})\in M_n(\F), \; r(A) = r \iff \\\\
    \exists \; 1 \leq i_1 < i_2 < \dots < i_r \leq n \\\\
    \exists \; 1 \leq j_1 < j_2 < \dots < j_r \leq n \; : \\\\
    \exists \; \begin{vmatrix}
        a_{i_1j_1} & \cdots & a_{i_1j_r}\\
        \vdots & \ddots & \vdots\\
        a_{i_nj_1} & \cdots & a_{i_nj_r}
    \end{vmatrix} \neq 0 \\\\
    \forall \; k \in \N \; : \; r < k \leq n \; \begin{vmatrix}
        a_{i_1j_1} & \cdots & a_{i_1j_k}\\
        \vdots & \ddots & \vdots\\
        a_{i_nj_1} & \cdots & a_{i_nj_k}
    \end{vmatrix} = 0\)
    \section{}
    Нека \(\F\) е поле и \subsection{\(A \in M_5(\F), \; \det A = 0 \implies 0 \leq r(A) < 5\)}
    \subsection{Th Руше}
    \(A \in \F_{m \times n}, \; B \in F^m \; Ax = B \text{ има решение} \iff r(A) = r(A|B)\)
    \subsection{\(x \in \F^n, \; Ax = B, \; r = r(A) = r(A|B) \\\\
    \implies \text{броя на неизвестните е } n - r\)}
    \subsection{Една хомогенна система с равен брой неизвестни и уравнения \(Ax = 0, \; A \in M_n(\F)\), която има ненулево решение е с \( \det A = 0\) (непълен ранг)}
    \subsection{Една хомогенна система с квадратна матрица на \(Ax = 0, \; A \in M_n(\F)\), има единствено решение \(\iff \det A \neq 0\) (пълен ранг)}
    \subsection{Определение за Фундаментална система решения на хомогенна система уравнения}
    \(\F\) - числово поле, \(A \in M_n(\F)\)\\\\
    Ф.С.Р. е всеки базис на пространството \(\{x \in \F^n \; | \; Ax = \OV\}\)
    \subsection{Размерността на множеството от решения\\на хомогенна система линейни уравнения,\\
    ако общият брой на неизвестните е \(n\) и рангът\\ на матрицата на системата е равен на \(r\)}
    \(\F\) - числово поле, \(A \in M_n(\F) \\\\
    dim \{x \in \F^n \; | \; Ax = \OV\} = n - r(A) = n - r\)
    \section{}
    \subsection{Подобни матрици}
    Нека \(\F\) - числово поле и \(A, \; B \in M_n(\F) \\\\
    A \sim B \iff \exists \; T \in M_n(\F) \; : \; \det T \neq 0, \; B = T^{-1}.A.T\)
    \subsection{Определение за характеристичен полином на матрица}
    Нека \(\F\) - числово поле и \(A \in M_n(\F) \\\\
    f_A(\lambda) = \det(A - \lambda E) \text{ - характеристичен полином на матрицата } A\)
    \subsection{Определение за характеристичен полином на линеен оператор}
    Нека \(\V\) - K.M.Л.П. над числовото поле \(\F, \; dim\V = n\) \\\\
    Нека \(\varphi \in Hom\V, \; A = M(\varphi) \in M_n(\F) \text{ е матрицата на } \varphi \text{, в кой да е базис на } \V \\\\
    \implies f_\varphi(\lambda) = f_A(\lambda) = \det(A - \lambda E) \text{ - характеристичен полином на Л.О. } \varphi\)
    \subsection{Определение за характеристичен корен}
    Нека \(\F\) - числово поле и \(A \in M_n(\F) \\\\
    \lambda_0 \text{ - характеристичен корен на характеристичния полином на } A \\\\
    \iff f_A(\lambda_0) = \det(A - \lambda_0 E) = 0\)
    \subsection{Определение за собствена стойност на линеен оператор}
    Нека \(\V\) - K.M.Л.П. над числовото поле \(\F\) \\\\
    Нека \(\varphi \in Hom\V \\\\
    \lambda_0 \in \F \text{ - собствена стойност на } \varphi \iff f_\varphi(\lambda_0) = \det(M(\varphi) - \lambda_0 E) = 0\)
    \subsection{Определение за собствен вектор на линеен оператор}
    Нека \(\V\) - K.M.Л.П. над числовото поле \(\F\) \\\\
    Нека \(\varphi \in Hom\V \\\\
    v_{\lambda_0} \in \V \text{ - собствен вектор на } \varphi \iff\\\\
    \begin{cases}
        (1) \quad v_{\lambda_0} \neq \OV & \text{вектора е ненулев}\\\\
        (2) \quad \exists \lambda_0 \in \F \; : \; f_\varphi(\lambda_0) = \det(M(\varphi) - \lambda_0 E) = 0 & \lambda_0 \text{ e собствена стойност}\\\\
        (3) \quad \varphi(v_{\lambda_0}) = \lambda_0.v_{\lambda_0} & ~
    \end{cases}\)
    \subsection{Всички собствени стойности, които са част от полето над, което е дефинирано линейното пространство, за което даденото линейно изображение е линеен оператор са негови характеристични корени}
    \subsection{Всички характеристични корени, които са част от полето над, което е дефинирано линейното пространство, за което даденото линейно изображение е линеен оператор са негови собствени стойности}
    \subsection{Векторите съотвестващи на различни собствени стойности на даден линеен оператор са линейно независими}
    Нека \(\V\) - K.M.Л.П. над числовото поле \(\F, \; dim\V = n\) \\\\
    Нека \(\varphi \in Hom\V, \; \varphi\) има \(n\) на брой различни собствени стойности \(\n{\lambda}\), \\\\
    тогава \(\exists\) базис \(\n{v}\) на \(\V\), в който \(\varphi\) има диагонална матрица, \\\\
    \(M_v(\varphi) = \begin{pmatrix}
        \lambda_1 & 0 & 0& \dots & 0\\
        0 & \lambda_2 & 0 & \dots & 0\\
        ~ & ~ & \dots & ~ & ~\\
        0 & 0 & \dots & 0 & \lambda_n
    \end{pmatrix}\)
    \subsection{Връзка между степен на характеристичен полином на линеен оператор и размерност на пространството}
    Нека \(\V\) - K.M.Л.П. над числовото поле \(\F\) \\\\
    Нека \(\varphi \in Hom\V \implies \deg(f_\varphi(\lambda)) = \dim\V\)
    \subsection{Връзка между степен на характеристичен полином на линеен оператор и брой различни собствени стойности}
    Нека \(\V\) - K.M.Л.П. над числовото поле \(\F\) и нека \(\varphi \in Hom\V \\\\
    \implies \deg(f_\varphi(\lambda)) \geq \text{ брой различни собствени стойности } \geq 0\)
    \subsection{Връзка между броя на различните характеристични корени на линеен оператор и размерността на пространството}
    Нека \(\V\) - K.M.Л.П. над числовото поле \(\F\) и нека \(\varphi \in Hom\V \\\\
    \implies \dim\V \geq \text{ брой различни характеристични корени } \geq 0\)
    \subsection{Матрицата на линеен оператор в базис от собствени вектори е диагонална}
    \section{}
    \subsection{Определение за евклидово пространство}
    Л.П. \(\V\) над \(\R\) се нарича евклидово пространство,\\
    ако в него е въведено скаларно произведение,\\
    тоест изображение \(( \; , \; ) : \V \times \V \to \R\), имащо свойствата:\\\\
    \(1. \; \forall a, \;  b, \; c \in \V \; ( a + b, \; c ) = ( a, \; c ) + ( b, \; c )\\
    \\2. \; \forall a, \; b \in \V, \; \forall \lambda \in \R \; ( \lambda a, \; b ) = \lambda ( a, \; b )\\
    \\3. \; \forall a, \; b \in \V \; ( a, \; b ) = ( b, \; a )\\
    \\4. \; \forall a \in \V \; ( a, \; a ) \geq 0, \; ( a, \; a ) = 0 \iff a = \OV\)
    \subsection{Определение за скаларно произведение}
    Нека \(\V\) е eвклидово пространство.\\\\
    Скаларно произведение е изображение \(( \; , \; ) : \V \times \V \to \R\), имащо свойствата:\\
    \(1. \; \forall a, \;  b, \; c \in \V \; ( a + b, \; c ) = ( a, \; c ) + ( b, \; c )\\
    \\2. \; \forall a, \; b \in \V, \; \forall \lambda \in \R \; ( \lambda a, \; b ) = \lambda ( a, \; b )\\
    \\3. \; \forall a, \; b \in \V \; ( a, \; b ) = ( b, \; a )\\
    \\4. \; \forall a \in \V \; ( a, \; a ) \geq 0, \; ( a, \; a ) = 0 \iff a = \OV\)
    \subsection{Определение за дължина на вектор}
    Нека \(\V\) е eвклидово пространство.\\\\
    Нека \(a \in \V \implies |a| = \sqrt{(a, \; a)}\) е дължината на вектора \(a\)
    \subsection{Определение за ортогонални вектори}
    Нека \(\V\) е евклидово пространство.\\\\
    Нека \(a, b \in \V, \; a, \; b\) са ортогонални \((a \perp b) \iff (a, \; b) = 0 \)
    \subsection{Определение за ортогонален базис}
    Нека \(\V\) е евклидово пространство. \(dim\V = n\) \\\\
    Базисът \(\n{v} \in \V\) е ортогонален \\\\
    \(\iff \forallij \; (v_i, \; v_j) = 0\)
    \subsection{Определение за ортонормиран базис}
    Нека \(\V\) е евклидово пространство. \(dim\V = n\) \\\\
    Базисът \(\n{e} \in \V\) е ортонормиран \\\\
    \(\iff \forall i, \; j \in \{1, \; \dots, \; n\} \; (e_i, \; e_j) = \delta_{ij}\)
    \subsection{Ортогонални по между си и ненулеви вектори в евклидово пространство са линейно независими}
    \subsection{Определение за ортогонално допълнение на линейно подпространство в евклидово пространство}
    Нека \(\V\) е К.М.Е.П. и \(U < V \implies U^\perp = \{\forall v \in \V \; | \; \forall u \in \UV \; (u, \; v) = 0\}\)
    \subsection{Th(Питагор)}
    Ако векторите \(\n{a}\) са два по два ортогонални, то \(\left|\displaystyle\sum_{i = 1}^n a_i\right|^2 = \displaystyle\sum_{i = 1}^n |a_i|^2\)
    \subsection{Метод на Грам-Шмид}
    Нека векторите \(\n{e}\) са ортогонални и ненулеви вектори \\\\
    и нека вектора \(a \notin l(\n{e})\). \\\\
    Тогава \(\exists \n{\lambda} \in \F \; : \; e_{n + 1} = a + \displaystyle\sum_{j = 1}^n \lambda_j e_j \\\\
    \implies \nplusone{e}\) са ортогонални и ненулеви вектори \\\\
    и \(l(\n{e}, a) = l(\nplusone{e})\)
    \section{}
    \subsection{Детерминанта на Грам}
    Нека \(\V\) е Е.П. и \(\m{a} \in \V\\\\
    \Gamma(\m{a}) = \begin{vmatrix}
        (a_1, \; a_1) & (a_1, \; a_2) & \dots & (a_1, \; a_m)\\
        (a_2, \; a_1) & (a_2, \; a_2) & \dots & (a_2, \; a_m)\\
        \vdots & \vdots & \dots & \vdots\\
        (a_m, \; a_1) & (a_m, \; a_2) & \dots & (a_m, \; a_m)\\ 
    \end{vmatrix}\)
    \subsection{\(\Gamma(\m{a}) = 0 \iff \m{a}\) са линейно зависими}
    \subsection{Неравенството на Коши-Буняковски}
    Нека \(\V\) е Е.П. и \(a, \; b \in \V \implies |(a, \; b)| \leq |a|.|b| \\\\
    |(a, \; b)| = |a|.|b| \iff a, \; b\) са линейно зависими.
    \subsubsection{Координатна форма на неравенството на Коши-Буняковски}
    Нека \(a = (\n{a}), \; b = (\n{b}) \in \R^n\). Тогава \\\\
    \(\left(\displaystyle\sum_{i = 1}^n a_ib_i \right)^2 \leq \left(\displaystyle\sum_{i = 1}^n a_i^2\right) \left(\displaystyle\sum_{i = 1}^n b_i^2\right) \)
    \subsubsection{Интегрална форма на неравенството на Коши-Буняковски}
    Нека \(a, \; b \in R \; : \; a < b\) и \(f, \; g \in C[a, \; b]\). Тогава \\\\
    \(\left(\displaystyle\int_a^b f(x)g(x) \; dx\right)^2 \leq \displaystyle\int_a^b f^2(x) \; dx.\displaystyle\int_a^b g^2(x) \; dx \)
    \subsection{Нека \(\V\) е Е.П., \(a, \; b \in \V \implies \cos\measuredangle(a, \; b) = \frac{(a, \; b)}{|a||b|} \\\\
    \implies \measuredangle(a, \; b) = \arccos\frac{(a, \; b)}{|a||b|}\)}
    \subsection{Неравенство на триъгълника за вектори в евклидово пространство}
    Нека \(\V\) е Е.П. и \(a, \; b \in \V \implies |a + b| \leq |a| + |b| \\\\
    |a + b| = |a| + |b| \iff a \uparrow \uparrow b \text{ или } a = \OV \text{ или } b = \OV\)
    \section{Симетричен Линеен оператор}
    \subsection{Симетрична матрица}
    Нека \(A \in M_n(\R), \; A \text{ е симетрична} \iff A^t = A\)
    \subsection{Определение за симетричен оператор}
    Нека \(\V\) е Е.П. и \(\varphi \in Hom\V, \; \varphi \text{ е симетричен} \\\\
    \iff \forall u, \; v \in \V \; (\varphi(v), \; u) = (v, \; \varphi(u))\)
    \subsection{Mатрицата на симетричен оператор в ортонормиран базис е симетрична}
    \subsection{Собствените стойности на симетричен оператор са реални и съвпадат с характеристичните му корени}
    \subsection{Собствените вектори на симетричен оператор, съответстващи на различни собствени стойности са ортогонални помежду си}
    \subsection{Основна Теорема за симетричен оператор}
    Нека \(\V\) е Е.П. и \(\varphi \in Hom\V, \; \varphi\) е симетричен оператор \\\\
    \(\implies \exists \n{v} \in \V\) ортонормиран базис на \(\V\), \\\\
    в който матрицата на \(\varphi\) е диагонална \\\\
    и по главния диагонал са собствените стойности на \(\varphi\).
    \section{Ортогонален Линеен оператор}
    \subsection{Ортогонална матрица}
    Нека \(A \in M_n(\R), \; A \text{ е ортогонална} \iff AA^t = E\)
    \subsection{Определение за ортогонален оператор}
    Нека \(\V\) е Е.П. и \(\varphi \in Hom\V, \; \varphi \text{ е ортогонален} \\\\
    \iff \forall u, \; v \in \V \; (\varphi(v), \varphi(u)) = (v, u)\)
    \subsection{Необходимото и достатъчно условие един оператор да е ортогонален е той да изпраща един ортнормиран базис в друг ортнормиран базис}
    \subsection{Mатрицата на ортогонален оператор в ортонормиран базис е ортогонална}
    \subsection{Собствените стойности на ортогонален оператор са равни на \(\pm 1\)}
    \subsection{Собствените вектори на ортогонален оператор, съответстващи на различни собствени стойности са ортогонални помежду си}
    \subsection{Основна Теорема за ортогонален оператор}
    Нека \(\V\) е Е.П. и \(\varphi \in Hom\V, \; \varphi\) е ортогонален оператор \\\\
    \(\implies \exists \n{v} \in \V\) ортонормиран базис на \(\V\), \\\\
    в който матрицата на \(\varphi\) е клетачно диагонална.
\end{document}