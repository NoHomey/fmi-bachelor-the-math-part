\documentclass{article}[12pt]
\usepackage{amsmath,amsthm}
\usepackage{amssymb}
\usepackage{lipsum}
\usepackage{stmaryrd}
\usepackage[T1,T2A]{fontenc}
\usepackage[utf8]{inputenc}
\usepackage[bulgarian]{babel}
\usepackage[normalem]{ulem}
\usepackage{xcolor}

\newcommand{\Lang}{\mathcal{L}}

\setlength{\parindent}{0mm}

\title{Линейни пространства}
\author{Иво Стратев}

\begin{document}

\maketitle

\tableofcontents

\pagebreak

\section{Еквивалентни дефиниции на поле}

\subsection{Стандартна със съществуване}

\((\mathbb{F}, +, .)\) е поле ако:

\begin{enumerate}
\item \(\mathbb{F}\) е непразно множество и \((\forall a \in \mathbb{F})(\forall b \in \mathbb{F})[\; a + b \in \mathbb{F} \; \land \; a.b \in \mathbb{F} \;]\);
\item \((\forall a \in \mathbb{F})(\forall b \in \mathbb{F})(\forall c \in \mathbb{F})[\; (a + b) + c = a + (b + c) \;]\);
\item \((\forall a \in \mathbb{F})(\forall b \in \mathbb{F})[\; a + b = b + a \;]\);
\item \((\exists 0 \in \mathbb{F})(\forall a \in \mathbb{F})[\; 0 + a = a \; \land \; (\exists -a \in \mathbb{F})[\; a + (-a) = 0 \;] \;]\);
\item \((\forall a \in \mathbb{F})(\forall b \in \mathbb{F})(\forall c \in \mathbb{F})[\; (a . b) . c = a . (b . c) \;]\);
\item \((\forall a \in \mathbb{F})(\forall b \in \mathbb{F})[\; a . b = b . a \;]\);
\item \((\exists 1 \in \mathbb{F})(\forall a \in \mathbb{F})[\; 1 . a = a \;]\);
\item \((\forall a \in \mathbb{F})[\; a \neq 0 \implies (\exists a^{-1} \in \mathbb{F})[\; a . a^{-1} = 1 \;] \;]\);
\item \((\forall a \in \mathbb{F})(\forall b \in \mathbb{F})(\forall c \in \mathbb{F})[\; a.(b + c) = a.b + a.c \;]\);
\end{enumerate}

\subsection{Еквивалентна с операции и константи (инфиксен запис)}

\((\mathbb{F}, 0, 1, +, -, .)\) е поле ако:

\begin{enumerate}
\item \(\mathbb{F}\) е непразно множество, \(0 \in \mathbb{F}\), \(1 \in \mathbb{F}\), \(+ \; : \; \mathbb{F} \times \mathbb{F} \to \mathbb{F}\), \(- \; : \; \mathbb{F} \to \mathbb{F}\) и \(. \; : \; \mathbb{F} \times \mathbb{F} \to \mathbb{F}\);
\item \((\forall a \in \mathbb{F})(\forall b \in \mathbb{F})(\forall c \in \mathbb{F})[\; (a + b) + c = a + (b + c) \;]\);
\item \((\forall a \in \mathbb{F})(\forall b \in \mathbb{F})[\; a + b = b + a \;]\);
\item \((\forall a \in \mathbb{F})[\; 0 + a = a \;]\);
\item \((\forall a \in \mathbb{F})[\; a + (-a) = 0 \;]\);
\item \((\forall a \in \mathbb{F})(\forall b \in \mathbb{F})(\forall c \in \mathbb{F})[\; (a . b) . c = a . (b . c) \;]\);
\item \((\forall a \in \mathbb{F})(\forall b \in \mathbb{F})[\; a . b = b . a \;]\);
\item \((\forall a \in \mathbb{F})[\; 1 . a = a \;]\);
\item \((\forall a \in \mathbb{F})[\; a \neq 0 \implies (\exists a^{-1} \in \mathbb{F})[\; a . a^{-1} = 1 \;] \;]\);
\item \((\forall a \in \mathbb{F})(\forall b \in \mathbb{F})(\forall c \in \mathbb{F})[\; a.(b + c) = a.b + a.c \;]\);
\end{enumerate}

\subsection{Еквивалентна с функции и константи (функционален запис)}

\((\mathbb{F}, 0, 1, add, inv, mul)\) е поле ако:

\begin{enumerate}
\item \(\mathbb{F}\) е непразно множество, \(0 \in \mathbb{F}\), \(1 \in \mathbb{F}\), \(add \; : \; \mathbb{F} \times \mathbb{F} \to \mathbb{F}\), \(inv \; : \; \mathbb{F} \to \mathbb{F}\) и \(mul \; : \; \mathbb{F} \times \mathbb{F} \to \mathbb{F}\);
\item \((\forall a \in \mathbb{F})(\forall b \in \mathbb{F})(\forall c \in \mathbb{F})[\; add(add(a, b), c) = add(a, add(b, c)) \;]\);
\item \((\forall a \in \mathbb{F})(\forall b \in \mathbb{F})[\; add(a, b) = add(b, a) \;]\);
\item \((\forall a \in \mathbb{F})[\; add(0, a) = a \;]\);
\item \((\forall a \in \mathbb{F})[\; add(a, inv(a)) = 0 \;]\);
\item \((\forall a \in \mathbb{F})(\forall b \in \mathbb{F})(\forall c \in \mathbb{F})[\; mul(mul(a, b), c) = mul(a, mul(b, c)) \;]\);
\item \((\forall a \in \mathbb{F})(\forall b \in \mathbb{F})[\; mul(a, b) = mul(b, a) \;]\);
\item \((\forall a \in \mathbb{F})[\; mull(1, a) = a \;]\);
\item \((\forall a \in \mathbb{F})[\; a \neq 0 \implies (\exists a^{-1} \in \mathbb{F})[\; mul(a,  a^{-1}) = 1 \;] \;]\);
\item \((\forall a \in \mathbb{F})(\forall b \in \mathbb{F})(\forall c \in \mathbb{F})[\; mul(a, add(b, c)) = add(mul(a, b), mul(a, c)) \;]\);
\end{enumerate}

\section{Еквивалентни дефиниции на линейно пространство}

\subsection{Стандартна със съществуване}

Нека \((\mathbb{F}, +, .)\) е поле.

\((\mathbb{V}, \oplus, \odot)\) е линейно пространство над полето \((\mathbb{F}, +, .)\), ако:

\begin{enumerate}
\item \(\mathbb{V}\) е непразно множество, \\
\((\forall a \in \mathbb{V})(\forall b \in \mathbb{V})[\; a \oplus b \in \mathbb{V}  \;]\)
и \((\forall \lambda \in \mathbb{F})(\forall v \in \mathbb{V})[\; \lambda \odot v \in \mathbb{V} \;]\);
\item \((\forall a \in \mathbb{V})(\forall b \in \mathbb{V})(\forall c \in \mathbb{V})[\; (a \oplus b) \oplus c = a \oplus (b \oplus c) \;]\);
\item \((\forall a \in \mathbb{V})(\forall b \in \mathbb{V})[\; a \oplus b = b \oplus a \;]\);
\item \((\exists \theta \in \mathbb{V})(\forall a \in \mathbb{V})[\; \theta \oplus a = a \; \land \; (\exists \ominus a \in \mathbb{V})[\; a \oplus (\ominus a) = \theta \;] \;]\);
\item \((\forall a \in \mathbb{V})[\; 1 \odot a = a \;]\);
\item \((\forall \lambda \in \mathbb{F})(\forall \mu \in \mathbb{F})(\forall v \in \mathbb{V})[\; (\lambda . \mu) \odot v = \lambda \odot (\mu \odot v) \;]\);
\item \((\forall \lambda \in \mathbb{F})(\forall b \in \mathbb{V})(\forall c \in \mathbb{V})[\; \lambda \odot (b \oplus c) = (\lambda \odot b) \oplus (\lambda \odot c) \;]\);
\item \((\forall \lambda \in \mathbb{F})(\forall \mu \in \mathbb{F})(\forall v \in \mathbb{V})[\; (\lambda + \mu) \odot v = (\lambda \odot v) \oplus (\mu \odot v) \;]\);
\end{enumerate}

\subsection{Еквивалентна с операции и константи (инфиксен запис)}

Нека \((\mathbb{F}, 0, 1, +, -, .)\) е поле.

\((\mathbb{V}, \theta, \oplus, \ominus, \odot)\) е линейно пространство над полето \((\mathbb{F}, 0, 1, +, -, .)\), ако:

\begin{enumerate}
\item \(\mathbb{V}\) е непразно множество, \\
\(\theta \in \mathbb{V}\), \(\oplus \; : \; \mathbb{V} \times \mathbb{V} \to \mathbb{V}\), \(\ominus \; : \; \mathbb{V} \to \mathbb{V}\) и \(\odot \; : \; \mathbb{F} \times \mathbb{V} \to \mathbb{V}\);
\item \((\forall a \in \mathbb{V})(\forall b \in \mathbb{V})(\forall c \in \mathbb{V})[\; (a \oplus b) \oplus c = a \oplus (b \oplus c) \;]\);
\item \((\forall a \in \mathbb{V})(\forall b \in \mathbb{V})[\; a \oplus b = b \oplus a \;]\);
\item \((\forall a \in \mathbb{V})[\; \theta \oplus a = a \;]\);
\item \((\forall a \in \mathbb{V})[\; a \oplus (\ominus a) = \theta \;]\);
\item \((\forall a \in \mathbb{V})[\; 1 \odot a = a \;]\);
\item \((\forall \lambda \in \mathbb{F})(\forall \mu \in \mathbb{F})(\forall v \in \mathbb{V})[\; (\lambda . \mu) \odot v = \lambda \odot (\mu \odot v) \;]\);
\item \((\forall \lambda \in \mathbb{F})(\forall b \in \mathbb{V})(\forall c \in \mathbb{V})[\; \lambda \odot (b \oplus c) = (\lambda \odot b) \oplus (\lambda \odot c) \;]\);
\item \((\forall \lambda \in \mathbb{F})(\forall \mu \in \mathbb{F})(\forall v \in \mathbb{V})[\; (\lambda + \mu) \odot v = (\lambda \odot v) \oplus (\mu \odot v) \;]\);
\end{enumerate}

\subsection{Еквивалентна с функции и константи (функционален запис)}

Нека \((\mathbb{F}, 0, 1, add, inv, mul)\) е поле.

\((\mathbb{V}, \theta, vecAdd, vecInv, scalMul)\) е линейно пространство над полето \((\mathbb{F}, 0, 1, add, inv, mul)\), ако:

\begin{enumerate}
\item \(\mathbb{V}\) е непразно множество, \\
\(\theta \in \mathbb{V}\), \(vecAdd \; : \; \mathbb{V} \times \mathbb{V} \to \mathbb{V}\), \(vecInv \; : \; \mathbb{V} \to \mathbb{V}\) и \(scalMul \; : \; \mathbb{F} \times \mathbb{V} \to \mathbb{V}\);
\item \((\forall a \in \mathbb{V})(\forall b \in \mathbb{V})(\forall c \in \mathbb{V})[\; vecAdd(vecAdd(a, b), c) = a vecAdd(a, vecAdd(b, c)) \;]\);
\item \((\forall a \in \mathbb{V})(\forall b \in \mathbb{V})[\; vecAdd(a, b) = vecAdd(b, a) \;]\);
\item \((\forall a \in \mathbb{V})[\; vecAdd(\theta, a) = a \;]\);
\item \((\forall a \in \mathbb{V})[\; vecAdd(a, vecInv(a)) = \theta \;]\);
\item \((\forall a \in \mathbb{V})[\; scalMul(1, a) = a\;]\);
\item \((\forall \lambda \in \mathbb{F})(\forall \mu \in \mathbb{F})(\forall v \in \mathbb{V})[\; scalMul(mul(\lambda, \mu), v) = scalMul(\lambda, scalMul(\mu, v)) \;]\);
\item \((\forall \lambda \in \mathbb{F})(\forall b \in \mathbb{V})(\forall c \in \mathbb{V})[\; scalMul(\lambda, vecAdd(b, c)) = vecAdd(scalMul(\lambda, b), scalMul(\lambda, c)) \;]\);
\item \((\forall \lambda \in \mathbb{F})(\forall \mu \in \mathbb{F})(\forall v \in \mathbb{V})[\; scalMul(add(\lambda + \mu), v) = vecAdd(scalMul(\lambda, v), scalMul(\mu, v)) \;]\);
\end{enumerate}

\section{НДУ за подпространство}

Нека \((\mathbb{V}, \oplus, \odot)\) е ЛП над поле \((\mathbb{F}, +, .)\).

\((\mathbb{U}, \oplus, \odot)\) е подпространство на \((\mathbb{V}, \oplus, \odot)\) ТСТК:

\begin{enumerate}
\item \(\mathbb{U} \subseteq \mathbb{V}\);
\item \((\forall a \in \mathbb{U})(\forall b \in \mathbb{U})[\; a \oplus b \in \mathbb{U}\;]\);
\item \((\forall \lambda \in \mathbb{F})(\forall u \in \mathbb{U})[\; \lambda \odot u \in \mathbb{U}\;]\).
\end{enumerate}

\section{Две доказателства на подпространства}

\subsection{Нечетни реални функции}

Да се докаже, че множеството от нечетните реални функции е линейно пространство
над полето на реалните числа относо стандартните операции за събиране на функции
и умножение на функция с число.

\subsubsection{Решение:}

Очевидно множеството от нечетните функции е подмножество на всички реални функции.
Показваме затвореност.

Нека \(f\) и \(g\) са нечетни функции. Тогава

\begin{align*}
(\forall x \in \mathbb{R})[\; f(-x) = -f(x) \;] \; \land \;
(\forall x \in \mathbb{R})[\; g(-x) = -g(x) \;]
\end{align*}

Нека \(x \in \mathbb{R}\) тогава \((f \oplus g)(-x) = f(-x) + g(-x) = -f(x) + (-g(x)) = -(f(x) + g(x)) = -(f \oplus g)(x)\).
Следователно \\ \((\forall x \in \mathbb{R})[\; (f \oplus g)(-x) = -(f \oplus g)(x) \;]\).
Тоест сума на нечетни функции е отново нечетна функция.
Тоест доказахме затвореност относно събирането.


Нека \(h\) е нечетна функция и нека \(\lambda \in \mathbb{R}\).
Нека \(x \in \mathbb{R}\) тогава \((\lambda \odot h)(-x) = \lambda.h(-x) = \lambda.(-h(x)) = -\lambda.h(x) = -(\lambda \odot h)(x)\).
Следователно \\ \((\forall x \in \mathbb{R})[\; (\lambda \odot h)(-x) = -(\lambda \odot h)(x) \;]\).
Тоест нечетна функция умножена с число отново е нечетна функция.
Тоест доказахме затвореност относно умножение с число.


Така множеството на нечетните функции относно стандартните операции е линейно подпространство.
В частност линейо пространство над реалните числа.

\subsection{"Интересни" \; функции}

Нека \(\mathbb{W} = \{(a \odot sinx) \oplus (b \odot x^2) \oplus c \; | \; (a, b, c) \in \mathbb{R}^3 \}\).
Да се докаже, че \(\mathbb{W}\) е линейно пространство относно стандартните операции за събиране на функции и умножение на функция с число.

\subsubsection{Решение:}

Очевидно всеки елемент на множеството \(\mathbb{W}\) е функция от реални числа в реални числа.
Тоест очевидно \(\mathbb{W} \subseteq \{f \; | \; f \; : \; \mathbb{R} \to \mathbb{R} \}\).

Ние знаем (може би още не съвсем, но вие ще го докажете, вярвам във вас!),
че относно стандартните операции за събиране на функции и умножение на функция с число
\(\{f \; | \; f \; : \; \mathbb{R} \to \mathbb{R} \}\) е ЛП над полето на реалните числа.

Тогава ни остава да докажем затвореност относно операциите!

Нека \(w_1 \in  \mathbb{W}\) и нека \(w_2 \in  \mathbb{W}\) и са произволни. \\
Тогава съществуват \((a_1, b_1, c_1) \in \mathbb{R}^3\), такива че \(w_1 = (a_1 \odot  sinx) \oplus (b_1 \odot x^2) \oplus c_1\)
и съществуват \((a_2, b_2, c_2) \in \mathbb{R}^3\), такива че \(w_2 = (a_2 \odot  sinx) \oplus (b_2 \odot x^2) \oplus c_2\).
Нека тогава \((a_1, b_1, c_1) \in \mathbb{R}^3\) и нека \((a_2, b_2, c_2) \in \mathbb{R}^3\)
и нека \(w_1 = (a_1 \odot  sinx) \oplus (b_1 \odot x^2) \oplus c_1\) и \(w_2 = (a_2 \odot  sinx) \oplus (b_2 \odot x^2) \oplus c_2\).
Така \(w_1 \oplus w_2 = ((a_1 \odot  sinx) \oplus (b_1 \odot x^2) \oplus c_1) \oplus ((a_2 \odot  sinx) \oplus (b_2 \odot x^2) \oplus c_2) =
((a_1 + a_2) \odot sinx) \oplus ((b_1 + b_2) \odot x^2) \oplus (c_1 + c_2)\)
\\  и \((a_1 + a_2, b_1 + b_2, c_1 + c_2) \in \mathbb{R}^3\).
Следователно \(w_1 \oplus w_2 \in  \mathbb{W}\). \\
Следователно \((\forall w_1 \in \mathbb{W})(\forall w_2 \in \mathbb{W})[\; w_1 \oplus w_2 \in \mathbb{W} \;]\).

Нека \(\mu \in \mathbb{R}\) и нека \(w \in  \mathbb{W}\) и са произволни.
Тогава съществуват \\
\((a, b, c) \in \mathbb{R}^3\), такива че \(w = (a \odot sinx) \oplus (b \odot x^2) \oplus c\).
Нека тогава \((a, b, c) \in \mathbb{R}^3\) и нека \(w = (a \odot sinx) \oplus (b \odot x^2) \oplus c\).
Така \(\mu \odot w = \mu \odot ((a \odot sinx) \oplus (b \odot x^2) \oplus c) = \\
((\mu.a) \odot sinx) \oplus ((\mu.b) \odot x^2) \oplus (\mu.c)\) и \((\mu.a, \mu.b, \mu.c) \in \mathbb{R}^3\).
Следователно \(\mu \odot w \in  \mathbb{W}\).
Следователно \((\forall \mu \in \mathbb{F})(\forall w \in \mathbb{W})[\; \mu \odot w \in \mathbb{W} \;]\).


Значи \((\mathbb{W}, \oplus, \odot)\) е подпространсто на функците от реални в реални числа.
В частност \((\mathbb{W}, \oplus, \odot)\) е линейно пространство.

\section{Задачи за домашно}

\subsection{Задачи, които са задължителни}

\subsubsection{Всяко поле е ЛП над себе си}

Докажете, че ако \((F, +, .)\) е поле, то \((F, +, .)\) e ЛП над полето \((F, +, .)\).

\subsubsection{Задачата за функциите}

Нека \(X\) е непразно множество и нека \((F, +, .)\) е поле.

Нека \(Func(X, F) = \{f \; | \; f \; : \; X \to F\}\).

Дефинираме естествените операции събиране на функции и умножение на функция с число по следните правила:

\begin{align*}
(\forall f \in Func(X, F))(\forall g \in Func(X, F))(\forall x \in X)[\; (f \oplus g)(x) = f(x) + g(x) \;] \\
(\forall \lambda \in F)(\forall f \in Func(X, F))(\forall x \in X)[\; (\lambda \odot f)(x) = \lambda.f(x) \;]
\end{align*}

Докажете, че \((Func(X, F), \oplus, \odot)\) е ЛП над \((F, +, .)\).


Асоциативността е по желание!

Равенство на функции се дефинира като равенство между функционалните им стойности. Тоест
\begin{align*}
(\forall f \in Func(X, F))(\forall g \in Func(X, F))[\; f = g \iff (\forall x \in X)[\; f(x) = g(x) \;] \; ]
\end{align*}

Нулевата функция се дефинира така:
\begin{align*}
(\forall x \in X)[\; \theta(x) = 0 \;] 
\end{align*}

Тоест константа \(0\), която е нулата на полето!

Противоположна функция се дефинира така:
\begin{align*}
(\forall f \in Func(X, F))(\forall x \in X)[\; (\ominus f)(x) = -f(x) \;]
\end{align*}

Тоест функцията, която връща противоположните стойности.

\subsection{Задачи за упражнение}

\subsubsection{Четните реални функции образуват ЛП}

\subsubsection{Полиномите състоящи се само от четни степени образуват ЛП}

\subsubsection{Множеството \(\{(p, 0, -p) \; | \; p \in \mathbb{Q}\}\) образува ЛП}

\subsubsection{Задача 3.9 от сборника}

\subsubsection{Задача 3.7 от сборника}

\subsection{Нека \((V, \oplus, \odot)\) е ЛП над \((F, +, .)\) и \(A\) е крайно подмножество на \(V\). Тогава \(l(A)\) е ЛП}

\end{document}