\documentclass[a4paper, 12pt, oneside]{article}
\usepackage[left=3cm,right=3cm,top=1cm,bottom=2cm]{geometry}
\usepackage{amsmath,amsthm}
\usepackage{amssymb}
\usepackage{lipsum}
\usepackage{stmaryrd}
\usepackage[T1,T2A]{fontenc}
\usepackage[utf8]{inputenc}
\usepackage[bulgarian]{babel}
\usepackage[normalem]{ulem}
                
\setlength{\parindent}{0mm}

\title{Предложения за контролна работа 3}
\author{Иво Стратев}

\begin{document}
\maketitle
\section*{Задача 4.}
Нека \(\mathbb{V}\) е линейно пространсво над полето \(\mathbb{F}\)
и нека \(\varphi \in \mathrm{Hom}\mathbb{V}\). \\
Нека \(A\) е матрицата на \(\varphi\) спрямо произволен базис.
Нека \(n = \mathrm{dim}\mathbb{V}\).\\\
Нека \(D = \begin{pmatrix}
    \lambda_1 & 0 & 0 & \cdots & 0 \\
    0 & \lambda_2 & 0 & \cdots & 0 \\
    0 & 0 & \lambda_3 & \cdots & 0 \\
    \vdots & \vdots & \vdots & \ddots & \vdots \\
    0 & 0 & 0 & \cdots & \lambda_n
\end{pmatrix} \in M_n(\mathbb{F})\). \\
а) Да се докаже, че \(D^k = \begin{pmatrix}
    \lambda_1^k & 0 & 0 & \cdots & 0 \\
    0 & \lambda_2^k & 0 & \cdots & 0 \\
    0 & 0 & \lambda_3^k & \cdots & 0 \\
    \vdots & \vdots & \vdots & \ddots & \vdots \\
    0 & 0 & 0 & \cdots & \lambda_n^k
\end{pmatrix}\) за произволно естествено число \(k\). \\
б) \(p\) е произволен полином с коефициенти от полето \(\mathbb{F}\).
Тоест \(p \in \mathbb{F}[x]\). \\
Нека \(\tau \; : \; \mathbb{F}[x] \to (M_n(\mathbb{F}) \to M_n(\mathbb{F}))\)
е естественото изображение, което на полином \(f\) от \(\mathbb{F}[x]\)
по естествен начин съпоставя "полином" със същите коефициенти като \(f\),
само че приемащ матрици, а не скалари. Тоест \\
\(\tau( a_0x^m + a_1 x^{m - 1} + \dots + a_{m - 1}x + a_m.1 ) = a_0X^m + a_1 X^{m - 1} + \dots + a_{m - 1}X + a_m E \). \\
Да се докаже, че ако \(A\) е подобна матрица на \(D\),
то  \((\tau(p))(A)\) е подобна на \((\tau(p))(D)\). \\
в) Да се докаже, че \((\tau(p))(D) = \begin{pmatrix}
    p(\lambda_1) & 0 & 0 & \cdots & 0 \\
    0 & p(\lambda_2) & 0 & \cdots & 0 \\
    0 & 0 & p(\lambda_3) & \cdots & 0 \\
    \vdots & \vdots & \vdots & \ddots & \vdots \\
    0 & 0 & 0 & \cdots & p(\lambda_n)
\end{pmatrix} \). \\
г) Нека \(\lambda_1, \; \lambda_2, \; \dots, \; \lambda_n\) са корени на \(p\).
Да се докаже, че \((\tau(p))(D) = \theta\). Тоест \(D\) е корен на \(\tau(p)\). \\
д) Нека \(f_A(\lambda) = \mathrm{det}(A - \lambda E)\) е характерестичния полином на матрицата \(A\). \\
И нека \(f_A\) има \(n\) на брой корена в полето \(\mathbb{F}\).
Тоест всички корени на \(f_A\) са в полето \(\mathbb{F}\). Да се докаже, че
\((\tau(f_A))(A) = \theta\). Тоест \(A\) е корен на \(\tau(f_A)\).
\end{document}