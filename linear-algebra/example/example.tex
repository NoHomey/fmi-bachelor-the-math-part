\documentclass[a4paper, 12pt, oneside]{article}
\usepackage[left=3cm,right=3cm,top=1cm,bottom=2cm]{geometry}
\usepackage{amsmath,amsthm}
\usepackage{amssymb}
\usepackage{lipsum}
\usepackage{stmaryrd}
\usepackage[T1,T2A]{fontenc}
\usepackage[utf8]{inputenc}
\usepackage[bulgarian]{babel}
\usepackage[normalem]{ulem}
                
\setlength{\parindent}{0mm}

\title{Примерно решение с описание на задача за домашна работа}
\author{Иво Стратев}

\begin{document}
\maketitle
В линейното пространство \(\mathbb{Q}^5[x]\) разглеждаме множествата
\begin{align*}
    \mathbb{U} = \{ax^4 + bx^3 + cx^2 + d \; | \; (a, b, c, d) \in \mathbb{Q}^4, \; a + b + c + d = 0\} \\
    \mathbb{W} = \{ax^3 + bx^2 + cx \; | \; (a, b, c) \in \mathbb{Q}^3, \; b = a + c\}
\end{align*}
а) Докажете, че \(\mathbb{U}\) и  \(\mathbb{W}\) са линейни подпространства на \(\mathbb{Q}^5[x]\). \\
б) Определете размерностите на \(\mathbb{U}\) и  \(\mathbb{W}\). \\
в) Намерете базис на \(\mathbb{U} \cap \mathbb{W}\). \\
г) Намерете базис на \(\mathbb{U} + \mathbb{W}\). \\
Решение: \\
а) Нека \(p(x) = ax^4 + bx^3 + cx^2 + d \in \mathbb{U}\), следователно \(a + b + c + d = 0\). \\
Тогава \(p(x) = ax^4 + bx^3 + cx^2 + 0x - (a + b + c) \in \mathbb{Q}^5[x]\), защото \\
\(a, b, c, 0, - (a + b + c) \in \mathbb{Q}\). Следователно \\
\(\mathbb{U} \subseteq \mathbb{Q}^5[x]\), понеже произволен елемент на \(\mathbb{U}\) е елемент на \(\mathbb{Q}^5[x]\). (1) \\
Нека \(f1(x) = a_1x^4 + b_1x^3 + c_1x^2 + d_1, \; f_2(x) = a_2x^4 + b_2x^3 + c_2x^2 + d_2 \in \mathbb{U}\). \\
Тогава \((f_1 + f_2)(x) = f_1(x) + f_2(x) = (a_1 + a_2)x^4 + (b_1 + b_2)x^3 + (c_1 + c_2)x^2 + (d_1 + d_2)\). \\
Трябва да проверим дали елемента \(f_1 + f_2\) е елемент на \(\mathbb{U}\). \\
Тоест трябва да проверим дали \(\mathbb{U}\) е затворено относно събиране на полиноми. \\
Очевидно \((a_1 + a_2, b_1 + b_2, c_1 + c_2, d_1 + d_2) \in \mathbb{Q}^4\). \\
\((a_1 + a_2) + (b_1 + b_2) + (c_1 + c_2) + (d_1 + d_2) = (a_1 + b_1 + c_1 + d_1) + (a_2 + b_2 + c_2 + d_2) = 0 + 0 = 0\), \\
защото \(a_1 + b_1 + c_1 + d_1 = 0\) и \(a_2 + b_2 + c_2 + d_2 = 0\), понеже \(f_1 \in \mathbb{U}\) и \(f_2 \in \mathbb{U}\).
Следователно \(f_1 + f_2 \in \mathbb{U}\). (2) \\
Остава да проверим, че \(\mathbb{U}\) е затворено относно умножение на полином с рационално число. \\
Нека \(f(x) = ax^4 + bx^3 + cx^2 + d \in \mathbb{U}\) и нека \(\mu \in \mathbb{{Q}}\). Тогава \((\mu f)(x) = \mu f(x) = \\
(\mu a)x^4 + (\mu b)x^3 + (\mu c)x^2 + (\mu d)\). Очевидно \((\mu a, \mu b, \mu c, \mu d) \in \mathbb{Q}^4\). \\
\(f \in \mathbb{U}\) следователно \(a + b + c + d = 0\). Тогава \\
\((\mu a) + (\mu b) + (\mu c) + (\mu d) = \mu (a + b + c + d) = \mu 0 = 0\). Следователно \(\mu f \in \mathbb{U}\). (3) \\
От (1), (2) и (3), следва, че \(\mathbb{U} \leq \mathbb{Q}^5[x]\). \\
Лесно се забелязва, че множеството \(\mathbb{W}\) е много подобно на множеството \(\mathbb{U}\). \\
По-точно \(\mathbb{W} = \{ax^3 + bx^2 + cx \; | \; (a, b, c) \in \mathbb{Q}^3, \; b = a + c\} = \\
\{ax^3 + bx^2 + cx \; | \; (a, b, c) \in \mathbb{Q}^3, \; a - b + c = 0\}\). Тогава \\
аналогично на доказателството, че \(\mathbb{U}\) е подпространство на \(\mathbb{Q}^5[x]\) се доказва, \\
че  \(\mathbb{W}\) е подпространство на \(\mathbb{Q}^5[x]\). \\
б) За да определим размерността на \(\mathbb{U}\) трябва да намерим един негов базис. \\
От разсъжденията в а) очевидно \(\mathbb{U} = \\
\{a.x^4 + b.x^3 + c.x^2 + (-a -b - c).1 \; | \; (a, b, c, d) \in \mathbb{Q}^4\}\) и отново очевидно \\
\(\mathbb{U} = \{a(x^4 - 1) + b(x^3 - 1) + c(x^2 - 1) \; | \; a, b, c \in \mathbb{Q}\} = l(x^4 - 1, x^3 - 1, x^2 - 1)\).
Тогава базис ще бъде една линейно независима подсистема на \(\{x^4 - 1, x^3 - 1, x^2 - 1\}\). \\
\textbf{Понеже не сме свикнали да правим изчисления с полиноми ще се възползваме от естественото еднозначно съотвествие между полином и неговите коефициенти}.
Тоест в конкретния случай неявно се възползваме от изображението \(sigma \; : \; \mathbb{Q}^5[x] \to \mathbb{Q}^5\).
действащо по следния начин \(\sigma(a_0x^4 + a_1x^3 + a_2x^2 + a_3x + a_4) = (a_0, a_1, a_2, a_3, a_4)\).
За домейн избрахме множеството \(\mathbb{Q}^5[x]\) понеже в задачата ни е казано, че "работим в него". Така задачата \\
свеждаме до позната за нас задача да определим МЛНЗП на системата \\
\(\{\sigma(x^4 - 1), \sigma(x^3 - 1), \sigma(x^2 - 1)\} = \{(1, 0, 0, 0, -1), (0, 1, 0, 0, -1), (0, 0, 1, 0, -1)\}\). \\
Прилагаме алгоритъма за намиране на МЛНЗП. \\
1) Нареждаме координатите на векторите в матрица по редове. \\
\begin{align*}
    \begin{pmatrix}
        1 & 0 & 0 & 0 & -1 \\
        0 & 1 & 0 & 0 & -1 \\
        0 & 0 & 1 & 0 & -1
    \end{pmatrix}
\end{align*}
2) Прилагаме метода на Гаус-Жордан \\
\begin{align*}
    \begin{pmatrix}
        \fbox{1} & 0 & 0 & 0 & -1 \\
        0 & 1 & 0 & 0 & -1 \\
        0 & 0 & 1 & 0 & -1
    \end{pmatrix} \to \begin{pmatrix}
        1 & 0 & 0 & 0 & -1 \\
        0 & \fbox{1} & 0 & 0 & -1 \\
        0 & 0 & 1 & 0 & -1
    \end{pmatrix} \to \begin{pmatrix}
        1 & 0 & 0 & 0 & -1 \\
        0 & 1 & 0 & 0 & -1 \\
        0 & 0 & \fbox{1} & 0 & -1
    \end{pmatrix} \to \begin{pmatrix}
        1 & 0 & 0 & 0 & -1 \\
        0 & 1 & 0 & 0 & -1 \\
        0 & 0 & 1 & 0 & -1
    \end{pmatrix}
\end{align*}
3) Взимаме наредни n-торки от ненулевите редове на последната матрица. \\
Тогава една МЛНЗ е \(\{(1, 0, 0, 0, -1), (0, 1, 0, 0, -1), (0, 0, 1, 0, -1)\}\). \\
\textbf{Използваме \(sigma^{-1}\) за да се върнем обратно към полиноми, понеже все пак нашите множества са от полиноми}. \\
Една МЛНЗ на \(\{x^4 - 1, x^3 - 1, x^2 - 1\}\) е \\
\(\{\sigma^{-1}((1, 0, 0, 0, -1)), \sigma^{-1}((0, 1, 0, 0, -1)), \sigma^{-1}((0, 0, 1, 0, -1))\} = \\
\{x^4 - 1, x^3 - 1, x^2 - 1\}\). Тогава един базис на \(\mathbb{U}\) са
\begin{align*}
    x^4 - 1 \\
    x^3 - 1 \\
    x^2 - 1
\end{align*}
Следователно размерността на \(\mathbb{U}\) е 3. \\
По аналогичен начин за \(\mathbb{W}\) получаваме. \\
\(\mathbb{W} = \{a(x^3 - x) + b(x^2 - x) \; | \; a, b \in \mathbb{Q}\} = l(x^3 - x, x^2 - x)\). \\
Директно почваме да работим с координатите на тези два полинома като прилагаме отново алгоритъма за намиране на МЛНЗП.
\begin{align*}
    \begin{pmatrix}
        0 & \fbox{1} & 0 & -1 & 0 \\
        0 & 0 & \fbox{1} & -1 & 0
    \end{pmatrix} \to \begin{pmatrix}
        0 & 1 & 0 & -1 & 0 \\
        0 & 0 & 1 & -1 & 0
    \end{pmatrix}
\end{align*}
Директно обръщаме ненулевите редове на полиноми и получаваме, че един базис на \(\mathbb{W}\)
е \(x^3 - x\) и \(x^2 - x\). Следователно размерността е 2. \\
в) \(a + b + c + d = 0\) е хомогенно линейно уравнение, освен това понеже липсва коефициент пред \(x\),
то очевидно той е 0, както видяхме и в а). 
Тогава очевидно множеството \(\mathbb{U}\) е зададено като онези полиноми, които са решение на хомогенната система
\begin{align*}
    x_1 + x_2 + x_3 + x_5 = 0 \\
    x_4 = 0
\end{align*}
Забележка: Разбера еквивалентна на тази система може да бъде получена от намерения в б) базис.
Нека видим как би изглеждало описанието на преминаването от линейна обвивка към еквалентно описание чрез хомогенна система,
понеже на второто контролно масово беше объркано кога какво се прави на Зад. 2.
Работим с координатите на базиса \(x^4 - 1, x^3 - 1, x^2 - 1\). \\
1) нареждаме координатите на координатните вектори по редове и решаваме съответната ХС.
\begin{align*}
    \begin{pmatrix}
        \fbox{1} & 0 & 0 & 0 & -1 \\
        0 & 1 & 0 & 0 & -1 \\
        0 & 0 & 1 & 0 & -1
    \end{pmatrix} \to \begin{pmatrix}
        1 & 0 & 0 & 0 & -1 \\
        0 & \fbox{1} & 0 & 0 & -1 \\
        0 & 0 & 1 & 0 & -1
    \end{pmatrix} \to \begin{pmatrix}
        1 & 0 & 0 & 0 & -1 \\
        0 & 1 & 0 & 0 & -1 \\
        0 & 0 & \fbox{1} & 0 & -1
    \end{pmatrix} \to \begin{pmatrix}
        1 & 0 & 0 & 0 & -1 \\
        0 & 1 & 0 & 0 & -1 \\
        0 & 0 & 1 & 0 & -1
    \end{pmatrix}
\end{align*}
Достигаме до системата
\begin{align*}
    x_1 - x_5 = 0 \\
    x_2 - x_5 = 0 \\
    x_3 - x_5 = 0
\end{align*}
Тя е зависима от 5 неизвестни, но имаме само 3 уравнения, следователно
ще ни трябват 2 параметъра. За параметри избираме колоните, от които не сме избирали водещ елемент.
Така достигаме, до
\begin{align*}
    x_1 = q \\
    x_2 = q \\
    x_3 = q \\
    x_4 = p \\
    x_5 = q
\end{align*}
2) Търсим ФСР \\
Даваме линейно независими стойности на параметрите (за най-лесно стандартния базис на пространството, на което принадлежат параметрите. Тук това е \(\mathbb{Q}^2\)). \\
При \((p, q) = (1, 0)\) получаваме \((0, 0, 0, 1, 0)\). \\
При \((p, q) = (1, 0)\) получаваме \((1, 1, 1, 0, 1)\). \\
3) Взимаме координатните вектори като коефициенти на хомогенна система и наистина получаваме
хомогенната система:
\begin{align*}
    x_1 + x_2 + x_3 + x_5 = 0 \\
    x_4 = 0
\end{align*}
По аналогичен начин разсъждаваме за \(\mathbb{W}\) \\
\(\mathbb{W} = \{ax^3 + bx^2 + cx \; | \; (a, b, c) \in \mathbb{Q}^3, \; a - b + c = 0\} = \\
\{a_0x^3 + a_1x^2 + a_2x \in \mathbb{Q}^5[x] \; | \;  a_0 - a_1 + a_2 = 0, a_3 = 0, a_4 = 0\}\) Тогава \\
елементите на \(\mathbb{W}\) са онези полиноми от \(\mathbb{Q}^5[x]\), чиито
коефициенти \((x_1, x_2, x_3, x_4, x_5)\) са решение на хомогенната система:
\begin{align*}
    x_1 - x_2 + x_3 = 0 \\
    x_4 = 0 \\
    x_5 = 0
\end{align*}
За базис на сечението \(\mathbb{U} \cap \mathbb{W}\) първо търсим ФСР на системата
\begin{align*}
    x_1 + x_2 + x_3 + x_5 = 0 \\
    x_4 = 0 \\
    x_1 - x_2 + x_3 = 0 \\
    x_4 = 0 \\
    x_5 = 0
\end{align*}
тоест на системата
\begin{align*}
    x_1 + x_2 + x_3 = 0 \\
    x_4 = 0 \\
    x_1 - x_2 + x_3 = 0 \\
    x_5 = 0
\end{align*}
\begin{align*}
    2x_1 + 2x_3 = 0 \\
    x_4 = 0 \\
    x_1 - x_2 + x_3 = 0\\
    x_5 = 0
\end{align*} 
\begin{align*}
    x_1 + x_3 = 0 \\
    x_4 = 0 \\
    x_2 = 0 \\
    x_5 = 0
\end{align*}
Достигнахме до система с 5 неизвестни и 4 уравнения.
Очевидно за параметър можем да изберем само или \(x_1\) или \(x_3\).
Избираме \(x_1\) и получаваме:
\begin{align*}
    x_1 = t \\
    x_2 = 0 \\
    x_3 = -t \\
    x_4 = 0 \\
    x_5 = 0
\end{align*}
Тогава едно ФСР на системата е: \((1, 0, -1, 0, 0)\).
Взимаме полиноми с кофициенти координатите на ФСР-то, защото все пак работим в полиномиални пространства.
Тогава един базис на сечението е \(x^4 - x^2\). \\
г) Преди да търсим базис на сумата нека си изготвим предположение на колко трябва да е равна размерността
на сумата на двете пространства с формулата вързваща размерностите на сумата и сечението на две пространства.
\begin{align*}
    dim(\mathbb{U} + \mathbb{W}) = dim(\mathbb{U}) + dim(\mathbb{U}) - dim(\mathbb{U} \cap \mathbb{W}) = \\
    3 + 2 - 1 = 4
\end{align*}
От б) знаем базис на всяко пространство. За базис на сумата търсим МЛНЗП на векторите в двата базиса. \\
1) Нареждаме коориднатите на двата базиса по редове в една обща матрица
\begin{align*}
    \begin{pmatrix}
        1 & 0 & 0 & 0 & -1 \\
        0 & 1 & 0 & 0 & -1 \\
        0 & 0 & 1 & 0 & -1 \\
        0 & 1 & 0 & -1 & 0 \\
        0 & 0 & 1 & -1 & 0
    \end{pmatrix}
\end{align*}
2) Прилагаме алгоритъма на Гаус-Жордан
\begin{align*}
    \begin{pmatrix}
        \fbox{1} & 0 & 0 & 0 & -1 \\
        0 & 1 & 0 & 0 & -1 \\
        0 & 0 & 1 & 0 & -1 \\
        0 & 1 & 0 & -1 & 0 \\
        0 & 0 & 1 & -1 & 0
    \end{pmatrix} \to
    \begin{pmatrix}
        1 & 0 & 0 & 0 & -1 \\
        0 & \fbox{1} & 0 & 0 & -1 \\
        0 & 0 & 1 & 0 & -1 \\
        0 & 1 & 0 & -1 & 0 \\
        0 & 0 & 1 & -1 & 0
    \end{pmatrix} \to \begin{pmatrix}
        1 & 0 & 0 & 0 & -1 \\
        0 & 1 & 0 & 0 & -1 \\
        0 & 0 & \fbox{1} & 0 & -1 \\
        0 & 0 & 0 & -1 & 1 \\
        0 & 0 & 1 & -1 & 0
    \end{pmatrix} \to \\
    \begin{pmatrix}
        1 & 0 & 0 & 0 & -1 \\
        0 & 1 & 0 & 0 & -1 \\
        0 & 0 & 1 & 0 & -1 \\
        0 & 0 & 0 & \fbox{-1} & 1 \\
        0 & 0 & 0 & -1 & 1
    \end{pmatrix} \to \begin{pmatrix}
        1 & 0 & 0 & 0 & -1 \\
        0 & 1 & 0 & 0 & -1 \\
        0 & 0 & 1 & 0 & -1 \\
        0 & 0 & 0 & -1 & 1 \\
        0 & 0 & 0 & 0 & 0
    \end{pmatrix} \to \begin{pmatrix}
        1 & 0 & 0 & 0 & -1 \\
        0 & 1 & 0 & 0 & -1 \\
        0 & 0 & 1 & 0 & -1 \\
        0 & 0 & 0 & 1 & -1
    \end{pmatrix}
\end{align*}
3) На всеки ненулев (оцелял) ред съпоставяме полином с коефициенти - коефициентите на реда.
Така получаваме за базис на сумата:
\begin{align*}
    x^4 - 1 \\
    x^3 - 1 \\
    x^2 - 1 \\
    x - 1
\end{align*}
Базиса е от 4 вектора, което не е гаранция, че нямаме грешка, но поне съвпада с очакването и вероятно нямаме грешка. \\
\textbf{Отговори:}\\
б) \(dim(\mathbb{U}) = 3\) и \(dim(\mathbb{W}) = 2\). \\
в) базис на \(\mathbb{U} \cap \mathbb{W}\): \(x^4 - x^2\). \\
г) базис на \(\mathbb{U} + \mathbb{W}\): \(x^4 - 1, x^3 - 1, x^2 - 1, x - 1\).
\end{document}
