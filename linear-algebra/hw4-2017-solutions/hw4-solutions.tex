\documentclass[a4paper, 12pt]{article}
    
\usepackage[left=3cm,right=3cm,top=1cm,bottom=2cm]{geometry}
\usepackage{amsmath}
\usepackage{amssymb}
\usepackage{pst-node}
\usepackage{stackengine}
\usepackage{fixltx2e}
\usepackage[T1,T2A]{fontenc}
\usepackage[utf8]{inputenc}
\usepackage[bulgarian]{babel}
\usepackage[normalem]{ulem}
\newcommand{\stkout}[1]{\ifmmode\text{\sout{\ensuremath{#1}}}\else\sout{#1}\fi}

\makeatletter
\renewcommand*\env@matrix[1][*\c@MaxMatrixCols c]{%
  \hskip -\arraycolsep
  \let\@ifnextchar\new@ifnextchar
  \array{#1}}
\makeatother

\newcommand\tab[1][1cm]{\hspace*{#1}}

\newcommand{\N}{\mathbb{N}}

\title{Решения на домашна работа № 4 по Линейна Алгебра за спец. Информатика 2017/18г.}
\author{Иво Стратев}

\begin{document}
    \maketitle
    \section*{Задача 1.} Да се пресметне детерминантата: \\\\
    $\Delta = \begin{vmatrix}
        -1 & -5 & -5 & -5 & -5 & \dots & -5\\
        -8 & 9 + n.1 & 9 & 9 & 9 & \dots & 9\\
        -8 & 9 & 9 + 1.2 & 9 & 9 & \dots & 9\\
        -8 & 9 & 9 & 9 + 2.3 & 9 & \dots & 9\\
        -8 & 9 & 9 & 9 & 9 + 3. 4 & \dots & 9\\
        \\
        ~ & ~ & ~ & \dots & ~ & ~ & ~\\
        \\
        -8 & 9 & 9 & 9 & 9 & \dots & 9 + (n - 1).n\\
    \end{vmatrix}$ \\\\
    
    Забелязваме, че ако например умножим първия ред с $\frac{9}{5}$
    и го прибавим към останалите ще получим много нули,
    което ни е основната цел при пресмятането на детерминанти. Така получаваме, че: \\\\

    $\Delta = \begin{vmatrix}
        -1 & -5 & -5 & -5 & -5 & \dots & -5\\
        -\frac{49}{5} & n.1 & 0 & 0 & 0 & \dots & 0\\
        -\frac{49}{5} & 0 & 1.2 & 0 & 0 & \dots & 0\\
        -\frac{49}{5} & 0 & 0 & 2.3 & 0 & \dots & 0\\
        -\frac{49}{5} & 0 & 0 & 0 & 3. 4 & \dots & 0\\
        \\
        ~ & ~ & ~ & \dots & ~ & ~ & ~\\
        \\
        -\frac{49}{5} & 0 & 0 & 0 & 0 & \dots & (n - 1).n\\
    \end{vmatrix}$ \\\\\\
    Лесно можем да съобразим, че получената и дадената детерминанта
    всъщност са детерминанти от така наречения "вид - Пачи крак". \\\\
    Забелязваме, че изключвайки елемента $-1$ елементите по главния диагонал
    са точно $n$ на брой (това лесно се вижда, например от факта, че всички
    са произведние на две числа и второто число в тези пройзведения са мени
    от $1$ до $n$). Тоест дадената ни детерминанта е от $n + 1$ ред. \\\\
    Същото така лесно можем да забележим, че елементите по главния диагонал
    започвайки от $1.2$ до $(n - 1).n$ имат еднакъв вид и могат да бъдат
    записани като $(i - 1).i$ за $i = 2, \; \dots, \; n$, а елементът $n.1$
    не е от техният вид, затова още тук ще се спрем и ще разгледаме частния
    случай, когато $n = 1$, тогава \\\\
    
    $\Delta = \begin{vmatrix}
        -1 & -5 \\
        -\frac{49}{5} & 1
    \end{vmatrix} = -1.1 - \left(-5.-\frac{49}{5}\right) = -1 - 49 = -50$ \\\\
    
    Сега нека $n > 1$. Тогава всички елементите по главния диагонал
    са различни от нула и спокойно можем да делим на всеки от тях.
    Съобразявайки, че както вече казахме вида на получената детерминанта е Пачи крак.
    То можем или да нулираме всички елементи на първия ред с изключение на елемента $-1$
    или да нулираме всички елементи на първия стълб с изключение на елемента $-1$.
    Понеже сме свикнали да работим по редове ще предпочетем да нулираме елементите
    на първия ред. За целта забелязваме, че всеки ред с изключение на първия трябва
    да разделим първо на елемента в главния диагонал, след това да го уножим с $5$
    и да го прибавим към първия. Последователно пресмятаме: \\\\ 

    $\Delta = \begin{vmatrix}
        -1 -\frac{49}{5}.\frac{5}{n.1} & 0 & -5 & -5 & -5 & \dots & -5\\
        -\frac{49}{5} & n.1 & 0 & 0 & 0 & \dots & 0\\
        -\frac{49}{5} & 0 & 1.2 & 0 & 0 & \dots & 0\\
        -\frac{49}{5} & 0 & 0 & 2.3 & 0 & \dots & 0\\
        -\frac{49}{5} & 0 & 0 & 0 & 3. 4 & \dots & 0\\
        \\
        ~ & ~ & ~ & \dots & ~ & ~ & ~\\
        \\
        -\frac{49}{5} & 0 & 0 & 0 & 0 & \dots & (n - 1).n\\
    \end{vmatrix} = \\\\\\
    = \begin{vmatrix}
        -1 -49.\frac{1}{n.1} -\frac{49}{5}.\frac{5}{1.2} & 0 & 0 & -5 & -5 & \dots & -5\\
        -\frac{49}{5} & n.1 & 0 & 0 & 0 & \dots & 0\\
        -\frac{49}{5} & 0 & 1.2 & 0 & 0 & \dots & 0\\
        -\frac{49}{5} & 0 & 0 & 2.3 & 0 & \dots & 0\\
        -\frac{49}{5} & 0 & 0 & 0 & 3. 4 & \dots & 0\\
        \\
        ~ & ~ & ~ & \dots & ~ & ~ & ~\\
        \\
        -\frac{49}{5} & 0 & 0 & 0 & 0 & \dots & (n - 1).n\\
    \end{vmatrix} = \\\\\\
    \begin{vmatrix}
        -1 -49.\frac{1}{n.1} -49.\frac{1}{1.2} -\frac{49}{5}.\frac{5}{2.3}  & 0 & 0 & 0 & -5 & \dots & -5\\
        -\frac{49}{5} & n.1 & 0 & 0 & 0 & \dots & 0\\
        -\frac{49}{5} & 0 & 1.2 & 0 & 0 & \dots & 0\\
        -\frac{49}{5} & 0 & 0 & 2.3 & 0 & \dots & 0\\
        -\frac{49}{5} & 0 & 0 & 0 & 3. 4 & \dots & 0\\
        \\
        ~ & ~ & ~ & \dots & ~ & ~ & ~\\
        \\
        -\frac{49}{5} & 0 & 0 & 0 & 0 & \dots & (n - 1).n\\
    \end{vmatrix} = \\\\\\
    = \begin{vmatrix}
        -1 -49.\frac{1}{n.1} -49.\frac{1}{1.2} -49.\frac{1}{2.3} -\frac{49}{5}.\frac{5}{3.4}  & 0 & 0 & 0 & 0 & \dots & -5\\
        -\frac{49}{5} & n.1 & 0 & 0 & 0 & \dots & 0\\
        -\frac{49}{5} & 0 & 1.2 & 0 & 0 & \dots & 0\\
        -\frac{49}{5} & 0 & 0 & 2.3 & 0 & \dots & 0\\
        -\frac{49}{5} & 0 & 0 & 0 & 3. 4 & \dots & 0\\
        \\
        ~ & ~ & ~ & \dots & ~ & ~ & ~\\
        \\
        -\frac{49}{5} & 0 & 0 & 0 & 0 & \dots & (n - 1).n\\
    \end{vmatrix} = \dots = \\\\\\
    = \begin{vmatrix}
        -1 -49.\frac{1}{n.1} -49.\frac{1}{1.2} + \dots + (-49).\frac{1}{(n - 1).n}  & 0 & 0 & 0 & 0 & \dots & 0\\
        -\frac{49}{5} & n.1 & 0 & 0 & 0 & \dots & 0\\
        -\frac{49}{5} & 0 & 1.2 & 0 & 0 & \dots & 0\\
        -\frac{49}{5} & 0 & 0 & 2.3 & 0 & \dots & 0\\
        -\frac{49}{5} & 0 & 0 & 0 & 3. 4 & \dots & 0\\
        \\
        ~ & ~ & ~ & \dots & ~ & ~ & ~\\
        \\
        -\frac{49}{5} & 0 & 0 & 0 & 0 & \dots & (n - 1).n\\
    \end{vmatrix}$ \\\\\\

    Забелязваме, че можем да изкараме общия множител $-49$ пред скоби,
    както и че сумата $\frac{1}{n.1} + \frac{1}{1.2} + \dots + \frac{1}{(n - 1).n}$
    можем да запишем като $\frac{1}{n.1} + \displaystyle\sum_{i = 2}^n \frac{1}{(i - 1).i}$
    използвайки наблюденията ни за общия вид на всички без първото събираемо. Получаваме, че \\\\
    $\Delta = \begin{vmatrix}
        -1 -49\left(\frac{1}{n.1} + \displaystyle\sum_{i = 2}^n \frac{1}{(i - 1).i}\right)  & 0 & 0 & 0 & 0 & \dots & 0\\
        -\frac{49}{5} & n.1 & 0 & 0 & 0 & \dots & 0\\
        -\frac{49}{5} & 0 & 1.2 & 0 & 0 & \dots & 0\\
        -\frac{49}{5} & 0 & 0 & 2.3 & 0 & \dots & 0\\
        -\frac{49}{5} & 0 & 0 & 0 & 3. 4 & \dots & 0\\
        \\
        ~ & ~ & ~ & \dots & ~ & ~ & ~\\
        \\
        -\frac{49}{5} & 0 & 0 & 0 & 0 & \dots & (n - 1).n\\
    \end{vmatrix}$ \\\\\\

    Развиваме детерминанта по първия ред, защото по този начин ще се отървем от първия стълб и ред (смъкваме размерността с единица) и получаваме: \\\\
    
    $\Delta = (-1)^{1 + 1}.\left[-1 -49\left(\frac{1}{n.1} + \displaystyle\sum_{i = 2}^n \frac{1}{(i - 1).i}\right)\right].\begin{vmatrix}
        n.1 & 0 & 0 & 0 & \dots & 0\\
        0 & 1.2 & 0 & 0 & \dots & 0\\
        0 & 0 & 2.3 & 0 & \dots & 0\\
        0 & 0 & 0 & 3. 4 & \dots & 0\\
        \\
        ~ & ~ & ~ & \dots & ~ & ~\\
        \\
        0 & 0 & 0 & 0 & \dots & (n - 1).n\\
    \end{vmatrix}$ \\\\\\

    Получената детерминанта от n-ти ред можем да развием или по първия ред или по първия стълб
    и в двата случая ще получим: \\\\
    
    $\Delta = (-1)^{1 + 1}.\left[-1 -49\left(\frac{1}{n.1} + \displaystyle\sum_{i = 2}^n \frac{1}{(i - 1).i}\right)\right].(n.1).\begin{vmatrix}
        1.2 & 0 & 0 & \dots & 0\\
        0 & 2.3 & 0 & \dots & 0\\
        0 & 0 & 3. 4 & \dots & 0\\
        \\
        ~ & ~ & ~ & \dots & ~ \\
        \\
        0 & 0 & 0 & \dots & (n - 1).n\\
    \end{vmatrix}$ \\\\\\

    Отново можем да развием по първи ред/стълб и ще получим: \\\\
    
    $\Delta = (-1)^{1 + 1}.\left[-1 -49\left(\frac{1}{n.1} + \displaystyle\sum_{i = 2}^n \frac{1}{(i - 1).i}\right)\right].(n.1).(1.2).\begin{vmatrix}
        2.3 & 0 & \dots & 0\\
        0 & 3. 4 & \dots & 0\\
        \\
        ~ & ~ & \dots & ~\\
        \\
        0 & 0 & \dots & (n - 1).n\\
    \end{vmatrix}$ \\\\\\

    Очевидно можем да повторим развитието по първи ред/стълб и ще получим: \\\\

    $\Delta = \left[-1 -49\left(\frac{1}{n.1} + \displaystyle\sum_{i = 2}^n \frac{1}{(i - 1).i}\right)\right].(n.1).(1.2).(2.3).(3.4).\dots.(n - 1).n$ \\\\\\

    Забелязваме, че произведението $(1.2).(2.3).(3.4).\dots.(n - 1).n$
    е произведение на елемените, които остановихме, че имат един и същ вид
    тогава можем да го запишем по-компактно, чрез символа за произведение.

    Тоест $(1.2).(2.3).(3.4).\dots.(n - 1).n = \displaystyle\prod_{i = 2}^n (i - 1).i$,
    това произведение можем да запишем като произведение на две произведения, използвайки, че
    произведението на числа е комутативна операция, тоест
    $\displaystyle\prod_{i = 2}^n (i - 1).i = \displaystyle\prod_{i = 1}^{n - 1}i.\displaystyle\prod_{i = 2}^ni$.
    Така получаваме
    
    \begin{align*}
        (n.1).(1.2).(2.3).(3.4).\dots.(n - 1).n = \\
        = (n.1).\displaystyle\prod_{i = 1}^{n - 1}i.\displaystyle\prod_{i = 2}^ni
        = \displaystyle\prod_{i = 1}^ni.\displaystyle\prod_{i = 1}^ni = (n!).(n!) = (n!)^2.
    \end{align*} \\

    Тоест получаваме, че: \\

    \begin{align*}
        \Delta = \left[-1 -49\left(\frac{1}{n.1} + \displaystyle\sum_{i = 2}^n \frac{1}{(i - 1).i}\right)\right].(n!)^2
    \end{align*} \\

    Сега единственото, което остава е да пресметнем стойността на израза
    
    \begin{align*}
        \frac{1}{n.1} + \displaystyle\sum_{i = 2}^n \frac{1}{(i - 1).i}
    \end{align*} \\
    
    Нека пресметнем първите четири последователни стойности на израза\\
    за $n = 1, \; 2, \; 3, \; 4$.
    
    \begin{align*}
        \frac{1}{1.1} + \displaystyle\sum_{i = 2}^1 \frac{1}{(i - 1).i} = 1 + 0 = 1\\
        \frac{1}{2.1} + \displaystyle\sum_{i = 2}^2 \frac{1}{(i - 1).i} = \frac{1}{2} + \frac{1}{1.2} = 1\\
        \frac{1}{3.1} + \displaystyle\sum_{i = 2}^3 \frac{1}{(i - 1).i} = \frac{1}{3} + \frac{1}{1.2} + \frac{1}{2.3} = \frac{2 + 3 + 1}{6} = 1\\
        \frac{1}{4.1} + \displaystyle\sum_{i = 2}^3 \frac{1}{(i - 1).i} = \frac{1}{4} + \frac{1}{1.2} + \frac{1}{2.3} + \frac{1}{3.4} = \frac{3 + 6 + 2 + 1}{12} = 1
    \end{align*} \\

    След като ги пресметнахме си изграждаме хипотезата, че стойността на израза е $1$ за всяко $n \in \N$.
    Ще докажем тази хипотеза чрез индукция по естествените числа. \\
    
    Нека $\exists k \in \N \; : \; \frac{1}{k} + \displaystyle\sum_{i = 2}^{k}\frac{1}{(i - 1).i} = 1$.\\

    Тогава  \begin{align*}
    \frac{1}{k + 1} + \displaystyle\sum_{i = 2}^{k + 1}\frac{1}{(i - 1).i} = \\\\
    = \frac{1}{k + 1} + \displaystyle\sum_{i = 2}^{k}\frac{1}{(i - 1).i} + \frac{1}{k.(k + 1)} = \\\\
    = \frac{1}{k + 1} - \frac{1}{k} + \frac{1}{k} + \displaystyle\sum_{i = 2}^{k}\frac{1}{(i - 1).i} + \frac{1}{k.(k + 1)} = \\\\
    = \frac{1}{k + 1} - \frac{1}{k} + 1 + \frac{1}{k.(k + 1)} = \frac{k - (k + 1) + k.(k + 1) + 1}{k(.k + 1)} = \\\\
    = \frac{(k + 1)-(k + 1) + k.(k + 1)}{k.(k + 1)} = \frac{k.(k + 1)}{k.(k + 1)} = 1 \\\\
    \implies \forall n \in \N \quad \frac{1}{n} + \displaystyle\sum_{i = 2}^{n}\frac{1}{(i - 1).i} = 1.
    \end{align*} \\\\

    Така получаваме, че

    \begin{align*}
        \Delta = \left[-1 -49\left(\frac{1}{n.1} + \displaystyle\sum_{i = 2}^n \frac{1}{(i - 1).i}\right)\right].(n!)^2 = \\\\
        = (-1 - 49.1).(n!)^2 = (-50).(n!)^2.
    \end{align*} \\

    Тоест $\forall n \in \N \quad \Delta = (-50).(n!)^2$
    \section*{Задача 2.}
    В линейното пространство $\mathbb{V}$ с базис $e_1, \; e_2, \; e_3$
    е даден линейният оператор $\varphi$ с матрица
    \begin{align*}
        \begin{pmatrix}
            -12 & 2 & 3\\
            6 & -13 & -6\\
            -9 & 6 & 0
        \end{pmatrix}.
    \end{align*}

    Да се намери базис, в който матрицата на $\varphi$ е диагонална, както и матрицата
    на оператора в този базис.
    \subsection*{Решение:}
    Нека $A = M_e(\varphi) = \begin{pmatrix}
        -12 & 2 & 3\\
        6 & -13 & -6\\
        -9 & 6 & 0
    \end{pmatrix}$ \\\\\\ 
    Тогава търсим корените на полинома $f_A(\lambda) = det(A - \lambda E) =  \begin{vmatrix}
        -12 - \lambda & 2 & 3\\
        6 & -13 - \lambda & -6\\
        -9 & 6 & - \lambda
    \end{vmatrix} = \\\\\\
    = \begin{vmatrix}
        -12 - \lambda & 2 & 3\\
        6 & -13 - \lambda & -6\\
        3 + \lambda & 4 & -3 - \lambda
    \end{vmatrix} = \begin{vmatrix}
        -12 - \lambda & 2 & 3\\
        9 + \lambda & -9 - \lambda & -9 - \lambda\\
        3 + \lambda & 4 & -3 - \lambda
    \end{vmatrix} = \\\\\\
    = (9 + \lambda)\begin{vmatrix}
        -12 - \lambda & 2 & 3\\
        1 & -1 & -1\\
        3 + \lambda & 4 & -3 - \lambda
    \end{vmatrix} = (9 + \lambda)\begin{vmatrix}
        -10 - \lambda & 0 & 1\\
        1 & -1 & -1\\
        7 + \lambda & 0 & -7 - \lambda
    \end{vmatrix} = \\\\\\
    = (9 + \lambda)(7 + \lambda)\begin{vmatrix}
        -10 - \lambda & 0 & 1\\
        1 & -1 & -1\\
        1 & 0 & -1
    \end{vmatrix} = (9 + \lambda)(7 + \lambda)\begin{vmatrix}
        -9 - \lambda & 0 & 0\\
        0 & -1 & 0\\
        1 & 0 & -1
    \end{vmatrix} = \\\\\\
    = (9 + \lambda)^2(7 + \lambda)\begin{vmatrix}
        -1 & 0 & 0\\
        0 & -1 & 0\\
        1 & 0 & -1
    \end{vmatrix} = (9 + \lambda)^2(7 + \lambda)\begin{vmatrix}
        -1 & 0 & 0\\
        0 & -1 & 0\\
        0 & 0 & -1
    \end{vmatrix} = -(9 + \lambda)^2(7 + \lambda)\\
    \\\\ \implies f_A(\lambda) = -(9 + \lambda)^2(7 + \lambda) = 0\\
    \\\\\implies \lambda_{1, 2} = -9, \; \lambda_3 = -7$ \\\\\\
    Търсим базис от собствени вектори на подпространстовото на $\mathbb{V}$:
    \begin{align*}
        \mathbb{V}_{-9} = \{v \in \mathbb{V} \; | \; \varphi(v) = (-9).v\} = \\\\
        = \{v \in \mathbb{V} \; | \; \varphi(v) - (-9).v = \theta \} = \\\\
        = \{v \in \mathbb{V} \; | \; \varphi(v) + 9.id(v) = \theta \} = \\\\
        = \{v \in \mathbb{V} \; | \; (\varphi + 9.id)(v) = \theta \} = \\\\
        = Ker(\varphi + 9.id)
    \end{align*}
    Спрямо дадния базис оператора $\varphi + 9.id$ има матрица $A - (-9).E$.
    Тогава търсим ФСР на хомогенната система $(A + 9.E).v = \theta$.\\\\
    $A + 9.E = \begin{pmatrix}
        -3 & 2 & 3\\
        6 & -4 & -6\\
        -9 & 6 & 9
    \end{pmatrix} \to \begin{pmatrix}
        -3 & 2 & 3\\
        0 & 0 & 0\\
        0 & 0 & 0
    \end{pmatrix} \to \begin{pmatrix}
        -3 & 2 & 3
    \end{pmatrix}$\\\\\\
    Достигнахме до едно ЛНЗ уравнение на три променливи,
    тоест решенията зависи от два параметъра и размерността
    му съвпада с кратността на корена. Тогава полагаме
    $v_2 = p, \; v_3 = q$. Тогава общия вид на решенията
    на хомогенната система е:
    \begin{align*}
        v = \begin{pmatrix}
            v_1 \\
            v_2 \\
            v_3
        \end{pmatrix} = \begin{pmatrix}
            \frac{2}{3}.p + q \\
            p \\
            q
        \end{pmatrix}
    \end{align*}
    При $p = 3, \; q = 0$ получаваме вектора $b_1 = 2.e_1 + 3.e_2$ \\\\
    При $p = 0, \; q = 1$ получаваме вектора $b_2 = e_1 + e_3$. \\\\
    Сега ще търсим базис на подпространстовото
    \begin{align*}
        \mathbb{V}_{-9} = \{v \in \mathbb{V} \; | \; \varphi(v) = (-7).v\} = Ker(\varphi + 7.id)
    \end{align*}
    Спрямо дадния базис оператора $\varphi + 7.id$ има матрица $A - (-7 ).E$.
    Тогава търсим ФСР на хомогенната система $(A + 7.E).v = \theta$.\\\\
    $A + 7.E = \begin{pmatrix}
        -5 & 2 & 3\\
        6 & -6 & -6\\
        -9 & 6 & 7
    \end{pmatrix} \to \begin{pmatrix}
        -5 & 2 & 3\\
        1 & -1 & -1\\
        -9 & 6 & 7
    \end{pmatrix} \to \begin{pmatrix}
        -3 & 0 & 1\\
        1 & -1 & -1\\
        -3 & 0 & 1
    \end{pmatrix} \to\\
    \\\\\to \begin{pmatrix}
        -3 & 0 & 1\\
        1 & -1 & -1\\
        0 & 0 & 0
    \end{pmatrix} \to \begin{pmatrix}
        -3 & 0 & 1\\
        1 & -1 & -1
    \end{pmatrix} \to \begin{pmatrix}
        -3 & 0 & 1\\
        -2 & -1 & 0
    \end{pmatrix}$ \\\\
    Получихме две ЛНЗ уравнения на три променливи,
    тоест решенията зависят от един параметъра и размерността
    му съвпада с кратността на корена. Тогава общия вид на решенията
    на хомогенната система е:
    \begin{align*}
        v = \begin{pmatrix}
            v_1 \\
            v_2 \\
            v_3
        \end{pmatrix} = \begin{pmatrix}
            t \\
            -2.t \\
            3.t
        \end{pmatrix}.
    \end{align*}
    Полагаме $t = 1$ и получаваме вектора $b_3 = e_1 - 2.e_2 + 3.e_3$.\\\\
    Размерността на $\mathbb{V}$ е 3, получихме три ЛНЗ собствени вектора
    (векторите $b_1$ и $b_2$ са ЛНЗ, защото така си ги избрахме и вектора $b_3$,
    защото векторите отговарящи на различни собствени стойности са ЛНЗ).
    Тогава векторите:
    \begin{align*}
        b_1 = 2.e_1 + 3.e_2 \\
        b_2 = e_1 + e_3 \\
        b_3 = e_1 - 2.e_2 + 3.e_3
    \end{align*}
    образуват базис на $\mathbb{V}$. \\\\
    Знаем също и че:
    \begin{align*}
        \varphi(b_1) = (-9).b_1 = (-9).b_1 + 0.b_2 + 0.b_3 \\
        \varphi(b_2) = (-9).b_2 = 0.b_1 + (-9).b_2 + 0.b_3 \\
        \varphi(b_3) = (-7).b_3 = 0.b_1 + 0.b_2 + (-7).b_3
    \end{align*}
    Тогава знаем, как действа $\varphi$ в базиса $b_1, \; b_2, \; b_3$.
    По-конкретно знаем как изглежда матрицата на $\varphi$ в базиса $b_1, \; b_2, \; b_3$,
    записвайки по стълбове координатите на образите получаваме: \\\\
    $M_b(\varphi) = \begin{pmatrix}
        -9 & 0 & 0\\
        0 & -9 & 0\\
        0 & 0 & -7
    \end{pmatrix}$
    \section*{Задача 4.}
    В Евклидовото пространство $\mathbb{R}^4$ са дадени векторите
    \begin{align*}
        a_1 = (3, \; -2, \; 4, \; 1), \quad a_2 = (-2, \; 4, \; -4, \; -1),\\
        a_3 = (2, \; -2, \; 3, \; 1), \quad a_4 = (4, \; -2, \; 5, \; 1)
    \end{align*}
    и векторът $a = (1, \; 7, \; 2, \; -1)$.\\
    а) Да се намери ортоногонален базис по метода
    на Грам-Шмид на $\mathbb{U} = l(a_1, \; a_2, \; a_3, \; a_4)$. \\
    б) Да се намери проекцията $a_0$ на вектора $a$ върху $\mathbb{U}$
    и перпендикулярът $h$ спуснат от $a$ към $\mathbb{U}$.
    \subsection*{Решение:}
    \subsubsection*{а)}
    Първо прилагаме метода на Гаус или Гаус-Жордан за да премахнем линейно
    зависимите и за да намалим коефициентите. \\\\
    $\begin{pmatrix}
         3 &  -2 & 4 &  1\\
        -2 &  4 & -4 & -1\\
         2 & -2 &  3 &  1\\
         4 & -2 &  5 &  1
    \end{pmatrix} \to \begin{pmatrix}
         1 & 0 &  1 &  0\\
         0 & 2 & -1 &  0\\
         2 &-2 &  3 &  1\\
         2 & 0 &  2 &  0
    \end{pmatrix} \to \begin{pmatrix}
         1 & 0 &  1 & 0\\
         0 & 2 & -1 & 0\\
         0 &-2 &  1 & 1\\
         0 & 0 &  0 & 0
    \end{pmatrix}$ \\\\
    Нека \begin{align*}
        b_1 = (1, \; 0, \; 1, \; 0) \\
        b_2 = (0, \; 2, \; -1, \; 0) \\
        b_3 = (0, \; -2, \; 1, \; 1)
    \end{align*}
    тогава $\mathbb{U} = l(b_1, \; b_2, \; b_3)$. \\\\
    
    Ще постройм ортогонален базис на $\mathbb{U}$.
    Понеже вектора $b_1$ е с най-малки коефициенти ще изберем
    да започнем от него (бихме могли да изберем произволен вектор измежду
    $b_1, \; b_2, \; b_3$). Тогава нека $u_1 = b_1$. Ще търсим вектор
    $u_2$ в следния вид $u_2 = b_2 + \lambda.u_1$, който да е ортогонален
    на вектора $u_1$, тоест $(u_2, \; u_1) = 0$. \\
    Пресмятаме
    \begin{align*}
        (u_2, \; u_1) = (b_2 + \lambda.u_1, \; u_1) = \\\\
        = (b_2, \; u_1) + (\lambda.u_1, \; u_1) = \\\\
        = (b_2, \; u_1) + \lambda.(u_1, \; u_1) = 0 \\\\
        \implies \lambda.(u_1, \; u_1) = -(b_2, \; u_1).
    \end{align*}
    Вектора $u_1$ е ненулев тогава $(u_1, \; u_1) > 0$. Следователно
    $\lambda = -\frac{(b_2, \; u_1)}{(u_1, \; u_1)}$. \\\\
    Пресмятаме скаларния квадрат на $u_1$, защото ще ни трябва
    и по-натам.
    \begin{align*}
        (u_1, \; u_1) = 1^2 + 0^2 + 1^2 + 0^2 = 2 
    \end{align*}
    Така получаваме
    \begin{align*}
        \lambda = -\frac{(b_2, \; u_1)}{2} = \\
        = -\frac{1.0 + 0.2 + 1.(-1) + 0.0}{2} = \\
        = -\frac{(-1)}{2} = \frac{1}{2}.
    \end{align*}
    Тоест $u_2 = b_2 + \frac{1}{2}.b_1
    = (\frac{1}{2}, \; 2, \; -\frac{1}{2}, \; 0) = \frac{1}{2}.(1, \; 4, \; -1, \; 0)$. \\\\
    Сега ще търсим вектор $u_3$ в следния вид
    $u_3 = b_3 + \alpha.u_1 + \beta.u_2$, който да е едновременно ортогонален
    на вектора $u_1$ и на $u_2$, тоест $(u_3, \; u_1) = 0$ и $(u_2, \; u_1) = 0$.
    Пресмятаме
    \begin{align*}
        (u_3, \; u_1) = (b_3 + \alpha.u_1 + \beta.u_2, \; u_1) = \\\\
        = (b_3, \; u_1) + (\alpha.u_1, \; u_1) + (\beta.u_2, \; u_1) = \\\\
        = (b_3, \; u_1) + \alpha.(u_1, \; u_1) + \beta.(u_2, \; u_1) = 0.
    \end{align*}
    Тук използваме, че $u_2$ го избрахме такъв, че да бъде ортогонален на $u_1$.
    Тоест $(u_2, \; u_1) = 0$. Отново ще използваме, че $(u_1, \; u_1) > 0$. \\\\
    Така получаваме $\alpha = -\frac{(b_3, \; u_1)}{(u_1, \; u_1)}
    = -\frac{0.1 + (-2).2 + 1.(-1) + 1.0}{2} = -\frac{(-5)}{2} = \frac{5}{2}$. \\\\
    За $(u_3, \; u_2)$ пресмятаме
    \begin{align*}
        (u_3, \; u_2) = (b_3 + \alpha.u_1 + \beta.u_2, \; u_2) = \\\\
        = (b_3, \; u_2) + (\alpha.u_1, \; u_2) + (\beta.u_2, \; u_2) = \\\\
        = (b_3, \; u_2) + \alpha.(u_1, \; u_2) + \beta.(u_2, \; u_2) = 0.
    \end{align*}
    Отнвово ползваме, че $(u_2, \; u_1) = 0$, както и че $u_2 \neq \theta \implies (u_2, \; u_2) > 0$.
    Получаваме $\beta = -\frac{(b_3, \; u_2)}{(u_2, \; u_2)}$.
    Отново пресмятаме отделно скаларния квадрат на $u_2$.
    \begin{align*}
        (u_2, \; u_2) = \frac{1}{2^2}.(1^2 + 4^2 + (-1)^2 + 0^2) = \frac{1}{4}.18 = \frac{9}{2}
    \end{align*} и така получаваме
    \begin{align*}
        \beta = -\frac{(b_3, \; u_2)}{(u_2, \; u_2)} = -\frac{2}{9}.(b_3, \; u_3) = \\
        = -\frac{2}{9}.\frac{1}{2}.((0, \; -2, \; 1, \; 1), \; (1, \; 4, \; -1, \; 0)) = \\
        = -\frac{1}{9}.(0.1 + (-2).4 + 1.(-1) + 1.0) = -\frac{1}{9}.(-9) = 1.
    \end{align*}
    Тоест $u_3 = b_3 + \frac{5}{2}.u_1 + u_2 =
    \frac{1}{2}.(2.(0, \; -2, \; 1, \; 1) + 5.(1, \; 0, \; 1, \; 0) + (1, \; 4, \; -1, \; 0)) = \\\\
    = \frac{1}{2}.(6, \; 0, \; 6, \; 2) = (3, \; 0, \; 3, \; 1)$. \\\\
    Така достигнахме до ортогоналният базис на $\mathbb{U}$
    \begin{align*}
        u_1 = (1, \; 0, \; 1, \; 0) \\
        u_2 = \frac{1}{2}.(1, \; 4, \; -1, \; 0) \\
        u_3 = (3, \; 0, \; 3, \; 1)
    \end{align*}

    От свойството на скаларното произведение получаваме:
    \begin{align*}
        (u_2, \; u_1) = \frac{1}{2}.((1, \; 4, \; -1, \; 0), \; u_1) = 0 \implies ((1, \; 4, \; -1, \; 0), \; u_1) = 0 \\\\
        (u_2, \; u_3) = \frac{1}{2}.((1, \; 4, \; -1, \; 0), \; u_1) = 0 \implies ((1, \; 4, \; -1, \; 0), \; u_3) = 0.
    \end{align*}
    Тоест за ортогонален базис на $\mathbb{U}$ можем да изберем, базисът
    \begin{align*}
        d_1 = (1, \; 0, \; 1, \; 0) \\
        d_2 = (1, \; 4, \; -1, \; 0) \\
        d_3 = (3, \; 0, \; 3, \; 1)
    \end{align*}
    \subsubsection*{б)}
    $a = (1, \; 7, \; 2, \; -1)$ \\\\
    $a_0$ е проекцията на вектора $a$ върху $\mathbb{U}$.
    Тогава $a_0 \in \mathbb{U}$. $h$ е перпендикулярът спуснат от $a$ към $\mathbb{U}$.
    Тогава $h \in \mathbb{U}^\perp$. От свойството на ортогоналното допълнение
    $\mathbb{R}^4 = U \oplus \mathbb{U}^\perp$ имаме, че $a = a_0 + h$.
    Тогава $h = a - a_0$, но $a_0 \in \mathbb{U}$ следователно вектора
    $a_0$ има координати спрямо базиса $d_1, \; d_2, \; d_3$. \\\\
    Нека тогава $a_0 = \gamma_1.d_1 + \gamma_2.d_2 + \gamma_3.d_3$.
    Тогава $h = a - \gamma_1.d_1 - \gamma_2.d_2 - \gamma_3.d_3$. \\\\
    $h \in \mathbb{U}^\perp \implies \forall i \in \{1, \; 2, \; 3\} \; (h, d_i) = 0
    \implies (a - \gamma_1.d_1 - \gamma_2.d_2 - \gamma_3.d_3, \; d_i) = \\\\
    = (a, \; d_i) - \displaystyle\sum_{j \in \{1, \; 2, \; 3\}} \gamma_j.(d_j, \; d_i) = 0.$ \\\\
    Тъйкато за $i \neq j \quad (d_j, \; d_i) = 0$, то
    $\forall i \in \{1, \; 2, \; 3\} \quad (h, d_i) = (a, \; d_i) - \gamma_i.(d_i, \; d_i) = 0 \\\\
    \implies \gamma_i = \frac{(a, \; d_i)}{(d_i, \; d_i)}.$
    \begin{align*}
        \gamma_1 = \frac{(a, \; d_1)}{(d_1, \; d_1)} = \frac{1.1 + 7.0 + 2.1 + (-1).0}{2} = \frac{3}{2} \\\\
        \gamma_2 = \frac{(a, \; d_2)}{(d_2, \; d_2)} = \frac{1.1 + 7.4 + 2.(-1) + (-1).0}{9} = \frac{31}{9} \\\\
        \gamma_3 = \frac{(a, \; d_3)}{(d_3, \; d_3)} = \frac{1.3 + 7.0 + 2.3 + (-1).1}{3^2 + 0^2 + 3^2 + 1^2} = \frac{10}{19}.
    \end{align*}
    Тогава $a_0 = \frac{3}{2}.d_1 + \frac{31}{9}.d_2 + \frac{10}{19}.d_3 
    = \frac{1}{342}.(172.d_1 + 18.d_2 + 38.d_3) = \\\\
    = \frac{1}{342}.(172 + 18 + 114, \; 18.3, \; 172 - 18 + 18.3, \; 38) = \\\\
    = \frac{1}{342}.(304, \; 54, \; 208, \; 38)$ и $h = a - a_0 = (1, \; 7, \; 2, \; -1) - \frac{1}{342}.(304, \; 54, \; 208, \; 38)$ 
\end{document}
