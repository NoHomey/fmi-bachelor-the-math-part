\documentclass{article}
\usepackage{amsmath}
\usepackage{amssymb}
\usepackage{pst-node}
\usepackage{stackengine}
\usepackage{fixltx2e}
\usepackage[T1,T2A]{fontenc}
\usepackage[utf8]{inputenc}
\usepackage[bulgarian]{babel}
\usepackage[normalem]{ulem}
\newcommand{\stkout}[1]{\ifmmode\text{\sout{\ensuremath{#1}}}\else\sout{#1}\fi}

\makeatletter
\renewcommand*\env@matrix[1][*\c@MaxMatrixCols c]{%
  \hskip -\arraycolsep
  \let\@ifnextchar\new@ifnextchar
  \array{#1}}
\makeatother

\title{Домашна работа 2, №45342, Martin, 1, I, Информатика}
\author{Иво Стратев}

\begin{document}
    \pagenumbering{gobble}
    \maketitle
    Задача 1.\\
    а) Да се пресметне \(A^{132}\) , където
    \(A =
    \begin{pmatrix}
        12 & 12\\
        0 & 12
    \end{pmatrix} \in M_2(\mathbb{R})\)\\
    Решение:\\
    \(\\A^2 = AA = \begin{pmatrix}
        12 & 12\\
        0 & 12
    \end{pmatrix}
    \begin{pmatrix}
        12 & 12\\
        0 & 12
    \end{pmatrix}
    = \begin{pmatrix}
        12^2 & 2.12^2\\
        0 & 12^2
    \end{pmatrix}\\
    \\
    \\
    A^3 = A^2A =\begin{pmatrix}
        12^2 & 2.12^2\\
        0 & 12^2
    \end{pmatrix}
    \begin{pmatrix}
        12 & 12\\
        0 & 12
    \end{pmatrix}
    = \begin{pmatrix}
        12^3 & 3.12^3\\
        0 & 12^3
    \end{pmatrix}\)\\
    \\ Решение по индукция:\\
    \(\\\\A^n \; ?= \begin{pmatrix}
        12^n & n.12^n\\
        0 & 12^n
    \end{pmatrix}\)\\
    \\ База на индукцията \(n = 1\):\\
    \(\\\\A^1 \; ?= \begin{pmatrix}
        12^1 & 1.12^1\\
        0 & 12^1
    \end{pmatrix}\\
    \\
    \\A^1 \; = \begin{pmatrix}
        12^1 & 1.12^1\\
        0 & 12^1
    \end{pmatrix} = \begin{pmatrix}
        12 & 12\\
        0 & 12
    \end{pmatrix} = A\)\\
    \\ Индуктивно предположение \(n = k\):\\
    \(\\A^k = A^k = \begin{pmatrix}
        12^k & k.12^k\\
        0 & 12^k
    \end{pmatrix}\)\\
    \\ Индуктивно заключение \(n = k + 1\):\\
    \(\\A^{k + 1} \; ?= \begin{pmatrix}
        12^{k + 1} & (k + 1)12^{k + 1}\\
        0 & 12^{k + 1}
    \end{pmatrix}\\
    \\\\A^{k + 1} = A^kA = \begin{pmatrix}
        12^k & k.12^k\\
        0 & 12^k
    \end{pmatrix}
    \begin{pmatrix}
        12 & 12\\
        0 & 12
    \end{pmatrix} = \\
    = \begin{pmatrix}
        12^{k + 1} & k.12^{k + 1} + 12^{k + 1}\\
        0 & 12^{k + 1}
    \end{pmatrix} = \begin{pmatrix}
        12^{k + 1} & (k + 1)12^{k + 1}\\
        0 & 12^{k + 1}
    \end{pmatrix}\\
    \\\\ \implies A^n = \begin{pmatrix}
        12^n & n12^n\\
        0 & 12^n
    \end{pmatrix} \; \forall n \in \mathbb{N}\\
    \\\\ \implies A^{132} = \begin{pmatrix}
        12^{132} & 132.12^{132}\\
        0 & 12^{132}
    \end{pmatrix}\)\\
    \\\\ Отговор:\\
    \(\\  A^{132} = \begin{pmatrix}
        12^{132} & 132.12^{132}\\
        0 & 12^{132}
    \end{pmatrix}\)
    \\\\\\ б) Да се намери матрицата \(B = f(A)\), където
    \(f(x) = -8x^4 + 4x^3 + 3x^2 - 6x + 2\\
    \\ B = -8A^4 + 4A^3 + 3A^2 - 6A + 2E\\
    \\ B = -8\begin{pmatrix}
        12^4 & 4.12^4\\
        0 & 12^4
    \end{pmatrix} + 4 \begin{pmatrix}
        12^3 & 3.12^3\\
        0 & 12^3
    \end{pmatrix} + \\
    \\ 3 \begin{pmatrix}
        12^2 & 2.12^2\\
        0 & 12^2
    \end{pmatrix} - 6 \begin{pmatrix}
        12 & 12\\
        0 & 12
    \end{pmatrix} + 2 \begin{pmatrix}
        1 & 0\\
        0 & 1
    \end{pmatrix} = \\
    \\\\ = \begin{pmatrix}
        -8.12^4 & -32.12^4\\
        0 & -8.12^4
    \end{pmatrix} + \begin{pmatrix}
        4.12^3 & 12^4\\
        0 & 4.12^3
    \end{pmatrix} + \\
    \\ \begin{pmatrix}
        3.12^2 & 6.12^2\\
        0 & 3.12^2
    \end{pmatrix} + \begin{pmatrix}
        -6.12 & -6.12\\
        0 & -6.12
    \end{pmatrix} + \begin{pmatrix}
        2 & 0\\
        0 & 2
    \end{pmatrix} = \\
    \\\\ = \begin{pmatrix}
        4.12^3.(-23) & 12^4.(-31)\\
        0 & 4.12^3.(-23)
    \end{pmatrix} + \begin{pmatrix}
        6.12.5 & 6.12.11\\
        0 & 6.12.5
    \end{pmatrix} + \begin{pmatrix}
        2 & 0\\
        0 & 2
    \end{pmatrix} = \\
    \\\\ = \begin{pmatrix}
        -92.12^3 & -31.12^4\\
        0 & -92.12^3
    \end{pmatrix} + \begin{pmatrix}
        \frac{5}{2}.12^2 & \frac{11}{2}.12^2\\
        0 & \frac{5}{2}.12^2
    \end{pmatrix} + \begin{pmatrix}
        2 & 0\\
        0 & 2
    \end{pmatrix} = \\
    \\\\ = \begin{pmatrix}
        12^2(\frac{5}{2} -92.12) & 12^2(\frac{11}{2} - 31.12^2)\\
        0 & 12^2(\frac{5}{2} -92.12)
    \end{pmatrix} + \begin{pmatrix}
        2 & 0\\
        0 & 2
    \end{pmatrix} = \\
    \\\\ = \begin{pmatrix}
        12^2(\frac{5 -92.24}{2}) & 12^2(\frac{11 - 62.12^2}{2})\\
        0 & 12^2(\frac{5 -92.24}{2})
    \end{pmatrix} + \begin{pmatrix}
        2 & 0\\
        0 & 2
    \end{pmatrix} = \\
    \\\\ = \begin{pmatrix}
        12^2(\frac{5 -2208}{2}) & 12^2(\frac{11 - 8928}{2})\\
        0 & 12^2(\frac{5 -2208}{2})
    \end{pmatrix} + \begin{pmatrix}
        2 & 0\\
        0 & 2
    \end{pmatrix} = \\
    \\\\ = \begin{pmatrix}
        12^2(\frac{-2203}{2}) & 12^2(\frac{-8917}{2})\\
        0 & 12^2(\frac{-2203}{2})
    \end{pmatrix} + \begin{pmatrix}
        2 & 0\\
        0 & 2
    \end{pmatrix} = \\
    \\\\ = \begin{pmatrix}
        -2203.72 & -8917.72\\
        0 & -2203.72
    \end{pmatrix} + \begin{pmatrix}
        2 & 0\\
        0 & 2
    \end{pmatrix} = \\
    \\\\ = \begin{pmatrix}
        -158616 & -642024\\
        0 & -158616
    \end{pmatrix} + \begin{pmatrix}
        2 & 0\\
        0 & 2
    \end{pmatrix} = \\
    \\\\ \begin{pmatrix}
        -158614 & -642024\\
        0 & -158614
    \end{pmatrix}\)\\
    \\\\ Отговор:\\
    \(\\ B = \begin{pmatrix}
        -158614 & -642024\\
        0 & -158614
    \end{pmatrix}\)
    \\\\\\ в) Да се провери дали матриците \(A\) и \(B\) комутират, т. е. дали
    \(AB = BA\\
    \\\\ AB = \begin{pmatrix}
        12 & 12\\
        0 & 12
    \end{pmatrix} \begin{pmatrix}
        -158614 & -642024\\
        0 & -158614
    \end{pmatrix} = \begin{pmatrix}
        -158614.12 & -(642024 + 158614)12\\
        0 & -158614.12
    \end{pmatrix}\\
    \\\\ BA = \begin{pmatrix}
        -158614 & -642024\\
        0 & -158614
    \end{pmatrix} \begin{pmatrix}
        12 & 12\\
        0 & 12
    \end{pmatrix} = \begin{pmatrix}
        -158614.12 & -(642024 + 158614)12\\
        0 & -158614.12
    \end{pmatrix} \\
    \\\\ \implies AB = BA\)\\
    \\ Задача 2.\\
    Да се реши матричното уравнение \(AXB = C\), където \\
    \(\\\\\\ A = \begin{pmatrix}
        1 & 1 & 1\\
        3 & 4 & 4\\
        3 & 6 & 5
    \end{pmatrix}, \quad B = \begin{pmatrix}
        -1 & -2 & -2\\
        2 & 5 & 6\\
        2 & 2 & 1
    \end{pmatrix}, \quad C = \begin{pmatrix}
        -27 & -45 & -45\\
        -102 & -175 & -178\\
        -129 & -229 & -238
    \end{pmatrix}\)\\
    \\\\ Решение:\\
    \(\\ AXB = C \; |B^{-1}\\
    AXBB^{-1} = CB^{-1}\\
    AXE = CB^{-1}\\
    AX = CB^{-1}, \text{ полг. } Y = CB^{-1}\\
    AX = Y \; | ()^t\\
    (AX)^t = Y^t\\
    X^tA^t = Y^t \; | (A^t)^{-1}\\
    X^tA^t(A^t)^{-1} = Y^t(A^t)^{-1}\\
    X^tE = Y^t(A^t)^{-1}\\
    X^t = Y^t(A^t)^{-1}\\
    \\ B^{-1}: (B | E) \to (E | B^{-1})\\
    \\\\ \begin{matrix}
        ~\\
        2\\
        2    
    \end{matrix}
    \begin{pmatrix}[ccc|ccc]
        -1 & -2 & -2 & 1 & 0 & 0\\
         2 &  5 &  6 & 0 & 1 & 0\\
         2 &  2 &  1 & 0 & 0 & 1
    \end{pmatrix} \to \\
    \\\\ \to \begin{matrix}
        2\\
        ~\\
        2    
    \end{matrix}
    \begin{pmatrix}[ccc|ccc]
        -1 & -2 & -2 & 1 & 0 & 0\\
         0 &  1 &  2 & 2 & 1 & 0\\
         0 & -2 & -3 & 2 & 0 & 1
    \end{pmatrix} \to \\
    \\\\ \to \begin{matrix}
        -2\\
        -2\\
        ~   
    \end{matrix}
    \begin{pmatrix}[ccc|ccc]
        -1 & 0 & 2 & 5 & 2 & 0\\
         0 & 1 & 2 & 2 & 1 & 0\\
         0 & 0 & 1 & 6 & 2 & 1
    \end{pmatrix} \to \\
    \\\\ \begin{pmatrix}[ccc|ccc]
        -1 & 0 & 0 & -7  & -2 & -2\\
         0 & 1 & 0 & -10 & -3 & -2\\
         0 & 0 & 1 &   6 &  2 & 1
    \end{pmatrix}
    \begin{matrix}
        -1\\
        ~\\
        ~   
    \end{matrix} \to \\
    \\\\ \begin{pmatrix}[ccc|ccc]
         1 & 0 & 0 &  7  &  2 & 2\\
         0 & 1 & 0 & -10 & -3 & -2\\
         0 & 0 & 1 &   6 &  2 & 1
    \end{pmatrix} \\
    \\\\ B^{-1} = \begin{pmatrix}
         7  &  2 & 2\\
        -10 & -3 & -2\\
          6 &  2 & 1
    \end{pmatrix} \\
    \\\\ CB^{-1} = \begin{pmatrix}
        -27 & -45 & -45\\
        -102 & -175 & -178\\
        -129 & -229 & -238
    \end{pmatrix}
    \begin{pmatrix}
         7  &  2 & 2\\
        -10 & -3 & -2\\
          6 &  2 & 1
    \end{pmatrix} = Y = \begin{pmatrix}
        y_{11} & y_{12} & y_{13}\\
        y_{21} & y_{22} & y_{23}\\
        y_{31} & y_{32} & y_{33}
    \end{pmatrix}\\
    \\\\ y_{11} = 7.-27 + 45.10 -45.6 = 7.-27 + 45(10 - 6) = 9(20 - 21) = -9\\
    y_{12} = 2.-27 + 3.45 - 2.45 = -9.6 + 9.5 = -9\\
    y_{13} = 2.-27 + 2.45 - 45 = -9.6 + 9.5 = -9\\
    y_{21} = 7.-102 + 10.175 + 6.-178 = -714 + 1750 - 1068 = -32\\
    y_{22} = 2.-102 + 3.175 + 2.-178 = -2.280 + 3.175 = 35(15 - 16) = -35\\
    y_{23} = 2.-102 + 2.175 + 1.-178 = 2.73 - 178 = -32\\
    y_{31} = 7.-129 + 10.229 -6.238 = -903 + 2290 - 1428 = -41\\
    y_{32} = 2.-129 + 3.229 + 2.-238 = -2.367 + 3.229 = -734 + 687 = -47\\
    y_{33} = 2.-129 + 2.229 + 1.-238 = 200 -238 = -38\\
    \\\\ CB^{-1} = Y = \begin{pmatrix}
         -9 &  -9 & -9\\
        -32 & -35 & -32\\
        -41 & -47 & -38
    \end{pmatrix} \\
    Y^t = \begin{pmatrix}
        -9 & -32 & -41\\
        -9 & -35 & -47\\
        -9 & -32 & -38
    \end{pmatrix} \\
    \\\\ A^t = \begin{pmatrix}
        1 & 3 & 3\\
        1 & 4 & 6\\
        1 & 4 & 5
    \end{pmatrix}\\
    \\\\ (A^t)^{-1}: (A^t | E) \to (E | (A^t)^{-1})\\
    \\\\ \begin{matrix}
        ~\\
        -\\
        -\\
    \end{matrix}
    \begin{pmatrix}[ccc|ccc]
        1 & 3 & 3 & 1 & 0 & 0\\
        1 & 4 & 6 & 0 & 1 & 0\\
        1 & 4 & 5 & 0 & 0 & 1
    \end{pmatrix} \to \\
    \\\\ \to \begin{matrix}
        -3\\
        ~\\
        1\\
    \end{matrix}
    \begin{pmatrix}[ccc|ccc]
        1 & 3 & 3 &  1 & 0 & 0\\
        0 & 1 & 3 & -1 & 1 & 0\\
        0 & 1 & 2 & -1 & 0 & 1
    \end{pmatrix} \to \\
    \\\\ \to \begin{matrix}
        -6\\
        3\\
        ~\\
    \end{matrix}
    \begin{pmatrix}[ccc|ccc]
        1 & 0 & -6 &  4 & -3 &  0\\
        0 & 1 &  3 & -1 &  1 &  0\\
        0 & 0 & -1 &  0 & -1  & 1
    \end{pmatrix} \to \\
    \\\\ \to \begin{pmatrix}[ccc|ccc]
        1 & 0 &  0 &  4 &  3 & -6\\
        0 & 1 &  0 & -1 & -2 &  3\\
        0 & 0 & -1 &  0 & -1 &  1
    \end{pmatrix}
    \begin{matrix}
        ~\\
        ~\\
        -1
    \end{matrix} \to \\
    \\\\ \begin{pmatrix}[ccc|ccc]
        1 & 0 & 0 &  4 &  3 & -6\\
        0 & 1 & 0 & -1 & -2 &  3\\
        0 & 0 & 1 &  0 &  1 & -1
    \end{pmatrix}\\
    \\\\ (A^t)^{-1} = \begin{pmatrix}
         4 & 3 & -6\\
        -1 & -2 &  3\\
         0 & 1 & -1
    \end{pmatrix} \\
    \\\\ X^t = Y^t(A^t)^{-1}\\
    \\\\ X^t = \begin{pmatrix}
        -9 & -32 & -41\\
        -9 & -35 & -47\\
        -9 & -32 & -38
    \end{pmatrix}
    \begin{pmatrix}
         4 & 3 & -6\\
        -1 & -2 &  3\\
         0 & 1 & -1
    \end{pmatrix} = \begin{pmatrix}
        x_{11} & x_{12} & x_{13}\\
        x_{21} & x_{22} & x_{23}\\
        x_{31} & x_{32} & x_{33}
    \end{pmatrix}\\
    \\\\ x_{11} = -9.4 + 32 + 0 = -36 + 32 = -4\\
    x_{12} = -9.3 + 32.2 - 41 = 64 - 68 = -4\\
    x_{13} = 96 - 3.32 + 41 = 59 + 41 - 96 = -1\\
    x_{21} = -9.4 + 35 + 0 = -36 + 35 = -1\\
    x_{22} = -3.9 + 2.35 - 47 = -27 - 47 + 70 = -74 + 70 = -4\\
    x_{23} = 9.6 - 3.35 + 47 = 54 + 47 - 105 = 101 - 105 = -4\\
    x_{31} = -9.4 + 32 + 0 = -36 + 32 = -4\\
    x_{32} = -3.9 + 64 - 38 = 64 - 27 -38 = 64 - 65 = -1\\
    x_{33} = 9.6 - 3.32 + 38 = 54 + 38 - 96 = 92 - 96 = -4\\
    \\\\ X^t = \begin{pmatrix}
        -4 -4 -1\\
        -1 -4 -4\\
        -4 -1 -4
    \end{pmatrix}\\
    \\\\ (X^t)^t = X = \begin{pmatrix}
        -4 -1 -4\\
        -4 -4 -1\\
        -1 -4 -4
    \end{pmatrix}\)\\\\
    \\\\ Отговор:\\
    \\ \(X = \begin{pmatrix}
        -4 -1 -4\\
        -4 -4 -1\\
        -1 -4 -4
    \end{pmatrix}\)\\
    \\\\\\\\ Задача 3.\\
    Да се намери рангът на матрицата:\\
    \(\\\\ \begin{pmatrix}
        1 & 3 & -1 & 3 & 1\\
        0 & -7 & -8 & -3 & -4\\
        4 & 20 & 7 & 15 & -3\lambda + \mu + 9\\
        -1 & -6 & -3 & -4 & -3\\
        2 & 11 & \lambda + 5 & 8 & 5\\
        -1 & -1 & 3 & -2 & 0
    \end{pmatrix}\)
    \\\\\\\\ Решение:\\
    \(\\\\ \begin{pmatrix}
        1 & 3 & -1 & 3 & 1\\
        0 & -7 & -8 & -3 & -4\\
        4 & 20 & 7 & 15 & -3\lambda + \mu + 9\\
        -1 & -6 & -3 & -4 & -3\\
        2 & 11 & \lambda + 5 & 8 & 5\\
        -1 & -1 & 3 & -2 & 0
    \end{pmatrix}
    \begin{matrix}
        ~\\
        -1\\
        ~\\
        -1\\
        ~\\
        ~
    \end{matrix} \to \\
    \\\\ \to \begin{matrix}
        ~\\
        ~\\
        -4\\
        -1\\
        -2\\
        1
    \end{matrix}
    \begin{pmatrix}
        1 & 3 & -1 & 3 & 1\\
        0 & 7 & 8 & 3 & 4\\
        4 & 20 & 7 & 15 & -3\lambda + \mu + 9\\
        1 & 6 & 3 & 4 & 3\\
        2 & 11 & \lambda + 5 & 8 & 5\\
        -1 & -1 & 3 & -2 & 0
    \end{pmatrix} \to \\
    \\\\ \to \begin{matrix}
        -3\\
        -3\\
        -3\\
        -1\\
        -2\\
        ~
    \end{matrix}
    \begin{pmatrix}
        1 & 3 & -1 & 3 & 1\\
        0 & 7 & 8 & 3 & 4\\
        0 & 8 & 11 & 3 & -3\lambda + \mu + 5\\
        0 & 3 & 4 & 1 & 2\\
        0 & 5 & \lambda + 7 & 2 & 3\\
        0 & 2 & 2 & 1 & 1
    \end{pmatrix} \to \\
    \\\\ \to \begin{pmatrix}
        1 & -3 & -7 & 0 & -2\\
        0 & 1 & 2 & 0 & 1\\
        0 & 2 & 5 & 0 & -3\lambda + \mu + 2\\
        0 & 1 & 2 & 0 & 1\\
        0 & 1 & \lambda + 3 & 0 & 1\\
        0 & 2 & 2 & 1 & 1
    \end{pmatrix} \to \\
    \\\\ \to \begin{matrix}
        3\\
        ~\\
        -2\\
        -1\\
        -2
    \end{matrix}
    \begin{pmatrix}
        1 & -3 & -7 & 0 & -2\\
        0 & 1 & 2 & 0 & 1\\
        0 & 2 & 5 & 0 & -3\lambda + \mu + 2\\
        0 & 1 & \lambda + 3 & 0 & 1\\
        0 & 2 & 2 & 1 & 1
    \end{pmatrix} \to \\
    \\\\ \to \begin{pmatrix}
        1 & 0 & -1 & 0 & 1\\
        0 & 1 & 2 & 0 & 1\\
        0 & 0 & 1 & 0 & -3\lambda + \mu\\
        0 & 0 & \lambda + 1 & 0 & 0\\
        0 & 0 & -2 & 1 & -1
    \end{pmatrix}\\
    \\\\ \lambda = -1 \implies r = 4\\
    \lambda \neq -1, \; \mu = 3\lambda \implies r = 4\\
    \lambda \neq -1, \; \mu \neq 3\lambda \implies r = 5\)\\
    \\\\ Отговор:\\
    \(\lambda = -1, \; r = 4\\
    \lambda \neq -1, \; \mu = 3\lambda, \; r = 4\\
    \lambda \neq -1, \; \mu \neq 3\lambda, \; r = 5\)\\
    \\ Задача 4.\\
    Да се намери обратната матрица на матрицата от \(n^{\text{ти}}\) ред, \(n \in \mathbb{N}: \\
    \\\\ \begin{pmatrix}[ccccc|ccccc]
        2 & 9 & 9 & \dots & 9 & 1 & 0 & 0 & \dots  & 0\\
        9 & 2 & 9 & \dots & 9 & 0 & 1 & 0 & \dots  & 0\\
        9 & 9 & 2 & \dots & 9 & 0 & 0 & 1 & \dots  & 0\\
        ~ & ~ & \dots & ~ & ~ & ~ & ~ & \dots & ~  & ~\\
        9 & 9 & 9 & \dots & 2 & 0 & 0 & 0 & \dots  & 1\\
    \end{pmatrix}\)\\
    \\\\ Решение:\\
    \(\\\\ \begin{matrix}
        ~\\
        -\\
        -\\
        -\\
        -
    \end{matrix} \begin{pmatrix}[ccccc|ccccc]
        2 & 9 & 9 & \dots & 9 & 1 & 0 & 0 & \dots  & 0\\
        9 & 2 & 9 & \dots & 9 & 0 & 1 & 0 & \dots  & 0\\
        9 & 9 & 2 & \dots & 9 & 0 & 0 & 1 & \dots  & 0\\
        ~ & ~ & \dots & ~ & ~ & ~ & ~ & \dots & ~  & ~\\
        9 & 9 & 9 & \dots & 2 & 0 & 0 & 0 & \dots  & 1\\
    \end{pmatrix} \to \\
    \\\\ \to \begin{pmatrix}[ccccc|ccccc]
        2 & 9 & 9 & \dots & 9 & 1 & 0 & 0 & \dots   & 0\\
        7 & -7 & 0 & \dots & 0 & -1 & 1 & 0 & \dots & 0\\
        7 & 0 & -7 & \dots & 0 & -1 & 0 & 1 & \dots & 0\\
        ~ & ~ & \dots & ~ & ~ & ~ & ~ & \dots & ~   & ~\\
        7 & 0 & 0 & \dots & -7 & -1 & 0 & 0 & \dots & 1\\
    \end{pmatrix}
    \begin{matrix}
        ~\\
        -\frac{1}{7}\\
        -\frac{1}{7}\\
        -\frac{1}{7}\\
        -\frac{1}{7}
    \end{matrix} \to \\
    \\\\ \to \begin{matrix}
        -9\\
        ~\\
        ~\\
        ~\\
        ~
    \end{matrix} \begin{pmatrix}[ccccc|ccccc]
        2 & 9 & 9 & \dots & 9 & 1 & 0 & 0 & \dots   & 0\\
        -1 & 1 & 0 & \dots & 0 & \frac{1}{7} & -\frac{1}{7} & 0 & \dots & 0\\
        -1 & 0 & 1 & \dots & 0 & \frac{1}{7} & 0 & -\frac{1}{7} & \dots & 0\\
        ~ & ~ & \dots & ~ & ~ & ~ & ~ & \dots & ~   & ~\\
        -1 & 0 & 0 & \dots & 1 & \frac{1}{7} & 0 & 0 & \dots & -\frac{1}{7}\\
    \end{pmatrix} \to \\
    \\\\ \to \begin{matrix}
        -9\\
        ~\\
        ~\\
        ~\\
        ~
    \end{matrix} \begin{pmatrix}[ccccc|ccccc]
        2 + 9 & 0 & 9 & \dots & 9 & 1 - \frac{9}{7} & \frac{9}{7} & 0 & \dots   & 0\\
        -1 & 1 & 0 & \dots & 0 & \frac{1}{7} & -\frac{1}{7} & 0 & \dots & 0\\
        -1 & 0 & 1 & \dots & 0 & \frac{1}{7} & 0 & -\frac{1}{7} & \dots & 0\\
        ~ & ~ & \dots & ~ & ~ & ~ & ~ & \dots & ~   & ~\\
        -1 & 0 & 0 & \dots & 1 & \frac{1}{7} & 0 & 0 & \dots & -\frac{1}{7}\\
    \end{pmatrix} \to \\
    \\\\ \to \begin{matrix}
        -9\\
        ~\\
        ~\\
        ~\\
        ~
    \end{matrix} \begin{pmatrix}[ccccc|ccccc]
        2 + 2.9 & 0 & 0 & \dots & 9 & 1 - 2.\frac{9}{7} & \frac{9}{7} & \frac{9}{7} & \dots   & 0\\
        -1 & 1 & 0 & \dots & 0 & \frac{1}{7} & -\frac{1}{7} & 0 & \dots & 0\\
        -1 & 0 & 1 & \dots & 0 & \frac{1}{7} & 0 & -\frac{1}{7} & \dots & 0\\
        ~ & ~ & \dots & ~ & ~ & ~ & ~ & \dots & ~   & ~\\
        -1 & 0 & 0 & \dots & 1 & \frac{1}{7} & 0 & 0 & \dots & -\frac{1}{7}\\
    \end{pmatrix} \to \\
    \\\\ \dots \\
    \\\\ \to \begin{matrix}
        -9\\
        ~\\
        ~\\
        ~\\
        ~
    \end{matrix} \begin{pmatrix}[ccccc|ccccc]
        2 + (n - 2).9 & 0 & 0 & \dots & 9 & 1 - (n - 2).\frac{9}{7} & \frac{9}{7} & \frac{9}{7} & \dots   & 0\\
        -1 & 1 & 0 & \dots & 0 & \frac{1}{7} & -\frac{1}{7} & 0 & \dots & 0\\
        -1 & 0 & 1 & \dots & 0 & \frac{1}{7} & 0 & -\frac{1}{7} & \dots & 0\\
        ~ & ~ & \dots & ~ & ~ & ~ & ~ & \dots & ~   & ~\\
        -1 & 0 & 0 & \dots & 1 & \frac{1}{7} & 0 & 0 & \dots & -\frac{1}{7}\\
    \end{pmatrix} \to \\
    \\\\ \to \begin{pmatrix}[ccccc|ccccc]
        2 + (n - 1).9 & 0 & 0 & \dots & 0 & 1 - (n - 1).\frac{9}{7} & \frac{9}{7} & \frac{9}{7} & \dots & \frac{9}{7}\\
        -1 & 1 & 0 & \dots & 0 & \frac{1}{7} & -\frac{1}{7} & 0 & \dots & 0\\
        -1 & 0 & 1 & \dots & 0 & \frac{1}{7} & 0 & -\frac{1}{7} & \dots & 0\\
        ~ & ~ & \dots & ~ & ~ & ~ & ~ & \dots & ~   & ~\\
        -1 & 0 & 0 & \dots & 1 & \frac{1}{7} & 0 & 0 & \dots & -\frac{1}{7}\\
    \end{pmatrix} \begin{matrix}
        \frac{1}{2 + (n - 1).9}\\
        ~\\
        ~\\
        ~\\
        ~
    \end{matrix} \to \\
    \\\\ \to \begin{pmatrix}[ccccc|ccccc]
        1 & 0 & 0 & \dots & 0 & \frac{1 - (n - 1).\frac{9}{7}}{2 + (n - 1).9} & \frac{9}{7} . \frac{1}{2 + (n - 1).9} & \frac{9}{7} . \frac{1}{2 + (n - 1).9} & \dots & \frac{9}{7} . \frac{1}{2 + (n - 1).9}\\
        -1 & 1 & 0 & \dots & 0 & \frac{1}{7} & -\frac{1}{7} & 0 & \dots & 0\\
        -1 & 0 & 1 & \dots & 0 & \frac{1}{7} & 0 & -\frac{1}{7} & \dots & 0\\
        ~ & ~ & \dots & ~ & ~ & ~ & ~ & \dots & ~   & ~\\
        -1 & 0 & 0 & \dots & 1 & \frac{1}{7} & 0 & 0 & \dots & -\frac{1}{7}\\
    \end{pmatrix} \to \\
    \\\\ 1 - (n - 1).\frac{9}{7} = 1 - n.\frac{9}{7} + \frac{9}{7} = \frac{16}{7} - \frac{9}{7}.n = \frac{16 - 9.n}{7}\\
    2 + (n - 1).9 = 2 + 9.n - 9 = 9.n - 7\\
    \\\\ \to \begin{matrix}
        ~\\
        +\\
        +\\
        +\\
        +
    \end{matrix} \begin{pmatrix}[ccccc|ccccc]
        1 & 0 & 0 & \dots & 0 & \frac{16 - 9.n}{7.(9n - 7)} & \frac{9}{7} . \frac{1}{9n - 7} & \frac{9}{7} . \frac{1}{9n - 7} & \dots & \frac{9}{7} . \frac{1}{9n - 7}\\
        0 & 1 & 0 & \dots & 0 & \frac{1}{7} + \frac{16 - 9.n}{7.(9n - 7)} &  \frac{9}{7} . \frac{1}{9n - 7}  -\frac{1}{7} & \frac{9}{7} . \frac{1}{9n - 7}  & \dots & \frac{9}{7} . \frac{1}{9n - 7} \\
        0 & 0 & 1 & \dots & 0 & \frac{1}{7} + \frac{16 - 9.n}{7.(9n - 7)} & \frac{9}{7} . \frac{1}{9n - 7}  & \frac{9}{7} . \frac{1}{9n - 7}  -\frac{1}{7} & \dots & \frac{9}{7} . \frac{1}{9n - 7} \\
        ~ & ~ & \dots & ~ & ~ & ~ & ~ & \dots & ~   & ~\\
        0 & 0 & 0 & \dots & 1 & \frac{1}{7} + \frac{16 - 9.n}{7.(9n - 7)} & \frac{9}{7} . \frac{1}{9n - 7}  & \frac{9}{7} . \frac{1}{9n - 7}  & \dots & \frac{9}{7} . \frac{1}{9n - 7}  -\frac{1}{7}\\
    \end{pmatrix} \to \\
    \\\\ \frac{1}{7} + \frac{16 - 9.n}{7.(9n - 7)} = \frac{\stkout{9n} - 7 + 16 \stkout{-9n}}{7.(9n - 7)} = \frac{9}{7.(9n - 7)} = \frac{9}{7} . \frac{1}{9n - 7}\\
    \frac{9}{7} . \frac{1}{9n - 7}  -\frac{1}{7} = \frac{9 - 9n + 7}{7.(9n - 7)} = \frac{16 - 9.n}{7.(9n - 7)} = \frac{16 - 9.n}{9} . \frac{9}{7.(9n - 7)}\\
    \\ \text{Полг. } s = \frac{9}{7.(9n - 7)}, \; p = \frac{16 - 9.n}{9} \\
    \\ \to \begin{pmatrix}[ccccc|ccccc]
        1 & 0 & 0 & \dots & 0 & s.p & s & s & \dots & s\\
        0 & 1 & 0 & \dots & 0 & s & s.p & s & \dots & s\\
        0 & 0 & 1 & \dots & 0 & s & s & s.p & \dots & s\\
        ~ & ~ & \dots & ~ & ~ & ~ & ~ & \dots & ~   & ~\\
        0 & 0 & 0 & \dots & 1 & s & s & s & \dots & s.p\\
    \end{pmatrix}\)\\
    \\\\ Отговор:\\
    \(\\ \begin{pmatrix}
        s.p & s & s & \dots & s\\
        s & s.p & s & \dots & s\\
        s & s & s.p & \dots & s\\
        ~ & ~ & \dots & ~   & ~\\
        s & s & s & \dots & s.p\\
    \end{pmatrix} \; , \; \text{където: } \; s = \frac{9}{7.(9n - 7)}, \; p = \frac{16 - 9.n}{9}\)\\
    \\\\ Задача 5.\\
    \(a_1 = (8, \; -3, \; -1, \; 1)\\
    a_2 = (4, \; 0, \; -20, \; 8)\\
    a_3 = (-8, \; 3, \; 1, \; -1)\\
    a_3 = (-3, \; 1, \; 2, \; -1)\\
    \\\\ \begin{pmatrix}
        8 & -3 & -1 & 1\\
        4 & 0 & -20 & 8\\
        -8 & 3 & 1 & -1\\
        -3 & 1 & 2 & -1
    \end{pmatrix} \to \\
    \\\\ \to \begin{pmatrix}
        8 & -3 & -1 & 1\\
        4 & 0 & -20 & 8\\
        0 & 0 & 0 & 0\\
        -3 & 1 & 2 & -1
    \end{pmatrix} \to \\
    \\\\ \to  \begin{pmatrix}
        8 & -3 & -1 & 1\\
        4 & 0 & -20 & 8\\
        -3 & 1 & 2 & -1
    \end{pmatrix} \begin{matrix}
        ~\
        \frac{1}{4}\\
        ~
    \end{matrix} \to \\
    \\\\ \to \begin{matrix}
        -8\\
        ~\\
        3
    \end{matrix} \begin{pmatrix}
        8 & -3 & -1 & 1\\
        1 & 0 & -5 & 2\\
        -3 & 1 & 2 & -1
    \end{pmatrix} \to \\
    \\\\ \to \begin{pmatrix}
        0 & -3 & -39 & -15\\
        1 & 0 & -5 & 2\\
        0 & 1 & -13 & 5
    \end{pmatrix} \to \\
    \\\\ \to \begin{pmatrix}
        0 & 0 & 0 & 0\\
        1 & 0 & -5 & 2\\
        0 & 1 & -13 & 5
    \end{pmatrix} \to \\
    \\\\ \to \begin{pmatrix}
        1 & 0 & -5 & 2\\
        0 & 1 & -13 & 5
    \end{pmatrix}\\
    \\u_1 = (1, \; 0, \; -5, \; 2)\\
    u_2 = (0, \; 1, \; -13, \; 5)\\
    u_1, u_2 \text{ - базис на } \mathbb{U}\\
    \\\\ \mathbb{U}: \begin{array}{lcl}
        5x_1 + 13x_2 + x_3 & = & 0\\
        -2x_1 - 5x_2 + x_4 & = & 0\\ 
    \end{array}\)\\
    \\\\ Задача 6.\\
    Нека векторите \(e_1, \; e_2, \; e_3\) са базис на линейното пространство \(\mathbb{V}\) и\\
    \(\\ \begin{array}{lcl}
        a_1 & = & 5e_3 - e_1\\
        a_2 & = & e_1 - e_2 - 3e_3\\
        a_3 & = & e_3
    \end{array} \begin{array}{lcl}
        b_1 & = & e_1 - e_3\\
        b_2 & = & -e_2\\
        b_3 & = & e_3 - 3e_2
    \end{array} \begin{array}{lcl}
        c_1 & = & -2e_1 - 3e_2 - 3e_3\\
        c_2 & = & 2e_2 + e_3\\
        c_3 & = & e_1 + 8e_2 + 5e_3
    \end{array}\)\\
    \\а) Да се докаже, че векторите \(a_1, \; a_2, \; a_3\) и \(b_1, \; b_2, \; b_3\) също образуват базис на \(\mathbb{V}\)\\
    \\ Решение:\\
    \(\\ \lambda_1 a_1 + \lambda_2 a_2 + \lambda_3 a_3 = 0\\
    \lambda_1 (5e_3 - e_1) + \lambda_2 (e_1 - e_2 - 3e_3) + \lambda_3 e_3 = 0\\
    5\lambda_1 e_3 - \lambda_1 e_1 + \lambda_2 e_2 - \lambda_2 e_2 - 3\lambda_2 e_3 + \lambda_3 e_3 = 0\\
    (\lambda_2 - \lambda_1) e_1 -\lambda_2 e_2 + (5\lambda_1 -3\lambda_2 + \lambda_3) e_3 = 0\\
    \\\\ \begin{array}{lcl}
        \lambda_2 - \lambda_1 & = & 0\\
        -\lambda_2 & = & 0\\
        5\lambda_1 -3\lambda_2 + \lambda_3 & = & 0
    \end{array} \implies \begin{pmatrix}
        -1 & 1 & 0\\
        0 & -1 & 0\\
        5 & -3 & 1
    \end{pmatrix}\\
    \\\\ \begin{pmatrix}
        -1 & 1 & 0\\
        0 & -1 & 0\\
        5 & -3 & 1
    \end{pmatrix} \begin{matrix}
        -1\\
        -1\\
        ~
    \end{matrix} \to \\
    \\\\ \to \begin{matrix}
        ~\\
        ~\\
        -5
    \end{matrix} \begin{pmatrix}
        1 & -1 & 0\\
        0 & 1 & 0\\
        5 & -3 & 1
    \end{pmatrix} \to \\
    \\\\ \to \begin{matrix}
        1\\
        ~\\
        -2
    \end{matrix} \begin{pmatrix}
        1 & -1 & 0\\
        0 & 1 & 0\\
        0 & 2 & 1
    \end{pmatrix} \to \\
    \\\\ \to  \begin{pmatrix}
        1 & 0 & 0\\
        0 & 1 & 0\\
        0 & 0 & 1
    \end{pmatrix} \implies (\lambda_1, \; \lambda_2, \; \lambda_3) = (0, \; 0, \; 0)\\
    \\ \implies a_1, \; a_2, \; a_3 \text{ са базис на } \mathbb{V}
    \\\\ \mu_1 b_1 + \mu_2 b_2 + \mu_3 b_3 = 0\\
    \mu_1 (e_1 - e_3) + \mu_2(-e_2) + \mu_3 (e_3 - 3e_2) = 0\\
    \mu_1 e_1 - \mu_1 e_3 - \mu_2 e_2 - \mu_3 e_3 - 3\mu_3 e_2\\
    \mu_1 e_1 -(\mu_2 + 3\mu_3) e_2 + (\mu_3 - \mu_1) e_3 = 0\\
    \\\\ \begin{array}{lcl}
        \mu_1 & = & 0\\
        -\mu_2 - 3\mu_3 & = & 0\\
        \mu_3 - \mu_1 & = & 0
    \end{array} \implies \begin{pmatrix}
        1 & 0 & 0\\
        0 & -1 & -3\\
        -1 & 0 & 1
    \end{pmatrix}\\
    \\\\ \begin{pmatrix}
        1 & 0 & 0\\
        0 & -1 & -3\\
        -1 & 0 & 1
    \end{pmatrix} \begin{matrix}
        ~\\
        -1\\
        ~
    \end{matrix} \to \\
    \\\\ \to \begin{matrix}
        ~\\
        ~\\
        1
    \end{matrix} \begin{pmatrix}
        1 & 0 & 0\\
        0 & 1 & 3\\
        -1 & 0 & 1
    \end{pmatrix} \to \\
    \\\\ \to \begin{matrix}
        ~\\
        -3\\
        ~
    \end{matrix} \begin{pmatrix}
        1 & 0 & 0\\
        0 & 1 & 3\\
        0 & 0 & 1
    \end{pmatrix} \to \\
    \\\\ \to  \begin{pmatrix}
        1 & 0 & 0\\
        0 & 1 & 0\\
        0 & 0 & 1
    \end{pmatrix} \implies (\mu_1, \; \mu_2, \; \mu_3) = (0, \; 0, \; 0)\\
    \\ \implies b_1, \; b_2, \; b_3 \text{ са базис на } \mathbb{V}\)\\
    \\б) Да се докаже, че \(\exists! \varphi \in Hom\mathbb{V}; \; \varphi(b_i) = c_i, \; i = 1, \; 2, \; 3\)\\
    \\ Решение:\\
    \(\\ B = \begin{pmatrix}
        1 & 0 & 0\\
        0 & -1 & -3\\
        -1 & 0 & 1\\
    \end{pmatrix} \in M_3(\mathbb{R})\\
    \\\\ C = \begin{pmatrix}
        -2 & 0 & 1\\
        -3 & 2 & 8\\
        -3 & 1 & 5\\
    \end{pmatrix} \in M_3(\mathbb{R})\\
    \\\\ \text{Нека } \tau, \psi \in Hom\mathbb{V};\\
    \quad Me(\tau) = B, \; \tau(e_i) = b_i, \; i = 1, \; 2, \; 3\\
    \quad Me(\psi) = C, \; \psi(e_i) = c_i, \; i = 1, \; 2, \; 3\\
    \\ \text{Допс. } \exists \varphi \in Hom\mathbb{V}; \; \varphi(b_i) = c_i, \; i = 1, \; 2, \; 3\\
    \implies (\varphi\tau)(e_i) = (\varphi\ \circ \tau)(e_i) = c_i = \psi(e_i), \; i = 1, \; 2, \; 3\\
    \implies \varphi \tau = \psi\\
    \mathbb{V} = l(e_1, \; e_2, \; e_3) \implies \dim \mathbb{V} = 3\\
    b_1, \; b_2, \; b_3 \text{ e базис на } \mathbb{V} \implies r(B) = 3 \implies \exists B^{-1}\\
    \\ B^{-1}: \\
    \\\\ \begin{pmatrix}[ccc|ccc]
        1 & 0 & 0 & 1 & 0 & 0\\
        0 & -1 & -3 & 0 & 1 & 0\\
        -1 & 0 & 1 & 0 & 0 & 1
    \end{pmatrix} \begin{matrix}
        ~\\
        -1\\
        ~
    \end{matrix} \to \\
    \\\\ \to \begin{matrix}
        ~\\
        ~\\
        1
    \end{matrix} \begin{pmatrix}[ccc|ccc]
        1 & 0 & 0 & 1 & 0 & 0\\
        0 & 1 & 3 & 0 & -1 & 0\\
        -1 & 0 & 1 & 0 & 0 & 1
    \end{pmatrix} \to \\
    \\\\ \to \begin{matrix}
        ~\\
        -3\\
        ~
    \end{matrix} \begin{pmatrix}[ccc|ccc]
        1 & 0 & 0 & 1 & 0 & 0\\
        0 & 1 & 3 & 0 & -1 & 0\\
        0 & 0 & 1 & 1 & 0 & 1
    \end{pmatrix} \to \\
    \\\\ \to \begin{pmatrix}[ccc|ccc]
        1 & 0 & 0 & 1 & 0 & 0\\
        0 & 1 & 0 & -3 & -1 & -3\\
        0 & 0 & 1 & 1 & 0 & 1
    \end{pmatrix} \\
    \\\\ \implies B^{-1} = \begin{pmatrix}
        1 & 0 & 0\\
        -3 & -1 & -3\\
        1 & 0 & 1
    \end{pmatrix}\\
    \\\\ \text{Нека } \tau' \to B^{-1}, \; \tau \tau' \to B B^{-1} = E, \; \tau' \tau \to B^{-1}B = E\\
    \implies \tau \tau' = \tau' \tau = \varepsilon \\
    \implies \exists! \tau^{-1}; \; \tau^{-1} = \tau', ; \; \tau^{-1}(b_i) = e_i , \; i = 1, \; 2, \; 3\\
    \\ \varphi \tau = \psi \; | \tau^{-1}\\
    \varphi \tau \tau^{-1} = \psi \tau^{-1}\\
    \varphi \varepsilon =  \psi \tau^{-1}\\
    \varphi =  \psi \tau^{-1}, \; \psi \to C, \; \tau^{-1} \to B^{-1}\\
    \implies \varphi \to C B^{-1}\\
    \exists! B^{-1}, \implies \exists! F = C B^{-1} \in M_3(\mathbb{R})\\
    \implies \exists! \varphi \in Hom\mathbb{V}; \; \varphi(b_i) = c_i, \; i = 1, \; 2, \; 3, \; \varphi \to C B^{-1}\)\\
    \\в) Да се намери матрицата на \(\varphi\) в базиса \(e_1, \; e_2, \; e_3\)\\
    \\ Решение:\\
    \(M_e(\varphi) \to C B^{-1}\\
    \\\\C B^{-1} = \begin{pmatrix}
        -2 & 0 & 1\\
        -3 & 2 & 8\\
        -3 & 1 & 5\\
    \end{pmatrix} \begin{pmatrix}
        1 & 0 & 0\\
        -3 & -1 & -3\\
        1 & 0 & 1
    \end{pmatrix} = \begin{pmatrix}
        -1 & 0 & 1\\
        -1 & -2 & 2\\
        -1 & -1 & 2
    \end{pmatrix}\)\\
    \\г) Да се намери матрицата на \(\varphi\) в базиса \(a_1, \; a_2, \; a_3\)\\
    \\ Решение:\\
    \(\\A = \begin{pmatrix}
        -1 & 1 & 0\\
        0 & -1 & 0\\
        5 & -3 & 1
    \end{pmatrix}\\
    \\\\ \text{Нека } \chi \in Hom\mathbb{V};\\\quad Me(\chi) = A, \; \chi(e_i) = a_i, \; i = 1, \; 2, \; 3\\
    \mathbb{V} = l(e_1, \; e_2, \; e_3) \implies \dim \mathbb{V} = 3\\
    a_1, \; a_2, \; a_3 \text{ e базис на } \mathbb{V} \implies r(A) = 3 \implies \exists A^{-1} \in M_3(\mathbb{R})\\
    \\\\A^{-1}:\\
    \begin{matrix}
        ~\\
        ~\\
        5
    \end{matrix} \begin{pmatrix}[ccc|ccc]
        -1 & 1 & 0 & 1 & 0 & 0\\
        0 & -1 & 0 & 0 & 1 & 0\\
        5 & -3 & 1 & 0 & 0 & 1
    \end{pmatrix} \to \\
    \\\\ \to \begin{matrix}
        1\\
        ~\\
        2
    \end{matrix} \begin{pmatrix}[ccc|ccc]
        -1 & 1 & 0 & 1 & 0 & 0\\
        0 & -1 & 0 & 0 & 1 & 0\\
        0 & 2 & 1 & 5 & 0 & 1
    \end{pmatrix} \to \\
    \\\\ \begin{pmatrix}[ccc|ccc]
        -1 & 0 & 0 & 1 & 1 & 0\\
        0 & -1 & 0 & 0 & 1 & 0\\
        0 & 0 & 1 & 5 & 2 & 1
    \end{pmatrix} \begin{matrix}
        -1\\
        -1\\
        ~
    \end{matrix} \to \\
    \\\\ \to \begin{pmatrix}[ccc|ccc]
        1 & 0 & 0 & -1 & -1 & 0\\
        0 & 1 & 0 & 0 & -1 & 0\\
        0 & 0 & 1 & 5 & 2 & 1
    \end{pmatrix}\\
    \\\\ A^{-1} = \begin{pmatrix}
        -1 & -1 & 0\\
        0 & -1 & 0\\
        5 & 2 & 1
    \end{pmatrix}\\
    \\\\ \text{Нека } \chi' \to A^{-1}, \; \chi \chi' \to A A^{-1} = E, \; \chi' \chi \to A^{-1}A = E\\
    \implies \chi \chi' = \chi' \chi = \varepsilon \\
    \implies \exists! \chi^{-1}; \; \chi^{-1} = \chi', ; \; \chi^{-1}(a_i) = e_i , \; i = 1, \; 2, \; 3\\
    \\\\ \varphi(a_i) = \lambda_{1i}a_1 + \lambda_{2i}a_2 + \lambda_{3i}a_3 \; | \chi^{-1}, \; i = 1, \; 2, \; 3\\
    \chi^{-1}\varphi(a_i) = \lambda_{1i}\chi^{-1}(a_1) + \lambda_{2i}\chi^{-1}(a_2) + \lambda_{3i}\chi^{-1}(a_3), \; i = 1, \; 2, \; 3\\
    \chi^{-1}\varphi(a_i) = \lambda_{1i}e_1 + \lambda_{2i}e_2 + \lambda_{3i}e_3 \; i = 1, \; 2, \; 3\\
    a_i = \chi(e_i), \; i = 1, \; 2, \; 3\\
    \chi^{-1}\varphi(\chi(e_i)) = \lambda_{1i}e_1 + \lambda_{2i}e_2 + \lambda_{3i}e_3 \; i = 1, \; 2, \; 3\\
    (\chi^{-1}\varphi\chi)(e_i) = \lambda_{1i}e_1 + \lambda_{2i}e_2 + \lambda_{3i}e_3 \; i = 1, \; 2, \; 3\\
    \implies (\chi^{-1}\varphi\chi \to A^{-1}M_e(\varphi)A = M_a(\varphi)\\
    \\M_a(\varphi) = A^{-1}M_e(\varphi)A = \begin{pmatrix}
        -1 & -1 & 0\\
        0 & -1 & 0\\
        5 & 2 & 1
    \end{pmatrix} \begin{pmatrix}
        -1 & 0 & 1\\
        -1 & -2 & 2\\
        -1 & -1 & 2
    \end{pmatrix} \begin{pmatrix}
        -1 & 1 & 0\\
        0 & -1 & 0\\
        5 & -3 & 1
    \end{pmatrix} = \\
    \\\\ = \begin{pmatrix}
        -1 & 1 & 0\\
        0 & -1 & 0\\
        5 & 2 & 1
    \end{pmatrix} \begin{pmatrix}
        6 & -4 & 1\\
        11 & -5 & 2\\
        11 & -6 & 2
    \end{pmatrix} = \begin{pmatrix}
        -17 & 9 & -3\\
        -11 & 5 & -2\\
        63 & -36 & 11
    \end{pmatrix}\)\\
    \\\\ Задача 7.\\
    Нека \(\mathbb{V} = M_2(\mathbb{F})\). Дадени са изображенията:\\
    \\а) \(\varphi(X) = \begin{pmatrix}
        1 & 3\\
        2 & 5
    \end{pmatrix}X + X\begin{pmatrix}
        -1 & 1\\
        -4 & 3
    \end{pmatrix}, \; X \in \mathbb{V}\)\\
    \\ Решение:\\
    \(\varphi \in Hom\mathbb{V} \iff \begin{matrix}
        \varphi(A + B) = \varphi(A) + \varphi(B) \; \forall A, B \in \mathbb{V}\\
        \varphi(\lambda A) = \lambda \varphi(A) \; \forall \lambda \in \mathbb{F}, \; \forall A \in \mathbb{V}
    \end{matrix}\\
    \\\\ A \in \mathbb{V}, \; A = \begin{pmatrix}
        a_{11} & a_{12}\\
        a_{21} & a_{22}
    \end{pmatrix}\\
    \\\\ \varphi(A) = \begin{pmatrix}
        1 & 3\\
        2 & 5
    \end{pmatrix}A + A\begin{pmatrix}
        -1 & 1\\
        -4 & 3
    \end{pmatrix} = \\
    \\\\ = \begin{pmatrix}
        1 & 3\\
        2 & 5
    \end{pmatrix} \begin{pmatrix}
        a_{11} & a_{12}\\
        a_{21} & a_{22}
    \end{pmatrix} + \begin{pmatrix}
        a_{11} & a_{12}\\
        a_{21} & a_{22}
    \end{pmatrix} \begin{pmatrix}
        -1 & 1\\
        -4 & 3
    \end{pmatrix} = \\
    \\\\ = \begin{pmatrix}
        a_{11} + 3a_{21} & a_{12} + 3a_{22}\\
        2a_{11} + 5a_{21} & 2a_{12} + 5a_{22}
    \end{pmatrix} + \begin{pmatrix}
        -a_{11} - 4a_{12} & a_{11} + 3a_{12}\\
        -a_{21} + -4a_{22} & a_{21} + 3a_{22}
    \end{pmatrix} = \\
    \\\\ = \begin{pmatrix}
        3a_{21} - 4a_{12} & a_{11} + 3a_{22} + 4a_{12}\\
        2a_{11} + 4(a_{21} - a_{22}) & a_{21} + 2a_{12} + 8a_{22}
    \end{pmatrix}\\
    \\\\ B \in \mathbb{V}, \; B = \begin{pmatrix}
        b_{11} & b_{12}\\
        b_{21} & b_{22}
    \end{pmatrix}\\
    \\\\ \varphi(B) = \begin{pmatrix}
        1 & 3\\
        2 & 5
    \end{pmatrix}B + B\begin{pmatrix}
        -1 & 1\\
        -4 & 3
    \end{pmatrix} = \\
    \\\\ = \begin{pmatrix}
        1 & 3\\
        2 & 5
    \end{pmatrix} \begin{pmatrix}
        b_{11} & b_{12}\\
        b_{21} & b_{22}
    \end{pmatrix} + \begin{pmatrix}
        b_{11} & b_{12}\\
        b_{21} & b_{22}
    \end{pmatrix} \begin{pmatrix}
        -1 & 1\\
        -4 & 3
    \end{pmatrix} = \\
    \\\\ = \begin{pmatrix}
        b_{11} + 3b_{21} & b_{12} + 3b_{22}\\
        2b_{11} + 5b_{21} & 2b_{12} + 5b_{22}
    \end{pmatrix} + \begin{pmatrix}
        -b_{11} - 4b_{12} & b_{11} + 3b_{12}\\
        -b_{21} + -4b_{22} & b_{21} + 3b_{22}
    \end{pmatrix} = \\
    \\\\ = \begin{pmatrix}
        3b_{21} - 4b_{12} & b_{11} + 3b_{22} + 4b_{12}\\
        2b_{11} + 4(b_{21} - b_{22}) & b_{21} + 2b_{12} + 8b_{22}
    \end{pmatrix}\\
    \\\\ \varphi(A) + \varphi(B) = \\
    \\ = \begin{pmatrix}
        3a_{21} - 4a_{12} & a_{11} + 3a_{22} + 4a_{12}\\
        2a_{11} + 4(a_{21} - a_{22}) & a_{21} + 2a_{12} + 8a_{22}
    \end{pmatrix}\\
    +\\
    \begin{pmatrix}
        3b_{21} - 4b_{12} & b_{11} + 3b_{22} + 4b_{12}\\
        2b_{11} + 4(b_{21} - b_{22}) & b_{21} + 2b_{12} + 8b_{22}
    \end{pmatrix} = \\
    \\\\ = \begin{pmatrix}
        3(a_{21} + b_{21}) - 4(a_{12} + b_{12}) & a_{11} + b_{11} + 3(a_{22} + b_{22}) + 4(a_{12} + b_{12})\\
        2(a_{11} + b_{11}) + 4(a_{21} - a_{22} + b_{21} - b_{22}) & a_{21} + b_{21} + 2(a_{12} + b_{12}) + 8(a_{22} + b_{22})
    \end{pmatrix}\\
    \\\\ A + B = \begin{pmatrix}
        a_{11} & a_{12}\\
        a_{21} & a_{22}
    \end{pmatrix} + \begin{pmatrix}
        b_{11} & b_{12}\\
        b_{21} & b_{22}
    \end{pmatrix} = \\
    \\\\ = \begin{pmatrix}
        a_{11} + b_{11} & a_{12} + b_{12}\\
        a_{21} + b_{21} & a_{22} + b_{22}
    \end{pmatrix}\\
    \\\\ \varphi(A + B) = \begin{pmatrix}
        1 & 3\\
        2 & 5
    \end{pmatrix}(A + B) + (A + B)\begin{pmatrix}
        -1 & 1\\
        -4 & 3
    \end{pmatrix} = \\
    \\\\ = \begin{pmatrix}
        1 & 3\\
        2 & 5
    \end{pmatrix} \begin{pmatrix}
        a_{11} + b_{11} & a_{12} + b_{12}\\
        a_{21} + b_{21} & a_{22} + b_{22}
    \end{pmatrix} + \begin{pmatrix}
        a_{11} + b_{11} & a_{12} + b_{12}\\
        a_{21} + b_{21} & a_{22} + b_{22}
    \end{pmatrix} \begin{pmatrix}
        -1 & 1\\
        -4 & 3
    \end{pmatrix} = \\
    \\\\ = \begin{pmatrix}
        a_{11} + b_{11} + 3(a_{21} + b_{21}) & a_{12} + b_{12} + 3(a_{22} + b_{22})\\
        2(a_{11} + b_{11}) + 5(a_{21} + b_{21}) & 2(a_{12} + b_{12}) + 5(a_{22} + b_{22})
    \end{pmatrix}\\
    +\\
    \begin{pmatrix}
        -(a_{11} + b_{11}) - 4(a_{12} + b_{12}) & a_{11} + b_{11} + 3(a_{12} + b_{12})\\
        -(a_{21} + b_{21}) + -4(a_{22} + b_{22}) & a_{21} + b_{21} + 3(a_{22} + b_{22})
    \end{pmatrix} = \\
    \\\\ = \begin{pmatrix}
        3(a_{21} + b_{21}) - 4(a_{12} + b_{12}) & a_{11} + b_{11} + 3(a_{22} + b_{22}) + 4(a_{12} + b_{12})\\
        2(a_{11} + b_{11}) + 4(a_{21} - a_{22} + b_{21} - b_{22}) & a_{21} + b_{21} + 2(a_{12} + b_{12}) + 8(a_{22} + b_{22})
    \end{pmatrix}\\
    \\\\ \implies \varphi(A + B) = \varphi(A) + \varphi(B) \; \forall A, B \in \mathbb{V} \quad (1)\\
    \\ \lambda \in \mathbb{F}\\
    \\ \lambda \varphi(A) = \lambda \begin{pmatrix}
        3a_{21} - 4a_{12} & a_{11} + 3a_{22} + 4a_{12}\\
        2a_{11} + 4(a_{21} - a_{22}) & a_{21} + 2a_{12} + 8a_{22}
    \end{pmatrix} = \\
    \\\\ \begin{pmatrix}
        \lambda(3a_{21} - 4a_{12}) & \lambda(a_{11} + 3a_{22} + 4a_{12})\\
        \lambda(2a_{11} + 4(a_{21} - a_{22})) & \lambda(a_{21} + 2a_{12} + 8a_{22})
    \end{pmatrix}\\
    \\\\ \lambda A = \lambda \begin{pmatrix}
        a_{11} & a_{12}\\
        a_{21} & a_{22}
    \end{pmatrix} = \begin{pmatrix}
        \lambda a_{11} & \lambda a_{12}\\
        \lambda a_{21} & \lambda a_{22}
    \end{pmatrix}\\
    \\\\ \varphi(\lambda A) = \begin{pmatrix}
        1 & 3\\
        2 & 5
    \end{pmatrix}\lambda A + \lambda A\begin{pmatrix}
        -1 & 1\\
        -4 & 3
    \end{pmatrix} = \\
    \\\\ = \begin{pmatrix}
        1 & 3\\
        2 & 5
    \end{pmatrix} \begin{pmatrix}
        \lambda a_{11} & \lambda a_{12}\\
        \lambda a_{21} & \lambda a_{22}
    \end{pmatrix} + \begin{pmatrix}
        \lambda a_{11} & \lambda a_{12}\\
        \lambda a_{21} & \lambda a_{22}
    \end{pmatrix} \begin{pmatrix}
        -1 & 1\\
        -4 & 3
    \end{pmatrix} = \\
    \\\\ = \begin{pmatrix}
        \lambda a_{11} + 3\lambda a_{21} & \lambda a_{12} + 3\lambda a_{22}\\
        2\lambda a_{11} + 5\lambda a_{21} & 2\lambda a_{12} + 5\lambda a_{22}
    \end{pmatrix} + \begin{pmatrix}
        -\lambda a_{11} - 4\lambda a_{12} & \lambda a_{11} + 3\lambda a_{12}\\
        -\lambda a_{21} + -4\lambda a_{22} & \lambda a_{21} + 3\lambda a_{22}
    \end{pmatrix} = \\
    \\\\ = \begin{pmatrix}
        3\lambda a_{21} - 4\lambda a_{12} & \lambda a_{11} + 3\lambda a_{22} + 4\lambda a_{12}\\
        2\lambda a_{11} + 4(\lambda a_{21} - \lambda a_{22}) & \lambda a_{21} + 2\lambda a_{12} + 8\lambda a_{22}
    \end{pmatrix} = \\
    \\\\ \begin{pmatrix}
        \lambda(3a_{21} - 4a_{12}) & \lambda(a_{11} + 3a_{22} + 4a_{12})\\
        \lambda(2a_{11} + 4(a_{21} - a_{22})) & \lambda(a_{21} + 2a_{12} + 8a_{22})
    \end{pmatrix}\\
    \\\\ \implies \varphi(\lambda A) = \lambda \varphi(A) \; \forall \lambda \in \mathbb{F}, \; \forall A \in \mathbb{V} \quad (2)\\
    \\ \text{От (1) и (2) } \implies \varphi \in Hom \mathbb{V}\\
    \\\\ \varphi(E_{11}) = \begin{pmatrix}
        1 & 3\\
        2 & 5
    \end{pmatrix}E_{11} + E_{11}\begin{pmatrix}
        -1 & 1\\
        -4 & 3
    \end{pmatrix} = \\
    \\\\ = \begin{pmatrix}
        1 & 3\\
        2 & 5
    \end{pmatrix} \begin{pmatrix}
        1 & 0\\
        0 & 0
    \end{pmatrix} + \begin{pmatrix}
        1 & 0\\
        0 & 0
    \end{pmatrix} \begin{pmatrix}
        -1 & 1\\
        -4 & 3
    \end{pmatrix} = \\
    \\\\ = \begin{pmatrix}
        1 & 0\\
        2 & 0
    \end{pmatrix} + \begin{pmatrix}
        -1 & 1\\
        0 & 0
    \end{pmatrix} = \begin{pmatrix}
        0 & 1\\
        2 & 0
    \end{pmatrix}\\
    \\\\ \varphi(E_{12}) = \begin{pmatrix}
        1 & 3\\
        2 & 5
    \end{pmatrix}E_{12} + E_{12}\begin{pmatrix}
        -1 & 1\\
        -4 & 3
    \end{pmatrix} = \\
    \\\\ = \begin{pmatrix}
        1 & 3\\
        2 & 5
    \end{pmatrix} \begin{pmatrix}
        0 & 1\\
        0 & 0
    \end{pmatrix} + \begin{pmatrix}
        0 & 1\\
        0 & 0
    \end{pmatrix} \begin{pmatrix}
        -1 & 1\\
        -4 & 3
    \end{pmatrix} = \\
    \\\\ = \begin{pmatrix}
        0 & 1\\
        0 & 2
    \end{pmatrix} + \begin{pmatrix}
        -4 & 3\\
        0 & 0
    \end{pmatrix} = \begin{pmatrix}
        -4 & 4\\
        0 & 2
    \end{pmatrix}\\
    \\\\ \varphi(E_{21}) = \begin{pmatrix}
        1 & 3\\
        2 & 5
    \end{pmatrix}E_{21} + E_{21}\begin{pmatrix}
        -1 & 1\\
        -4 & 3
    \end{pmatrix} = \\
    \\\\ = \begin{pmatrix}
        1 & 3\\
        2 & 5
    \end{pmatrix} \begin{pmatrix}
        0 & 0\\
        1 & 0
    \end{pmatrix} + \begin{pmatrix}
        0 & 0\\
        1 & 0
    \end{pmatrix} \begin{pmatrix}
        -1 & 1\\
        -4 & 3
    \end{pmatrix} = \\
    \\\\ = \begin{pmatrix}
        3 & 0\\
        5 & 0
    \end{pmatrix} + \begin{pmatrix}
        0 & 0\\
        -1 & 1
    \end{pmatrix} = \begin{pmatrix}
        3 & 0\\
        4 & 1
    \end{pmatrix}\\
    \\\\ \varphi(E_{22}) = \begin{pmatrix}
        1 & 3\\
        2 & 5
    \end{pmatrix}E_{22} + E_{22}\begin{pmatrix}
        -1 & 1\\
        -4 & 3
    \end{pmatrix} = \\
    \\\\ = \begin{pmatrix}
        1 & 3\\
        2 & 5
    \end{pmatrix} \begin{pmatrix}
        0 & 0\\
        0 & 1
    \end{pmatrix} + \begin{pmatrix}
        0 & 0\\
        0 & 1
    \end{pmatrix} \begin{pmatrix}
        -1 & 1\\
        -4 & 3
    \end{pmatrix} = \\
    \\\\ = \begin{pmatrix}
        0 & 3\\
        0 & 5
    \end{pmatrix} + \begin{pmatrix}
        0 & 0\\
        -4 & 3
    \end{pmatrix} = \begin{pmatrix}
        0 & 3\\
        -4 & 8
    \end{pmatrix}\\
    \\\\ M_{E_2}(\varphi) = \begin{pmatrix}
        0 & -4 & 3 & 0\\
        1 & 4 & 0 & 3\\
        2 & 0 & 4 & -4\\
        0 & 2 & 1 & 8
    \end{pmatrix}\)\\
    \\\\ б) \(\psi(X) = X\begin{pmatrix}
        1 & 3\\
        2 & 5
    \end{pmatrix} + \begin{pmatrix}
        -1 & 1\\
        -4 & 3
    \end{pmatrix}, \; X \in \mathbb{V}\)\\
    \\ Решение:\\
    \(\psi \in Hom\mathbb{V} \iff \begin{matrix}
        \psi(A + B) = \psi(A) + \psi(B) \; \forall A, B \in \mathbb{V}\\
        \psi(\lambda A) = \lambda \psi(A) \; \forall \lambda \in \mathbb{F}, \; \forall A \in \mathbb{V}
    \end{matrix}\\
    \\\\ A \in \mathbb{V}, \; A = \begin{pmatrix}
        a_{11} & a_{12}\\
        a_{21} & a_{22}
    \end{pmatrix}\\
    \\\\ \psi(A) = A\begin{pmatrix}
        1 & 3\\
        2 & 5
    \end{pmatrix} + \begin{pmatrix}
        -1 & 1\\
        -4 & 3
    \end{pmatrix} = \\
    \\\\ = \begin{pmatrix}
        a_{11} & a_{12}\\
        a_{21} & a_{22}
    \end{pmatrix} \begin{pmatrix}
        1 & 3\\
        2 & 5
    \end{pmatrix} + \begin{pmatrix}
        -1 & 1\\
        -4 & 3
    \end{pmatrix} = \\
    \\\\ = \begin{pmatrix}
        a_{11} + 2a_{12} & 3a_{11} + 5a_{12}\\
        a_{21} + 2a_{22} & 3a_{21} + 5a_{22}
    \end{pmatrix} + \begin{pmatrix}
        -1 & 1\\
        -4 & 3
    \end{pmatrix} = \\
    \\\\ = \begin{pmatrix}
        a_{11} + 2a_{12} - 1 & 3a_{11} + 5a_{12} + 1\\
        a_{21} + 2a_{22} - 4 & 3a_{21} + 5a_{22} + 3
    \end{pmatrix}\\
    \\\\ B \in \mathbb{V}, \; B = \begin{pmatrix}
        b_{11} & b_{12}\\
        b_{21} & b_{22}
    \end{pmatrix}\\
    \\\\ \psi(B) = B\begin{pmatrix}
        1 & 3\\
        2 & 5
    \end{pmatrix} + \begin{pmatrix}
        -1 & 1\\
        -4 & 3
    \end{pmatrix} = \\
    \\\\ = \begin{pmatrix}
        b_{11} & b_{12}\\
        b_{21} & b_{22}
    \end{pmatrix} \begin{pmatrix}
        1 & 3\\
        2 & 5
    \end{pmatrix} + \begin{pmatrix}
        -1 & 1\\
        -4 & 3
    \end{pmatrix} = \\
    \\\\ = \begin{pmatrix}
        b_{11} + 2b_{12} & 3b_{11} + 5b_{12}\\
        b_{21} + 2b_{22} & 3b_{21} + 5b_{22}
    \end{pmatrix} + \begin{pmatrix}
        -1 & 1\\
        -4 & 3
    \end{pmatrix} = \\
    \\\\ = \begin{pmatrix}
        b_{11} + 2b_{12} - 1 & 3b_{11} + 5b_{12} + 1\\
        b_{21} + 2b_{22} - 4 & 3b_{21} + 5b_{22} + 3
    \end{pmatrix}\\
    \\\\ \psi(A) + \psi(B) = \begin{pmatrix}
        a_{11} + 2a_{12} - 1 & 3a_{11} + 5a_{12} + 1\\
        a_{21} + 2a_{22} - 4 & 3a_{21} + 5a_{22} + 3
    \end{pmatrix} + \begin{pmatrix}
        b_{11} + 2b_{12} - 1 & 3b_{11} + 5b_{12} + 1\\
        b_{21} + 2b_{22} - 4 & 3b_{21} + 5b_{22} + 3
    \end{pmatrix} = \\
    \\\\ = \begin{pmatrix}
        a_{11} + b_{11} + 2(a_{12} + b_{12} - 1) & 3(a_{11} + b_{11}) + 5(a_{12} + b_{12}) + 2\\
        a_{21}  + b_{21} + 2(a_{22} + b_{22} - 4) & 3(a_{21} + b_{21} + 2) + 5(a_{22} + b_{22})
    \end{pmatrix}\\
    \\\\ A + B = \begin{pmatrix}
        a_{11} & a_{12}\\
        a_{21} & a_{22}
    \end{pmatrix} + \begin{pmatrix}
        b_{11} & b_{12}\\
        b_{21} & b_{22}
    \end{pmatrix} = \\
    \\\\ = \begin{pmatrix}
        a_{11} + b_{11} & a_{12} + b_{12}\\
        a_{21} + b_{21} & a_{22} + b_{22}
    \end{pmatrix}\\
    \\\\ \psi(A + B) = (A + B)\begin{pmatrix}
        1 & 3\\
        2 & 5
    \end{pmatrix} + \begin{pmatrix}
        -1 & 1\\
        -4 & 3
    \end{pmatrix} = \\
    \\\\ = \begin{pmatrix}
        a_{11} + b_{11} & a_{12} + b_{12}\\
        a_{21} + b_{21} & a_{22} + b_{22}
    \end{pmatrix} \begin{pmatrix}
        1 & 3\\
        2 & 5
    \end{pmatrix} + \begin{pmatrix}
        -1 & 1\\
        -4 & 3
    \end{pmatrix} = \\
    \\\\ = \begin{pmatrix}
        a_{11} + b_{11} + 2(a_{12} + b_{12}) & 3(a_{11} + b_{11}) + 5(a_{12} + b_{12})\\
        a_{21} + b_{21} + 2(a_{22} + b_{22}) & 3(a_{21} + b_{21}) + 5(a_{22} + b_{22})
    \end{pmatrix} + \begin{pmatrix}
        -1 & 1\\
        -4 & 3
    \end{pmatrix} = \\
    \\\\ = \begin{pmatrix}
        a_{11} + b_{11} + 2(a_{12} + b_{12}) - 1 & 3(a_{11} + b_{11}) + 5(a_{12} + b_{12}) + 1\\
        a_{21} + b_{21} + 2(a_{22} + b_{22} - 2) & 3(a_{21} + b_{21} + 1) + 5(a_{22} + b_{22})
    \end{pmatrix} \\
    \\\\ \implies \psi(A + B) \neq \psi(A) + \psi(B) \; \forall A, B \in \mathbb{V} \implies \psi \not\in Hom\mathbb{V}\)\\
    \\\\ Задача 8.\\
    \\ В линейното пространство \(\mathbb{V}\) с базис \(e_1, \; e_2, \; e_3, \; e_4\), \; \(A \in Hom\mathbb{V}\):\\
    \(\\ \begin{matrix}
        A(\xi_1 e_1, \; \xi_2 e_2, \; \xi_3 e_3, \; \xi_4 e_4) = & ~ & ~\\
        ~ & = (\xi_1 - \xi_2 - 3\xi_3 + \xi_4)e_1 + (3\xi_2 + \xi_3)e_2 \; + & ~ \\
        ~ & + \; (-\xi_1 -2\xi_2 + 2\xi_3 - \xi_4)e_3 & + \; (4\xi_1 + 5\xi_2 - 9\xi_3 + 4\xi_4)e_4
    \end{matrix}\)\\
    \\\\ Решение:\\
    \\Полг.\\
    \(\xi_1 = 1, \; \xi_2 = \xi_3 = \xi_4 = 0 ; \; A(e_1) = e_1 - e_3 + 4e_4\\
    \xi_2 = 1, \; \xi_1 = \xi_3 = \xi_4 = 0 ; \; A(e_2) = -e_1 + 3e_2 - 2e_3 + 5e_4\\
    \xi_3 = 1, \; \xi_1 = \xi_2 = \xi_4 = 0 ; \; A(e_3) = -3e_1 + e_2 + 2e_3 - 9e_4\\
    \xi_1 = 4, \; \xi_1 = \xi_2 = \xi_3 = 0 ; \; A(e_4) = e_1 - e_3 + 4e_4\\
    \\ M_e(A) = \begin{pmatrix}
        1 & -1 & -3 & 1\\
        0 & 3 & 1 & 0\\
        -1 & -2 & 2 & -1\\
        4 & 5 & -9 & 4
    \end{pmatrix}\\
    \\ KerA:\\
    \\\\ \begin{matrix}
        ~\\
        ~\\
        1\\
        -4
    \end{matrix} \begin{pmatrix}
        1 & -1 & -3 & 1\\
        0 & 3 & 1 & 0\\
        -1 & -2 & 2 & -1\\
        4 & 5 & -9 & 4
    \end{pmatrix} \to \\
    \\\\ \to \begin{pmatrix}
        1 & -1 & -3 & 1\\
        0 & 3 & 1 & 0\\
        0 & -3 & -1 & 0\\
        0 & 9 & 3 & 0
    \end{pmatrix} \to \\
    \\\\ \to \begin{pmatrix}
        1 & -1 & -3 & 1\\
        0 & 3 & 1 & 0\\
        0 & 0 & 0 & 0\\
        0 & 0 & 0 & 0
    \end{pmatrix} \to \\
    \\\\ \to \begin{matrix}
        3\\
        ~
    \end{matrix} \begin{pmatrix}
        1 & -1 & -3 & 1\\
        0 & 3 & 1 & 0
    \end{pmatrix} \to \\
    \\\\ \to \begin{pmatrix}
        1 & 8 & 0 & 1\\
        0 & 3 & 1 & 0\\
        ~ & ~ & - & -
    \end{pmatrix}\\
    \\\\
    a_1 = (\frac{8}{3}, \; -\frac{1}{3}, \; 1, \; 0)\\
    a_2 = (-1, \; 0, \; 0, \; 1)\\
    a_1, a_2 \text{ - базис на } KerA\\
    d(A) = dimKerA = 2\\
    \\ ImA:\\
    \\\\ M_e(A)^t = \begin{pmatrix}
        1 & -1 & -3 & 1\\
        0 & 3 & 1 & 0\\
        -1 & -2 & 2 & -1\\
        4 & 5 & -9 & 4
    \end{pmatrix}^t = \begin{pmatrix}
        1 & 0 & -1 & 4\\
        -1 & 3 & -2 & 5\\
        -3 & 1 & 2 & -9\\
        1 & 0 & -1 & 4
    \end{pmatrix}\\
    \\\\ \begin{pmatrix}
        1 & 0 & -1 & 4\\
        -1 & 3 & -2 & 5\\
        -3 & 1 & 2 & -9\\
        1 & 0 & -1 & 4
    \end{pmatrix} \to \\
    \\\\ \to \begin{pmatrix}
        1 & 0 & -1 & 4\\
        -1 & 3 & -2 & 5\\
        -3 & 1 & 2 & -9\\
        0 & 0 & 0 & 0
    \end{pmatrix} \to \\
    \\\\ \to \begin{matrix}
        ~\\
        1\\
        3
    \end{matrix} \begin{pmatrix}
        1 & 0 & -1 & 4\\
        -1 & 3 & -2 & 5\\
        -3 & 1 & 2 & -9
    \end{pmatrix} \to \\
    \\\\ \to \begin{pmatrix}
        1 & 0 & -1 & 4\\
        0 & 3 & -3 & 9\\
        0 & 1 & -1 & 3
    \end{pmatrix} \to \\
    \\\\ \to \begin{pmatrix}
        1 & 0 & -1 & 4\\
        0 & 0 & 0 & 0\\
        0 & 1 & -1 & 3
    \end{pmatrix} \to \\
    \\\\ \to \begin{pmatrix}
        1 & 0 & -1 & 4\\
        0 & 1 & -1 & 3
    \end{pmatrix}\\
    \\\\
    b_1 = (1, \; 0, \; -1, \; 4)\\
    b_2 = (0, \; 1, \; -1, \; 3)\\
    b_1, b_2 \text{ - базис на } ImA\\
    r(A) = dimImA = 2\)
\end{document}
