\documentclass{article}
\usepackage{amsmath}
\usepackage{amssymb}
\usepackage{pst-node}
\usepackage{stackengine}
\usepackage{fixltx2e}
\usepackage[T1,T2A]{fontenc}
\usepackage[utf8]{inputenc}
\usepackage[bulgarian]{babel}
\usepackage[normalem]{ulem}
\newcommand{\stkout}[1]{\ifmmode\text{\sout{\ensuremath{#1}}}\else\sout{#1}\fi}

\makeatletter
\renewcommand*\env@matrix[1][*\c@MaxMatrixCols c]{%
  \hskip -\arraycolsep
  \let\@ifnextchar\new@ifnextchar
  \array{#1}}
\makeatother

\newcommand\tab[1][1cm]{\hspace*{#1}}

\title{Домашна работа 3, №45342, Martin, 1, I, Информатика}
\author{Иво Стратев}

\begin{document}
    \pagenumbering{gobble}
    \maketitle
    \section{Задача 1.}
    \(\begin{vmatrix}
        -1 & -5 & -5 & -5 & -5 & \dots & -5\\
        -8 & 9 + n.1 & 9 & 9 & 9 & \dots & 9\\
        -8 & 9 & 9 + 1.2 & 9 & 9 & \dots & 9\\
        -8 & 9 & 9 & 9 + 2.3 & 9 & \dots & 9\\
        -8 & 9 & 9 & 9 & 9 + 3. 4 & \dots & 9\\
        \\
        ~ & ~ & ~ & \dots & ~ & ~ & ~\\
        \\
        -8 & 9 & 9 & 9 & 9 & \dots & 9 + (n - 1).n\\
    \end{vmatrix} =\\
    \\\\= \begin{vmatrix}
        -1 & -5 & -5 & -5 & -5 & \dots & -5\\
        -\frac{49}{5} & n.1 & 0 & 0 & 0 & \dots & 0\\
        -\frac{49}{5} & 0 & 1.2 & 0 & 0 & \dots & 0\\
        -\frac{49}{5} & 0 & 0 & 2.3 & 0 & \dots & 0\\
        -\frac{49}{5} & 0 & 0 & 0 & 3. 4 & \dots & 0\\
        \\
        ~ & ~ & ~ & \dots & ~ & ~ & ~\\
        \\
        -\frac{49}{5} & 0 & 0 & 0 & 0 & \dots & (n - 1).n\\
    \end{vmatrix} = \Delta\\
    \\\\\varsigma_{ij} = \begin{cases}
        1 & i \geq j\\
        0 & i < j
    \end{cases}\\
    \\\\\Delta = \begin{vmatrix}\\
        -1 -49.(\frac{1}{n} + \varsigma_{n2}\sum_{i = 2}^{n}\frac{1}{(i - 1)i}) & -5 & -5 & -5 & -5 & \dots & -5\\
        0 & n.1 & 0 & 0 & 0 & \dots & 0\\
        0 & 0 & 1.2 & 0 & 0 & \dots & 0\\
        0 & 0 & 0 & 2.3 & 0 & \dots & 0\\
        0 & 0 & 0 & 0 & 3. 4 & \dots & 0\\
        \\
        ~ & ~ & ~ & \dots & ~ & ~ & ~\\
        \\
        0 & 0 & 0 & 0 & 0 & \dots & (n - 1).n\\
    \end{vmatrix}\\
    \\\\\Delta = [-1 -49.(\frac{1}{n} + \varsigma_{n2}\sum_{i = 2}^{n}\frac{1}{(i - 1)i})]n!^2\\
    \\\forall n \in \mathbb{N} \quad \frac{1}{n} + \varsigma_{n2}\sum_{i = 2}^{n}\frac{1}{(i - 1)i} \overset{?}{=} 1\\
    \\n = 1 \implies \frac{1}{1} + 0 = 1\\
    \\n = 2 \implies \frac{1}{2} + \frac{1}{1.2} = \frac{2}{2} = 1\\
    \\n = k \implies \frac{1}{k} + \varsigma_{k2}\sum_{i = 2}^{k}\frac{1}{(i - 1)i} = 1\\
    \\n = k + 1 \implies \frac{1}{k + 1} + \varsigma_{(k + 1)2}\sum_{i = 2}^{k + 1}\frac{1}{(i - 1)i} =\\
    \\= \frac{1}{k + 1} + \varsigma_{k2}\sum_{i = 2}^{k}\frac{1}{(i - 1)i} + \frac{1}{k(k + 1)} =\\
    \\= \frac{1}{k + 1} - \frac{1}{k} + \frac{1}{k} + \varsigma_{k2}\sum_{i = 2}^{k}\frac{1}{(i - 1)i} + \frac{1}{k(k + 1)} =\\
    \\= \frac{1}{k + 1} - \frac{1}{k} + 1 + \frac{1}{k(k + 1)} = \frac{k -(k + 1) + k(k + 1) + 1}{k(k + 1)} =\\
    \\= \frac{(k + 1)-(k + 1) + k(k + 1)}{k(k + 1)} = \frac{k(k + 1)}{k(k + 1)} = 1\\
    \\\implies \forall n \in \mathbb{N} \quad \frac{1}{n} + \varsigma_{n2}\sum_{i = 2}^{n}\frac{1}{(i - 1)i} = 1\\
    \\\implies \Delta = (-1 -49)n!^2 = -50.n!^2\)
    \section{Задача 2.}
    \(A = M_e(\Lambda)\\
    \\A = \begin{pmatrix}
        -12 & 2 & 3\\
        6 & -13 & -6\\
        -9 & 6 & 0
    \end{pmatrix}\\
    \\\\f_A(\lambda) = \begin{vmatrix}
        -12 - \lambda & 2 & 3\\
        6 & -13 - \lambda & -6\\
        -9 & 6 & - \lambda
    \end{vmatrix} = \begin{vmatrix}
        -12 - \lambda & 2 & 3\\
        6 & -13 - \lambda & -6\\
        3 + \lambda & 4 & -3 - \lambda
    \end{vmatrix} =\\
    \\\\= \begin{vmatrix}
        -12 - \lambda & 2 & 3\\
        9 + \lambda & -9 - \lambda & -9 - \lambda\\
        3 + \lambda & 4 & -3 - \lambda
    \end{vmatrix} = (9 + \lambda)\begin{vmatrix}
        -12 - \lambda & 2 & 3\\
        1 & -1 & -1\\
        3 + \lambda & 4 & -3 - \lambda
    \end{vmatrix} =\\
    \\\\= (9 + \lambda)\begin{vmatrix}
        -10 - \lambda & 0 & 1\\
        1 & -1 & -1\\
        7 + \lambda & 0 & -7 - \lambda
    \end{vmatrix} = (9 + \lambda)(7 + \lambda)\begin{vmatrix}
        -10 - \lambda & 0 & 1\\
        1 & -1 & -1\\
        1 & 0 & -1
    \end{vmatrix} =\\
    \\\\= (9 + \lambda)(7 + \lambda)\begin{vmatrix}
        -9 - \lambda & 0 & 0\\
        0 & -1 & 0\\
        1 & 0 & -1
    \end{vmatrix} = (9 + \lambda)^2(7 + \lambda)\begin{vmatrix}
        -1 & 0 & 0\\
        0 & -1 & 0\\
        1 & 0 & -1
    \end{vmatrix} =\\
    \\\\= (9 + \lambda)^2(7 + \lambda)\begin{vmatrix}
        -1 & 0 & 0\\
        0 & -1 & 0\\
        0 & 0 & -1
    \end{vmatrix} = -(9 + \lambda)^2(7 + \lambda)\\
    \\\\f_A(\lambda) = 0 = -(9 + \lambda)^2(7 + \lambda)\\
    \\\\\implies \lambda_{1, 2} = -9, \; \lambda_3 = -7\\
    \\\lambda_{1, 2} = -9\\
    \\\begin{pmatrix}
        -3 & 2 & 3\\
        6 & -4 & -6\\
        -9 & 6 & 9
    \end{pmatrix} \to \begin{pmatrix}
        -3 & 2 & 3\\
        0 & 0 & 0\\
        0 & 0 & 0
    \end{pmatrix} \to \begin{pmatrix}
        -3 & 2 & 3
    \end{pmatrix}\\
    \\\\v_1 = (1, 0, 1)\\
    v_2 = (2, 3, 0)\\
    \\\\\begin{pmatrix}
        -5 & 2 & 3\\
        6 & -6 & -6\\
        -9 & 6 & 7
    \end{pmatrix} \to \begin{pmatrix}
        -5 & 2 & 3\\
        1 & -1 & -1\\
        -9 & 6 & 7
    \end{pmatrix} \to \begin{pmatrix}
        -3 & 0 & 1\\
        1 & -1 & -1\\
        -3 & 0 & 1
    \end{pmatrix} \to\\
    \\\\\to \begin{pmatrix}
        -3 & 0 & 1\\
        1 & -1 & -1\\
        0 & 0 & 0
    \end{pmatrix} \to \begin{pmatrix}
        -3 & 0 & 1\\
        1 & -1 & -1
    \end{pmatrix} \to \begin{pmatrix}
        -3 & 0 & 1\\
        -2 & -1 & 0
    \end{pmatrix}\\
    \\\\v_3 = (1, -2, 3)\\
    \\\\v_1 = (1, 0, 1)\\
    v_2 = (2, 3, 0)\\
    v_3 = (1, -2, 3)
    \\\\M_v(\Lambda) = \begin{pmatrix}
        -9 & 0 & 0\\
        0 & -9 & 0\\
        0 & 0 & -7
    \end{pmatrix}\)
    \section{Задача 3.}
    \(\mathbb{V} = \mathbb{R}^2 \text{ e линейното пространство над \(\mathbb{R}\)}\\
    \\A = \begin{pmatrix}
        7 & 6\\
        6 & 9\\
    \end{pmatrix} \in M_2(\mathbb{V}), \; v = (3, -1) \in \mathbb{V}\\
    \\\\\forall x = (x_1, x_2), \; y = (y_1, y_2) \in \mathbb{V}\\
    \\\lbrack x, y \rbrack_1 := xAy^t = 7x_1y_1 + 6x_2y_1 + 6x_1y_2 + 9x_2y_2\\
    \\\lbrack x, y \rbrack_2 := -xAy^t = -7x_1y_1 - 6x_2y_1 - 6x_1y_2 - 9x_2y_2\\
    \\\lbrack x, y \rbrack : \mathbb{V} \times \mathbb{V} \to \mathbb{R} \text{ е скаларно произведение ако:}\\
    \\1. \; \forall a, b, c \in \mathbb{V} \; \lbrack a + b, c \rbrack = \lbrack a, c \rbrack + \lbrack b, c \rbrack\\
    \\2. \; \forall a, b \in \mathbb{V} \;, \forall \lambda \in \mathbb{R} \; \lbrack \lambda a, b \rbrack = \lambda \lbrack a, b \rbrack\\
    \\3. \; \forall a, b \in \mathbb{V} \; \lbrack a, b \rbrack = \lbrack b, a \rbrack\\
    \\4. \; \forall a \in \mathbb{V} \; \lbrack a, a \rbrack \geq 0, \; \lbrack a, a \rbrack = 0 \iff a = \theta\\
    \\\lbrack x, y \rbrack_2 \text{ е скаларно произведение ако } \forall a \in \mathbb{V} \implies \lbrack a, a \rbrack_2 \geq 0\\
    \\v \in \mathbb{V} \implies \lbrack v, v \rbrack_2 \text{ трябва да е } \geq 0\\
    \\\lbrack v, v \rbrack_2 = -(7.9 - 6.3 - 6.3 + 9) = -(8.9 - 4.9) = -(4.9) = -36\\
    \\\lbrack v, v \rbrack_2 < 0 \implies \lbrack x, y \rbrack_2 \text{ не е скаларно произведение}\\
    \\\forall a, b, c \in \mathbb{V} \; \lbrack a + b, c \rbrack_1 = (a + b)Ac^t = aAc^t + bAc^t\\
    \\\forall a, b \in \mathbb{V} \;, \forall \lambda \in \mathbb{R} \; \lbrack \lambda a, b \rbrack_1 = (\lambda a)Ab^t = \lambda(aAb^t) = \lambda \lbrack a, b \rbrack_1\\
    \\\text{Свойства 1. и 2. са вярни от свойства на матриците}\\
    \\\forall a = (a_1, a_2), b = (b_1, b_2) \in \mathbb{V} \; \lbrack b, a \rbrack_1 = 7b_1a_1 + 6b_2a_1 + 6b_1a_2 + 9b_2a_2 =\\
    \\= 7a_1b_1 + 6a_1b_2 + 6a_2b_1 + 9a_2b_2 = 7a_1b_1 + 6a_2b_1 + 6a_1b_2 + 9a_2b_2 = \lbrack a, b \rbrack_1\\
    \\\text{Свойствo 3. e вярнo от факта, че \(\mathbb{V}\) е Л.П. над \(\mathbb{R}\), което е числово поле}\\
    \\\forall a = (a_1, a_2) \in \mathbb{V} \; \lbrack a, a \rbrack_1 = 7a_1a_1 + 6a_2a_1 + 6a_1a_2 + 9a_2a_2 =\\
    \\= 7a_1^2 + 12a_1a_2 + 9a_2^2 = G(a)\\
    \\ a = \theta \implies G(a) = 0\\
    \\ a \neq \theta, \; \text{ Б.О.О. } a_2 \neq 0\\
    \\7a_1^2 + 12a_1a_2 + 9a_2^2 = 0 \; | \frac{1}{a_2^2}\\
    \\7\left(\frac{a_1}{a_2}\right)^2 + 12\frac{a_1}{a_2} + 9 = 0\\
    \\H(a) = 7\left(\frac{a_1}{a_2}\right)^2 + 12\frac{a_1}{a_2} + 9\\
    \\D_H = 4.4.3.3 - 4.7.9 = 36(4 - 7) = -3.36 = -108\\
    \\D_H < 0, \; 7 > 0 \implies \forall a; \; a_2 \neq 0 \; H(a) > 0\\
    \\\implies G(a) > 0 \; \forall a; \; a_2 \neq 0\\
    \\\implies \forall a = (a_1, a_2) \in \mathbb{V} \; \lbrack a, a \rbrack_1 \geq 0, \; \lbrack a, a \rbrack_1 = 0 \iff a = \theta\\
    \\\implies \forall x, \; y \in \mathbb{V} \; \lbrack x, y \rbrack_1 \text{ е скаларно произведение}\\
    \\\implies \lbrack v, v \rbrack_1 = G(v) = 7.9 - 12.3 + 9 = 7.9 - 4.9 + 9 = (8 - 4)9 = 36\\
    \\|v| = \sqrt{ \lbrack v, v \rbrack_1} = \sqrt{36} = 6\)
    \section{Задача 4.}
    \(a_1 = (-22, 17, -27, -8), \quad a_2 = (-2, -6, 4, -4),\\
    a_3 = (-15, -2, -6, -9), \quad a_4 = (-3, -1, -1, -3)\\
    \\\mathbb{U} = l(a_1, a_2, a_3, a_4)\\
    \\\begin{pmatrix}
        -22 & 17 & -27 & -8\\
        -2 & -6 & 4 & -4\\
        -15 & -2 & -6 & -9\\
        -3 & -1 & -1 & -3
    \end{pmatrix} \begin{matrix}
        -1\\
        -\frac{1}{2}\\
        -2\\
        -2
    \end{matrix} \to  \begin{matrix}
        -4\\
        ~\\
        -9\\
        -3
    \end{matrix} \begin{pmatrix}
        22 & -17 & 27 & 8\\
        1 & 3 & -2 & 2\\
        30 & 4 & 12 & 18\\
        6 & 2 & 2 & 6
    \end{pmatrix} \to\\
    \\\\\to \begin{pmatrix}
        18 & -29 & 35 & 0\\
        1 & 3 & -2 & 2\\
        21 & -23 & 30 & 0\\
        3 & -7 & 8 & 0
    \end{pmatrix} \begin{matrix}
        ~\\
        3\
        ~\\
        ~
    \end{matrix} \to \begin{matrix}
        -6\\
        -1\\
        -7\\
        ~
    \end{matrix} \begin{pmatrix}
        18 & -29 & 35 & 0\\
        3 & 9 & -6 & 6\\
        21 & -23 & 30 & 0\\
        3 & -7 & 8 & 0
    \end{pmatrix} \to\\
    \\\\\to \begin{pmatrix}
        0 & 13 & -13 & 0\\
        0 & 16 & -14 & 6\\
        0 & 26 & -26 & 0\\
        3 & -7 & 8 & 0
    \end{pmatrix} \begin{matrix}
        \frac{1}{13}\\
        \frac{1}{2}\\
        \frac{1}{26}\\
        ~
    \end{matrix} \to \begin{pmatrix}
        0 & 1 & -1 & 0\\
        0 & 8 & -7 & 3\\
        0 & 1 & -1 & 0\\
        3 & -7 & 8 & 0
    \end{pmatrix} \to\\
    \\\\\to \begin{pmatrix}
        0 & 0 & 0 & 0\\
        0 & 8 & -7 & 3\\
        0 & 1 & -1 & 0\\
        3 & -7 & 8 & 0
    \end{pmatrix} \to \begin{matrix}
        -7\\
        ~\\
        8
    \end{matrix} \begin{pmatrix}
        0 & 8 & -7 & 3\\
        0 & 1 & -1 & 0\\
        3 & -7 & 8 & 0
    \end{pmatrix} \to\\
    \\\\\to \begin{pmatrix}
        0 & 1 & 0 & 3\\
        0 & 1 & -1 & 0\\
        3 & 1 & 0 & 0
    \end{pmatrix}\\
    \\\\v_1 = (0, 1, 0, 3)\\
    v_2 = (0, 1, -1, 0)\\
    v_3 = (3, 1, 0, 0)\\
    \\g_1 = v_1 = (0, 1, 0, 3), \quad (g_1, g_1) = 1 + 9 = 10\\
    \\g_2 = v_2 + \lambda g_1 \; |(\quad, g_1) \; \lambda \in \mathbb{R}\\
    \\0 = (v_2, g_1) + \lambda(g_1, g_1)\\
    \\\lambda = -\frac{(v_2, g_1)}{(g_1, g_1)} = -\frac{1}{10}\\
    \\\implies g_2 = (0, 1, -1, 0) + (0, -\frac{1}{10}, 0, -\frac{3}{10}) = (0, \frac{9}{10}, -1, -\frac{3}{10})\\
    \\(g_2, g_2) = \frac{81}{100} + \frac{100}{100} + \frac{9}{100} = \frac{190}{100} = \frac{19}{10}\\
    \\g_3 = v_3 + \mu_1 g_1 + \mu_2 g_2\\
    \\g_3 = v_3 + \mu_1 g_1 + \mu_2 g_2 \; |(\quad, g_1)\\
    \\0 = (v_3, g_1) + \mu_1(g_1, g_1) + 0\\
    \\\mu_1 = -\frac{(v_3, g_1)}{(g_1, g_1)} = -\frac{1}{10}\\
    \\g_3 = v_3 + \mu_1 g_1 + \mu_2 g_2 \; |(\quad, g_2)\\
    \\0 = (v_3, g_2) + 0 + \mu_2(g_2, g_2)\\
    \\\mu_2 = -\frac{(v_3, g_2)}{(g_2, g_2)} = -\frac{9}{\stkout{10}}\frac{\stkout{10}}{19} = -\frac{9}{19}\\
    \\\implies g_3 = (3, 1, 0, 0) + -\frac{1}{10}(0, 1, 0, 3) + -\frac{9}{19}(0, \frac{9}{10}, -1, -\frac{3}{10}) =\\
    \\= (3, 1, 0, 0) + (0, -\frac{1}{10}, 0, -\frac{3}{10}) + (0, -\frac{81}{190}, \frac{9}{19}, \frac{27}{190}) =\\
    \\= (3, \frac{9}{10}, 0, -\frac{3}{10}) + (0, -\frac{81}{190}, \frac{9}{19}, \frac{27}{190}) = (3, \frac{9.19 - 81}{190}, \frac{9}{19}, \frac{27 - 3.19}{190}) =\\
    \\ = (3, \frac{90}{190}, \frac{9}{19}, -\frac{30}{190}) = (3, \frac{9}{19}, \frac{9}{19}, -\frac{3}{19})\\
    \\b_1 = g_1 = (0, 1, 0, 3)\\
    \\b_2 = 10g_2 = (0, 9, -10, -3)\\
    \\b_3 = \frac{19}{3}g_3 = (19, 3, 3, -1)\\
    \\\\a = (1, 7, -15, -5) \in \mathbb{R}^4\\
    \\a_0 \in \mathbb{U}, \; h \in \mathbb{U}^\perp\\
    \\dimU = 3, \; \mathbb{U}^\perp = \{\forall v \in \mathbb{R}^4 \, | \, \forall u \in \mathbb{U} \; (u, v) = 0\}\\
    \\\forall u \in \mathbb{U} \; \exists \lambda_i \in \mathbb{R}, \; i = 1, 2, 3 \, ; \; u = \sum_{i = 1}^3 \lambda_i b_i, \; u \neq \theta\\
    \\\forall u' \in \mathbb{U} \; \exists \mu_i \in \mathbb{R}, \; i = 1, 2, 3 \, ; \; u' = \sum_{i = 1}^3 \mu_i b_i\\
    \\(u, u') = \sum_{i = 1}^3\lambda_i \mu_i (b_i, b_i), \; (u, u') = 0 \iff u' = \theta\\
    \\\implies \mathbb{U}^\perp = \{\mathbb{R}^4 \cap \mathbb{U}\} \cup \{\theta\}, \; \mathbb{U} < \mathbb{R}^4 \implies dim \mathbb{U}^\perp = 4 - 3 + 0 = 1\\
    \\\implies \mathbb{U}^\perp \cap \mathbb{U} = \{\{\mathbb{R}^4 \cap \mathbb{U}\} \cup \{\theta\}\} \cap \mathbb{U} = \{\theta\}\\
    \\\implies R^4 = \mathbb{U} \oplus \mathbb{U}^\perp\\
    \\\implies a = a_0 + h \implies h = a - a_0\\
    \\a_0 \in \mathbb{U} \implies \exists \rho_i \in \mathbb{R}, \; i = 1, 2, 3 \, ; \; a_0 =  \sum_{i = 1}^3 \rho_i b_i\\
    \\h \in \mathbb{U}^\perp \implies (h, b_i) = 0 , \; i = 1, 2, 3\\
    \\\implies (a - a_0, b_i) = 0 , \; i = 1, 2, 3\\
    \\\implies (a - \sum_{j = 1}^3 \rho_j b_j, b_i) = 0 , \; i = 1, 2, 3\\
    \\\implies (a, b_i) - \sum_{j = 1}^3 \rho_j (b_j, b_i) = 0 , \; i = 1, 2, 3\\
    \\\implies \rho_i = \frac{(a, b_i)}{(b_i, b_i)} , \; i = 1, 2, 3 \; ((b_j, b_i) = 0 \iff i \neq j, \; j = 1, 2, 3)\\
    \\\rho_1 = \frac{(a, b_1)}{(b_1, b_1)} = -\frac{8}{10} = -\frac{4}{5}\\
    \\\rho_2 = \frac{(a, b_2)}{(b_2, b_2)} = \frac{7.9 + 10.15 + 3.5}{81 + 100 + 9} = \frac{228}{190} = \frac{12}{10} = \frac{6}{5}\\
    \\\rho_3 = \frac{(a, b_3)}{(b_3, b_3)} =\frac{19 + 21 - 45 + 5}{19^2 + 9 + 9 + 1} = \frac{40 - 40}{361 + 19} = \frac{0}{380} = 0\\
    \implies a_0 = -\frac{4}{5}b_1 + \frac{6}{5}b_2 = -\frac{4}{5}(0, 1, 0, 3) + \frac{6}{5}(0, 9, -10, -3) =\\
    \\ = (0, -\frac{4}{5}, 0, -\frac{12}{5}) + (0, \frac{54}{5}, -\frac{60}{5}, -\frac{18}{5}) = (0, \frac{50}{5}, -\frac{60}{5}, -\frac{30}{5}) =\\
    \\= (0, 10, -12, -6)\\
    \\\implies h = a - a_0 = (1, 7, -15, -5) - (0, 10, -12, -6) = (1, -3, -3, 1)\)
    \section{Задача 5.}
    \(M_e(\varphi) = A\\
    \\A = \begin{pmatrix}
        1 & -9 & 3\\
        -9 & 1 & -3\\
        3 & -3 & -7
    \end{pmatrix}\\
    \\\\f_\varphi(\lambda) = f_A(\lambda) = -\lambda^3 + trA\lambda^2 -(\Delta_{23} + \Delta_{13} + \Delta_{12})\lambda + detA\\
    \\trA = 1 + 1 - 7 = -5\\
    \\\Delta_{23} = \begin{vmatrix}
        1 & -3\\
        -3 & -7\\
    \end{vmatrix} = -7 - 9 = -16\\
    \\\Delta_{13} = \begin{vmatrix}
        1 & 3\\
        3 & -7
    \end{vmatrix} = -7 - 9 = -16\\
    \\\Delta_{12} = \begin{vmatrix}
        1 & -9\\
        -9 & 1
    \end{vmatrix} = 1 - 81 = -80\\
    \\detA = \begin{vmatrix}
        1 & -9 & 3\\
        -9 & 1 & -3\\
        3 & -3 & -7
    \end{vmatrix} = -7 + 9.9 + 9.9 - 9 - 9 + 7.9.9 =\\
    \\= 9^3 - 5^2 = (9.3)^2 - 5^2 = (27 - 5)(27 + 5) = 22.32 = 704\\
    \\\implies f_\varphi(\lambda) = -\lambda^3 + (-5)\lambda^2 -(-16 - 16 - 80)\lambda + 704 =\\
    \\= -\lambda^3 -5\lambda^2 + 112\lambda + 704\\
    \\\text{Потенциални корени на } f_\varphi:\\
    \\\pm 1, \; \pm 2, \; \pm 4, \; \pm 8, \; \pm 11, \; \pm 16, \; \pm 22, \; \pm 32, \; \pm 44, \; \pm 64, \; \pm 88, \; \pm 176, \; \pm 352, \; \pm 704\\
    \\f_\varphi(1) = -1 -5 + 112 + 704 \neq 0\\
    \\f_\varphi(-1) = 1 - 5 - 112 + 704 \neq 0\\
    \\f_\varphi(2) = -8 - 20 + 242 + 704 \neq 0\\
    \\f_\varphi(-2) = 8 - 20 - 242 + 704 \neq 0\\
    \\f_\varphi(4) = -64 - 80 + 448 + 704 \neq 0\\
    \\f_\varphi(-4) = 64 - 80 - 448 + 704 \neq 0\\
    \\f_\varphi(8) = -64.8 -5.64 + 112.8 + 704 \neq 0\\
    \\f_\varphi(-8) = 64.8 - 5.64 - 112.8 + 704 = 8.64 - 5.64 - 14.64 + 11.64 = (19 - 19).64 = 0\\
    \\\implies \lambda_1 = -8 \implies (\lambda + 8) \text{ е делител на } f_\varphi(\lambda)\\
    \\-\lambda^3 -5\lambda^2 + 112\lambda + 704 : \lambda + 8 = -\lambda^2 + 3\lambda + 88\\
    -\\
    -\lambda^3 - 8\lambda^2\\
    \noindent\rule{5cm}{0.4pt}\\
    \tab 3\lambda^2 + 112\lambda + 704\\
    \tab -\\
    \tab 3\lambda^2 + 24\lambda\\
    \tab \noindent\rule{4cm}{0.4pt}\\
    \tab \tab 88\lambda + 704\\
    \tab \tab -\\
    \tab \tab 88\lambda + 704\\
    \tab \tab \noindent\rule{3cm}{0.4pt}\\
    \tab \tab \tab 0\\
    \\\implies f_\varphi(\lambda) = (\lambda + 8)(-\lambda^2 + 3\lambda + 88)\\
    \\f_{\varphi_1}(\lambda) = -\lambda^2 + 3\lambda + 88\\
    \\f_{\varphi_1}(-8) = -64 - 24 + 88 = 0\\
    \\\implies \lambda_2 = -8\\
    \\\lambda_2 + \lambda_3 = 3 \quad -8 + \lambda_3 = 3\\
    \\\lambda_2.\lambda_3 = -88 \quad -8.\lambda_3 = -88\\
    \\\implies \lambda_3 = 11\\
    \\\lambda_{1,2} = -8\\
    \\\begin{pmatrix}
        9 & -9 & 3\\
        -9 & 9 & -3\\
        3 & -3 & 1
    \end{pmatrix} \to \begin{pmatrix}
        0 & 0 & 0\\
        0 & 0 & 0\\
        3 & -3 & 1
    \end{pmatrix} \to \begin{pmatrix}
        3 & -3 & 1
    \end{pmatrix}\\
    \\v_1 = (1, 0, -3)\\
    \\v_2 = (0, 1, 3)\\
    \\g_1 = v_1 = (0, 1, 3)\\
    \\g_2 = v_2 + \mu g_1 \; |(\quad , g_1)\\
    \\0 = (v_2, g_1) + \mu(g_1, g_1)\\
    \\\mu = -\frac{(v_2, g_1)}{(g_1, g_1)} = -\frac{-9}{10} = \frac{9}{10}\\
    \\g_2 = (0, 1, 3) + (\frac{9}{10}, 0, -\frac{27}{10}) = (\frac{9}{10}, 1, -\frac{3}{10})\\
    \\\lambda_3 = 11\\
    \\\begin{pmatrix}
        -10 & -9 & 3\\
        -9 & -10 & -3\\
        3 & -3 & -18
    \end{pmatrix} \to \begin{pmatrix}
        -10 & -9 & 3\\
        -9 & -10 & -3\\
        1 & -1 & -6
    \end{pmatrix} \to \begin{pmatrix}
        0 & -19 & 3 - 10.6\\
        0 & -19 & -3 - 9.6\\
        1 & -1 & -6
    \end{pmatrix}\\
    \\\\\begin{pmatrix}
        0 & 0 & 0\\
        0 & -19 & -57\\
        1 & -1 & -6
    \end{pmatrix} \to  \begin{pmatrix}
        0 & -1 & -3\\
        1 & -1 & -6
    \end{pmatrix} \to  \begin{pmatrix}
        0 & -1 & -3\\
        1 & 0 & -3
    \end{pmatrix}\\
    \\v_3 = (3, -3, 1)\\
    \\f_1 = (1, 0, -3), \quad e_1' = \frac{1}{|f_1|}f_1 = (\frac{1}{\sqrt{10}}, 0, -\frac{3}{\sqrt{10}})\\
    \\f_2 = (9, 10, 3), \quad e_2' = \frac{1}{|f_2|}f_2 = (\frac{9}{\sqrt{190}}, \frac{10}{\sqrt{190}}, \frac{3}{\sqrt{190}})\\
    \\f_3 = (3, -3, 1), \quad e_3' = \frac{1}{|f_3|}f_3 = (\frac{3}{\sqrt{19}}, -\frac{3}{\sqrt{19}}, \frac{1}{\sqrt{19}})\\
    \\M_{e'}(\varphi) = \begin{pmatrix}
        -8 & 0 & 0\\
        0 & -8 & 0\\
        0 & 0 & 11
    \end{pmatrix}
    \)
\end{document}