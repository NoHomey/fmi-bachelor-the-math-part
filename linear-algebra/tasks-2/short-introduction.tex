\documentclass[12pt]{article}
\usepackage[left=3cm,right=3cm,top=1cm,bottom=2cm]{geometry}
    
\usepackage{amsmath,amsthm}
\usepackage{lipsum}
\usepackage{amssymb}
\usepackage{stmaryrd}
\usepackage[T1,T2A]{fontenc}
\usepackage[utf8]{inputenc}
\usepackage[bulgarian]{babel}
\usepackage[normalem]{ulem}
    
\setlength{\parindent}{0mm}

\newcommand{\R}{\mathbb{R}}

\newcommand{\Q}{\mathbb{Q}}

\newcommand*{\mcirc}[1]{\text{\raisebox{.5pt}{\textcircled{\raisebox{-.9pt} {#1}}}}}

\title{Практическо ръководство за решане на задачи за втората част от курса. Кратка версия}
\author{Иво Стратев}
    
\begin{document}
\maketitle

\section*{Алгоритъм за намиране на максимална лин. независима подсистема на система вектори}

Работим в линейно простраство $V$, което е описано по някакъв начин
(най-често чрез един свой базис) \\

\textbf{Вход}: Дадени са ни вектори $v_1, \; v_2, \; \dots, \; v_n \in V$ \\

\textbf{Изход}: Максимално лин. независима подсистема

на системата вектори $\{v_1, \; v_2, \; \dots, \; v_n\} \subset V$ \\

\textbf{Процедура}: Нареждаме векторите от координати на дадените вектори по редове.
Прилагаме Метода на Гаус или Гаус-Жордан, ако има линейно зависими вектори сред дадените,
то на някои от стъпките ще получаваме нулеви вектори (редове само от нули).
Максимално лин. независима подсистема се получава като за всеки ненулев ред след приключването
на избрания от нас метод вземем вектора $v_i$ от дадените, чийто индекс отговаря на ненулевя ред,
спрямо началното подреждане на редовете. \\

\textbf{Практическо приложение}: Нека $U = l(v_1, \; v_2, \; \dots, \; v_n)$.
Тогава алгоритъма връща множество от вектори, което е минимално по брой вектори и линейнаата му
обвивка съвпада с линейната обвивка на дадените вектори. Тоест алгоритъма дава директен отговор
на въпроса коя е една минимална (по брой в нея вектори) линейната обвивка,
която да съвпада с линейната обвивка на дадената сиситема. \\

Алгоритъма също така "строй" \; множество от вектори, чиято линейна обвивка отново съвпада с линейната обвивка
на дадената сиситема, но има повече на брой нули (ако използваме Метода на Гаус-Жордан, броят на нулите се
максимилизира, това е практически ценно когато дадено линейно пространство е зададено
като линейна обвивка на някакви вектори.) Това множество са векторите получени след прилагането на
избрания от нас метод, тоест получените ненулеви редове, които използваме за да определим
макс. лин. независимата подсистема. \\

\textbf{Пример}: В лин. пространство $\R^5$ спрямо стандартния базис са ни дадени векторите: \\

$a_1 = (0, \; 6, \; 6, \; 1, \; 0) \\
a_2 = (3, \; 1, \; 1, \; 0, \; 0) \\
a_3 = (1, \; -1, \; 3, \; 1, \; -2) \\
a_4 = (-2, \; 3, \; 1, \; 0, \; 1) \\
a_5 = (2, \; 3, \; 5, \; 1, \; -1) \\
a_6 = (1, \; -6, \; 4, \; 2, \; -5)$ \\

Да се намери една максимална лин. независима подсистема на системата вектори: $a_1, \; a_2, \; a_3, \; a_4, \; a_5$. \\

\textbf{Решение}: \\

Дадените вектори представляват координати вектори сами на себе си спрямо стандартния базис.
Тогава матрицата от която стартираме е: \\

$\begin{pmatrix}
     0 &  6 & 6 & 1 &  0 \\
     3 &  1 & 1 & 0 &  0 \\
     \mcirc{1} & -1 & 3 & 1 & -2 \\
    -2 &  3 & 1 & 0 &  1 \\
     2 &  3 & 5 & 1 & -1 \\
     1 & -6 & 4 & 2 & -5
\end{pmatrix} \begin{matrix}
    \\
    R_2 - 3R_3 \\
    \\
    R_4 + 2R_3 \\
    R_5 - 2R_3 \\
    R_6 -  R_3
\end{matrix} \to
\begin{pmatrix}
    0 &  6 &  6 &  1 &  0 \\
    0 &  4 & -8 & -3 &  6 \\
    1 & -1 &  3 &  1 & -2 \\
    0 &  1 &  7 &  2 & -3 \\
    0 &  5 & -1 & -1 &  3 \\
    0 & -5 &  1 & \mcirc{1} & -3
\end{pmatrix} \begin{matrix}
    R_1 -  R_6 \\
    R_2 + 3R_6 \\
    R_3 -  R_6 \\
    R_4 - 2R_6 \\
    R_5 +  R_6 \\
    \\
\end{matrix} \to \\\\\\
\begin{pmatrix}
    0 &  11 &  5 & 0 &  3 \\
    0 & -11 & -5 & 0 & -3 \\
    1 &   4 &  2 & 0 &  1 \\
    0 &  11 &  5 & 0 &  3 \\
    0 &   0 &  0 & 0 &  0 \\
    0 &  -5 &  1 & 1 & -3
\end{pmatrix} \begin{matrix}
    \\
    \\
    5R_3 \\
    \\
    \\
    -5R_6 \\
\end{matrix} \to \begin{pmatrix}
    0 &  11 &  5 &  0 &  3 \\
    0 & -11 & -5 &  0 & -3 \\
    1 &  20 & 10 &  0 &  5 \\
    0 &  11 & \mcirc{5} & 0 & 3 \\
    0 &   0 &  0 &  0 &  0 \\
    0 &  25 & -5 & -5 & 15
\end{pmatrix} \begin{matrix}
    R_1 -  R_4 \\
    R_2 +  R_4 \\
    R_3 - 2R_4 \\
    \\
    \\
    R_6 +  R_4 \\
\end{matrix} \to \\\\\\
\begin{pmatrix}
    0 &  0 &  0 &  0 &  0 \\
    0 &  0 & 0 &  0 & 0 \\
    1 &  -2 & 0 &  0 & -1 \\
    0 &  11 & 5 & 0 & 3 \\
    0 &   0 &  0 &  0 &  0 \\
    0 &  36 & 0 & -5 & 18
\end{pmatrix}$ \\\\

\textbf{Отговор}: Една максимална лин. независима под система на дадената е
системата вектори: $\{a_3, \; a_4, \; a_6\}$.

\section*{Алгоритъм за намиране на ранг на система вектори}

Работим в линейно простраство $V$, което е описано по някакъв начин
(най-често чрез един свой базис) \\

\textbf{Вход}: Дадени са ни вектори $v_1, \; v_2, \; \dots, \; v_n \in V$ \\

\textbf{Изход}: рангът на системата вектори $\{v_1, \; v_2, \; \dots, \; v_n\} \subset V$ \\

\textbf{Процедура}: Прилагаме алгоритъма, който търси максимална лин. независима подсистема на системата
от вектори $\{v_1, \; v_2, \; \dots, \; v_n\} \subset V$. Рангът на системата е броят на векторите
в максималната лин. независима подсистема. \\

\textbf{Пример}: В лин. пространство $\R^5$ спрямо стандартния базис са ни дадени векторите: \\

$a_1 = (0, \; 6, \; 6, \; 1, \; 0) \\
a_2 = (3, \; 1, \; 1, \; 0, \; 0) \\
a_3 = (1, \; -1, \; 3, \; 1, \; -2) \\
a_4 = (-2, \; 3, \; 1, \; 0, \; 1) \\
a_5 = (2, \; 3, \; 5, \; 1, \; -1) \\
a_6 = (1, \; -6, \; 4, \; 2, \; -5)$ \\

Да се намери рангът на системата вектори: $a_1, \; a_2, \; a_3, \; a_4, \; a_5$. \\

\textbf{Решение}: \\

Векторите са същите като в предния пример и вече знаем една максимална лин. независима подсистема
на дадената, това е системата от вектори: $\{a_3, \; a_4, \; a_6\}$. Тогава рангът на системата е $3$. \\

\textbf{Пример}: Да се намери една МЛНЗП и рангът на системата от следните полиноми с рационални коефициенти: \\

$p_1(x) = -3x^2  + x + 2 \\
p_2(x) = -5x^2 + x + 3 \\
p_3(x) = -7x^2 + 1 \\
p_4(x) = -x^2 + 2x + 4\\
p_5(x) = -2x^2 + 1$ \\

\textbf{Решение}: \\

Знаем, че ако $F$ е поле, то
$F^{n + 1}[x] = \left\{\displaystyle\sum_{k = 0}^n a_kx^k \; | \; a_0, \; \dots, \; a_n \in F  \right\}$
- множеството от полиномите с коефициенти от $F$ от степен не по-голяма от $n$ е линейно пространство с размерност $n + 1$. \\

Тоест знаем, че множеството $Q^3[x]$ е линейно пространство с нареден стандартен базис: $(1, \; x, \; x^2)$.
Тогава векторите от координати на дадените полиноми са: \\

$c_1 = (2, \; 1, \; -3) \\
c_2 = (3, \; 1, \; -5) \\
c_3 = (1, \; 0, \; -7) \\
c_4 = (4, \; 2, \; -1) \\
c_5 = (1, \; 0, \; -2)$ \\

Тяхната матрица от координати (написана по редове) е: \\

$\begin{pmatrix}
    2 & 1 & -3 \\
    3 & 1 & -5 \\
    1 & 0 & -7 \\
    4 & 2 & -1 \\
    1 & 0 & -2
\end{pmatrix}$ \\

Първо прилагаме алгоритъма за намеране на МЛНЗП, тоест: \\

$\begin{pmatrix}
    2 & \mcirc{1} & -3 \\
    3 & 1 & -5 \\
    1 & 0 & -7 \\
    4 & 2 & -1 \\
    1 & 0 & -2
\end{pmatrix} \begin{matrix}
    \\
    R_2 - R_1 \\
    \\
    R_4 - 2R_1 \\
    \\
\end{matrix} \to \begin{pmatrix}
    2 & 1 & -3 \\
    \mcirc{1} & 0 & -2 \\
    1 & 0 & -7 \\
    0 & 0 &  5 \\
    1 & 0 & -2
\end{pmatrix} \begin{matrix}
    R_1 - 2R_2 \\
    \\
    R_3 - R_2 \\
    \\
    R_5 - R_1 \\
\end{matrix} \to \begin{pmatrix}
    0 & 1 &  1 \\
    1 & 0 & -2 \\
    0 & 0 &  5 \\
    0 & 0 &  5 \\
    0 & 0 &  0
\end{pmatrix} \frac{1}{5} R_3 \to \\\\\\
\begin{pmatrix}
    0 & 1 &  1 \\
    1 & 0 & -2 \\
    0 & 0 &  \mcirc{1} \\
    0 & 0 &  5 \\
    0 & 0 &  0
\end{pmatrix} \begin{matrix}
    R_1 - R_3 \\
    R_2 + 2R_3
    \\
    R_4 - 5R_3\\
    \\
\end{matrix} \to \begin{pmatrix}
    0 & 1 &  1 \\
    1 & 0 &  0 \\
    0 & 0 &  1 \\
    0 & 0 &  0 \\
    0 & 0 &  0
\end{pmatrix}$ \\\\

Тогава една МЛНЗП на дадената системата от полиноми е: $\{p_1, \; p_2, \; p_3\}$. \\

Следователно рангът на системата е $3$.

\section*{Алгоритъм за намиране на ранг на матрица}

\textbf{Вход}: Дадени са ни е матрица $A \in F^{m \times n}$ - матрица с $m$ реда и $n$ стълба с елементи от дадено поле $F$. \\

\textbf{Изход}: рангът на матрицата $A$. \\

\textbf{Процедура}: Прилагаме алгоритъма за ранг на система вектори за редовете на матрицата $A$,
като си мислим, че редовете са вектори спрямо стандатния базис на лин. пространство $F^n$. \\

\textbf{Разяснение}: Ако трябва да търсим ранг на матрица, това е същото като за някое лин. пространство
спрямо фиксиран базис на това пространство да са ни дадени вектори като координатни вектори (редовете на матрицата $A$). \\

\textbf{Примери}: \\

Да се намери рангът на матрицата: $\begin{pmatrix}
    0 &  6 & 6 & 1 &  0 \\
    3 &  1 & 1 & 0 &  0 \\
    1 & -1 & 3 & 1 & -2 \\
   -2 &  3 & 1 & 0 &  1 \\
    2 &  3 & 5 & 1 & -1 \\
    1 & -6 & 4 & 2 & -5
\end{pmatrix}$ \\\\\\

\textbf{Решение}: \\

Вече търсихме рангът на системата вектори от редовете на матрица в първия пример след алгоритъма за ранг на система вектори.
Отговорът на задачата беше, че рангът на системата е $3$. \\

\textbf{Отговор}: $3$.

\section*{Алгоритъм за допълване на дадена система от вектори до базис базис на някое над пространство}

Работим в линейно простраство $V$, което е описано по някакъв начин
(най-често чрез един свой базис) \\

\textbf{Вход}: Дадено ни е подпростанство $U \leq V$ \\
и са ни дадени вектори $v_1, \; v_2, \; \dots, \; v_n \in U$ \\

\textbf{Изход}: Вектори, които допълват системата от дадените вектори да базис на $U$. \\

\textbf{Процедура}: Прилагаме алгоритъма за намиране на максимална лин. независима подсистема на система вектори
на системата. За допълване можем да използваме вектори от стандартния базис на $U$, като гледаме на този базис като
нареден базис (наредбата трябва да е същата, като тази изпозлвана за конструирането на началната матрица) и избираме
онези вектори от наредения базис с които като добавим към векторите от МЛНЗП (или тези от последната стъпка на избрания от нас метод)
и приложим метода на Гаус-Жордан да можем да получим единичната матрица (матрицата, която има единици по главния диагонал и нули под и над главния диагонал). \\

\textbf{Забележка}: Ако за намирането на МЛНЗП използваме метода на Гаус-Жордан допълването до базис става с
вектори от стандартния базис, чиийто индекс отговаря на стълбовете, в които не сме избирали водещ елемент (тоест. елемент с който да правим нули). \\

\textbf{Практическо приложение}: Можем да приемем, че алгоритъма дава отговор на следния въпрос,
кое е линейното пространство $X$ със свойството: $X \oplus U = V$. Отговорът на този въпрос е: \\

$X = l(d_1, \; \dots, d_k)$, където $d_1, \; \dots, d_k$ са векторите, които допълват $\{v_1, \; v_2, \; \dots, \; v_n\}$
до базис на $U$. Разбира се ако се окаже, че $l(v_1, \; v_2, \; \dots, \; v_n) = U$, то допълнението е празното множество,
чиято линейна обвивка съвпада с нулевото лин. пространство. (тоест $l(\emptyset) = \{\theta\}$). \\

\textbf{Пример}: \\

Ще работим в пространстовото $\Q^{2 \times 3}$. \\

Нека $
M_1 = \begin{pmatrix}
     1 & 0 & 2 \\\
    -1 & 0 & 1
\end{pmatrix} \quad M_2 = \begin{pmatrix}
    -2 & 0 & -4 \\\
     2 & 0 & -2
\end{pmatrix} \quad M_3 = \begin{pmatrix}
    3 & 0 & 5 \\\
    1 & 0 & 2
\end{pmatrix} \quad M_4 = \begin{pmatrix}
    0 & 0 & 0 \\\
    1 & 0 & -3
\end{pmatrix}$ \\

Да се допълни системата $M_1, \; M_2, \; M_3, \; M_4$ до базис на пространството \\

$U = \left\{\begin{pmatrix}
    a_{11} & 0 & a_{13} \\\
    a_{21} & 0 & a_{23}
\end{pmatrix} \; | \; a_{11}, \; a_{13}, \; a_{21}, \; a_{23} \in \Q \right\}$ \\

\textbf{Решение}: \\

Стандартният базис на $U$ са матриците: $E_{11}, \; E_{13}, \; E_{21}, \; E_{23}$.
Без проблем можем да намерим координатите вектори на дадените матрици като изразим всяка
матрица като лин. комбинация на стандартния базис на $U$. \\

$M_1 = 1.E_{11} + 2E_{13} +  (-1).E_{21} + 1.E_{23} \\
M_2 = (-2).E_{11} + (-4).E_{13} + 2.E_{21} + (-2).E_{23} \\
M_3 = 3.E_{11} + 5.E_{13} + 1.E_{21} + 2.E_{23} \\
M_4 = 0.E_{11} + 0.E_{13} + 1.E_{21} + (-3).E_{23}$ \\

Тогава матрицата от координатните вектори на дадените матирци е: \\

$\begin{pmatrix}
     1 &  2 & -1 &  1 \\
    -2 & -4 &  2 & -2 \\
     3 &  5 &  1 &  2 \\
     0 &  0 &  1 & -3 
\end{pmatrix}$ \\\\

Досега все прилагахме метода на Гаус-Жордан, този път ще приложим метода на Гаус и
освен, че ще допълним до базис ще намерим една МЛНЗП на дадена система, както и рангът ѝ. \\

$\begin{pmatrix}
    \mcirc{1} &  2 & -1 &  1 \\
   -2 & -4 &  2 & -2 \\
    3 &  5 &  1 &  2 \\
    0 &  0 &  1 & -3 
\end{pmatrix} \begin{matrix}
    \\
    R_2 + 2R_1 \\
    R_3 - 3R_1
    \\
\end{matrix} \to \begin{pmatrix}
    1 &  2 & -1 &  1 \\
    0 &  0 &  0 &  0 \\
    0 & -1 &  4 & -1 \\
    0 &  0 &  1 & -3 
\end{pmatrix} swap(R_2, \; R_3) \to \\\\\\
\begin{pmatrix}
    1 &  2 & -1 &  1 \\
    0 & -1 &  4 & -1 \\
    0 &  0 &  0 &  0 \\
    0 &  0 &  1 & -3 
\end{pmatrix} swap(R_3, \; R_4) \to \begin{pmatrix}
    1 &  2 & -1 &  1 \\
    0 & -1 &  4 & -1 \\
    0 &  0 &  1 & -3 \\
    0 &  0 &  0 &  0
\end{pmatrix}$ \\\\

Методът на Гаус спира до тук, тогава една МЛНЗП на дадената система е системата от матрици:
$\{M_1, \; M_3, \; M_4\}$, което значи че рангът е $3$. Стандартният базис на $U$ са матриците:
$E_{11}, \; E_{13}, \; E_{21}, \; E_{23}$. Тогава едно допълнение до базис на $U$ е матрицата $E_{13}$. \\

Нека се убедим, че с нейна помощ при прилагането на метода на Гаус-Жордан, можем да стигнем до единичната матрица. \\

$\begin{pmatrix}
    1 &  2 & -1 &  1 \\
    0 & -1 &  4 & -1 \\
    0 &  0 &  1 & -3 \\
    0 &  0 &  0 & \mcirc{1}
\end{pmatrix} \begin{matrix}
    R_1 - R_4 \\
    R_2 + R_4 \\
    R_3 + 3R_4
    \\
\end{matrix} \to \begin{pmatrix}
    1 &  2 & -1 & 0 \\
    0 & -1 &  4 & 0 \\
    0 &  0 &  1 & 0 \\
    0 &  0 &  0 & 1
\end{pmatrix} \begin{matrix}
    R_1 + R_3 \\
    R_2 - 4R_3 \\
    \\
    \\
\end{matrix} \to \\\\\\
\begin{pmatrix}
    1 &  2 & 0 & 0 \\
    0 & -1 & 0 & 0 \\
    0 &  0 & 1 & 0 \\
    0 &  0 & 0 & 1
\end{pmatrix} \begin{matrix}
    R_1 + 2R_2 \\
    \\
    \\
    \\
\end{matrix} \to \begin{pmatrix}
    1 &  0 & 0 & 0 \\
    0 & -1 & 0 & 0 \\
    0 &  0 & 1 & 0 \\
    0 &  0 & 0 & 1
\end{pmatrix} -1R_2 \to \begin{pmatrix}
    1 & 0 & 0 & 0 \\
    0 & 1 & 0 & 0 \\
    0 & 0 & 1 & 0 \\
    0 & 0 & 0 & 1
\end{pmatrix}$ \\\\

Получаваме, че $l(M_1, \; M_3, \; M_4) \oplus l(E_{13}) = U = l(E_{11}, \; E_{13}, \; E_{21}, \; E_{23})$. \\

\textbf{Отговор}: $E_{13}$. \\

Да разгледаме пак прилагането на метода на Гаус-Жордан в най-първия пример: \\

$\begin{pmatrix}
    0 &  6 & 6 & 1 &  0 \\
    3 &  1 & 1 & 0 &  0 \\
    \mcirc{1} & -1 & 3 & 1 & -2 \\
   -2 &  3 & 1 & 0 &  1 \\
    2 &  3 & 5 & 1 & -1 \\
    1 & -6 & 4 & 2 & -5
\end{pmatrix} \begin{matrix}
   \\
   R_2 - 3R_3 \\
   \\
   R_4 + 2R_3 \\
   R_5 - 2R_3 \\
   R_6 -  R_3
\end{matrix} \to
\begin{pmatrix}
   0 &  6 &  6 &  1 &  0 \\
   0 &  4 & -8 & -3 &  6 \\
   1 & -1 &  3 &  1 & -2 \\
   0 &  1 &  7 &  2 & -3 \\
   0 &  5 & -1 & -1 &  3 \\
   0 & -5 &  1 & \mcirc{1} & -3
\end{pmatrix} \begin{matrix}
   R_1 -  R_6 \\
   R_2 + 3R_6 \\
   R_3 -  R_6 \\
   R_4 - 2R_6 \\
   R_5 +  R_6 \\
   \\
\end{matrix} \to \\\\\\
\begin{pmatrix}
   0 &  11 &  5 & 0 &  3 \\
   0 & -11 & -5 & 0 & -3 \\
   1 &   4 &  2 & 0 &  1 \\
   0 &  11 &  5 & 0 &  3 \\
   0 &   0 &  0 & 0 &  0 \\
   0 &  -5 &  1 & 1 & -3
\end{pmatrix} \begin{matrix}
   \\
   \\
   5R_3 \\
   \\
   \\
   -5R_6
\end{matrix} \to \begin{pmatrix}
   0 &  11 &  5 &  0 &  3 \\
   0 & -11 & -5 &  0 & -3 \\
   1 &  20 & 10 &  0 &  5 \\
   0 &  11 & \mcirc{5} & 0 & 3 \\
   0 &   0 &  0 &  0 &  0 \\
   0 &  25 & -5 & -5 & 15
\end{pmatrix} \begin{matrix}
   R_1 -  R_4 \\
   R_2 +  R_4 \\
   R_3 - 2R_4 \\
   \\
   \\
   R_6 +  R_4
\end{matrix} \to \\\\\\
\begin{pmatrix}
   0 &  0 &  0 &  0 &  0 \\
   0 &  0 & 0 &  0 & 0 \\
   1 &  -2 & 0 &  0 & -1 \\
   0 &  11 & 5 & 0 & 3 \\
   0 &   0 &  0 &  0 &  0 \\
   0 &  36 & 0 & -5 & 18
\end{pmatrix}$ \\\\

Работихме с първи, четвърти и трети стълб. От забележката следва,
че едно гарантирано допълване до базис на $\R^5$ на дадената система
са векторите от стандартния базис на $\R^5$: $e_2, \; e_5$.
Нека отново се убедим, че с тяхна помощ можем да стигнем до единичната матрица
при прилагане на метода на Гаус-Жордан върху получената матрица: \\

$\begin{pmatrix}
    0 &  \mcirc{1} &  0 &  0 &  0 \\
    0 &  0 & 0 &  0 & 1 \\
    1 &  -2 & 0 &  0 & -1 \\
    0 &  11 & 5 & 0 & 3 \\
    0 &  36 & 0 & -5 & 18
\end{pmatrix} \begin{matrix}
    \\
    \\
    R_3 + 3R_1 \\
    R_4 - 11R_1 \\
    R_5 - 36R_1
\end{matrix} \to \begin{pmatrix}
    0 & 1 & 0 &  0 &  0 \\
    0 & 0 & 0 &  0 &  \mcirc{1} \\
    1 & 0 & 0 &  0 & -1 \\
    0 & 0 & 5 &  0 &  3 \\
    0 & 0 & 0 & -5 & 18
\end{pmatrix} \begin{matrix}
    \\
    \\
    R_3 + R_2 \\
    R_4 - 3R_2 \\
    R_5 - 18R_2
\end{matrix} \to \\\\\\
\begin{pmatrix}
    0 & 1 & 0 &  0 & 0 \\
    0 & 0 & 0 &  0 & 1 \\
    1 & 0 & 0 &  0 & 0 \\
    0 & 0 & 5 &  0 & 0 \\
    0 & 0 & 0 & -5 & 0
\end{pmatrix} \begin{matrix}
    \\
    \\
    \\
    \frac{1}{5}R_4 \\
    -\frac{1}{5}R_5 \\
\end{matrix} \to \begin{pmatrix}
    0 & 1 & 0 & 0 & 0 \\
    0 & 0 & 0 & 0 & 1 \\
    1 & 0 & 0 & 0 & 0 \\
    0 & 0 & 1 & 0 & 0 \\
    0 & 0 & 0 & 1 & 0
\end{pmatrix} \to \begin{pmatrix}
    1 & 0 & 0 & 0 & 0 \\
    0 & 1 & 0 & 0 & 0 \\
    0 & 0 & 1 & 0 & 0 \\
    0 & 0 & 0 & 1 & 0 \\
    0 & 0 & 0 & 0 & 1
\end{pmatrix} $

\section*{Дефиниция за ФСР - Фундаментална система решения на хомогенна система}

ФСР наричаме всеки базис линейното пространство от решения на хомогенната система.
Ако решението на дадена хомогенна система е нулевото (тоест тя има еднствено решение) казваме,
че системата не притежава ФСР или, че ФСР на системата е празното множество.

\section*{Алгоритъм за намиране на ФСР}

\textbf{Вход}: Дадена ни е хомогенна система. \\

\textbf{Изход}: ФСР на дадената система. \\

\textbf{Процедура}: Решавайки дадена хомогенна система накрая получаваме решенията и в зависимост
от някакви параметри (ако решението не е единствено). Можем да си мислим, че полученото решение,
коеот зависи (или не зависи) от параметри задава изображение. Нека системата е в общ вид с
$m$ уравнения и $n$ нейзвестни с коефициенти от полето $F$,
след прилагането на метода на Гаус-Жордан върху матрицата от коефициенти на системата достигаме
до $k$ на брой уравнения. Тогава броят на параметрите, от
които зависи решението на системата е $n - k$. Тогава решенията могат да се представят като избораженеи:
$f : F^{n - k} \to F^n$, ако $n = k$, то $f \equiv \theta$. Броят на векторите във ФСР на системата е
точно $n - k$ и всеки се получава по следния начин $b_i = f(e_i), \; i = 1, \; \dots, \; n - k$,
където векторите $e_1, \; \dots, \; e_{n - k}$ са стандартният базис на пространството $F^{n - k}$. \\

\textbf{Практическо приложение}: Ако вземем линейна обвивка на векторите от ФСР на дадена система
получаваме еквивалетно представяне на множеството от решенията ѝ. Тоест намираме еквивалетно
представяне на едно пространство зададено като хомогенна система под формата на линейна обвивка. \\

\textbf{Пример}: \\

Да представи като линекна обвивка линейното пространство $W$, задено като решения на хомогенна система: \\

$W : \begin{array}{|l@{}}
    x_1 + x_2 + 2x_3 + 2x_4 = 0 \\
    x_1 + x_2 - 2x_3 - 2x_4 = 0
\end{array}$ \\

\textbf{Решение}: \\

$\begin{pmatrix}
    \mcirc{1} & 1 &  2 &  2 \\
    1 & 1 & -2 & -2
\end{pmatrix} R_2 - R_1 \to \begin{pmatrix}
    1 & 1 &  2 &  2 \\
    0 & 0 & -4 & -4
\end{pmatrix} -\frac{1}{4}R_2 \to \\\\\\
\begin{pmatrix}
    1 & 1 & 2 & 2 \\
    0 & 0 & \mcirc{1} & 1
\end{pmatrix} R_1 - 2R_2 \to \begin{pmatrix}
    1 & 1 & 0 & 0 \\
    0 & 0 & 1 & 1
\end{pmatrix} $ \\\\

Чрез метода на Гаус-Жордан получихме еквивалентна хомогенна система на дадена, тоест: \\

$W : \begin{array}{|lccccccr}
    x_1 & + & x_2 & ~ & ~   & ~ & ~   & = 0 \\
    ~   & ~ & ~   & ~ & x_3 & + & x_4 & = 0
\end{array}$ \\

За параметри си избираме онези $x_i$ такива, че не сме избирали елемент от $i$-тия стълб за водещ.
Тоест полагаме $x_2 = p, \; x_4 = q$. Тогава общия вид на решенията на системата е: \\

$(x_1, \; x_2, \; x_3, \; x_4) = (-p, \; p, \; -q, \; q)$.
Да кажем, че ни интересуват само реалните решения на системата (тоест решаваме системата над полето на реалните числа).
Тогава функцията, която описва решенията е: $f : \R^2 \to \R^4 \; : \; f(p, \; q) = (-p, \; p, \; -q, \; q)$. \\

Тогава едно ФСР на системата са векторите: \\

$p = 1, \; q = 0 \implies b_1 = (-1, \; 1, \; 0, \; 0) \\
p = 0, \; q = 1 \implies b_2 = (0, \; 0, \; -1, \; 1)$. \\

С други думи: \\

$W = l(b_1, \; b_2) = l(f(e_1), \; f(e_2)) = l(f(1, \; 0), \; f(0, \; 1)) = l((-1, \; 1, \; 0, \; 0), (0, \; 0, \; -1, \; 1))$

\section*{Алгоритъм за преминаване от лин. обвивка в хом. система}

Алгоритъм за намиране на хомогенна система, пространството от решения,
което съвпада с пространство зададено като лин. обвивка. \\

\textbf{Вход}: $U = l(a_1, \; \dots, \; a_n)$. \\

\textbf{Изход}: хомогенна система, пространството от решения,
което съвпада с $U$. \\

\textbf{Процедура}: построяваме матрица от координатите на дадените вектори записани по редове,
тази система задава хомогенна система. Намираме ФСР на хомогенната система и използваме координатите
на векторите във ФСР на системата за коефициенти на хомогенна система, това е търсената система. \\

\textbf{Пример}: \\

Да се намери хомогенна система, пространството от решения,
което съвпада с пространство $U = l(a_1, \; a_2)$, където: \\

$a_1 = (1, \; 1, \; 2, \; 2), \; a_2 = (1, \; -1, \; 2, \; -2)$. \\

\textbf{Решение}: \\

$\begin{pmatrix}
    \mcirc{1} &  1 & 2 &  2 \\
    1 & -1 & 2 & -2
\end{pmatrix} R_2 - R_1 \to \begin{pmatrix}
    1 &  1 & 2 &  2 \\
    0 & -2 & 0 & -4
\end{pmatrix} -\frac{1}{2} R_2 \to \\\\\\
\begin{pmatrix}
    1 &  1 & 2 &  2 \\
    0 & \mcirc{1} & 0 & 2
\end{pmatrix} R_1 - R_2 \to \begin{pmatrix}
    1 & 0 & 2 & 0 \\
    0 & 1 & 0 & 2
\end{pmatrix}$ \\\\

От тук получаваме че $U = l((1, \; 0, \; 2, \; 0), (0, \; 1, \; 0, \; 2))$ \\

За свободни параметри избираме $x_3 = p, \; x_4 = q$. Общото решение на системата е : \\

$(x_1, \; x_2, \; x_3, \; x_4) = (-2p, \; -2q, \; p, \; q)$. \\

Тогава ФСР на системата с коефициенти векторите $a_1, \; a_2$ са: \\

$p = 1, \; q = 0 \implies b_1 = (-2, \; 0, \; 1, \; 0) \\
p = 0, \; q = 1 \implies b_2 = (0, \; -2, \; 0, \; 1)$ \\

Значи $U$ съвпада с решенията на хомогенната система: \\

$U : \begin{array}{|lccccccr}
    -2x_1 & ~     & ~ & + & x_3 & ~ & ~   & = 0 \\
    ~     & -2x_2 & ~ & ~ & ~   & + & x_4 & = 0
\end{array}$

\section*{Алгоритъм намиране на базиси на $U + W$ и $U \cap W$}

Работим в линейно простраство $V$, което е описано по някакъв начин
(или е ясно от контекста на задачата) \\

\textbf{Вход}: $U, \; W$ - две лин. подпространства на $V$ . \\

\textbf{Изход}: базиси на $U + W$ и $U \cap W$. \\

\textbf{Процедура}: За намирането на базис на $U + W$ ни трябват двете подпространства
да са зададени като лин. обвивки. Ако някое от тях е зададено като хомогенна система
му търсим ФСР, което веднага ни дава негово представяне като лин. обвивка.
Записваме координатите на векторите от двете лин. обвивки по редове и прилагаме
метода на Гаус или Гаус-Жордан, базис на $U + W$ са всички ненулеви вектори. \\

За намирането на базис на $U \cap W$ ни трябват двете подпространства
да са зададени като пространства от решенията на хомогнна система.
Ако някое от тях е зададено като лин. обвивка прилагаме
Алгоритъм за преминаване от лин. обвивка в хом. система за него.
Взимаме сечението на двете системи като ги запишем една под друга и си мислим, че са една система,
на която тръсим ФСР, то е базис на $U \cap W$. \\

\textbf{Забележка}: Понякога можем да си спестим сметки или да имаме очакване за резултата ако използваме
формулата за размерностите: \\

$dim(U + W) = dim(U) + dim(W) - dim(U \cap W)$ \\

\textbf{Пример}: Да се намерят базиси на $U + W$ и $U \cap W$, ако: \\

$U = l(a_1, \; a_2)$, където: $a_1 = (1, \; 1, \; 2, \; 2), \; a_2 = (1, \; -1, \; 2, \; -2)$. \\

$W : \begin{array}{|l@{}}
    x_1 + x_2 + 2x_3 + 2x_4 = 0 \\
    x_1 + x_2 - 2x_3 - 2x_4 = 0
\end{array}$ \\

\textbf{Решение}: От предходните два примера получихме представяне на $W$
като еквивалентна хомогенна система и линейна обвивка (ФСР) и на $U$
намерихме представяне като еквивалентна лин. обвивка и хомогенна система, тоест: \\

$W : \begin{array}{|lccccccr}
    x_1 & + & x_2 & ~ & ~   & ~ & ~   & = 0 \\
    ~   & ~ & ~   & ~ & x_3 & + & x_4 & = 0
\end{array}$ \quad $U : \begin{array}{|lccccccr}
    -2x_1 & ~     & ~ & + & x_3 & ~ & ~   & = 0 \\
    ~     & -2x_2 & ~ & ~ & ~   & + & x_4 & = 0
\end{array} \\\\
W = l((-1, \; 1, \; 0, \; 0), (0, \; 0, \; -1, \; 1)) \quad U = l((1, \; 0, \; 2, \; 0), (0, \; 1, \; 0, \; 2))$ \\\\

Първо ще намир базис на $U + W$: \\

$\begin{pmatrix}
     \mcirc{1} & 0 &  2 & 0 \\
     0 & 1 &  0 & 2 \\
    -1 & 1 &  0 & 0 \\
     0 & 0 & -1 & 1
\end{pmatrix} R_3 + R_1 \to \begin{pmatrix}
    1 & 0 &  2 & 0 \\
    0 & \mcirc{1} &  0 & 2 \\
    0 & 1 & 2 & 0 \\
    0 & 0 & -1 & 1
\end{pmatrix} R_3 - R_2 \to \\\\\\
\begin{pmatrix}
    1 & 0 &  2 &  0 \\
    0 & 1 &  0 &  2 \\
    0 & 0 &  2 & -2 \\
    0 & 0 & \mcirc{-1} &  1
\end{pmatrix} \begin{matrix}
    R_1 + 2R_4 \\
    \\
    R_3 + 2R_4
\end{matrix} \to \begin{pmatrix}
    1 & 0 &  0 & 2 \\
    0 & 1 &  0 & 2 \\
    0 & 0 &  0 & 0 \\
    0 & 0 & -1 & 1
\end{pmatrix}$ \\\\

Един базис на $U + W$ са векторите: \\

$(1, \; 0, \; 0, \; 2) \\
(0, \; 1, \; 0, \; 2) \\
(0, \; 0, \; -1, \; 1)$ \\

От формулата за размерностите получаваме: \\

$dim(U + W) = dim(U) + dim(W) - dim(U \cap W) \implies \\\\
3 = 2 + 2 - dim(U \cap W) \implies dim(U \cap W) = 1$ \\

Това значи, че  решенията на системата, която се получава при слепянето на двете системи зависи от един параметър
или матрицата от коефициенти на уравненията има три лин. независими реда. \\

$\begin{pmatrix}
     \mcirc{1} &  1 & 0 & 0 \\
     0 &  0 & 1 & 1 \\
    -2 &  0 & 1 & 0 \\
     0 & -2 & 0 & 1
\end{pmatrix} R_3 + 2R_1 \to \begin{pmatrix}
    1 & 1 & 0 & 0 \\
    0 & 0 & 1 & 1 \\
    0 & 0 & \mcirc{1} & 0 \\
    0 & 0 & 0 & 1
\end{pmatrix} R_2 - R_3 \to \\\\\\
\begin{pmatrix}
    1 & 1 & 0 & 0 \\
    0 & 0 & 0 & 1 \\
    0 & 0 & 1 & 0 \\
    0 & 0 & 0 & \mcirc{1}
\end{pmatrix} R_2 - R_4 \to \begin{pmatrix}
    1 & 1 & 0 & 0 \\
    0 & 0 & 0 & 0 \\
    0 & 0 & 1 & 0 \\
    0 & 0 & 0 & 1
\end{pmatrix} \implies \\\\\\
x_2 = t \implies (x_1, \; x_2, \; x_3, \; x_4) = (-t, \; t, \; 0, \; 0)$. \\

Тогава един базис на $U \cap W$ е вектора $(-1, \; 1, \; 0, \; 0)$.

\end{document}