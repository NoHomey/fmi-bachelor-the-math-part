\documentclass[12pt]{article}
\usepackage[utf8]{inputenc}

\usepackage[T2A]{fontenc}
\usepackage[english,bulgarian]{babel}
\def\frak#1{\cal #1}

\usepackage{amsmath}
\usepackage{amsthm}
\usepackage{amssymb}
\usepackage{graphicx}
\usepackage{shuffle}
\usepackage{alltt}
\usepackage{enumerate}
\newtheorem{theorem}{Теорема}%[section]
\newtheorem{problem}{Задача}%[section]
\newtheorem{remark}{{Забележка}}%[section]
\newtheorem{example}{Пример}%[section]
\newtheorem{lemma}{{Лема}}%[section]
\newtheorem{proposition}{Твърдение}%[section]
\def\proof{\textbf {Доказателство: }}%[section]
\newtheorem{corollary}{Следствие}%[section]
\newtheorem{fact}{Факт}%[
\newtheorem{definition}{Определение}%[section]

\author{Иво Стратев}
\title{Отговори на теоритично контролно №2, Линейна алгебра, Информатика}

\begin{document}
\maketitle

\begin{definition}[Линейно изображение]
Нека \((V, \underset{V}{+}, \underset{V}{.})\) и \((W, \underset{W}{+}, \underset{W}{.})\) са Л.П-ва над поле \((F, \underset{F}{+}, \underset{F}{.})\).
Нека \(\varphi \; : \; V \to W\) е изображение от \(V\) във \(W\).
\(\varphi\) е линейно изображение, ако:
\begin{align*}
(\forall v_1 \in V)(\forall v_2 \in V)[\; \varphi(v_1 \underset{V}{+} v_2) = \varphi(v_1) \underset{W}{+} \varphi(v_2)\;] \\
(\forall \lambda \in F)(\forall v \in V)[\; \varphi(\lambda \underset{V}{.} v) = \lambda \underset{W}{.} \varphi(v)\;] 
\end{align*}
\end{definition}

\begin{remark}
Пишем \(\varphi \in Hom(V, W)\), ако \(\varphi \; : \; V \to W\) е линейно изображение.
\end{remark}

\begin{definition}[Линеен оператор]
Нека \((V, \oplus, \odot)\) е Л.П. над поле \((F, +, .)\).
Нека \(\varphi \; : \; V \to V\) е изображение от \(V\) във \(V\).
\(\varphi\) е линеен оператор, ако:
\begin{align*}
(\forall v_1 \in V)(\forall v_2 \in V)[\; \varphi(v_1 \oplus v_2) = \varphi(v_1) \oplus \varphi(v_2)\;] \\
(\forall \lambda \in F)(\forall v \in V)[\; \varphi(\lambda \odot v) = \lambda \odot \varphi(v)\;] 
\end{align*}
\end{definition}

\begin{remark}
Пишем \(\varphi \in Hom(V)\), ако \(\varphi\) е линеен оператор.
\end{remark}

\begin{theorem}[Съществуване и единственост на Л.И.]
Нека \((V, \underset{V}{+}, \underset{V}{.})\) и \((W, \underset{W}{+}, \underset{W}{.})\) са Л.П-ва над поле \((F, \underset{F}{+}, \underset{F}{.})\).
\\Нека \(V\) е крайномерно и \(n = dim(V)\).
Нека \(b_1, b_2, \dots, b_n\) е базис на \(V\).
\\Нека \(w_1, w_2, \dots, w_n \in W\).
Тогава съществува единствено \(\varphi \in Hom(V, W)\), такова че
\begin{align*}
(\forall i \in \{1, 2, \dots, n\})[\; \varphi(b_i) = w_i\;]
\end{align*}
\end{theorem}

\begin{definition}[Изоморфизъм на Л.П-ва]
Нека \((V, \underset{V}{+}, \underset{V}{.})\) и \((W, \underset{W}{+}, \underset{W}{.})\) са Л.П-ва над поле \((F, \underset{F}{+}, \underset{F}{.})\).
\((V, \underset{V}{+}, \underset{V}{.})\) е изоморфно с \((W, \underset{W}{+}, \underset{W}{.})\),
ако съществува \(\varphi \in Hom(V, W)\), което е биекция.
\end{definition}

\begin{remark}
Пишем \((V, \underset{V}{+}, \underset{V}{.}) \cong (W, \underset{W}{+}, \underset{W}{.})\), ако
\((V, \underset{V}{+}, \underset{V}{.})\) е изоморфно с \((W, \underset{W}{+}, \underset{W}{.})\).
\end{remark}

\begin{proposition}[Н.Д.У. за изоморфизъм на К.М.Л.П-ва]
Нека \((V, \underset{V}{+}, \underset{V}{.})\) и \((W, \underset{W}{+}, \underset{W}{.})\) са К.М.Л.П-ва над поле \((F, \underset{F}{+}, \underset{F}{.})\).
В сила е:
\begin{align*}
(V, \underset{V}{+}, \underset{V}{.}) \cong (W, \underset{W}{+}, \underset{W}{.}) \iff dim(V) = dim(W)
\end{align*}
\end{proposition}

\begin{proposition}[Образа на нулевия вектор е нулевия при Л.И.]
Нека \((V, \underset{V}{+}, \underset{V}{.})\) и \((W, \underset{W}{+}, \underset{W}{.})\) са Л.П-ва над поле \((F, \underset{F}{+}, \underset{F}{.})\).
Нека \(\varphi \in Hom(V, W)\). Тогава \(\varphi(\theta_V) = \theta_W\).
\end{proposition}

\begin{proof}
Нека \(v \in V\). Тогава
\begin{align*}
\varphi(\theta_V) = \varphi(0 \underset{V}{.} v ) = 0 \underset{W}{.} \varphi(v) = \theta_W \qed
\end{align*}
\end{proof}

\begin{proposition}[Образа на противоположния вектор е противоположния на образа при Л.И.]
Нека \((V, \underset{V}{+}, \underset{V}{.})\) и \((W, \underset{W}{+}, \underset{W}{.})\) са Л.П-ва над поле \((F, \underset{F}{+}, \underset{F}{.})\).
Нека \(\varphi \in Hom(V, W)\). Тогава \((\forall v \in V)[\; \varphi(\underset{V}{-}v) = \underset{W}{-}\varphi(v) \; ]\).
\end{proposition}

\begin{proof}
Нека \(v \in V\). Тогава
\begin{align*}
\varphi(\underset{V}{-}v) = \varphi((-1) \underset{V}{.} v ) = (-1) \underset{W}{.} \varphi(v) = \underset{W}{-}\varphi(v) \qed
\end{align*}
\end{proof}

\begin{proposition}[Едно Л.И. изпраща Л.З-ми в Л.З-ми]
Нека \((V, \underset{V}{+}, \underset{V}{.})\) и \((W, \underset{W}{+}, \underset{W}{.})\) са Л.П-ва над поле \((F, \underset{F}{+}, \underset{F}{.})\).
Нека \(\varphi \in Hom(V, W)\). \\
Нека \(k \in \mathbb{N}^+\). Нека \(v_1, v_2, \dots, v_k \in V\) са Л.З-ми.
Тогава \(\varphi(v_1), \varphi(v_2), \dots, \varphi(v_k)\) са Л.З-ми.
\end{proposition}

\begin{proof}
\(v_1, v_2, \dots, v_k \in V\) са Л.З-ми. Тогава
\begin{align*}
(\exists \lambda_1 \in F)(\exists \lambda_2 \in F)\dots(\exists \lambda_k \in F)[\; (\lambda_1, \lambda_2, \dots, \lambda_k) \neq (0, 0, \dots, 0) \\
\& 
\\ \lambda_1 \underset{V}{.} v_1 \underset{V}{+} \lambda_2 \underset{V}{.} v_2 \underset{V}{+} \dots \underset{V}{+} \lambda_k \underset{V}{.} v_k  = \theta_V \;]
\end{align*}
Нека \(\lambda_1, \lambda_2 \dots, \lambda_k \in F\) са такива, че \((\lambda_1, \lambda_2, \dots, \lambda_k) \neq (0, 0, \dots, 0)\)
и \(\lambda_1 \underset{V}{.} v_1 \underset{V}{+} \lambda_2 \underset{V}{.} v_2 \underset{V}{+} \dots \underset{V}{+} \lambda_k \underset{V}{.} v_k  = \theta_V\).
Тогава от
\begin{align*}
    \varphi(\lambda_1 \underset{V}{.} v_1 \underset{V}{+} \lambda_2 \underset{V}{.} v_2 \underset{V}{+} \dots \underset{V}{+} \lambda_k \underset{V}{.} v_k) =
    \lambda_1 \underset{W}{.} \varphi(v_1) \underset{W}{+} \lambda_2 \underset{W}{.} \varphi(v_2) \underset{W}{+} \dots \underset{W}{+} \lambda_k \underset{W}{.} \varphi(v_k) \\
    \& \\
    \lambda_1 \underset{V}{.} v_1 \underset{V}{+} \lambda_2 \underset{V}{.} v_2 \underset{V}{+} \dots \underset{V}{+} \lambda_k \underset{V}{.} v_k  = \theta_V \\
    \& \\
    \varphi(\theta_V) = \theta_W
\end{align*}
Получаваме \(\lambda_1 \underset{W}{.} \varphi(v_1) \underset{W}{+} \lambda_2 \underset{W}{.} \varphi(v_2) \underset{W}{+} \dots \underset{W}{+} \lambda_k \underset{W}{.} \varphi(v_k) = \theta_W\).
\\
Но имаме \((\lambda_1, \lambda_2, \dots, \lambda_k) \in F^k \setminus \{(0, 0, \dots, 0)\}\),
следователно \(\varphi(v_1), \varphi(v_2), \dots, \varphi(v_k)\) са Л.З-ми. \(\qed\)
\end{proof}

\begin{definition}[Сума на Л.И.]
Нека \((V, \underset{V}{+}, \underset{V}{.})\) и \((W, \underset{W}{+}, \underset{W}{.})\) са Л.П-ва над поле \((F, \underset{F}{+}, \underset{F}{.})\).
Нека \(\varphi \in Hom(V, W)\) и \(\psi \in Hom(V, W)\).
Тогава \(\varphi + \psi \; : \; V \to W\) е сумата на \(\varphi\) и \(\psi\)
и е в сила:
\begin{align*}
(\forall v \in V)[\; (\varphi + \psi)(v) = \varphi(v) \underset{W}{+} \psi(v) \;]
\end{align*}
\end{definition}

\begin{definition}[Умножение на Л.И. със скалар]
Нека \((V, \underset{V}{+}, \underset{V}{.})\) и \((W, \underset{W}{+}, \underset{W}{.})\) са Л.П-ва над поле \((F, \underset{F}{+}, \underset{F}{.})\).
Нека \(\varphi \in Hom(V, W)\) и нека \(\lambda \in F\).
Тогава \(\lambda . \varphi \; : \; V \to W\) е умножението на \(\varphi\) с \(\lambda\)
и е в сила:
\begin{align*}
(\forall v \in V)[\; (\lambda . \varphi)(v) = \lambda \underset{W}{.} \varphi(v) \;]
\end{align*}
\end{definition}

\begin{definition}[Произведение на Л.И.]
Нека \((V, \underset{V}{+}, \underset{V}{.})\),
\((W, \underset{W}{+}, \underset{W}{.})\)
и \((U, \underset{U}{+}, \underset{U}{.})\)
са Л.П-ва над поле \((F, \underset{F}{+}, \underset{F}{.})\)
Нека \(\varphi \in Hom(V, W)\) и \(\psi \in Hom(U, V)\).
Тогава \(\varphi \circ \psi \; : \; U \to W\) е произведението на \(\varphi\) и \(\psi\)
и е в сила:
\begin{align*}
(\forall v \in V)[\; (\varphi \circ \psi)(v) = \varphi(\psi(v)) \;]
\end{align*}
\end{definition}

\begin{definition}[Матрица на Л.И. спрямо фиксирани базиси]
Нека \((V, \underset{V}{+}, \underset{V}{.})\) и \((W, \underset{W}{+}, \underset{W}{.})\) са К.М.Л.П-ва над поле \((F, \underset{F}{+}, \underset{F}{.})\).
Нека \(\varphi \in Hom(V, W)\).
Нека \(n = dim(V)\) и \(m = dim(W)\).
Нека \(b_1, b_2, \dots, b_n\) е базис на \(V\)
и нека \(a_1, a_2, \dots, a_m\) е базис на \(W\).
Нека
\begin{align*}
\varphi(b_i) = \gamma_{1i}\underset{W}{.}a_1 \underset{W}{+} \gamma_{2i}\underset{W}{.}a_2 \underset{W}{+} \dots \underset{W}{+} \gamma_{mi}\underset{W}{.}a_m
\end{align*}
за \(i \in \{1, 2, \dots, n\}\).
Тогава матрицата на \(\varphi\)
спрямо базисите \(b_1, b_2, \dots, b_n\) и \(a_1, a_2, \dots, a_m\) е
\begin{align*}
\begin{pmatrix}
    \gamma_{11} & \gamma_{12} & \dots & \gamma_{1n} \\
    \gamma_{21} & \gamma_{22} & \dots & \gamma_{2n} \\
    \vdots    & \vdots  & \ddots & \vdots \\
    \gamma_{m1} & \gamma_{m2} & \dots & \gamma_{mn}
\end{pmatrix}
\end{align*}
\end{definition}

\begin{remark}
Матрицата на \(\varphi\)
спрямо базисите \(b_1, b_2, \dots, b_n\) и \(a_1, a_2, \dots, a_m\) ще белим с \(\mathcal{M}_{a_1, a_2, \dots, a_m}^{b_1, b_2, \dots, b_n}(\varphi)\).
\end{remark}

\begin{definition}[Действие на матрицата на Л.И.]
Нека \((V, \underset{V}{+}, \underset{V}{.})\) и \((W, \underset{W}{+}, \underset{W}{.})\) са К.М.Л.П-ва над поле \((F, \underset{F}{+}, \underset{F}{.})\).
Нека \(\varphi \in Hom(V, W)\).
Нека \(n = dim(V)\) и \(m = dim(W)\).
Нека \(b_1, b_2, \dots, b_n\) е базис на \(V\)
и нека \(a_1, a_2, \dots, a_m\) е базис на \(W\).
Нека \(v = x_1\underset{V}{.}b_1 \underset{V}{+} x_2\underset{V}{.}b_2 \underset{V}{+} \dots \underset{V}{+} x_n\underset{V}{.}b_n \in V\)
и нека 
\begin{align*}
\varphi(b_i) = \gamma_{1i}\underset{W}{.}a_1 \underset{W}{+} \gamma_{2i}\underset{W}{.}a_2 \underset{W}{+} \dots \underset{W}{+} \gamma_{mi}\underset{W}{.}a_m
\end{align*}
за \(i \in \{1, 2, \dots, n\}\).
Нека \(y_1\underset{W}{.}a_1 \underset{W}{+} y_2\underset{W}{.}a_2 \underset{W}{+} \dots \underset{W}{+} y_m\underset{W}{.}a_m\)
Тогава матрицата на \(\varphi\)
спрямо базисите \(b_1, b_2, \dots, b_n\) и \(a_1, a_2, \dots, a_m\) е
\begin{align*}
\begin{pmatrix}
    \gamma_{11} & \gamma_{12} & \dots & \gamma_{1n} \\
    \gamma_{21} & \gamma_{22} & \dots & \gamma_{2n} \\
    \vdots    & \vdots  & \ddots & \vdots \\
    \gamma_{m1} & \gamma_{m2} & \dots & \gamma_{mn}
\end{pmatrix}
\end{align*}
и връзката между координатите на един вектор и неговия образ е:
\begin{align*}
\begin{pmatrix}
y_1 \\
y_2 \\
\vdots \\
y_m
\end{pmatrix}
=
\begin{pmatrix}
    \gamma_{11} & \gamma_{12} & \dots & \gamma_{1n} \\
    \gamma_{21} & \gamma_{22} & \dots & \gamma_{2n} \\
    \vdots    & \vdots  & \ddots & \vdots \\
    \gamma_{m1} & \gamma_{m2} & \dots & \gamma_{mn}
\end{pmatrix}
.
\begin{pmatrix}
x_1 \\
x_2 \\
\vdots \\
x_n
\end{pmatrix}
\end{align*}
\end{definition}

\begin{remark}
Нека \(V, \oplus, \odot\) е K.M.Л.П. над поле \((F, +, .)\).
Нека \(n = dim(V)\) и нека \(b_1, b_2, \dots, b_n\) е базис на \(V\).
Нека \(v = \lambda_1 \odot b_1 \oplus \lambda_2 \odot b_2 \oplus \dots \oplus \lambda_n \odot b_n \in V\).
Тогава с \([v]_{b_1, b_2, \dots, b_n}\) ще бележим координатния стълб съотвестващ на вектора \(v\).
Тоест
\begin{align*}
[v]_{b_1, b_2, \dots, b_n}
=
\begin{pmatrix}
x_1 \\
x_2 \\
\vdots \\
x_n
\end{pmatrix}
\end{align*}
\end{remark}

\begin{remark}
При така направените означения връзката между координатите от предното определение може да бъде записана
съкратено, но не и на ТК-то, като
\begin{align*}
[\varphi(v)]_{a_1, a_2, \dots, a_m}
=
\mathcal{M}_{a_1, a_2, \dots, a_m}^{b_1, b_2, \dots, b_n}(\varphi)
.
[v]_{b_1, b_2, \dots, b_n}
\end{align*}
\end{remark}

\begin{proposition}[Матрица на сума на Л.И.]
Нека \((V, \underset{V}{+}, \underset{V}{.})\) и \((W, \underset{W}{+}, \underset{W}{.})\) са К.М.Л.П-ва над поле \((F, \underset{F}{+}, \underset{F}{.})\).
Нека \(\varphi \in Hom(V, W)\) и \(\psi \in Hom(V, W)\).
Нека \(n = dim(V)\) и \(m = dim(W)\).
Нека \(b_1, b_2, \dots, b_n\) е базис на \(V\)
и нека \(a_1, a_2, \dots, a_m\) е базис на \(W\).
Тогава
\begin{align*}
\mathcal{M}_{a_1, a_2, \dots, a_m}^{b_1, b_2, \dots, b_n}(\varphi + \psi)
=
\mathcal{M}_{a_1, a_2, \dots, a_m}^{b_1, b_2, \dots, b_n}(\varphi)
+
\mathcal{M}_{a_1, a_2, \dots, a_m}^{b_1, b_2, \dots, b_n}(\psi)
\end{align*}
\end{proposition}

\begin{proposition}[Матрица на умножение на Л.И. със скалар]
Нека \((V, \underset{V}{+}, \underset{V}{.})\) и \((W, \underset{W}{+}, \underset{W}{.})\) са К.М.Л.П-ва над поле \((F, \underset{F}{+}, \underset{F}{.})\).
Нека \(\varphi \in Hom(V, W)\) и нека \(\lambda \in F\).
Нека \(n = dim(V)\) и \(m = dim(W)\).
Нека \(b_1, b_2, \dots, b_n\) е базис на \(V\)
и нека \(a_1, a_2, \dots, a_m\) е базис на \(W\).
Тогава
\begin{align*}
\mathcal{M}_{a_1, a_2, \dots, a_m}^{b_1, b_2, \dots, b_n}(\lambda . \varphi)
=
\lambda
.
\mathcal{M}_{a_1, a_2, \dots, a_m}^{b_1, b_2, \dots, b_n}(\varphi)
\end{align*}
\end{proposition}

\begin{proposition}[Матрица на произведение на Л.И.]
Нека \((V, \underset{V}{+}, \underset{V}{.})\),
\((W, \underset{W}{+}, \underset{W}{.})\)
и \((U, \underset{U}{+}, \underset{U}{.})\)
са К.М.Л.П-ва над поле \((F, \underset{F}{+}, \underset{F}{.})\)
Нека \(\varphi \in Hom(V, W)\) и \(\psi \in Hom(U, V)\).
Нека \(n = dim(V)\), \(m = dim(W)\) и \(k = dim(U)\).
Нека \(b_1, b_2, \dots, b_n\) е базис на \(V\),
нека \(a_1, a_2, \dots, a_m\) е базис на \(W\)
и нека \(c_1, c_2, \dots, c_k\) е базис на \(U\).
Тогава
\begin{align*}
\mathcal{M}_{a_1, a_2, \dots, a_m}^{c_1, c_2, \dots, c_k}(\varphi \circ \psi)
=
\mathcal{M}_{a_1, a_2, \dots, a_m}^{b_1, b_2, \dots, b_n}(\varphi)
.
\mathcal{M}_{b_1, b_2, \dots, b_n}^{c_1, c_2, \dots, c_k}(\psi)
\end{align*}
\end{proposition}

\begin{proposition}[Размерност на Л.П-во на Л.И-ния между две К.К.Л.П-ва]
Нека \((V, \underset{V}{+}, \underset{V}{.})\) и \((W, \underset{W}{+}, \underset{W}{.})\) са К.М.Л.П-ва над поле \((F, \underset{F}{+}, \underset{F}{.})\).
Тогава
\begin{align*}
dim(Hom(V, W)) = dim(V).dim(W)
\end{align*}
\end{proposition}

\begin{definition}[Ядро на Л.И.]
Нека \((V, \underset{V}{+}, \underset{V}{.})\) и \((W, \underset{W}{+}, \underset{W}{.})\) са Л.П-ва над поле \((F, \underset{F}{+}, \underset{F}{.})\).
Нека \(\varphi \in Hom(V, W)\).
Тогава ядро на \(\varphi\) наричаме следното множество:
\begin{align*}
Ker(\varphi) = \{v \in V \; | \; \varphi(v) = \theta_W\}
\end{align*}
\end{definition}

\begin{definition}[Образ на Л.И.]
Нека \((V, \underset{V}{+}, \underset{V}{.})\) и \((W, \underset{W}{+}, \underset{W}{.})\) са Л.П-ва над поле \((F, \underset{F}{+}, \underset{F}{.})\).
Нека \(\varphi \in Hom(V, W)\).
Тогава образ на \(\varphi\) наричаме следното множество:
\begin{align*}
Im(\varphi) = \{\varphi(v) \; | \; v \in V\}
\end{align*}
\end{definition}

\begin{definition}[Дефект на Л.И.]
Нека \((V, \underset{V}{+}, \underset{V}{.})\) и \((W, \underset{W}{+}, \underset{W}{.})\) са Л.П-ва над поле \((F, \underset{F}{+}, \underset{F}{.})\).
Нека \(\varphi \in Hom(V, W)\).
Тогава дефект на \(\varphi\) наричаме \(dim(Ker(\varphi))\).
Бележим с \(d(\varphi)\), тоест \(d(\varphi) = dim(Ker(\varphi))\).
\end{definition}

\begin{definition}[Ранг на Л.И.]
Нека \((V, \underset{V}{+}, \underset{V}{.})\) и \((W, \underset{W}{+}, \underset{W}{.})\) са Л.П-ва над поле \((F, \underset{F}{+}, \underset{F}{.})\).
Нека \(\varphi \in Hom(V, W)\).
Тогава ранг на \(\varphi\) наричаме \(dim(Im(\varphi))\).
Бележим с \(r(\varphi)\), тоест \(r(\varphi) = dim(Im(\varphi))\).
\end{definition}

\begin{theorem}[Теорема за ранга и дефекта]
Нека \((V, \underset{V}{+}, \underset{V}{.})\) и \((W, \underset{W}{+}, \underset{W}{.})\) са Л.П-ва над поле \((F, \underset{F}{+}, \underset{F}{.})\).
Нека \(\varphi \in Hom(V, W)\).
Тогава
\begin{align*}
r(\varphi) + d(\varphi) = dim(V)
\end{align*}
\end{theorem}

\begin{proposition}[Връзка между ранга на Л.И. и ранга на матрицата му спрямо произволни базиси]
Нека \((V, \underset{V}{+}, \underset{V}{.})\) и \((W, \underset{W}{+}, \underset{W}{.})\) са К.М.Л.П-ва над поле \((F, \underset{F}{+}, \underset{F}{.})\).
Нека \(\varphi \in Hom(V, W)\).
Нека \(n = dim(V)\) и \(m = dim(W)\).
Нека \(b_1, b_2, \dots, b_n\) е базис \(V\)
и нeка \(a_1, a_2, \dots, a_m\) е базис \(W\).
Тогава
\begin{align*}
r(\varphi) = r\left(\mathcal{M}_{a_1, a_2, \dots, a_m}^{b_1, b_2, \dots, b_n}(\varphi)\right)
\end{align*}
\end{proposition}

\begin{definition}[Обратимо линейно изобрабжение]
Нека \((V, \underset{V}{+}, \underset{V}{.})\) и \((W, \underset{W}{+}, \underset{W}{.})\) са Л.П-ва над поле \((F, \underset{F}{+}, \underset{F}{.})\).
Нека \(\varphi \in Hom(V, W)\), \(\varphi\) е обратимо линейно изобрабжение,
ако
\begin{align*}
(\exists \psi \in Hom(W, V))[\; \varphi \circ \psi = id_W \; \& \; \psi \circ \varphi = id_V \;]
\end{align*}
\end{definition}

\begin{definition}[Обратно линейно изобрабжение]
Нека \((V, \underset{V}{+}, \underset{V}{.})\) и \((W, \underset{W}{+}, \underset{W}{.})\) са Л.П-ва над поле \((F, \underset{F}{+}, \underset{F}{.})\).
Нека \(\varphi \in Hom(V, W)\) е обратимо и нека \(\psi \in Hom(W, V)\).
\(\psi\) е обратното на \(\varphi\), ако
\begin{align*}
\varphi \circ \psi = id_W \; \& \; \psi \circ \varphi = id_V
\end{align*}
\end{definition}

\begin{remark}
Доказва се, че обратното изображение на \(\varphi\) е единствено и за това го бележим с \(\varphi^{-1}\).
\end{remark}

\begin{proposition}[Обратния линеен оператор на един обратим линеен оператор е също обратим]
Нека \((V, \oplus, \odot)\) е Л.П. над поле \((F, +, .)\).
Нека \(\varphi \in Hom(V)\) е обратим.
Тогава \(\varphi^{-1}\) също е обратим.
\end{proposition}

\begin{proof}
В сила е \(\varphi \circ \varphi^{-1} = id_V\) и \(\varphi^{-1} \circ \varphi = id_V\),
тоест \(\varphi^{-1} \circ \varphi = id_V\) и \(\varphi \circ \varphi^{-1} = id_V\).
Следователно \(\varphi^{-1}\) е обратим, защото \(\varphi \in Hom(V)\). \(\qed\)
\end{proof}

\begin{proposition}[Едно Л.И. е инекция Т.С.Т.К. ядрото му е нулевото]
Нека \((V, \underset{V}{+}, \underset{V}{.})\) и \((W, \underset{W}{+}, \underset{W}{.})\) са Л.П-ва над поле \((F, \underset{F}{+}, \underset{F}{.})\).
Нека \(\varphi \in Hom(V, W)\). Тогава
\(\varphi\) е инекция Т.С.Т.К. \(Ker(\varphi) = \{\theta_V\}\).
\end{proposition}

\begin{proof}
1. Нека \(\varphi\) е инекция.
Нека \(v \in Ker(\varphi)\). Тогава \(\varphi(v) = \theta_W\),
но \(\varphi(\theta_V) = \theta_W\) и \(\varphi\) е инекция.
Следователно \(v = \theta_V\). Така \(\{\theta_V\} \subseteq Ker(\varphi) \subseteq \{\theta_V\}\).
Значи \(Ker(\varphi) = \{\theta_V\}\).
\newline
\vspace{2mm}
\newline
2. Нека \(Ker(\varphi = \{\theta_V\})\).
Нека \(v_1 \in V\) и \(v_2 \in V\) и са такива, че
\(\varphi(v_1) = \varphi(v_2)\). От тук последователно получаваме:
\begin{align*}
\varphi(v_1) \underset{W}{-} \varphi(v_2) = \theta_W \\
\varphi(v_1 \underset{V}{-} v_2) = \theta_W \\
v_1 \underset{V}{-} v_2 \in Ker(\varphi) = \{\theta_V\} \\
v_1 \underset{V}{-} v_2 = \theta_V \\
v_1 = v_2
\end{align*}
Следователно
\begin{align*}
(\forall a \in V)(\forall b \in V)[\; \varphi(a) = \varphi(b) \implies a = b \;]
\end{align*}
тоест
\begin{align*}
(\forall a \in V)(\forall b \in V)[\; a \neq b \implies \varphi(a) \neq \varphi(b) \;]
\end{align*}
и значи \(\varphi\) е инекция. \(\qed\)
\end{proof}

\begin{proposition}[Едно обратимо Л.И. изпраща Л.Н.З. в Л.Н.З.]
Нека \((V, \underset{V}{+}, \underset{V}{.})\) и \((W, \underset{W}{+}, \underset{W}{.})\) са Л.П-ва над поле \((F, \underset{F}{+}, \underset{F}{.})\).
Нека \(\varphi \in Hom(V, W)\) е обратимо.
Нека \(k \in \mathbb{N}^+\) и нека \(v_1, v_2, \dots, v_k \in V\) са Л.Н.З.
Тогава \(\varphi(v_1), \varphi(v_2), \dots, \varphi(v_k)\) също са Л.Н.З. 
\end{proposition}

\begin{proof}
Да допуснем, че \(\varphi(v_1), \varphi(v_2), \dots, \varphi(v_k)\) са Л.З.
Тогава нека \(\lambda_1, \lambda_2, \dots, \lambda_k \in F\) и \((\lambda_1, \lambda_2, \dots, \lambda_k) \neq (0, 0, \dots, 0)\)
са такива, че \(\lambda_1 \underset{W}{.} \varphi(v_1) \underset{W}{+} \lambda_2 \underset{W}{.} \varphi(v_2) \underset{W}{+} \dots \underset{W}{+} \lambda_k \underset{W}{.} \varphi(v_k) = \theta_W\).
Тогава \(\varphi^{-1}(\lambda_1 \underset{W}{.} \varphi(v_1) \underset{W}{+} \lambda_2 \underset{W}{.} \varphi(v_2) \underset{W}{+} \dots \underset{W}{+} \lambda_k \underset{W}{.} \varphi(v_k)) = \varphi^{-1}(\theta_W)\),
тоест \(\lambda_1 \underset{V}{.} \varphi^{-1}(\varphi(v_1)) \underset{V}{+} \lambda_2 \underset{V}{.} \varphi^{-1}(\varphi(v_2)) \underset{V}{+} \dots \underset{V}{+} \lambda_k \underset{V}{.} \varphi^{-1}(\varphi(v_k)) = \theta_V\)
и значи \(\lambda_1 \underset{V}{.} v_1 \underset{V}{+} \lambda_2 \underset{V}{.} v_2 \underset{V}{+} \dots \underset{V}{+} \lambda_k \underset{V}{.} v_k = \theta_V\),
но \((\lambda_1, \lambda_2, \dots, \lambda_k) \neq (0, 0, \dots, 0)\).
Следователно \(v_1, v_2, \dots, v_k\) са Л.З., но това е Абсурд, защото са Л.Н.З. \(\qed\)
\end{proof}

\begin{definition}[Матрица на прехода от един базис към друг]
Нека \((V, \oplus, \odot)\) е K.M.Л.П. над поле \((F, +, .)\).
Нека \(n = dim(V)\) и нека
\(b_1, b_2, \dots, b_n\) и \(a_1, a_2, \dots, a_n\) са два базиса на \(V\). \\
Нека още \(a_i = \lambda_{1i} \odot b_1 \oplus \lambda_{2i} \odot b_2 \oplus \dots \oplus \lambda_{ni} \odot b_n\).
Тогава
\begin{align*}
T_{b_1, b_2, \dots, b_n \to a_1, a_2, \dots, a_n}
=
\begin{pmatrix}
    \lambda_{11} & \lambda_{12} & \dots & \lambda_{1n} \\
    \lambda_{21} & \lambda_{22} & \dots & \lambda_{2n} \\
    \vdots    & \vdots  & \ddots & \vdots \\
    \lambda_{n1} & \lambda_{n2} & \dots & \lambda_{nn}
\end{pmatrix}
\end{align*}
е матрицата на прехода от \(b_1, b_2, \dots, b_n\) към \(a_1, a_2, \dots, a_n\).
\end{definition}

\begin{proposition}[Връзка между координатите при смяна на базиса]
Нека \((V, \oplus, \odot)\) е K.M.Л.П. над поле \((F, +, .)\).
Нека \(n = dim(V)\) и нека
\(b_1, b_2, \dots, b_n\) и \(a_1, a_2, \dots, a_n\) са два базиса на \(V\). \\
Нека още \(a_i = \lambda_{1i} \odot b_1 \oplus \lambda_{2i} \odot b_2 \oplus \dots \oplus \lambda_{ni} \odot b_n\). \\
Нека \(v = y_1 \odot b_1 \oplus y_2 \odot b_2 \oplus \dots \oplus y_n \odot b_n
= x_1 \odot a_1 \oplus x_2 \odot a_2 \oplus \dots \oplus x_n \odot a_n\).
Тогава в сила е връзката:
\begin{align*}
\begin{pmatrix}
y_1 \\
y_2 \\
\vdots \\
y_n
\end{pmatrix}
=
\begin{pmatrix}
    \lambda_{11} & \lambda_{12} & \dots & \lambda_{1n} \\
    \lambda_{21} & \lambda_{22} & \dots & \lambda_{2n} \\
    \vdots    & \vdots  & \ddots & \vdots \\
    \lambda_{n1} & \lambda_{n2} & \dots & \lambda_{nn}
\end{pmatrix}
.
\begin{pmatrix}
x_1 \\
x_2 \\
\vdots \\
x_n
\end{pmatrix}
\end{align*}
\end{proposition}

\begin{remark}
Горната връзка може да се запише още така при въведените означения,
но не и на ТК-то :)
\begin{align*}
[v]_{b_1, b_2, \dots, b_n}
=
T_{b_1, b_2, \dots, b_n \to a_1, a_2, \dots, a_n}
.
[v]_{a_1, a_2, \dots, a_n}
\end{align*}
\end{remark}

\begin{proposition}[Промяна на матрицата на Л.И. при смяна на базисите]
Нека \((V, \underset{V}{+}, \underset{V}{.})\) и \((W, \underset{W}{+}, \underset{W}{.})\) са К.М.Л.П-ва над поле \((F, \underset{F}{+}, \underset{F}{.})\).
Нека \(\varphi \in Hom(V, W)\).
Нека \(n = dim(V)\) и \(m = dim(W)\).
Нека \(b_1, b_2, \dots, b_n\) и \(d_1, d_2, \dots d_n\) са два базиса на \(V\)
и нeка \(a_1, a_2, \dots, a_m\) и \(c_1, c_2, \dots c_m\) са два базиса на \(W\).
Тогава в сила е връзката:
\begin{align*}
\mathcal{M}_{c_1, c_2, \dots, c_m}^{d_1, d_2, \dots, d_n}(\varphi)
=
T_{c_1, c_2, \dots, c_m \to a_1, a_2, \dots, a_m}
.
\mathcal{M}_{a_1, a_2, \dots, a_m}^{b_1, b_2, \dots, b_n}(\varphi)
.
T_{b_1, c_2, \dots, b_n \to d_1, d_2, \dots, d_n}
\end{align*}
\end{proposition}

\begin{proposition}[Промяна на матрицата на Л.О. при смяна на базисa]
Нека \((V, \oplus, \odot)\) е K.M.Л.П. над поле \((F, +, .)\).
Нека \(n = dim(V)\) и нека
\(b_1, b_2, \dots, b_n\) и \(a_1, a_2, \dots, a_n\) са два базиса на \(V\). \\
Тогава в сила е връзката:
\begin{align*}
\mathcal{M}_{a_1, a_2, \dots, a_n}(\varphi)
=
\left(T_{b_1, b_2, \dots, b_n \to a_1, a_2, \dots, a_n}\right)^{-1}
.
\mathcal{M}_{b_1, b_2, \dots, b_n}(\varphi)
.
T_{b_1, b_2, \dots, b_n \to a_1, a_2, \dots, a_n}
\end{align*}
\end{proposition}

\begin{theorem}[Първа теорема за ранг на матрици]
Нека \((F, +, .)\) е поле и \(m, n \in \mathbb{N}^+\) и \(A \in M_{m \times n}(F)\).
Тогава \(r(A) = r(A^t)\).
\end{theorem}

\end{document}