\documentclass{article}
\usepackage{amsmath}
\usepackage{amssymb}
\usepackage[T1,T2A]{fontenc}
\usepackage[utf8]{inputenc}
\usepackage[bulgarian]{babel}
\usepackage[normalem]{ulem}
\newcommand{\stkout}[1]{\ifmmode\text{\sout{\ensuremath{#1}}}\else\sout{#1}\fi}
\newcommand{\x}[2]{\text{\tiny{\(#1 \times #2\)}}}
\newcommand{\V}{\mathbb{V}}
\newcommand{\F}{\mathbb{F}}
\newcommand{\W}{\mathbb{W}}
\newcommand{\UV}{\mathbb{U}}
\newcommand{\n}[1]{#1_1, \dots, #1_n}
\newcommand{\m}[1]{#1_1, \dots, #1_m}
\newcommand{\OV}{\theta}
\newcommand{\ieqn}{i = 1, \dots, n}

\title{Теоритично контролно №2 1, I, Информатика}
\author{Иво Стратев}

\begin{document}
    \pagenumbering{gobble}
    \maketitle
    \section{Линейно изображение и линеен оператор}
    \subsection{Определение линеен оператор}
    \(\V\) - Л.П над полето \(\F, \; dim\V = n\\
    \varphi : \V \to \V, \; \varphi \in Hom\V\\
    \n{v} \in \V, \; \n{\lambda}, \in \F\\
    \implies \varphi(\sum_{i = n}^n \lambda_i v_i) = \sum_{i = n}^n \lambda_i \varphi(v_i) \in \V\)
    \subsection{Определение линейно изображение}
    \(\V, \; \W\) - Л.П над полето \(\F, \; dim\V = n\\
    \varphi : \V \to \W, \; \varphi \in Hom(\V,\W) \\
    \n{v} \in \V, \; \n{\lambda}, \in \F\\
    \implies \varphi(\sum_{i = n}^n \lambda_i v_i) = \sum_{i = n}^n \lambda_i \varphi(v_i) \in \W\)
    \subsection{Th \(\exists! \; \varphi \in Hom(\V, \W)\)}
    \(\V, \; \W\) - Л.П над полето \(\F, \; dim\V = n\\
    \n{e} - \text{ базис на } \V\\
    \n{w} - \text{ вектори от } \W\\
    \implies \exists! \; \varphi \in Hom(\V, \W); \; \varphi(e_i) = w_i, \; \ieqn\)
    \subsection{\(\V \cong \W \iff dim\V = dim\W\)}
    \section{Доказателства за линейни изображения}
    \subsection{\(\forall \; \varphi \in Hom(\UV, \V) \implies \varphi(\OV_\UV) = \OV_\V\)}
    \(\OV_\V = 0\varphi(u) = \varphi(0u) = \varphi(\OV_\UV) \; u \in \UV\)
    \subsection{\(\forall \; \varphi \in Hom(\UV, \V), \; \forall u \in \UV \implies \varphi(-u) = -\varphi(u)\)}
    \(\varphi \in Hom(\UV, \V), \; \forall u \in \UV, \; \forall \lambda \in \F \implies \varphi(\lambda u) = \lambda \varphi(u)\\
    \implies \lambda = -1 \implies \varphi(-1u) = -1\varphi(u) \implies \varphi(-u) = -\varphi(u)\)
    \subsection{\(\forall \; \varphi \in HomV \implies \varphi(\OV) = \OV\)}
    \(\varphi(\OV) = \varphi(\OV + \OV) = \varphi(\OV) + \varphi(\OV) \; | + (-\varphi(\OV))\\
    \varphi(\OV) - \varphi(\OV) = \varphi(\OV) + \varphi(\OV) - \varphi(\OV)\\
    \OV = \OV + \varphi(\OV) \implies \varphi(\OV) = \OV\)
     \subsection{\(\forall \; \varphi \in Hom\V, \; \forall v \in \V \implies \varphi(-v) = -\varphi(v)\)}
    \(\varphi \in Hom\V, \; \forall v \in \V, \; \forall \lambda \in \F \implies \varphi(\lambda v) = \lambda \varphi(v)\\
    \implies \lambda = -1 \implies \varphi(-1v) = -1\varphi(v) \implies \varphi(-v) = -\varphi(v)\)
    \subsection{Едно линейно изображение изпраща линейно зависими вектори в линейно зависими вектори)}
    \(\V, \; \W\) - Л.П над полето \(\F, \; dim\V = n\\
    \varphi \in Hom(\V,\W) \implies \varphi(\OV_\V) = \OV_\W\\
    \n{e} - \text{ базис на } \V\\
    v \in \V, \; \n{\lambda} \in \F; \; v = \sum_{i = 1}^n \lambda_i e_i = \OV_\V, \; (\n{\lambda}) \neq (0, \dots, 0)\\
    v = \OV_\V \; |\varphi\\
    \varphi(v) = \OV_\W\\
    \varphi(\sum_{i = 1}^n \lambda_i e_i) = \OV_\W\\
    \sum_{i = 1}^n \lambda_i \varphi(e_i) = \OV_\W, \; (\n{\lambda}) \neq (0, \dots, 0)\)
    \subsection{Един линеен оператор изпраща линейно зависими вектори в линейно зависими вектори)}
    \(\V\) - Л.П над полето \(\F, \; dim\V = n\\
    \varphi \in Hom\V \implies \varphi(\OV) = \OV\\
    \n{e} - \text{ един базис на } \V\\
    v \in \V, \; \n{\lambda} \in \F; \; v = \sum_{i = 1}^n \lambda_i e_i = \OV, \; (\n{\lambda}) \neq (0, \dots, 0)\\
    v = \OV \; |\varphi\\
    \varphi(v) = \OV\\
    \varphi(\sum_{i = 1}^n \lambda_i e_i) = \OV\\
    \sum_{i = 1}^n \lambda_i \varphi(e_i) = \OV, \; (\n{\lambda}) \neq (0, \dots, 0)\)
    \section{Действия с линейни изображения}
    \subsection{Определение за сума на линейни изображения}
    \(\V, \; \W\) - Л.П над полето \(\F, \; \varphi, \psi \in Hom(\V, \W)\\
    \varphi + \psi : \V \to \W; \; \forall v \in \V (\varphi + \psi)(v) = \varphi(v) + \psi(v) \in \W\)
    \subsection{Определение за произведение на линейно изображение със скалар}
    \(\V, \; \W\) - Л.П над полето \(\F, \; \varphi \in Hom(\V, \W), \; \lambda \in \F\\
    \lambda\varphi : \V \to \W; \; \forall v \in \V (\lambda\varphi)(v) = \lambda\varphi(v) \in \W\)
    \subsection{Определение за произведение на линейни изображения}
    \(\V, \; \W, \; \UV\) - Л.П над полето \(\F, \; \in Hom(\V, \W), \psi \in Hom(\W, \UV)\\
    \psi\varphi : \V \to \UV; \; \forall v \in \V (\psi\varphi)(v) = (\psi \circ \varphi)(v) = \psi(\varphi(v)) \in \UV\)
    \subsection{Определението за матрица на линейно изображение}
    \(\V, \; \W\) - Л.П над полето \(\F, \; \varphi \in Hom(\V, \W) , \; n = dim\V, \; m = dim\W\\
    \n{e} = \text{ базис на } \V\\
    \m{f} = \text{ базис на } \W\\
    A = (\lambda_{ji})\x{m}{n} = M_{e \to f}(\varphi) \in \F\x{m}{n}\\
    \varphi(e_i) = \sum_{j = 1}^m \lambda_{ji}f_j \in \W, \ieqn, \; \lambda_{ji} \in \F\)
    \subsection{изобразяване на координатите на образа\\
    на вектор под действието на линейно изображение\\
    чрез координатите на вектора и матрицата\\
    на линейното изображение}
    \(\V, \; \W\) - КМЛП над полето \(\F, \; n = dim\V, \; m = dim\W\\
    \n{e} - \text{ базис на } \V\\
    \m{f} - \text{ базис на } \W\\
    A = M_{e \to f}(\varphi) \in \F\x{m}{n}\\
    v \in \V \implies v = \sum{i = 1}^n \lambda_i e_i, \; \lambda_i \in \F, \; \ieqn\\
    (\n{\lambda}) - \text{ координатите на \(v\) спрямо базиса \(e\) на } \V\\
    \implies \exists(\m{\mu}) \in \F^m - \text{ координатите на \(v\) спрямо базиса \(f\) на } \W\\
    (\m{\mu})^t = A(\n{\lambda})^t\)
    \section{Матрици на линейни изображения, получени след действия с ЛИ}
    \subsection{Определение за сума на линейни изображения}
    \(\V, \; \W\) - Л.П над полето \(\F, \; \varphi, \psi \in Hom(\V, \W)\\
    \varphi + \psi : \V \to \W; \; \forall v \in \V (\varphi + \psi)(v) = \varphi(v) + \psi(v) \in \W\)
    \subsection{Определение за матрица на линейно изображение, което е сумата на две линейни изображения}
    \(\V, \; \W\) - KMЛП над полето \(\F, \; \varphi, \psi \in Hom(\V, \W), \; n = dim\V, \; m = dim\W\\
    \tau = \varphi + \psi \in Hom(\V, \W)\\
    A = M(\varphi) \in \F\x{m}{n}\\
    B = M(\psi) \in \F\x{m}{n}\\
    \implies C = M(\tau) = M(\varphi + \psi) = M(\varphi) + M(\psi) = A + B \in \F\x{m}{n}\)
    \subsection{Определение за матрица на линейно изображение, което е произведение на линейно изображение със скалар}
    \(\V, \; \W\) - KMЛП над полето \(\F, \; \varphi \in Hom(\V, \W), \; \lambda \in \F, \; n = dim\V, \; m = dim\W\\
    \psi = \lambda\varphi \in Hom(\V, \W)\\
    A = M(\varphi) \in \F\x{m}{n}\\
    B = M(\psi) = M(\lambda\varphi) = \lambda M(\varphi) = \lambda A \in \F\x{m}{n}\)
    \subsection{Определение за матрица на линейно изображение, което е произведение на две линейни изображения}
    \(\V, \; \W, \; \UV\) - KMЛП над полето \(\F, \; \varphi \in Hom(\V, \W), \psi \in Hom(\W, \UV)\\
    n = dim\V, \; m = dim\W, \; k = dim\UV\\
    \tau = \psi\varphi \in Hom(\V, \UV)\\
    A = M(\varphi) \in \F\x{m}{n}\\
    B = M(\psi) \in \F\x{k}{m}\\
    \implies C = M(\tau) = M(\psi\varphi) = M(\psi)M(\varphi) = BA \in \F\x{k}{n}\)
    \subsection{\(\V, \; \W\) - KMЛП над полето \(\F, \; n = dim\V, \; m = dim\W\\
    dimHom(\V, \W) = m . n\)}
    \section{Ядро и Образ на Линейно изображение}
    \(\V, \; \W\) - Л.П над полето \(\F, \; \varphi \in Hom(\V, \W)\)
    \subsection{\(Ker \varphi = \{v \in \V \; | \; \varphi(v) =\OV\}\)}
    \subsection{\(Im \varphi = \{\varphi(v) \; | \; v \in \V\}\)}
    \subsection{\(r(\varphi) = dimIm\varphi\)}
    \subsection{\(d(\varphi) = dimKer\varphi\)}
    \subsection{Th За ранга и дефекта}
    \(\UV, \; \mathbb{S}\) - KMЛП над полето \(\F, \; \psi \in Hom(\UV, \mathbb{S})\\
    dim\UV = p \implies r(\psi) + d(\psi) = p\)
    \subsection{\(A = M(\varphi) \implies r(\varphi) = r(A)\)}
    \section{Обратомост на ЛИ и ЛО}
    \subsection{Определение за обратимо линейно изображение}
    \(\V, \; \W\) - KMЛП над полето \(\F, \; \varphi \in Hom(\V, \W)\\
    \exists \varphi^{-1} \in Hom(\W, \V); \; \varphi.\varphi^{-1} = \varphi^{-1}.\varphi = \varepsilon \iff \varphi \text{ е биекция}\)
    \subsection{Определение за обратното линейно изображение на дадено линейно изображение}
    \(\V, \; \W\) - KMЛП над полето \(\F, \; \varphi \in Hom(\V, \W)\\
    \text{Ако } \varphi \text{ е обратимо ЛИ, то}\\
    \implies \exists! \; \varphi^{-1} \in Hom(\W, \V); \; \varphi.\varphi^{-1} = \varphi^{-1}.\varphi = \varepsilon\\
    \varphi^{-1} \text{ е обратното ЛИ на } \varphi\)
    \subsection{Обратният на обратим линеен оператор също е обратим}
    \(\V\) - KMЛП над полето \(\F, \; \varphi \in Hom\V\\
    \varphi \text{ - обратим ЛО } \implies \varphi \circ \varphi^{-1} = \varphi^{-1} \circ \varphi = \varepsilon_\V\\
    \implies\varphi^{-1} \circ (\varphi^{-1})^{-1} = (\varphi^{-1})^{-1} \circ \varphi^{-1} = \varepsilon_\V\\
    \implies (\varphi^{-1})^{-1} = \varphi\)
    \subsection{}
    \(\V, \; \W\) - KMЛП над полето \(\F, \; \varphi \in Hom(\V, \W)\\
    \varphi \text{ е интективно } \iff Ker\varphi = \{\OV\}\\
    (\implies) \; \{\OV\} \subset Ker\varphi, \; \text{ако } v \in Ker\varphi\\
    \implies \varphi(v) = \OV_\W = \varphi(\OV_\V)\\
    \varphi \text{ - инективно } \implies v = \OV_\V\\
    \implies Ker\varphi \subset	\{\OV\} \implies Ker\varphi = \{\OV\}\\
    (\; \Leftarrow \;) \; u, \; v \in \V ; \; \varphi(u) = \varphi(v)\\
    \implies \OV = \varphi(u) - \varphi(v) = \varphi(u - v)\\
    \implies u - v = \OV = \{\OV\} = Ker\varphi\\
    \implies u = v \implies \varphi \text{ е инективно}\)
    \subsection{Обратимо линейно изображение изпраща линейно независими вектори в линейно независими вектори}
    \(\V, \; \W\) - ЛП над полето \(\F, \; , divm\V = n, \; \varphi \in Hom(\V, \W) \text{ - обратимо ЛИ}\\
    \n{v} \text{ - л.нз. вектори } \in \V\\
    \text{Нека } \sum_{i = 1}^n \lambda_i \varphi(v_i) = \OV_\W \in \W, \; \lambda_i \in \F \; |\varphi^{-1}\\
    \sum_{i = 1}^n \lambda_i \varphi^{-1}(\varphi(v_i)) = \varphi^{-1}(\OV_\W)\\
    \sum_{i = 1}^n \lambda_i v_i = \OV_\V, \; \n{v} \text{ - л.нз. вектори }\\
    \implies (\n{\lambda}) = (0, \dots, 0)\\
    \implies \varphi(v_1), \dots, \varphi(v_1) \text{ - л.нз. вектори}\)
    \section{Смяна на базиса}
    \subsection{Определението за матрица на прехода между два базиса}
    \(\V\) - KMЛП над полето \(\F, \; dim\V = n \in \mathbb{N}\\
    \n{e} \text { - един базис на } \V\\
    \n{f} \text { - друг базис на } \V\\
    f_i = \sum_{j = 1}^n \tau_{ji} e_j, \; \ieqn, \; \tau_{ji} \in \F\\
    T = (\tau_{ji})\x{n}{n} \in M_n \text{ е матрица на прехода между базисите } e, \; f \text{ на } \V\)
    \subsection{Промяна на координатите на вектор при смяна на базиса}
    \(\V\) - KMЛП над полето \(\F, \; dim\V = n \in \mathbb{N}\)\\
    Нека \(T = (\tau_{ji})\x{n}{n} \in M_n\) е матрицата на прехода от \(e \to f\\
    v = \sum_{i = 1}^n \lambda_i e_i = \sum_{i = 1}^n \mu_i e_i \in \V, \; \lambda_i, \mu_u \in \F, \; \ieqn\\
    (\n{\lambda})^t = T(\n{\mu})^t\)
    \subsection{Промяна на матрицата на линейно изображение при смяна на базиса}
    \(\V, \; \W\) - KMЛП над полето \(\F, \; \varphi \in Hom(\V, \W), \; n = dim\V, \; m = dim\W\\
    \n{e} \text { - един базис на } \V\\
    \n{g} \text { - друг базис на } \V\\
    \m{f} \text { - един базис на } \W\\
    \m{h} \text { - друг базис на } \W\\
    A = (a_{ij})\x{m}{n} \in M\x{m}{n} \text{ - матрица на } \varphi \text{ между базисите } e, \; f\\
    B = (b_{ij})\x{m}{n} \in M\x{m}{n} \text{ - матрица на } \varphi \text{ между базисите } g, \; h\\
    T = (\tau_{ij})\x{n}{n} \in M_n \text{ е матрица на прехода между базисите } e, \; g \text{ на } \V\\
    K = (\kappa_{ij})\x{m}{m} \in M_n \text{ е матрица на прехода между базисите } f, \; h \text{ на } \W\\
    B = T^{-1}AK \; (M_{g \to h} = M_{g \to e}M_{e \to f}M_{f \to h})\)
    \subsection{Промяна на матрицата на линеен оператор при смяна на базиса}
    \(\V\) - KMЛП над полето \(\F, \; \varphi \in Hom\V, \; n = dim\V\\
    \n{e} \text { - един базис на } \V\\
    \n{f} \text { - друг базис на } \V\\
    A = (a_{ij})\x{n}{n} \in M_n \text{ - матрица на } \varphi \text{ в базиса } e\\
    B = (b_{ij})\x{n}{n} \in M_n \text{ - матрица на } \varphi \text{ в базиса } f\\
    T = (\tau_{ij})\x{n}{n} \in M_n \text{ е матрица на прехода между базисите } e, \; f\\
    B = T^{-1}AT \; (M_f = M_{f \to e} M_e M_{e \to f})\)
    \subsection{\(r(A) = r(A^t)\)}
    \section{Дуалност}
    \subsection{Определение за дуалното пространство на дадено линейно пространство}
    \(\V\) - ЛП над полето \(\F\\
    V^* = Hom(\V, \F) \text{ е дуалното пространство на ЛП } \V\)
    \subsection{Определение за линеен функционал}
    \(\V, \; \W\) - ЛП над полето \(\F\\
    f \text{ е линеен функционал на } \V \iff f \in V^*\)
    \subsection{Определение за дуалното изображение на дадено линейно изображение}
    \(\V, \; \W\) - ЛП над полето \(\F\\
    \V^*, \; \W^* \text{ - Дуалните пространства на ЛП } \V, \; \W\\
    \text{Ако } \varphi \in Hom(\V, \W) \implies \varphi^* \in Hom(\W^*, \V^*)\\
    \varphi^*(f) = f \circ \varphi, \; f \in \W^*\)
    \subsection{Определение за за дуален базис}
    \(\V\) - ЛП над полето \(\F, \; dim\V = n, \; \V^* \text{ - дуалното пространство на } \V\\
    \n{e} \text{ - базис на } \V\\
    f^1, \dots, f^n \text{ - дуален базис на } \V^*\\
    f^i(e_j) = \delta_{ij} = \begin{cases}
        1 & i = j\\
        0 & i \neq j
    \end{cases} \; , \; j, \; \ieqn\)
    \subsection{Дуално изображение на произведението на две линейни изображения}
    \(\V, \; \W, \; \UV\) - ЛП над полето \(\F, \; \in Hom(\V, \W), \psi \in Hom(\W, \UV)\\
    \psi\varphi \in Hom(\V, \UV) \implies (\psi\varphi)^* \in Hom(\UV^*, \V^*)\\
    (\psi\varphi)^* = f \circ (\psi\varphi) = (f \circ \psi) \circ \varphi =\\
    = \varphi^* \circ (f \circ \psi) = \varphi^*(\psi^*(f)) = (\varphi^*\psi^*)(f), \; f\in \UV^*\\
    \implies (\psi\varphi)^* = \varphi^*\psi^*\)
    \subsection{Връзката между матриците на едно линейно изображение и неговото дуално изображение}
    \(\V, \; \W\) - ЛП над полето \(\F\\
    \V^*, \; \W^* \text{ - Дуалните пространства на ЛП } \V, \; \W\\
    \varphi \in Hom(\V, \W), \; \varphi^* \in Hom(\W^*, \V^*)\\
    M(\varphi^*) = (M(\varphi))^t\)
    \subsection{Определение за анихилатор \(\UV^0\) на едно линейно подпространсво \(\UV\) на едно линейно пространство \(\V\)}
    \(\V\) - ЛП над полето \(\F\\
    \UV < \V \implies \UV^0 = \{f \in \V^* \; | \; \forall u \in \UV \; f(u) = 0\}\)
\end{document}