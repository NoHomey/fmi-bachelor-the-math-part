\documentclass{article}
\usepackage{amsmath}
\usepackage{amsthm}
\usepackage{amssymb}
\usepackage{stmaryrd}
\usepackage{tikz}
\usetikzlibrary{matrix}
\usepackage[T1,T2A]{fontenc}
\usepackage[utf8]{inputenc}
\usepackage[bulgarian]{babel}
\usepackage[normalem]{ulem}
\newcommand{\stkout}[1]{\ifmmode\text{\sout{\ensuremath{#1}}}\else\sout{#1}\fi}
\newcommand{\x}[2]{\text{\tiny{\(#1 \times #2\)}}}
\newcommand{\V}{\mathbb{V}}
\newcommand{\N}{\mathbb{N}}
\newcommand{\F}{\mathbb{F}}
\newcommand{\W}{\mathbb{W}}
\newcommand{\UV}{\mathbb{U}}
\newcommand{\n}[1]{#1_1, \dots, #1_n}
\newcommand{\m}[1]{#1_1, \dots, #1_m}
\newcommand{\OV}{\theta}
\newcommand{\ieqn}{i = 1, \dots, n}

\title{Теоритично контролно №2 1, I, Информатика}
\author{Иво Стратев}

\begin{document}
    \maketitle
    \section{Линейно изображение и линеен оператор}
    \subsection{Определение линеен оператор}
    Нека \(\V\) - Л.П. над полето \(\F, \; \varphi : \V \to \V \\\\
    \forall n \in \N, \; \forall \n{v} \in \V, \; \forall \n{\lambda} \in \F \\\\
    \varphi\left(\displaystyle\sum_{i = 1}^n \lambda_i v_i\right)
    = \displaystyle\sum_{i = 1}^n \lambda_i \varphi(v_i) \implies \varphi \in \mathrm{Hom}\V \)
    \subsection{Определение линейно изображение}
    Нека \(\V, \; \W\) - Л.П. над полето \(\F, \; \varphi : \V \to \W \\\\
    \forall n \in \N, \; \forall \n{v} \in \V, \; \forall \n{\lambda} \in \F \\\\
    \varphi\left(\displaystyle\sum_{i = 1}^n \lambda_i v_i\right)
    = \displaystyle\sum_{i = 1}^n \lambda_i \varphi(v_i) \implies \varphi \in \mathrm{Hom}(\V, \; \W) \)
    \subsection{Теорема \(\exists! \; \varphi \in \mathrm{Hom}(\V, \; \W)\)}
    Нека \(\V, \; \W\) - Л.П. над полето \(\F, \; \mathrm{dim}\V = n \\\\
    \n{e} - \; \text{ базис на } \V \\\\
    \n{w} - \; \text{ произволни вектори от } \W \\\\
    \implies \exists! \; \varphi \in \mathrm{Hom}(\V, \; \W) \; : \; \ieqn \quad \varphi(e_i) = w_i\)
    \subsection{Определение за изоморфизъм на линейни пространства}
    Нека \(\V, \; \W\) - Л.П. над полето \(\F\) и \(\varphi \; : \; \V \to \W\) е изображение. \\\\
    \(\varphi\) е изоморфизъм между \(\V\) и \(\W\) (\(\V \cong \W\)), ако: \\\\
    1) \(\varphi \in  \mathrm{Hom}(\V, \; \W)\) (\(\varphi\) е лин. изображение) \\\\
    2) \(\varphi\) е биекция
    \subsection{Н.Д.У две крайно мерни Л.П. да са изоморфни}
    Нека \(\V, \; \W\) - K.M.Л.П. над полето \(\F\) \\\\
    \(\V \cong \W \iff \mathrm{dim}\V = \mathrm{dim}\W \in \N\)
    \section{Доказателства за линейни изображения}
    \subsection{Докажете, че \(\forall \; \varphi \in \mathrm{Hom}(\UV, \; \V) \implies \varphi(\OV_\UV) = \OV_\V\)}
    Доказателство 1: \\\\
    Нека \(u \in \UV \; \OV_\V = 0\varphi(u) = \varphi(0u) = \varphi(\OV_\UV) \qed \) \\\\
    Доказателство 2: \\\\
    Нека \(u \in \UV \; \varphi(\OV_\UV) = \varphi(u - u) = \varphi(u + (-1)u) = \\\\
    = \varphi(u) + (-1)\varphi(u) = \varphi(u) - \varphi(u) = \OV_\V \qed \) \\\\
    Доказателство 3: \\\\
    \(\varphi(\OV_\UV) = \varphi(\OV_\UV + \OV_\UV) = \varphi(\OV_\UV) + \varphi(\OV_\UV) \; | \; -\varphi(\OV_\UV) \implies \\\\
    \varphi(\OV_\UV) - \varphi(\OV_\UV) = \varphi(\OV_\UV) + \varphi(\OV_\UV) - \varphi(\OV_\UV) \implies \\\\
    \OV_\V = \varphi(\OV_\UV) \qed \)
    \subsection{Докажете, че \(\forall \; \varphi \in \mathrm{Hom}(\UV, \; \V) \\
    \forall u \in \UV \implies \varphi(-u) = -\varphi(u)\)}
    Доказателство: \\\\
    \(\forall \varphi \in \mathrm{Hom}(\UV, \; \V), \; \forall u \in \UV, \; \forall \lambda \in \F \implies \varphi(\lambda u) = \lambda \varphi(u) \\\\
    \implies \lambda = -1 \implies \varphi(-u) = \varphi(-1u) = -1\varphi(u) = -\varphi(u) \\\\
    \implies \varphi(-u) = -\varphi(u) \qed\)
    \subsection{Докажете, че \(\forall \; \varphi \in \mathrm{Hom}(\V) \implies \varphi(\OV) = \OV\)}
    Доказателство 1: \\\\
    Нека \(v \in \V \; \OV = 0\varphi(v) = \varphi(0v) = \varphi(\OV) \qed \) \\\\
    Доказателство 2: \\\\
    Нека \(v \in \V \; \varphi(\OV) = \varphi(v - v) = \varphi(v + (-1)v) = \\\\
    = \varphi(v) + (-1)\varphi(v) = \varphi(v) - \varphi(v) = \OV \qed \) \\\\
    Доказателство 3: \\\\
    \(\varphi(\OV) = \varphi(\OV + \OV) = \varphi(\OV) + \varphi(\OV) \; | \; -\varphi(\OV) \implies \\\\
    \varphi(\OV) - \varphi(\OV) = \varphi(\OV) + \varphi(\OV) - \varphi(\OV) \implies \\\\
    \OV = \varphi(\OV) \qed \)
    \subsection{Докажете, че \(\forall \; \varphi \in \mathrm{Hom}(\V) \\
    \forall v \in \V \implies \varphi(-v) = -\varphi(v)\)}
    Доказателство: \\\\
    \(\forall \varphi \in \mathrm{Hom}(\V), \; \forall v \in \V, \; \forall \lambda \in \F \implies \varphi(\lambda v) = \lambda \varphi(v) \\\\
    \implies \lambda = -1 \implies \varphi(-v) = \varphi(-1v) = -1\varphi(v) = -\varphi(v) \\\\
    \implies \varphi(-v) = -\varphi(v) \qed\)
    \subsection{Докажете, че едно линейно изображение изпраща линейно зависими вектори в линейно зависими вектори}
    Доказателство: \\\\
    Нека \(\V, \; \W\) - Л.П. над полето \(\F, \; \varphi \in \mathrm{Hom}(\V, \; \W)\) \\\\
    Нека \(n \in \N\) и нека \(v_1, \; \dots, \; v_n \in \V\) - (линейно зависими) \\\\
    \(\implies \exists \n{\lambda} \in \F \; : \;  (\n{\lambda}) \neq (0, \dots, 0) \; : \; \displaystyle\sum_{i = 1}^n \lambda_i v_i = \OV_\V\) \\\\
    Нека \(v = \displaystyle\sum_{i = 1}^n \lambda_i v_i \implies v = \OV_\V \; | \;\varphi
    \implies \varphi(v) = \varphi(\OV_\V) = \OV_\W \implies \\\\
    \varphi(v) = \varphi\left(\displaystyle\sum_{i = 1}^n \lambda_i v_i\right) = \displaystyle\sum_{i = 1}^n \lambda_i \varphi(v_i) = \OV_\W \implies \\\\
    \displaystyle\sum_{i = 1}^n \lambda_i \varphi(v_i) = \OV_\W, \; (\n{\lambda}) \neq (0, \dots, 0) \implies \) \\\\
    Векторите \(\varphi(v_1), \; \dots, \; \varphi(v_n)\) (образите на векторите \(v_1, \; \dots, \; v_n\))
    са линейно зависими \(\qed\) 
    \subsection{Докажете, че един линеен оператор изпраща линейно зависими вектори в линейно зависими вектори)}
    Доказателство: \\\\
    Нека \(\V\) - Л.П. над полето \(\F, \; \varphi \in \mathrm{Hom}(\V)\) \\\\
    Нека \(n \in \N\) и нека \(v_1, \; \dots, \; v_n \in \V\) - (линейно зависими) \\\\
    \(\implies \exists \n{\lambda} \in \F \; : \;  (\n{\lambda}) \neq (0, \dots, 0) \; : \; \displaystyle\sum_{i = 1}^n \lambda_i v_i = \OV\) \\\\
    Нека \(v = \displaystyle\sum_{i = 1}^n \lambda_i v_i \implies v = \OV \; | \;\varphi
    \implies \varphi(v) = \varphi(\OV) = \OV \implies \\\\
    \varphi(v) = \varphi\left(\displaystyle\sum_{i = 1}^n \lambda_i v_i\right) = \displaystyle\sum_{i = 1}^n \lambda_i \varphi(v_i) = \OV \implies \\\\
    \displaystyle\sum_{i = 1}^n \lambda_i \varphi(v_i) = \OV, \; (\n{\lambda}) \neq (0, \dots, 0) \implies \) \\\\
    Векторите \(\varphi(v_1), \; \dots, \; \varphi(v_n)\) (образите на векторите \(v_1, \; \dots, \; v_n\))
    са линейно зависими \(\qed\) 
    \section{Действия с линейни изображения}
    \subsection{Определение за сума на линейни изображения}
    Нека \(\V, \; \W\) - Л.П. над полето \(\F, \; \varphi, \psi \in \mathrm{Hom}(\V, \; \W) \\\\
    \varphi + \psi : \V \to \W \; : \; \forall v \in \V \; (\varphi + \psi)(v) = \varphi(v) + \psi(v)\)
    \subsection{Определение за произведение на линейно изображение със скалар}
    Нека \(\V, \; \W\) - Л.П. над полето \(\F, \; \varphi \in \mathrm{Hom}(\V, \; \W), \; \lambda \in \F \\\\
    \lambda\varphi : \V \to \W \; : \; \forall v \in \V \; (\lambda\varphi)(v) = \lambda.\varphi(v)\)
    \subsection{Определение за произведение на линейни изображения}
    \(\V, \; \W, \; \UV\) - Л.П. над полето \(\F, \; \varphi \in \mathrm{Hom}(\V, \; \W), \psi \in \mathrm{Hom}(\W, \; \UV) \\\\
    \psi\varphi : \V \to \UV \; : \; \forall v \in \V \; (\psi\varphi)(v) = (\psi \circ \varphi)(v) = \psi(\varphi(v))\)
    \subsection{Определението за матрица на линейно изображение}
    Нека \(\V, \; \W\) - К.М.Л.П. над полето \(\F, \; \varphi \in \mathrm{Hom}(\V, \; \W) , \; n = \mathrm{dim}\V, \; m = \mathrm{dim}\W \\\\
    \n{e} - \text{ базис на } \V \\\\
    \m{f} - \text{ базис на } \W \\\\
    \ieqn \quad \varphi (e_i) = \displaystyle\sum_{j = 1}^m \lambda_{ji}f_j, \quad \lambda_{ji} \in \F \\\\
    A = (\lambda_{ji})\x{m}{n} = M_{e \to f}(\varphi) \in \F\x{m}{n}\) - матрица на \(\varphi\) в базисите \(e, \; f\) \\\\
    Тоест стълбовете на матрицата \(A\) са образите на векторите \(\n{e}\) \\\\
    \(A = (\varphi(e_1) \; \dots \; \varphi(e_n))\)
    \subsection{изобразяване на координатите на образа\\
    на вектор под действието на линейно изображение\\
    чрез координатите на вектора и матрицата\\
    на линейното изображение}
    Нека \(\V, \; \W\) - К.М.Л.П. над полето \(\F, \; \varphi \in \mathrm{Hom}(\V, \; \W) , \; n = \mathrm{dim}\V, \; m = \mathrm{dim}\W \\\\
    \n{e} - \text{ базис на } \V \\\\
    \m{f} - \text{ базис на } \W \\\\
    A = (\lambda_{ji})\x{m}{n} = M_{e \to f}(\varphi) \in \F\x{m}{n}\) - матрица на \(\varphi\) в базисите \(e, \; f\) \\\\
    Нека \(v \in \V \implies v = \displaystyle\sum_{i = 1}^n \lambda_i e_i, \quad \n{\lambda} \in \F\) \\\\
    \((\n{\lambda})\) - координатите на \(v\) спрямо базиса \(e\) на \(\V\) \\\\
    \(\varphi(v) \in \W \implies \exists(\m{\mu}) \in \F^m \; : \; \varphi(v)
    = \displaystyle\sum_{i = 1}^n \lambda_i \varphi(e_i) = \displaystyle\sum_{i = 1}^m \mu_i f_i \) \\\\
    \((\m{\mu})\) - координатите на образа на \(v\) спрямо базиса \(f\) на \(\W\) \\\\
    Тогава \((\m{\mu})^t = A(\n{\lambda})^t\)
    \section{Матрици на линейни изображения, получени след действия с ЛИ}
    \subsection{Определение за сума на линейни изображения}
    Нека \(\V, \; \W\) - Л.П. над полето \(\F, \; \varphi, \psi \in \mathrm{Hom}(\V, \; \W) \\\\
    \varphi + \psi : \V \to \W \; : \; \forall v \in \V \; (\varphi + \psi)(v) = \varphi(v) + \psi(v)\)
    \subsection{Определение за матрица на линейно изображение, което е сумата на две линейни изображения}
    Нека \(\V, \; \W\) - K.M.Л.П. над полето \(\F, \; \varphi, \; \psi \in \mathrm{Hom}(\V, \; \W), \; n = \mathrm{dim}\V, \; m = \mathrm{dim}\W\) \\\\
    Нека \(\tau = \varphi + \psi \in \mathrm{Hom}(\V, \; \W)\) \\\\
    Нека \(\n{e} - \text{ базис на } \V\) \\\\
    Нека \(\m{f} - \text{ базис на } \W\) \\\\
    Нека \(A = M_{e \to f}(\varphi) \in \F\x{m}{n}\) - матрицата на \(\varphi\) спрямо базисите \(e, \; f\) \\\\
    Нека \(B = M_{e \to f}(\psi) \in \F\x{m}{n}\) - матрицата на \(\psi\) спрямо базисите \(e, \; f\) \\\\
    Нека \(C = M_{e \to f}(\tau) = M_{e \to f}(\varphi + \psi) = M_{e \to f}(\varphi) + M_{e \to f}(\psi) = A + B \in \F\x{m}{n}\) \\\\
    Тогава \(C\) е матрицата на \(\tau = \varphi + \psi\) спрямо базисите \(e, \; f\)
    \subsection{Определение за матрица на линейно изображение, което е произведение на линейно изображение със скалар}
    Нека \(\V, \; \W\) - K.M.Л.П. над полето \(\F, \; \lambda \in \F, \; \varphi \in \mathrm{Hom}(\V, \; \W), \; n = \mathrm{dim}\V, \; m = \mathrm{dim}\W\) \\\\
    Нека \(\tau = \lambda\varphi \in \mathrm{Hom}(\V, \; \W)\) \\\\
    Нека \(\n{e} - \text{ базис на } \V\) \\\\
    Нека \(\m{f} - \text{ базис на } \W\) \\\\
    Нека \(A = M_{e \to f}(\varphi) \in \F\x{m}{n}\) - матрицата на \(\varphi\) спрямо базисите \(e, \; f\) \\\\
    Нека \(C = M_{e \to f}(\tau) = M_{e \to f}(\lambda\varphi) = \lambda.M_{e \to f}(\varphi) = \lambda.A \in \F\x{m}{n}\) \\\\
    Тогава \(C\) е матрицата на \(\tau = \lambda.\varphi\) спрямо базисите \(e, \; f\)
    \subsection{Определение за матрица на линейно изображение, което е произведение на две линейни изображения}
    Нека \(\V, \; \W, \; \UV\) - K.M.Л.П. над полето \(\F \\\\
    \varphi \in \mathrm{Hom}(\V, \; \W), \; \psi \in \mathrm{Hom}(\W, \; \UV), \;  n = \mathrm{dim}\V, \; m = \mathrm{dim}\W, \; k = \mathrm{dim}\UV\) \\\\
    Нека \(\tau = \psi\varphi \in \mathrm{Hom}(\V, \; \UV)\) \\\\
    Нека \(\n{e} - \text{ базис на } \V\) \\\\
    Нека \(\m{f} - \text{ базис на } \W\) \\\\
    Нека \(g_1, \; \dots, \; g_k - \text{ базис на } \UV\) \\\\
    Нека \(A = M_{e \to f}(\varphi) \in \F\x{m}{n}\) - матрицата на \(\varphi\) спрямо базисите \(e, \; f\) \\\\
    Нека \(B = M_{f \to g}(\psi) \in \F\x{k}{m}\) - матрицата на \(\psi\) спрямо базисите \(f, \; g\) \\\\
    Нека \(C = M_{e \to g}(\tau) = M_{e \to g}(\psi\varphi) = M_{e \to g}(\psi \circ \varphi) = \\\\
    = M_{f \to g}(\psi).M_{e \to f}(\varphi) = B.A \in \F\x{k}{n}\) \\\\
    Тогава \(C\) е матрицата на \(\tau = \psi\varphi\) спрямо базисите \(e, \; g\)
    \subsection{Размерност на Л.П. на всички лин. изображения между две крайно мерни Л.П}
    Нека \(\V, \; \W\) - K.M.Л.П. над полето \(\F, \; n = \mathrm{dim}\V, \; m = \mathrm{dim}\W\) \\\\
    Тогава \(\mathrm{dim}\mathrm{Hom}(\V, \; \W) = m . n\)
    \section{Ядро и Образ на Линейно изображение}
    \(\V, \; \W\) - Л.П. над полето \(\F, \; \varphi \in \mathrm{Hom}(\V, \W)\)
    \subsection{Определение за ядро на лин. изображение}
    \(\mathrm{Ker} \varphi = \{v \in \V \; | \; \varphi(v) =\OV\}\)
    \subsection{Определение за образ за лин. изображение}
    \(\mathrm{Im} \varphi = \{\varphi(v) \; | \; v \in \V\} = \{w \in \W \; | \; \exists v \in \V \; : \; \varphi(v) = w \}\)
    \subsection{Определение за образ на подпространство}
    Нека \(\mathbb{Y} \leq \V\), тогава образа на \(\mathbb{Y}\) под действието на \(\varphi\) се дефинира като: \\\\
    \(\varphi(\mathbb{Y}) = \mathrm{Im}\varphi_{|\mathbb{Y}}
    = \{\varphi(v) \; | \; v \in \mathbb{Y}\} = \{w \in \W \; | \; \exists y \in \mathbb{Y} \; : \; \varphi(y) = w \}\)
    \subsection{Определение за ранг на лин. изображение}
    \(r(\varphi) = \mathrm{dim}\mathrm{Im}\varphi\)
    \subsection{Определение за дефект на лин. изображение}
    \(d(\varphi) = \mathrm{dim}\mathrm{Ker}\varphi\)
    \subsection{Теорема(За ранга и дефекта)}
    \(\UV, \; \mathbb{S}\) - K.M.Л.П. над полето \(\F, \; \psi \in \mathrm{Hom}(\UV, \; \mathbb{S}) \\\\
    dim\UV = p \implies r(\psi) + d(\psi) = p\)
    \subsection{Връзката между ранга на едно лин. изображение и ранга на една неговата матрица относно един всеки (в частност и един) базис е:}
    Нека \(\n{e} - \text{ произволен базис на } \V\) \\\\
    Нека \(\m{f} - \text{ произволен базис на } \W\) \\\\
    Ако \(A = M_{e \to f}(\varphi) \implies r(\varphi) = r(A)\)
    \section{Обратомост на ЛИ и ЛО}
    \subsection{Определение за обратимо линейно изображение}
    Нека \(\V, \; \W\) - Л.П. над полето \(\F, \; \varphi \in \mathrm{Hom}(\V, \; \W)\) \\\\
    \(\varphi\) е обратимо Л.И, ако \(\exists \psi \in \mathrm{Hom}(\W, \; \V) \; : \; \varphi.\psi = \mathrm{id}_\W, \; \psi.\varphi = \mathrm{id}_\V\)
    \subsection{Определение за обратното линейно изображение на дадено линейно изображение}
    Нека \(\V, \; \W\) - Л.П. над полето \(\F, \; \varphi \in \mathrm{Hom}(\V, \; \W)\) \\\\
    Ако \(\varphi\) е обратимо Л.И, то \\\\
    \(\exists! \; \varphi^{-1} \in \mathrm{Hom}(\W, \; \V) \; : \; \varphi.\varphi^{-1} = \mathrm{id}_\W, \; \varphi^{-1}.\varphi = \mathrm{id}_\V \\\\
    \varphi^{-1} \text{ е обратното Л.И. на } \varphi\)
    \subsection{Доказателство обратният на обратим линеен оператор също е обратим}
    Нека \(\V\) - Л.П. над полето \(\F, \; \varphi \in \mathrm{Hom}\V \\\\
    \varphi \text{ - обратим Л.О. } \implies \varphi.\varphi^{-1} = \varphi^{-1}.\varphi = \mathrm{id}_\V\) \\\\
    Ако \(\varphi^{-1}\) е обратим Л.О, то \((\varphi^{-1})^{-1}.\varphi^{-1} = \varphi^{-1}.(\varphi^{-1})^{-1} = \mathrm{id}_\V \\\\
    \varphi = \varphi.\mathrm{id}_\V = \varphi(\varphi^{-1}.(\varphi^{-1})^{-1}) = (\varphi.\varphi^{-1}).(\varphi^{-1})^{-1} = \mathrm{id}_\V.(\varphi^{-1})^{-1} = (\varphi^{-1})^{-1} \\\\
    \implies (\varphi^{-1})^{-1} = \varphi \)
    \subsection{Теорема}
    Нека \(\V, \; \W\) - Л.П. над полето \(\F, \; \varphi \in \mathrm{Hom}(\V, \; \W) \\\\
    \varphi \text{ е интективно } \iff \mathrm{Ker}\varphi = \{\OV_\V\}\) \\\\
    Доказателство (в двете посоки): \\\\
    Доказателство \((\implies) \; \{\OV_\V\} \subseteq \mathrm{Ker}\varphi \; (\OV_\V \in \mathrm{Ker}\varphi)\) \\\\
    Нека \(v \in \mathrm{Ker}\varphi \implies \varphi(v) = \OV_\W = \varphi(\OV_\V) \\\\
    \varphi \text{ - инективно } \implies v = \OV_\V \implies \mathrm{Ker}\varphi \subseteq \{\OV_\V\} \implies \mathrm{Ker}\varphi = \{\OV\}\) \\\\
    Доказателство \((\; \Leftarrow \;)\) \\\\
    Нека \(u, \; v \in \V \; : \; \varphi(u) = \varphi(v) \implies \\\\
    \OV_\V = \varphi(u) - \varphi(v) = \varphi(u - v) \implies \\\\
    u - v = \OV_\V, \; \{\OV_\V\} = \mathrm{Ker}\varphi \implies \\\\
    u = v \implies \varphi \text{ е инективно}\)
    \subsection{Обратимо линейно изображение изпраща линейно независими вектори в линейно независими вектори}
    Нека \(\V, \; \W\) - Л.П. над полето \(\F, \; \mathrm{dim}\V = n, \; \varphi \in \mathrm{Hom}(\V, \; \W)\) - обратимо Л.И. \\\\
    Нека \(k \in \N \; : \; k \leq n\) и нека \(v_1, \; \dots v_k \in \V\) са лин. независи вектори \\\\
    Допускаме, че техните образи са лин. зависими, тоест: \\\\
    \(\exists \lambda_1, \; \dots, \; \lambda_k \in \F \; : \; (\lambda_1, \; \dots, \; \lambda_k) \neq (0, \; \dots, \; 0) \quad \displaystyle\sum_{i = 1}^k \lambda_i \varphi(v_i) = \OV_\W \; | \; \varphi^{-1} \implies \\\\
    \varphi^{-1}\left(\displaystyle\sum_{i = 1}^k \lambda_i \varphi(v_i)\right) = \displaystyle\sum_{i = 1}^k \lambda_i \varphi^{-1}(\varphi(v_i))
    = \displaystyle\sum_{i = 1}^k \lambda_i v_i = \varphi^{-1}(\OV_\W) = \OV_\V \implies \\\\
    \displaystyle\sum_{i = 1}^n \lambda_i v_i = \OV_\V \implies v_1, \; \dots v_k \text{ - лин. зависими } \implies \lightning \\\\
    \implies \varphi(v_1), \; \dots \varphi(v_k) \) - лин. независими
    \section{Смяна на базиса}
    \subsection{Определението за матрица на прехода между два базиса}
    Нека \(\V\) - K.M.Л.П. над полето \(\F, \; \mathrm{dim}\V = n\\
    \n{e} \text { - един базис на } \V \\\\
    \n{f} \text { - друг базис на } \V \\\\
    \ieqn \quad f_i = \displaystyle\sum_{j = 1}^n \tau_{ji} e_j, \quad  j, \; \ieqn, \; \tau_{ji} \in \F \\\\
    T_{e \to f} = (\tau_{ji})\x{n}{n} \in M_n \text{ е матрица на прехода между базисите } e, \; f \text{ на } \V\)
    \subsection{Промяна на координатите на вектор при смяна на базиса}
    Нека \(\V\) - K.M.Л.П. над полето \(\F, \; \mathrm{dim}\V = n\) \\\\
    Нека \(\n{e}\) - един базис на \(\V\) \\\\
    Нека \(\n{f}\) - друг базис на \(\V\) \\\\
    Нека \(v = \displaystyle\sum_{i = 1}^n \lambda_i e_i = \displaystyle\sum_{i = 1}^n \mu_i f_i \in \V, \quad \ieqn \; \lambda_i, \; \mu_i \in \F \\\\
    (\n{\lambda})^t = T_{e \to f}(\n{\mu})^t, \; T_{e \to f}\) - матрицата на прехода между базисите \(e\) и \(f\).
    \subsection{Промяна на матрицата на линейно изображение при смяна на базиса}
    Нека \(\V, \; \W\) - K.M.Л.П. над полето \(\F, \; \mathrm{dim}\V = n, \; \mathrm{dim}\W = m, \; \varphi \in \mathrm{Hom}(\V, \; \W)\) \\\\
    Нека \(\n{s}\) е един базис на \(\V\) \\\\
    Нека \(\n{s'}\) е друг базис на \(\V\) \\\\
    Нека \(\m{u}\) е базис на \(\W\) \\\\
    Нека \(\m{u'}\) е друг базис на \(\W\) \\\\
    Нека \(A = M_{s \to u}(\varphi)\) е матрицата на оператора \(\varphi\) спрямо базисите \(s, \; u\) \\\\
    и нека \(B = T_{s \to s'}\) е матрицата на прехода между базисите \(s\) и \(s'\) \\\\
    и нека \(C = T_{u \to u'}\) е матрицата на прехода между базисите \(u\) и \(u'\) \\\\
    и нека \(D = M_{s' \to u'}(\varphi)\) е матрицата на оператора \(\varphi\) спрямо базисите \(s', \; u'\). \\\\
    Тогава е в сила равенството \(D = CAB^{-1}\). \\\\
    \textit{Бележка}: \(M_{s' \to u'}(\varphi) = T_{u \to u'} \circ M_{s \to u}(\varphi) \circ T_{s' \to s}\). \\\\
    \textit{Комутативна диаграма} \\\\
    \begin{tikzpicture}
        \matrix (m) [matrix of math nodes,row sep=3em,column sep=4em,minimum width=2em]
        {V_s & V_{s'} \\
             W_u & W_{u'} \\};
        \path[-stealth]
                (m-1-1) edge node [left] {$A$} (m-2-1)
        (m-2-1.east|-m-2-2) edge node [below] {$C$} (m-2-2)
        (m-1-2) edge node [right] {$D$} (m-2-2);
        \path[-stealth]
            (m-1-2) edge node [above] {$B^{-1}$} (m-1-1);
    \end{tikzpicture}
    \subsection{Промяна на матрицата на линеен оператор при смяна на базиса}
    Нека \(\V\) - K.M.Л.П. над полето \(\F, \; \mathrm{dim}\V = n, \; \varphi \in \mathrm{Hom}\V\) \\\\
    Нека \(\n{b}\) - един базис на \(\V\) \\\\
    Нека \(\n{b'}\) - друг базис на \(\V\) \\\\
    Нека \(A = M_b(\varphi)\) - матрицата на оператора \(\varphi\) в базиса \(b\) \\\\
    и нека \(B = T_{b \to b'}\) - матрицата на прехода между базисите \(b\) и \(b'\). \\\\
    Тогава \(C = B^{-1}AB\) е матрицата на оператора \(\varphi\) в базиса \(b'\). \\\\
    Тоест \(C = T_{b' \to b}M_b(\varphi)T_{b \to b'} = M_{b'}(\varphi)\).
    \subsection{Първа теорема за ранг на матрици}
    Нека \(m, \; n \in \N, \; A \in \F_{m \times n} \implies r(A) = r(A^t)\)
    \section{Дуалност}
    \subsection{Определение за дуалното пространство на дадено линейно пространство}
    Нека \(\V\) - Л.П. над полето \(\F\) \\\\
    Тогава \(\V^* = \mathrm{Hom}(\V, \; \F)\) е дуалното пространство на Л.П. на \(\V\)
    \subsection{Определение за линеен функционал}
    Нека \(\V, \; \W\) - Л.П. над полето \(\F\) \\\\
    \(f\) е линеен функционал на \(\V \iff f \in \V^*\)
    \subsection{Определение за дуалното изображение на дадено линейно изображение}
    Нека \(\V, \; \W\) - Л.П. над полето \(\F\) \\\\
    \(\V^*, \; \W^*\) - Дуалните пространства на Л.П. \(\V, \; \W \) \\\\
    Ако \(\varphi \in \mathrm{Hom}(\V, \; \W) \implies \varphi^* \in \mathrm{Hom}(\W^*, \; \V^*) \\\\
    \forall f \in \W^* \; \varphi^*(f) = f \circ \varphi\)
    \subsection{Определение за за дуален базис}
    Нека \(\V\) - Л.П. над полето \(\F, \; \mathrm{dim}\V = n, \; \V^* \text{ - дуалното пространство на } \V \\\\
    \n{e} \text{ - базис на } \V \\\\
    f^1, \dots, f^n \text{ - дуален базис на базиса } \n{e} \\\\
    j, \; \ieqn \quad f^i(e_j) = \delta_{ij} = \begin{cases}
        1 & i = j\\
        0 & i \neq j
    \end{cases}\)
    \subsection{Дуално изображение на произведението на две линейни изображения}
    Нека \(\V, \; \W, \; \UV\) - Л.П. над полето \(\F, \; \in \mathrm{Hom}(\V, \; \W), \psi \in \mathrm{Hom}(\W, \; \UV) \\\\
    \psi\varphi \in \mathrm{Hom}(\V, \; \UV) \implies (\psi\varphi)^* \in Hom(\UV^*, \V^*) \\\\
    \forall f \in \UV^* \; (\psi\varphi)^*(f) = f \circ (\psi\varphi) = (f \circ \psi) \circ \varphi = \\\\
    = \varphi^*(f \circ \psi) = \varphi^*(\psi^*(f)) = (\varphi^*\psi^*)(f) \\\\
    \implies (\psi\varphi)^* = \varphi^*\psi^*\)
    \subsection{Връзката между матриците на едно линейно изображение и неговото дуално изображение}
    Нека \(\V, \; \W\) - K.M.Л.П. над полето \(\F \\\\
    \varphi \in \mathrm{Hom}(\V, \; \W), \; \varphi^* \in \mathrm{Hom}(\W^*, \V^*)\) \\\\
    Нека \(\n{e} - \text{ базис на } \V\) \\\\
    Нека \(\n{e'} - \text{ дуален базис на базиса } \n{e}\) \\\\
    Нека \(\m{f} - \text{ базис на } \W\) \\\\
    Нека \(\m{f'} - \text{ дуален базис на базиса } \m{f}\) \\\\
    Тогава \(M_{f' \to e'}(\varphi^*) = (M_{e \to f}(\varphi))^t\)
    \subsection{Определение за анихилатор \(\UV^0\) на едно линейно подпространсво \(\UV\) на едно линейно пространство \(\V\)}
    Нека \(\V\) - Л.П. над полето \(\F \\\\
    \UV < \V \implies \UV^0 = \{f \in \V^* \; | \; \forall u \in \UV \; f(u) = 0\}\)
\end{document}