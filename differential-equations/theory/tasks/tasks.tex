\documentclass[a4paper, 12pt, oneside]{article}
    
\usepackage[left=3cm,right=3cm,top=1cm,bottom=2cm]{geometry}
\usepackage{amsmath,amsthm}
\usepackage{amssymb}
\usepackage{lipsum}
\usepackage{stmaryrd}
\usepackage[T1,T2A]{fontenc}
\usepackage[utf8]{inputenc}
\usepackage[bulgarian]{babel}
\usepackage[normalem]{ulem}
\usepackage{hyperref}
\hypersetup{
    colorlinks=true,
    linkcolor=blue,
    filecolor=magenta,      
    urlcolor=cyan,
}
         
\urlstyle{same}
        
\newcommand{\R}{\mathbb{R}}
\newcommand{\N}{\mathbb{N}}
        
\setlength{\parindent}{0mm}
                
\title{Решения на примерни теоретични задачи по ДУПРИЛ}
\author{Иво Стратев}
                
\begin{document}
\maketitle
    
\tableofcontents

\section{Задача 1.} Колко решения има задачата

\subsection{а)}
\begin{align*}
\begin{cases}
    y' = x^3 + y^3 \\
    y(3) = 3
\end{cases}
\end{align*}
Решение: \\
\begin{align*}
    x^3 + y^3 \in C^\infty (\R^2)
\end{align*}
Уравнението е от първи ред. Следователно има единствено решение. (от Теоремата за съществуване и единственост).
\subsection{б)}
\begin{align*}
    \begin{cases}
        y'^2 - 2xyy' = x^2y^2 + 1\\
        y(2) = 1
    \end{cases}
\end{align*}
Решение: \\

Прехвърляме всичко от едната страна и получаваме
\begin{align*}
    \begin{cases}
        F(x, \; y, \; y') = y'^2 - 2xyy' - x^2y^2 - 1 = 0 \\
        y(2) = 1
    \end{cases}
\end{align*}
Проверяваме вида на точка $(2, \; 1)$. Заместваме в даното уравнение с \\

$r = y'(2), \; y(2), \; x = 2$, тоест $F(2, \; y(2), \; r) = 0$ получаваме \\

$r^2 - 2.2.1.r - 2^2.1^2 - 1 = 0 \implies r^2 - 4r - 5 = 0 \implies (r - 5)(r + 1) = 0$. \\

Пресмятаме $\frac{\partial}{\partial r} F(2, \; 1, \; r) = (r^2 - 4r - 5)' = 2r - 4 = 2(r - 2)$. \\

$F(2, \; 1, \; 2) \neq 0$. Тогава точката е обикновенна и
следователно задачата на Коши има точно две решения (от Лемата за редукцията).
\subsection{в)}
\begin{align*}
    \begin{cases}
        F(x, \; y, \; y') = y'^2 - xyy' + x^2 + y = 0\\
        y(1) = 2
    \end{cases}
\end{align*}
Решение: \\

Търсим на колко е равно $y'(1)$. Заместваме $F(1, \; y(1), \; y'(1)) = 0 \implies \\\\
(y'(1))^2 - 1.2.y'(1) + 1^2 + 2 = 0 \implies (y'(1))^2 - 2.y'(1) + 3 \implies \\\\
D(F(1, \; y(1), \; y'(1))) = 4 - 12 = -8 < 0 \implies \lnot\exists y'(1)$. \\\\
Следователно задачата няма реално решение.

\section{Задача 2}
Колко са решенията на системата
\subsection{а)}
\begin{align*}
    \begin{cases}
        y' = xy^2 + 4 \\
        y(1) = 2 \\
        y'(1) = 3
    \end{cases}
\end{align*}
Решение: \\

$3 = y'(1) = 1.y(1)^2 + 4 = 1.2^2 + 4 = 8 \implies 3 = 8 \implies \lightning$ \\

Следователно системата няма решение.
\subsection{б)}
\begin{align*}
    \begin{cases}
        y' = 2y^2 - 5x \\
        y(1) = 2 \\
        y'(1) = 3
    \end{cases}
\end{align*}
Решение: \\

$3 = y'(1) = 2.y(1)^2 - 5.1 = 8 - 5  = 3 \implies 3 = 3$ \\
Системата 
\begin{align*}
    \begin{cases}
        y' = 2y^2 - 5x \\
        y(1) = 2 
    \end{cases}
\end{align*}
предасталява задача на Коши, която има едно единствено решение. Уловието $y'(1) = 3$
е изпълнено. Тогава общата система има единствено решение.

\subsection{в)}
\begin{align*}
    \begin{cases}
        y'' = 2xy' + 3y + x \\
        y(1) = 2 \\
        y'(1) = 3
    \end{cases}
\end{align*}
Решение: \\

Преобразуваме системата до еквивалентната ѝ
\begin{align*}
    \begin{cases}
        y'' - 2xy' - 3y = x \\
        y(1) = 2 \\
        y'(1) = 3
    \end{cases}
\end{align*}
$-2x, \; -3, \; x \in C^\infty(\R)$  тогава дадената система представлява
задача на Коши за нехомогенно линейно уравнение от втори ред,
следователно системата има единствено решение (при това то е глобално).
\subsection{г)}
\begin{align*}
    \begin{cases}
        y''' = xy'' + (x + 1)y\\
        y(1) = 2 \\
        y'(1) = 3
    \end{cases}
\end{align*}
Решение: \\

Преобразуваме системата до еквивалентната ѝ
\begin{align*}
    \begin{cases}
        y''' - xy'' - (x + 1)y = 0\\
        y(1) = 2 \\
        y'(1) = 3
    \end{cases}
\end{align*}
$-x, \; -(x + 1) \in C^\infty(\R)$.
Уравнението $y''' - xy'' - (x + 1)y = 0$ е линейно уравнение от 3-ти ред.
Множеството от решенията му представлява линейно пространство изоморфно на $\R^3$.
Имаме дадени две начални условия тогава размерността на линейно пространство,
съвпадащо с решенията на системата е $3 - 2 = 1$. Тоест то е изоморфно на цялата
реална права следователно системата има безброй много решения (тя е неопределена).

\section{Задача 3.}
За уравнението \begin{align*}
    (*) \quad y' + y^2 + x^2 = 2xy + 5
\end{align*}
намерете частно решение $y_1(x)$ от вида $ax + b$.
Уравнение от какъв тип за $z(x)$ се получава след полагането
$y(x) = z(x) + y_1(x)$ в $(*)$? \\

Решение: \\

Частното решение $y_1(x) = ax + b$ ще намерим като заместим в даденото уравнение.
\begin{align*}
    y_1' + y_1^2 + x^2 = 2xy_1 + 5 \implies \\\\
    y_1' + y_1^2 + x^2 - 2xy_1 - 5 = 0 \implies \\\\
    a + (ax + b)^2 + x^2 - 2x(ax + b) - 5 = 0 \implies \\\\
    a + a^2x^2 + 2abx + b^2 + x^2 -2ax^2 -2bx - 5 = 0 \implies \\\\
    (a^2 - 2a + 1)x^2 + (2ab - 2b)x + (a + b^2 - 5) = 0 \implies \\\\
    \begin{cases}
        a^2 - 2a + 1 = 0 \\
        2(ab - b) = 0 \\
        a + b^2 - 5 = 0
    \end{cases} \implies \begin{cases}
        a = 1 \\
        b = b
        b^2 = 4
    \end{cases} \implies \begin{cases}
        a = 1 \\
        b = \pm 2
    \end{cases}
\end{align*}
Избираме $y_1(x) = x + 2$. Правим смяната $y(x) = z(x) + y_1(x) = z(x) + x + 2$ \\\\
Получаваме $(z(x) + x + 2)' + (z(x) + x + 2)^2 + x^2 = 2x(z(x) + x + 2) + 5 \implies \\\\
z'(x) + 1 + z^2(x) + 2(x + 2)z(x) + x^2 + 4x + 4 + x^2 = 2xz(x) + 2x^2 + 4x + 5 \\\\
z'(x) + z^2(x) + 2z(x) = 0 \implies z' = z^2 + 2z$.
Това е уравнение с разделящи се променливи, понеже не зависи от $x$,
както и Бернулиево за $\alpha(x) = 2$ и $\beta(x) = 1$. ($z' = \alpha z + \beta z^2$).

\section{Задача 4.}
Дадено е уравнението
\begin{align*}
    (**) \quad y' = \displaystyle\frac{x + 2y + 1}{2x + y - 1}
\end{align*}
а) Намерете пресечната точка $(a, \; b)$ на двете прави
Решение: \\

\begin{align*}
    \begin{cases}
        x + 2y + 1 = 0 \\
        2x + y - 1 = 0
    \end{cases}
\end{align*}
Решение: \\
\begin{align*}
\begin{cases}
    x + 2y + 1 = 0 \\
    2x + y - 1 = 0
\end{cases} \implies
\begin{cases}
    x = -2y - 1 \\
    -3y - 3 = 0
\end{cases} \implies
\begin{cases}
    x = -2y - 1 \\
    y = -1
\end{cases} \implies \begin{cases}
    x = 1 \\
    y = -1
\end{cases}
\end{align*}

б) Уравнение от какъв вид се получава за $z(t) = y(x - a) -b$, като
направите смяната на променливите $x = t + a$ и $y = z + b$ в $(**)$? \\

Решение: \\

Правим смяната
\begin{align*}
\begin{cases}
    x = t + 1 \\
    y = z - 1
\end{cases}
\end{align*} и заместваме в $(**)$
\begin{align*}
    \left(z - 1\right)' = \displaystyle\frac{\left(t + 1\right) + 2\left(z - 1\right) + 1}{2\left(t + 1\right) + \left(z - 1\right) - 1} \implies \\\\
    z' = \displaystyle\frac{t + 2z}{2t + z} = \displaystyle\frac{t\left(1 + 2\frac{z}{t}\right)}{t\left(2 + \frac{z}{t}\right)} = \displaystyle\frac{1 + 2\frac{z}{t}}{2 + \frac{z}{t}}
\end{align*}
Вида на полученото уравнение е хомогенно. След полагане на $p = zt$ се получава уравнение с разделящи се променливи.

\section{Задача 5.}
Дадена е задачата на Коши
\begin{align*}
    \begin{cases}
        y' = a(x)y + b(x)\\
        y(0) = y_0
    \end{cases}
\end{align*}
където $y_0 \in \R$, а коефициентите $a(x)$ и $b(x)$ са непрекъснати функции в интервала $(-6, \; 6)$.
Проверете, че функцията 
\begin{align*}
    y(x) = e^{\displaystyle\int_0^x a(t) \; dt}\left(y_0 + \displaystyle\int_0^x b(t) e^{-\displaystyle\int_0^t a(s) \; ds} \; dt\right)
\end{align*}
е решение на задачата на Коши в интервала $(-6, \; 6)$. \\

Решение: \\

Първо ще проверим, че дадената функция изпълнява условието 
\begin{align*}
    y(x)' = a(x)y(x) + b(x)
\end{align*}
за целта я диференцираме и получаваме
\begin{align*}
    y(x)' = \left[e^{\displaystyle\int_0^x a(t) \; dt}\left(y_0 + \displaystyle\int_0^x b(t) e^{-\displaystyle\int_0^t a(s) \; ds} \; dt\right)\right]' = \\\\
    = \left(e^{\displaystyle\int_0^x a(t) \; dt}\right)'\left(y_0 + \displaystyle\int_0^x b(t) e^{-\displaystyle\int_0^t a(s) \; ds} \; dt\right) \\\\
    + \quad e^{\displaystyle\int_0^x a(t) \; dt}\left(y_0 + \displaystyle\int_0^x b(t) e^{-\displaystyle\int_0^t a(s) \; ds} \; dt\right)' = \\\\
    = \left(\displaystyle\int_0^x a(t) \; dt\right)'e^{\displaystyle\int_0^x a(t) \; dt}\left(y_0 + \displaystyle\int_0^x b(t) e^{-\displaystyle\int_0^t a(s) \; ds} \; dt\right) \\\\
    + \quad e^{\displaystyle\int_0^x a(t) \; dt}\left(\displaystyle\int_0^x b(t) e^{-\displaystyle\int_0^t a(s) \; ds} \; dt\right)' = \\\\
    = a(x)y(x) + e^{\displaystyle\int_0^x a(t) \; dt}\left(b(x) e^{-\displaystyle\int_0^x a(s) \; ds}\right) = \\\\
    = a(x)y(x) + e^{\displaystyle\int_0^x a(t) \; dt}e^{-\displaystyle\int_0^x a(t) \; dt} b(x) = \\\\
    = a(x)y(x) + b(x) \implies y'(x) = a(x)y(x) + b(x).
\end{align*}
Остава да проверим, че $y(0) = y_0$. Ако то е изпълнено дадената функция ще бъде решение на задачата на Коши в интервала $(-6, \; 6)$,
защото там функциите $a(x)$ и $b(x)$ са непрекъснати.
\begin{align*}
    y(0) = e^{\displaystyle\int_0^0 a(t) \; dt}\left(y_0 + \displaystyle\int_0^0 b(t) e^{-\displaystyle\int_0^t a(s) \; ds} \; dt\right) = \\\\
    = e^0(y_0 + 0) = 1.y_0 = y_0.
\end{align*}
Следователно \begin{align*}
    y(x) = e^{\displaystyle\int_0^x a(t) \; dt}\left(y_0 + \displaystyle\int_0^x b(t) e^{-\displaystyle\int_0^t a(s) \; ds} \; dt\right)
\end{align*} е решение на задачата на Коши
\begin{align*}
    \begin{cases}
        y' = a(x)y + b(x)\\
        y(0) = y_0
    \end{cases}
\end{align*}
в интервала $(-6, \; 6)$.

\section{Задача 6.}
Сведете задачата на Коши
\begin{align*}
    \begin{cases}
        y' = y^2 - 2x \\
        y(0) = 1
    \end{cases}
\end{align*}
до интегрално уравнение. Намерете първите три последователни
приближения $(y_0, \; y_1, \; y_2)$ за решението на задачата на Коши. \\

Решение: \\

В даденото уравнение $\frac{\partial}{\partial x} y = y^2 - 2x$ ще сменим променливата $x$ с $v$
получаваме уравнието $\frac{\partial}{\partial v} y = y^2 - 2v$. Интегрираме го в граници от началното условие до $x$.

\begin{align*}
    \frac{\partial}{\partial v} y(v) = y^2(v) - 2v \quad \Big| \quad \displaystyle\int_0^x \; dv \implies \\\\
    \displaystyle\int_0^x \frac{\partial}{\partial v} y(v) \; dv = \displaystyle\int_0^x y^2(v) - 2v \; dv \implies \\\\
    \displaystyle\int_0^x dy = \displaystyle\int_0^x y^2(v) - 2v \; dv \implies \\\\
    y(x) - y(0) = \displaystyle\int_0^x y^2(v) - 2v \; dv \implies \\\\
    y(x) = 1 + \displaystyle\int_0^x y^2(v) - 2v \; dv.
\end{align*}
Полученото интегрално уравнение
\begin{align*}
    y(x) = 1 + \displaystyle\int_0^x y^2(v) - 2v \; dv.
\end{align*} е еквивалентно на задачата на Коши
\begin{align*}
    \begin{cases}
        y' = y^2 - 2x \\
        y(0) = 1
    \end{cases}
\end{align*}
Дефинираме следната редица от последователни приближения
на решението на задачата на Коши по метода на Пикар
\begin{align*}
    y_0(x) \equiv y(0) = 1 \\\\
    \forall n \in \N \quad y_{n + 1}(x) = 1 + \displaystyle\int_0^x y_n(v)^2 - 2v \; dv
\end{align*}
Пресмятаме първите три приближения.
\begin{align*}
    y_0(x) \equiv 1 \quad \text{(константа функция)} \\\\
    y_1(x) = 1 + \displaystyle\int_0^x y_0(v)^2 - 2v \; dv = 1 + \displaystyle\int_0^x 1 - 2v \; dv = \\\\
    = 1 + v\displaystyle|_0^x -2\displaystyle\int_0^x v \; dv = 1 + x -2\frac{v^2}{2}\displaystyle|_0^x = 1 + x - x^2 \\\\
    y_2(x) = 1 + \displaystyle\int_0^x y_1(v)^2 - 2v \; dv = 1 + \displaystyle\int_0^x (1 + v - v^2)^2 - 2v \; dv = \\\\
    = 1 + \displaystyle\int_0^x 1 + 2v + v^2 -2(1 + v)v^2 + v^4 - 2v \; dv = \\\\
    = 1 + \displaystyle\int_0^x 1 -v^2 -2v^3 + v^4 \; dv
    = 1 + x -\frac{x^3}{3} - \frac{x^4}{2} + \frac{x^5}{5}
\end{align*}
Отговор: \\
Интегралното уравнение е $ y(x) = 1 + \displaystyle\int_0^x y^2(v) - 2v \; dv$. \\

Първите триприближения са:
\begin{align*}
    y_0(x) = 1 \\\\
    y_1(x) = 1 + x - x^2 \\\\
    y_2(x) = 1 + x -\frac{x^3}{3} - \frac{x^4}{2} + \frac{x^5}{5}
\end{align*}

\section{Задача 7.}
Приложете теоремата за съществуване и единственост в правоъгълника \\
$\Pi = \{(x, \; y) \; | \; |x| \leq 2, \; |y| \leq 1\}$, за да намерите
интервал, в който съществува решение на задачата на Коши
\begin{align*}
    \begin{cases}
        y' = y^2 - x + 1 \\
        y(0) = 0
    \end{cases}
\end{align*}

Решение: \\

$f(x, \; y) = y^2 - x + 1 \in C^\infty(\R^2)$, тоест $f(x, \; y)$ е безкрайно гладка и в частност е липшищова.
В случая е лесно да се докаже, че $f(x, \; y)$ по втория аргумент в $\Pi$, което ни е необходимо. \\\\

Очевидно $\Pi = \{(x, \; y) \; | \; |x| \leq 2, \; |y| \leq 1\} = [-2, \; 2] \times [-1, \; 1]$. \\

Ще си мислим, че сме фиксирали първия аргумент и ще разглеждаме $F_x(y) = f(x, \; y)$
за фиксирано $x \in [-2, \; 2]$. От теоремата за крайните нараствания приложена за $F_x$
и $y_1, \; y_2 \in [-1, \; 1]$ получаваме $\exists \xi \in [-1, \; 1] \; :$ \\
\begin{align*}
    F_x(y_1) - F_x(y_2) = F_x'(\xi)(y_1 - y_2) \implies \\\\
    |F_x(y_1) - F_x(y_2)| = |F_x'(\xi)(y_1 - y_2)| \implies \\\\
    |F_x(y_1) - F_x(y_2)| = |F_x'(\xi)||y_1 - y_2| \leq \displaystyle\max_{y \in [-1, \; 1]}|F_x'(y)||y_1 - y_2| \implies \\\\
    |F_x(y_1) - F_x(y_2)| \leq \displaystyle\max_{y \in [-1, \; 1]}|2y||y_1 - y_2| = 2|y_1 - y_2| \implies \\\\
    |F_x(y_1) - F_x(y_2)| \leq 2|y_1 - y_2|
\end{align*}
Така чрез теоремата за крайните нараствания доказахме, че наистина $F_x$ е липшищова,
от където у $f$ е липшищова по втория си аргумент. \\\\
Можем и дирекно да докажем, че $F_x$ е липшищова, без да използваме теорамата за крайните нараствания.
Нека отново $y_1, \; y_2 \in [-1, \; 1]$ ще намерим оценка отгоре за $|F_x(y_1) - F_x(y_2)|$
\begin{align*}
    |F_x(y_1) - F_x(y_2)| = |y_1^2 - x + 1 - (y_2^2 - x + 1)| = \\\\
    = |y_1^2 - y_2^2 -x + 1 + x - 1| = |y_1^2 - y_2^2| = \\\\
    = |(y_1 - y_2)(y_1 + y_2)| = |y_1 + y_2||y_1 - y_2| \implies \\\\
    |F_x(y_1) - F_x(y_2)| = |y_1 + y_2||y_1 - y_2| \leq \displaystyle\max_{c, \; d \in [-1, \; 1]}|c + d||y_1 - y_2| = \\\\
    = |\displaystyle\max_{c \in [-1, \; 1]} c + \displaystyle\max_{d \in [-1, \; 1]} d||y_1 - y_2| = |1 + 1||y_1 - y_2| = 2|y_1 - y_2| \implies \\\\
    |F_x(y_1) - F_x(y_2)| \leq 2|y_1 - y_2|
\end{align*}
И в двата случая е ясно, че $\forall x \in [-2, \; 2] \; F_x$ е липшищова, което значи, че $f$ е липшищова по втория си аргумент. \\

Нека $M = \displaystyle\max_{(x, \; y) \in \Pi} f(x, \; y) = \displaystyle\max_{(x, \; y) \in \Pi} y^2 -x + 1
    = \displaystyle\max_{y \in [-1, \; 1]} y^2 + \displaystyle\max_{x \in [-2, \; 2]} (-x) + 1 = \\\\
    = 1^2 -\displaystyle\min_{x \in [-2, \; 2]} x + 1 = 2 -(-2) = 2 + 2 = 4 $. \\\\
Нека $h = \min(2, \; \frac{1}{M}) = \min(2, \frac{1}{4}) = \frac{1}{4}$.
Тогава съществува решение на задачата на Коши в интервала $(0 - h, \; 0 + h) = \left(-\frac{1}{4}, \; \frac{1}{4}\right)$.

\section{Задача 8.}
Докажете, че решението на задачата на Коши
\begin{align*}
    \begin{cases}
        y' = xy^2 - x^3 \\
        y(0) = 1
    \end{cases}
\end{align*}
е четна функция. \\

Решение: \\

Нека $\varphi(x)$ е непрекъснато решение на задачата на Коши в интервала $(\alpha, \; \beta)$.
Тоест нека $\varphi'(x) = x\varphi^2(x) - x^3$ и $\varphi(0) = 1$.
Нека $\psi(x) = \varphi(-x)$. Тогава $\psi(0) = \varphi(-0) = \varphi(0) = 1$.
\begin{align*}
    \psi'(x) = (\varphi(-x)) = \varphi'(-x)(-x)' = -\varphi'(-x) = \\\\
    = -((-x)\varphi^2(-x) - (-x)^3) = -(-x\varphi^2(-x) + x^3) = \\\\
    = x\varphi^2(-x) - x^3 = x\psi^2(x) - x^3 \implies \psi'(x) = x\psi^2(x) - x^3
\end{align*}
Следователно $\psi(x)$ е решение на същата задача на Коши в интервала $(\gamma, \; \delta)$.
От теоремата за съществуване и единственост следва, че \\

$\forall x \in (\alpha, \; \beta) \cap (\gamma, \; \delta) \; \varphi(x) \equiv \psi(x) \equiv \varphi(-x)$. \\

Следователно $\varphi(x)$ е четна функция и $(\gamma, \; \delta) = (\alpha, \; \beta)$.
\section{Задача 9.}
За уравнението $(x^2 + y^2 + x)dx + ydy = 0$ намерете интегриращ множите от вида $\mu(x^2 + y^2)$. \\

Решение: \\
Нека $P(x, \; y) = x^2 + y^2 + x$ и $Q(x, \; y) = y$ и $t(x, \; y) = x^2 + y^2$
Ако $\mu(t)$ е интегриращ множеител за $Pdx + Qdy = 0$, то
\begin{align*}
    \frac{\partial}{\partial y}\mu P =  \frac{\partial}{\partial x}\mu Q \implies \\\\
    \mu'yP + \mu P'y = \mu'xQ + \mu Q'x \implies \\\\
    \mu(P'y - Q'x) = \mu'xQ - \mu'yP \implies \\\\
    \mu(P'y - Q'x) = \mu't.t'x.Q - \mu't.t'y.P \implies \\\\
    \mu(P'y - Q'x) = \mu't(t'x.Q - t'y.P) \implies \\\\
    \frac{\mu't}{\mu} = \frac{P'y - Q'x}{t'x.Q - t'y.P} = \frac{2y - 0}{2xy - 2y(x^2 + y^2 + x)} \implies \\\\
    \frac{\mu't}{\mu} = -\frac{1}{x^2 + y^2} = -\frac{1}{t} \implies \\\\
    \frac{\partial}{\partial t}\mu(t) \frac{1}{\mu(t)} = -\frac{1}{t} \; \Big| \; \displaystyle\int \; \partial t \implies \\\\
    \displaystyle\int \frac{1}{\mu(t)} \frac{\partial}{\partial t}\mu(t) \; \partial t = -\displaystyle\int \frac{1}{t}\; \partial t \implies \\\\
    \displaystyle\int \frac{1}{\mu} \; \partial \mu = -\ln |t| + c \implies \\\\
    \ln|\mu| = - \ln|t| + C \implies \ln|\mu.t| = c \; | \; e \implies \\\\
    |\mu.t| = e^c \implies \mu.t = C \implies \mu = \frac{C}{t} = \frac{C}{x^2 + y^2}.
\end{align*}
Тогава един интегриращ множител е $\mu(x^2 + y^2) = \displaystyle\frac{1}{x^2 + y^2}$.

\section{Задача 10.}
Намерете особените точки за уравнението
\begin{align*}x(y'^2 + 4) = 2yy'\end{align*}
Има ли уравнението особени решения? \\

Решение: \\

Нека $F(x, \; y, \; z) = x(z^2 + 4) -2yz$ и нека $(x_0, \; y_0) \in \R^2$.
Тогава търсим корените на уравнението $F(x_0, \; y_0, \; z) = 0$. Тоест $x_0(z^2 + 4) -2y_0z = 0$. \\

Да разгледаме частния случай когато $x_0 = 0$. Тогава получаваме, че $y_0z = 0$. \\

Ако $y_0 = 0$ то получаваме, че всяко $z \in \R$ е корен. \\
Тогава имаме и че $\frac{\partial}{\partial z} F(0, \; 0, \; z) = \frac{\partial}{\partial z} 0 = 0$. Тоест точката $(0, \; 0)$ е особена.\\

Ако $y_0 \neq 0$ то директно следва, че корен е $z = 0$. Тогава пресмятаме
$\left(\frac{\partial}{\partial z} F(0, \; y_0, \; z)\right)(0) = (-2y_0)(0) = -2y_0 \neq 0$. Тогава точките от вида $(0, \; y_0) \; : \; y_0 \in \R\backslash\{0\}$ са обикновени.\\

Ако $x_0 \neq 0$ то $F(x_0, \; y_0, \; z) = x_0z^2 - 2y_0z + 4x_0 = 0$. \\
$D_{F(x_0, \; y_0, \; z)} = 4y_0^2 - 16x_0^2 = 4(y_0^2 - 4x_0^2)$. \\

Ако $y_0^2 - 4x_0^2 < 0$ то уравнението няма реални корени.\\

Ако $y_0^2 - 4x_0^2 = 0$, тоест $y_0 = \pm 2x_0$
Получаваме един корен на уравнението, който е $z_{1, \; 2} = \frac{y_0}{x_0}$.
Пресмятаме $\left(\frac{\partial}{\partial z} F(x_0, \; y_0, \; z)\right)\left(\frac{y_0}{x_0}\right) = (2x_0z - 2y_0)\left(\frac{y_0}{x_0}\right) = 0$.
Тогава точките $\{(r, \; \pm 2r) \; | \; r \in \R\backslash\{0\}\}$ са особени.
Но ние получихме, че точката $(0, \; 0)$ също е особена. Тогава правите $y(x) = 2x$
и $y(x) = -2x$ са криви изцяло от особени точки. Заместваме в уравнението $x(y'^2 + 4) = 2yy'$ с $y = 2x$.
Получаваме $x(4 + 4) = 8x = 2.2x.2$. Тогава правата $y(x) = 2x$ е особено решение.
Заместваме и с $y = -2x$ и получаваме $x(4 + 4) = 8x = 2.(-2x).(-2)$, което значи и че
правата $y = -2x$ е особено решение на даденото уравнение. \\

Ако $y_0^2 - 4x_0^2 > 0$ то уравнението $F(x_0, \; y_0, \; z) = 0$ има два корена \\
$z_{1, \; 2} = \displaystyle\frac{y_0 \pm \sqrt{y_0^2 - 4x_0^2 }}{x_0}$.
Пресмятаме $\left(\frac{\partial}{\partial z} F(x_0, \; y_0, \; z)\right)(z_{1, \; 2}) = (2x_0z - 2y_0)(z_{1, \; 2}) = \\\\
= 2(y_0 \pm \sqrt{y_0^2 - 4x_0^2}) - 2y_0 = \pm2\sqrt{y_0^2 - 4x_0^2} \neq 0$ тогава точките
$\{(c, \; d) \; | \; c \in \R\backslash\{0\}, \; |c| > 2|d|\}$ са обикновени. \\

Отговор: \\

Всички особени точки са точките от вида $\{(r, \; \pm 2r) \; | \; r \in \R\}$
и уравнението $x(y'^2 + 4) = 2yy'$ има две особени решения $y(x) = 2x$ и $y(x) = -2x$.

\section{Задача 11}
Като използвате теоремата за единственост на решението
на задачата на Коши за линейни уравнения, докажете, че
всяко решение на уравнението $y'' + y = 0$ може да се
представи като линейна комбинация на функциите $\sin x$
и $\cos x$. \\

Решение: \\

Ние знаем, че $\sin x$ е решение на задачата на Коши
\begin{align*}
    \begin{cases}
        y'' + y = 0 \\
        y(0) = 0 \\
        y'(0) = 1
    \end{cases}
\end{align*}
и $\cos x$ е решение на задатачата на Коши
\begin{align*}
    \begin{cases}
        y'' + y = 0 \\
        y(0) = 1 \\
        y'(0) = 0
    \end{cases}
\end{align*}
Също така детерминанта на Вронски на двете функции \\
$\forall x \in \R \; \begin{vmatrix}
    \cos x & \sin x \\
    -\sin x & \cos x
\end{vmatrix} = \sin^2 x + \cos^2 x = 1 \neq 0$ тоест те са линейно независими
тогава всяко решение на уравнението $y'' + y = 0$ се представя като тяхна линейна
комбинация.  Сега ще докажем само чрез теоремата за единственост на решението
на задачата на Коши за линейни уравнения, че всяко решение на $y'' + y = 0$
се представя като линейна комбинация на $\sin x$ и $\cos x$.
Нека
\begin{align*}
    \varphi(x) = a\sin x + b \cos x \\\\
    \varphi(0) = a.0 + b.1 = b \\\\
    \varphi'(0) = a.1 + b.0 = a \implies \\\\
    \varphi(x) = \varphi'(0)\sin x + \varphi(0)\cos x \\\\
    \varphi''(x) = -a\sin x - b\cos x = -\varphi(x) \implies \\\\
    \varphi''(x) + \varphi(x) = -\varphi(x) + \varphi(x) = 0
\end{align*} следователно $\varphi(x)$ е единствено решение на задачата на Коши
\begin{align*}
    \begin{cases}
        y'' + y = 0 \\
        y(0) = a \\
        y'(0) = b
    \end{cases}
\end{align*}
$\varphi(x) = \begin{pmatrix}
    \varphi(0)\cos x \\
    \varphi'(0)\sin x
\end{pmatrix} \in \displaystyle l \left(\begin{pmatrix}
    \cos(0) \\
    \cos'(0)
\end{pmatrix} \cos x , \; \begin{pmatrix}
    \sin(0) \\
    \sin'(0)
\end{pmatrix} \sin x\right)$

\section{Задача 12.}
Възможно ли е  да се допират графиките на две различни решения
на уеавнението $y'' -xy' + x^2y = 1$? Защо? \\

Решение: \\

Нека $y_1, \; y_2$ са две решения на уравнението $y'' -xy' + x^2y = 1$,
които се допират. Тогава $\exists a, \; b, \; c \in \R \; : \; y_1(a) = y_2(a) = b$ (графиките им минават през една обща точка)
и $y_1'(a) = y_2'(a) = c$ (допират се, имат обаща допирателна в точката $(a, \; b)$).
Допускаме, че $y_1 \not\equiv y_2$. За $y_1$ е изпълнено
\begin{align*}
    \begin{cases}
        y_1'' -xy_1' + x^2y_1 = 1 \\
        y_1(a) = b \\
        y_1'(a) = c
    \end{cases}
\end{align*} за $y_2$ е изпълнено
\begin{align*}
    \begin{cases}
        y_2'' -xy_2' + x^2y_2 = 1 \\
        y_2(a) = b \\
        y_2'(a) = c
    \end{cases}
\end{align*}, но тогава $y_1, \; y_2$ са решение на една и съща задача на Коши
\begin{align*}
    \begin{cases}
        y'' -xy' + x^2y = 1 \\
        y(a) = b \\
        y'(a) = c
    \end{cases}
\end{align*}
От теоремата за съществуване и единственост на задача на Коши за линейно уравнение
следва, че $y_1 \equiv y_2$, но това противоречи на допускането. Следователно не е
възможно графиките на две различни решения на уравнението $y'' -xy' + x^2y = 1$ да
се допират.

\section{Задача 13.}
Нека $\varphi(x)$ е непродължимото решение на задачата на Коши
\begin{align*}
    \begin{cases}
        y'' = x^2y + 1 \\
        y(0) = 1 \\
        y'(0) = 0
    \end{cases}
\end{align*}

а) Какъв е дефиниционния интервал на $\varphi(x)$? \\

Решение: \\

$x^2, \; 1 \in C^\infty(\R)$. Реда на уравнението е втори
и са дадени две начални условия в една и съща точка тогава
дадената задача на Коши има единствено решение $\varphi$ дефинирано
в цялото $\R$. \\

б) Каква е най-малката стойност на $\varphi$? Защо ? \\

Решение: \\

$\varphi(0) = 0$ и $\varphi''(0) = 0^2\varphi(0) + 1 = 1 > 0$ следователно
в $0$ се реализира локален минимум за $\varphi$. Ако $\varphi(x) \geq 0 \implies \varphi''(x) = x^2\varphi(x) + 1 \geq 1 > 0$.
$\varphi(0) = 1 > 0$. След като $\varphi(0) = 1 > 0$ то в околност на $0$ $\varphi > 0 \implies \varphi'' > 0 \implies \varphi$
е изпъкнала функция. Тогава $\varphi$ няма други екстремуми освен в $0$, в който има локален
минимум тогава най-малката стойснот е $\varphi(0) = 1$.

\section{Задача 14.}

На чертежаса изобразени графиките на три непрекъснати
в интервала $[a, \; b]$ функции $f_1(x), \; f_2(x), \; f_3(x)$.
Линейно зависими ли са функциите $f_1(x), \; f_2(x), \; f_3(x)$
в интервала $[a, \; b]$? Защо? \\

а) \\

На графиката се вижда, че
$\exists c, \; d \in [a, \; b]$, за които да е изпълено
\begin{align*}
    f_3(c) = f_3'(c) = f_3'(d) = f_3(d) = 0 \\
    0 < f_1(c) < f_2(c) \\
    f_1(d) > f_2(d) > 0
\end{align*}

Да допуснем, че $f_1$, $f_2$ и $f_3$ са линейно зависими то тогава \\
$\forall x \in [a, \; b] \; \exists c_1, \; c_2, \; c_3 \in \R \; : \; (c_1, \; c_2, \; c_3) \neq (0, \; 0, \; 0) \; \land \;
 c_1f_1(x) + c_2f_2(x) + c_3f_3(x) = 0$, но тогава $(c_1f_1(x) + c_2f_2(x) + c_3f_3(x))' = (0)' = 0 \implies c_1f_1'(x) + c_2f_2'(x) + c_3'f_3(x) = 0$
Следователно
\begin{align*}
    \begin{cases}
        c_1f_1(c) + c_2f_2(c) = 0 \\
        c_1f_1(d) + c_2f_2(d) = 0
    \end{cases} \implies \begin{cases}
        c_1f_1(c) = -c_2f_2(c) \\
        c_1f_1(d) = -c_2f_2(d)
    \end{cases} \implies \begin{cases}
        c_1 \geq -c_2 \\
        c_1 \leq c_2
    \end{cases} \implies \\\\
    c_1 = c_2 = 0 \; \text{(линейно независими са)}
\end{align*}
Така получаваме, че $\forall x \in [a, \; b] \; c_3f_3(x) = 0$,
от графиката е ясно, че $f_3 \not\equiv 0$ тогава $c_3 = 0$,
но това значи, че $f_1$, $f_2$ и $f_3$ са линейно независими. \\

б) \\

На графиката се вижда, че $f_1$ и $f_2$ са линейни фунции
с ненулев наклон пресичащи се в една обаща точка, в която
и двете функции се нулират. Тогава
\begin{align*}
    f_1(x) = p_1x + q_1 \; \land \; p_1 \neq 0 \\
    f_2(x) = p_2x + q_2 \; \land \; p_2 \neq 0 \\
    \exists x_0 \in [a, \; b] \; : \; f_1(x_0) = f_2(x_0) = 0 \implies \\
    p_1x_0 + q_1 = 0 \; \land \; p_2x_0 + q_2 = 0 \implies \\
    q_1 = -p_1x_0  \; \land \; q_2 = -p_2x_0 \implies \\
    f_1(x) = p_1(x - x_0) \\
    f_2(x) = p_2(x - x_0) \implies \\
    p_2f_1(x) - p_1f_2(x) = p_2p_1(x - x_0) - p_1p_2(x - x_0) \equiv 0
\end{align*}
$p_1, \; p_2 \neq 0$  тогава $f_1, \; f_2$ са линейно зависими,
но тогава $f_1$, $f_2$ и $f_3$ са линейно зависими.

\section{Задача 15.}
Пресметнете детерминанта на Вронски на двойката функции
$y_1(x) = 2 - 3x^2$ и $y_2(x) = 2x^3 + 1$. Могат ли $y_1(x)$
и $y_2(x)$ да са решения в интервала $(-2, \; 2)$ на едно и
също линейно уравнение 
\begin{align*}
    y''(x) + a(x)y'(x) + b(x)y = 0
\end{align*}
с непрекъснати коефициенти $a(x), \; b(x) \in C(-2, \; 2)$? Защо? \\

Решение: \\

\begin{align*}
    W(y_1, \; y_2)(x) = \begin{vmatrix}
        y_1(x) & y_2(x) \\
        y_1'(x) & y_2'(x)
    \end{vmatrix} = \begin{vmatrix}
        2 - 3x^2 & 2x^3 + 1 \\
        -6x & 6x
    \end{vmatrix} = \\\\
    = 6x(2 - 3x^2 + 2x^3 + 1) = 6x(3 - 3x^2 + 2x^3)
\end{align*}
Получаваме $W(y_1, \; y_2)(0) = 0 \neq 6(3 - 3 + 2) = 12 = W(y_1, \; y_2)(1)$.
Следователно $y_1(x)$ и $y_2(x)$ не са решение на $y''(x) + a(x)y'(x) + b(x)y = 0$
в интервала $(-2, \; 2)$.

\section{Задача 16.}
При какви стойности на реалния параметър $k$ уравнението
\begin{align*}
    y'' + ky = \sin \pi x
\end{align*}
няма нито едно периодично решение? \\

Решение: \\

Нулевата функция е решение на хомогенното уравнение $y'' + ky = 0$.
Но това значи, че частно решение на $y'' + ky = \sin \pi x$ е решение
на същото уравние не зависимо от стойността на параметъра $k$. \\
Очевидно $\sin \pi x \in l(\sin \pi x, \; \cos \pi x) = l(e^{0 + i\pi}) = l(e^{0 - i\pi})$.
Тогава частното решение е от видa $x^s(a\sin \pi x + b\cos \pi x)$, където
$s$ е кратността на двойката комплексно спрегнати корени $\pm i\pi$ на характеристичното уравнение $\lambda^2 + k = 0$.
Ако $s = 0$ частното решение е периодично, това се случва, когато $k \neq \pi^2$.
Тогава ако $k = \pi^2$ частното решение е непериодично.
Тогава решението на уравнението е от следния вид $y(x) = c_1\sin \pi x + c_2\cos \pi x + x(a\sin \pi x + b\cos \pi x)$,
което очевидно е непериодично. \\

Отговор $k = \pi^2$.

\section{Задача 17.}
Дадено е уравнението
\begin{align*}
    y'' + ay' + 4y = 0
\end{align*}
където $a$ е реален параметър. \\

а) При какви стойности на $a$ всички решения на уравнението са ограничени
за $x \in \R$? \\

б) При какви стойности на $a$ всички решения на уравнението клонят към $0$
при $x \to -\infty?$ \\

в) При какви стойности на $a$ уравнението има поне едно периодично решение различно
от $y(x) \equiv 0$?

Решение: \\

Даденото уравнение от хомогенно линейно втори ред. Тогава то или има два различни реални корена $\lambda_1, \; \lambda_2$
или има един двоен реален корен $\lambda$ или има двойка комплексно спрегнати корени $\gamma, \; \overline{\gamma}$. \\

В случая на два реални и различни корена общия вид на решенията е $y(x) = c_1e^{\lambda_1x} + c_2e^{\lambda_2x}$,
и двете базисини функции $e^{\lambda_1x}, \; e^{\lambda_2x}$ са неограничени, което значи, че и общото решение е неограничено.
Това се случва когато е изпълнено неравенството $a^2 - 16 > 0$ или $|a| > 4$. За да клони общото решение към $0$ при $x \to -\infty$. 
То трябва да имаме поне един положителен корен и той да е по-голям по абсолютна стойност от отрцателния, ако има такъв.
Двата корена на характеристичното уравнение са \\

$\lambda_{1, \; 2} = \displaystyle\frac{-a \pm \sqrt{a^2 - 16}}{2}$. Ако $a > 4$ това няма как да е изпълнено.
Тогава за да имаме по-голям по абсолютна стойност положителен корен със сигурност трябва да изпълнено неравенството
$a < -4$, което е и достатъчно условие положителния корен да е по абсолютна стойност по-голям от отрицателния. \\

В случая двоен корен, ФСР на системaта е $\{e^{\lambda x}, \; xe^{\lambda x}\}$ и двете
фунцкиции отново са неограничени, тогава и решението е неограничено. Двойният корен е $\lambda = -2$.
Тогава общото решение на уравнението е $y(x) = c_1e^{-2x} + c_2xe^{-2x}$ и в такъв случай
$\displaystyle\lim_{x \to -\infty} y(x) = \displaystyle\lim_{x \to -\infty} c_1e^{-2x} + c_2xe^{-2x} = \infty \neq 0$. \\

В случая на двойка комплексно спрегнати корени от вида $\alpha \pm i \beta$
ФСР на уравнението е $\{e^\alpha\cos \beta x, \; e^\alpha\sin \beta x\}$ общото решение
е тяхна линейна комбинация, която е ограничена и периодична функция.
Това е изпълнено, когато $a^2 - 16 < 0$ или $|a| < 4$. \\

Отговори: \\

a) $|a| < 4$ \\

б) $a < -4$ \\

в) $|a| < 4$

\section{Задача 18.}
Дадено е уравниението
\begin{align*}
    (x - 1)y'' + (x - 2)y' - y = 0, \; x > 0
\end{align*}

а) Намерете две частни решения на уравнението от вида $y_1(x) = e^{ax}$
и $y_2(x) = bx + c, \; b \neq 0$. \\

Решение: \\

Заместваме в уравнението с $y_1(x) = e^{ax}$ и получаваме
\begin{align*}
    (x - 1)a^2e^{ax} + (x - 2)ae^{ax} - e^{ax} = 0 \implies \\\\
   a^2(x - 1) + (x - 2)a - 1 = 0 \implies \\\\
   (a^2 + a)x - (a^2 + 2a + 1) = 0 \implies \\\\
    \begin{cases}
        a(a + 1) = 0 \\
        (a + 1)^2 = 0
    \end{cases} \implies a = -1
\end{align*}
Тогава $y_1(x) = e^{-x}$ е частно решение. Заместваме и с $y_2(x) = bx + c$
\begin{align*}
    (x - 1).0 + (x - 2)b - bx - c = 0 \implies \\\\
    bx -2b -bx - c = 0 \implies \\\\
    2b + c = 0 \implies c = -2b
\end{align*}
Имаме условието $b \neq 0$ тогава избираме $b = 1$
и получаваме решението \\
$y_2(x) = x - 2$. \\

б) Покажете, че намерените частни решения $y_1(x)$ и $y_2(x)$
са линейно независими в интервала $(1, \; +\infty)$. \\

Решение: \\

Пресмятаме детерминанта на вронски на намерените частни решения.
\begin{align*}
    W(y_1, \; y_2)(x) = \begin{vmatrix}
        y_1(x) & y_2(x) \\
        y_1'(x) & y_2'(x)
    \end{vmatrix} = \begin{vmatrix}
        e^{-x}& x - 2 \\
        -e^{-x} & 1
    \end{vmatrix} = e^{-x}(1 + x - 2) = e^{-x}(x - 1) \\\\
    \forall x \in (1, \; +\infty) \; e^{-x} \neq 0 \implies e^{-x}(x - 1) = 0 \iff (x - 1) = 0 \iff x = 1 \\\\
    \implies \forall x \in (1, \; +\infty) \quad 0 \neq e^{-x}(x - 1) = W(y_1, \; y_2)(x)
\end{align*}
Следователно $y_1$ и $y_2$ са линейно независими в интервала $(1, \; +\infty)$. \\

в) Намерете общото решение на уравнението. \\

Доказахме, че $y_1$ и $y_2$ са линейно независими в интервала $(1, \; +\infty)$.
Те са две на брой и са решения на на лийено хомогенно уравнение от втори ред с
непрекъснати коефициенти, тогава те образуват ФСР на линейното пространство съвпадащо
с решенията на хомогенното линейно уравнение. Следователно общото решение на уравнеието един
\begin{align*}
    y(x) = c_1e^{-x} + c_2(x - 2)
\end{align*}

\section{Задача 19.}
Нека $x(t)$ и $y(t)$ са решение на системата
\begin{align*}
    \begin{cases}
        \dot{x} = x + 2y + e^t \\
        \dot{y} = 4x + 3y
    \end{cases}
\end{align*}
Намерете линейно диференциално уравнение с постоянни коефициенти,
което се удовлетворява от функцията $x(t)$. \\

Решение: \\

Диференцираме уравнението с лява част съдържаща $x$ и получаваме
\begin{align*}
    \ddot{x} = \dot{x} + 2\dot{y} + e^t
    = \dot{x} + 8x + 6y + e^t
\end{align*}
От същото уравнение изразяваме $y$ и получаваме $2y = \dot{x} -x - e^t$.
Заместваме в полученото уравнение и получаваме
\begin{align*}
    \ddot{x} = \dot{x} + 8x + 3(\dot{x} -x - e^t) + e^t \implies \\\\
    \ddot{x} = 4\dot{x} + 5x - 2e^t \implies \ddot{x} - 4\dot{x} - 5x = 2e^t
\end{align*} \\

Отговор: $\ddot{x} - 4\dot{x} - 5x = 2e^t$ \\

\section{Задача 20.}
Приложете теоремата за съществуване и единственост в цилиндъра \\
$G = \{(t, \; x, \; y) \; | \; |t| \leq 2, \; x^2 + y^2 \leq 1\}$,
за да намерите интервал, в който съществува решение на задачата на Коши
\begin{align*}
    \begin{cases}
        \dot{x} = x + 3y \\
        \dot{y} = 2x^2 + y \\
        x(0) = 0 \\
        y(0) = 0
    \end{cases}
\end{align*}

Решение: \\

Нека $M = \displaystyle\max_{x^2 + y^2 \leq 1} \sqrt{(x + 3y)^2 + (2x^2 + y)} \leq \sqrt{(1 + 3)^2 + (2 + 1)^2} = \sqrt{16 + 9} = \sqrt{25} = 5$
и нека $h = \min\left(2, \; \frac{1}{M}\right) = \min\left(2, \; \frac{1}{5}\right) = \frac{1}{5}$.
Тогава $t \in (-h, \; h) = \left(-\frac{1}{5}, \; \frac{1}{5}\right)$

\section{Задача 21.}

Сведете задачата на Коши
\begin{align*}
    \begin{cases}
        \dot{x} = 1 - y \\
        \dot{y} = x + t \\
        x(0) = 0 \\
        y(0) = 1
    \end{cases}
\end{align*}
до система от две интегрални уравнения. Намерете първите три последователни
приближения $(x_0, \; y_0 ; \; x_1, \; y_1 \; x_2, \; y_2)$ на решението
на задачата на Коши. \\

Решение: \\

$\dot{x}(t) = 1 - y \implies
\displaystyle\int_0^t \dot{x}(s) \; ds = \displaystyle\int_0^t 1 - y(s) \; ds \implies
x(t) - x(0) = \displaystyle\int_0^t 1 - y(s) \; d \implies x(t) = \displaystyle\int_0^t 1 - y(s) \; ds$ \\

$\dot{y} = x + t \implies \displaystyle\int_0^t \dot{y} \; dt = \displaystyle\int_0^t x(s) + s \; ds \implies
y(t) - y(0) = \displaystyle\int_0^t x(s) + s \; ds \implies y(t) = 1 + \displaystyle\int_0^t x(s) + s \; ds$
Тогава задачата на Коши е еквивалентната със следнта система от две интегрални уравнениея
\begin{align*}
    \begin{cases}
        x(t) = \displaystyle\int_0^t 1 - y(s) \; ds \\
        y(t) = 1 + \displaystyle\int_0^t x(s) + s \; ds
    \end{cases}
\end{align*}
Дефинираме следните функционални редици \\
$\forall n \in \N \; x_n(x) = \begin{cases}
    0 & , \; n = 0 \\
    \displaystyle\int_0^t 1 - y_{n - 1}(s) \; ds & , \; n > 0
\end{cases}$
$\forall n \in \N \; y_n(x) = \begin{cases}
    1 & , \; n = 0 \\
    1 + \displaystyle\int_0^t x_{n - 1}(s) + s \; ds & , \; n > 0
\end{cases}$ \\

Тогава $x_0 = 0, \; y_0 = 1$. Пресмятаме последователно $x_1, \; y_1, \; x_2, \; y_2$. \\

$x_1 = \displaystyle\int_0^t 1 - 1 \; ds = 0$, \; $y_1 1 + \displaystyle\int_0^t s \; ds = 1 + \frac{t^2}{2}$,
$x_2 = \displaystyle\int_0^t 1 - (1 + \frac{s^2}{2}) \; ds = -\frac{s^3}{6}$ \\\\

$y_2 = 1 + \displaystyle\int_0^t s \; ds = 1 + \frac{t^2}{2}$

\section{Задача 22}
Нека функциите $x(t)$ и $y(t)$ удовлетворяват системата
\begin{align*}
    \begin{cases}
        \dot{x} = xy \\
        \dot{y} = x^2
    \end{cases}
\end{align*}

а) Покажете, че $x^2(t) - y^2(t)$ не зависи от $t$. \\

Решение: \\

Пресмятаме $\displaystyle\frac{\partial}{\partial t} x^2(t) - y^2(t) = 2x\dot{x} - 2y\dot{y} = 2x^2y - 2yx^2 = 0$ \\

Следователно $x^2 - y^2 = C$, тоест $x^2 - y^2$ не зависи от $t$, което
означава, че функцията $u(x, \; y) = x^2 - y^2$ е пръв интеграл на системата. \\

б) Определете равновесните точки на системата. Начертайте фазов портрет на системата.
Кои равновесни точки са устойчиви? \\

Решение: \\

Особените точки на системата са решенията на системата
\begin{align*}
    \begin{cases}
        \dot{x} = xy = 0 \\
        \dot{y} = x^2 = 0
    \end{cases}
\end{align*}
Следователно ординатата ос е права от особени точки и други особени точки няма. \\

Понеже $u(x, \; y) = x^2 - y^2$ е пръв интеграл на системата то всички фазови криви са от вида
$x^2 - y^2 = C$, които представляват хипербули. Системата
\begin{align*}
    \begin{cases}
        \dot{x} = xy \\
        \dot{y} = x^2
    \end{cases}
\end{align*}
задава векторно поле, което определя посоката на движение по хипербулите.

\section{Задача 23.}

\section{Задача 24.}
На чертежа са изобразени няколко фазови криви и всички
равновесни точки $A, \; B, \; C, \; D, \; E$ на системата
\begin{align*}
    \begin{cases}
        \dot{x} = f(x, \; y) \\
        \dot{y} = g(x, \; y)
    \end{cases}
\end{align*}
където $f(x, \; y), \; g(x, \; y) \in C^1(\R)$.
За кои от равновесните точки можем със сигурност
да твърдим, че са неустойчиви? Кои от равновесните
точки е възможно да са устойчиви? \\

Решение: \\

От четрежа се вижда, че движийки се по начертаните фазови криви
се отдалечаваме от $C$ и $D$ (има стрелки излизащи от тях).
Тогава за $C$ и $D$ със сигурност можем да твърдим, че са неустойчиви.
Останлите точки, може да са всякакви понеже не са дадени всички фазови
криви и не знам с тях какво точно се случва. \\

Отговор: \\

Със сигурност неустойчиви: $C$ и $D$. \\

Възможно устойчиви: $A, \; B$ и $E$.
\end{document}