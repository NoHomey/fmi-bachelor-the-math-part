\documentclass[a4paper, 12pt, oneside]{article}
    
\usepackage[left=3cm,right=3cm,top=1cm,bottom=2cm]{geometry}
\usepackage{amsmath,amsthm}
\usepackage{amssymb}
\usepackage{lipsum}
\usepackage{stmaryrd}
\usepackage[T1,T2A]{fontenc}
\usepackage[utf8]{inputenc}
\usepackage[bulgarian]{babel}
\usepackage[normalem]{ulem}
        
\newcommand{\R}{\mathbb{R}}
\newcommand{\N}{\mathbb{N}}
        
\setlength{\parindent}{0mm}
                
\begin{document}
\section{Формулирайте интерполационната задача на Лагранж.
        Докажете единствеността.
        Изведете интерполационната формула на Лагранж.}

\section{Формулирайте и докажете теоремата за оценка на грешката при интерполация на Лагранж.}

\section{Дайте определение за полином на Чебишов. Напишете рекурентната връзка.
        Намерете нулите на полинома на Чебишов от n-та степен.}

\section{Напишете и докажете интерполационната формула на Нютон с разделени разлики.
        Напишете задачата, която се решава с тази формула.}

\section{Напишете и докажете формулата на Нютон с крайни разлики за интерполация напред.
        Напишете задачата, която се решава с тази формула.}

\section{Формулирайте интерполационната задача на Ермит.
        Докажете, че задачата има единствено решение.}

\section{Формулирайте и докажете рекурентната връзка за разделени разлики с кратни възли.
        Включително случая, в който всички възли съвпадат.}

\section{Напишете и докажете, формулата за интерполационния тригонометричен полином
при произволно разположени интерполационни възли в $[0, 2\pi)$.
        Напишете задачата, която се решава с тази формула.}
\pagebreak
\section{Формулирайте и докажете теоремата за представяне на сплайн функция,
като линейна комбинация на полиноми и отсечени степенни функции.}

\section{Напишете и докажете рекуретната връзка за B-сплайните}

\section{Формулирайте теоремата на Чебишов за алтернанса. Докажете достатъчността.}

\section{Формулирайте теоремата на Вайерщрас. Докажете я като използвате полиномите на Бернщайн.}

\section{Напишете и докажете тричленната рекуретната връзка за редица от ортогонални полиноми.}

\section{Формулирайте и докажете теоремата за характеризация на елемента на най-добро ориближение в хилбартово пространство. (НДУ)}

\section{Изведете формулата от вида $f'(a) ~ C_0 f(a - h) + C_1 f(a + h)$ и грешката $O(h^2)$ при положение, че $f$ е достатъчно гладка.
        Обосновете порядъка на грешката.}

\section{Изведете елементарната квадратурна формула на трапеца и оценката на грешката при подходящи предположения за подинтегралната функция.}

\section{Формулирайте и докажете теоремата за квадратурната формула на Гаус.}

\section{Формулирайте и докажете теоремата за приближено решаване
на нелинейно изображение по метода свиващото изображение.}
\end{document}
