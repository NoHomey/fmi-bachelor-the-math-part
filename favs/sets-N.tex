\documentclass[a4paper, 12pt, oneside]{article}
\usepackage[left=3cm,right=3cm,top=1cm,bottom=2cm]{geometry}
\usepackage{amsmath,amsthm}
\usepackage{amssymb}
\usepackage{lipsum}
\usepackage{stmaryrd}
\usepackage[T1,T2A]{fontenc}
\usepackage[utf8]{inputenc}
\usepackage[bulgarian]{babel}
\usepackage[normalem]{ulem}

\newcommand{\N}{\mathbb{N}}
\newcommand{\Q}{\mathbb{Q}}
\newcommand{\LN}{\N_<}

\setlength{\parindent}{0mm}

\title{Някои изброимо безкрайни множества}
\author{Иво Стратев}

\begin{document}
\maketitle
\section*{Увод}
С този документ целя да докажа, че няколко множества са изброимо безкрайни.
Идеята се зароди от желанието ми да докажа, че \(\N\times\N\) и \(\Q\) са изброимо безкрайни.
Ще се опитам да подам идеите как се достига до съответните биекции,
но преди това ще формулирам и докажа няколко помощни твърдения, които ще използвам в доказателствата на твърденията.
\section*{Помощно твърдение 1.}
Нека разгледаме следното множество \(\LN = \{(i, j) \in \N\times\N \; | \; i < j\}\),
това същност е релацията "по-малко" над множеството на естествените числа.
Няма да се интересуваме от този факт. Ще разглеждаме \(\LN\) като множество и ще докажем,
че е изброимо безкрайно.
\subsection*{Твърдение:}
Множеството \(\LN\) е изброимо безкрайно.
\subsection*{Идея за биекция:}
Очевидно е вярно \(\LN = \{(i, j) \; | \; j \in \N, \; j > 0, \; i \in \{0, 1, \dots, j - 1\}\}\).
Нека фиксираме \(n \in \N\) и нека \(n > 0\), тогава наредените двойки, които участват в \(\LN\)
и втората им компонента е \(n\) са \((0, n), (1, n), \dots (n - 1, n)\).
Те очевидно са \(n\) на брой и очевидно броят на тези с втора компонента строго по-малка от \(n\)
e \(1 + 2 + \dots + n - 1\). Тоест \(\displaystyle\frac{(n - 1)n}{2}\) на брой.
Тогава на наредената двойка \((i, j)\) от \(\LN\) ще съпоставим естественото число \(\displaystyle\frac{(j - 1)j}{2} + i\). \\
\subsection*{Доказателство:}
Дефинираме функцията \(f := \left\{\left((i, j), \displaystyle\frac{(j - 1)j}{2} + i\right) \; \Big| \; (i, j) \in \LN \right\}\)
Ще докажем, че \(f\) е биекция между \(\LN\) и \(\N\). Това, че \(f\) е функция от \(\LN\) в \(\N\) вземаме за очевидно.
Очевидно е също, че \(f((0, 1)) = 0, f((0, 2)) = 1, f((1, 2)) = 2\) и така нататък.
\subsubsection*{\(f\) е инекция:}
Нека \((i, j)\) и \((l, k)\) са различни елементи на \(\LN\).
Тогава имаме два случая: \\
\textbf{Случай 1 \(j \neq k\):} \\
Нека без ограничение на общноста \(j < k\). Тогава
\begin{align*}
    f((i, j)) = \displaystyle\frac{(j - 1)j}{2} + i = \displaystyle\sum_{m = 1}^{j - 1} m  + i < \displaystyle\sum_{m = 1}^{j - 1} m + j = \\
    \displaystyle\sum_{m = 1}^j m \leq \displaystyle\sum_{m = 1}^{k - 1} m = \displaystyle\frac{(k - 1)k}{2} \leq \displaystyle\frac{(k - 1)k}{2} + l = f((l, k)).
\end{align*}
Така \(f((i, j)) < f((l, k))\) тоест \(f((i, j)) \neq f((l, k))\). \\
\textbf{Случай 2 \(j = k \; \land \; i \neq l\):} \\
Нека \(n = j\), тогава
\begin{align*}
    f((i, j)) = f((i, n)) = \displaystyle\frac{(n - 1)n}{2} + i \neq \displaystyle\frac{(n - 1)n}{2} + l = f((l, n)) = f((l, k)).
\end{align*}
Тоест \(f((i, j)) \neq f((l, k))\). \\
Така в и двата случая доказахме, че е в сила импликацията
\begin{align*}
    (i, j) \neq (l, k) \longrightarrow f((i, j)) \neq f((l, k)).
\end{align*}
Понеже \((i, j)\) и \((l, k)\) бяха произволни е в сила
\begin{align*}
    \forall (i, j) \in \LN \; \forall (l, k) \in \LN \; (i, j) \neq (l, k) \longrightarrow f((i, j)) \neq f((l, k)).
\end{align*}
Тоест \(f\) е инекция.
\subsubsection*{\(f\) е сюрекция:}
Трябва да докажем \(\forall n \in \N \; \exists (i, j) \in \LN \; f((i, j)) = n\)
Доказателството ще извършим по индукция. \\
\textbf{База:} \\
\begin{align*}
    f((0, 1)) = \displaystyle\frac{(1 - 1)1}{2} + 0 = 0 + 0 = 0.
\end{align*}
\textbf{Индукционна хипотеза:} \\
Допускаме, че е вярно \(\exists m \in \N \; \exists (i, j) \in \LN \; f((i, j)) = m\). \\
\textbf{Индукционна стъпка:}
Ще докажем за \(n = m + 1\). От Индукционната хипотеза имаме \(\exists (i, j) \in \LN \; f((i, j)) = m\).
Нека тогава \((i, j) \in \LN \; f((i, j)) = m\). Следното \(f((i, j)) + 1 = m + 1 = n\).
Има две възможности: \\
\textbf{Случай 1 \(j = i + 1\):} \\
Тогава
\begin{align*}
    n = f((i, j)) + 1 = f((i, i + 1)) + 1 = \left(\displaystyle\frac{(i + 1 - 1)(i + 1)}{2} + i\right) + 1 = \\
    \displaystyle\frac{i(i + 1)}{2} + i + 1 = \displaystyle\frac{(i + 1)(i + 2)}{2} = \displaystyle\frac{j(j + 1)}{2} = \displaystyle\frac{(j + 1 - 1)(j + 1)}{2} + 0 = f((0, j + 1)).
\end{align*}
Очевидно \((0, j + 1) \in \LN\) и \(f((0, j + 1)) = n\). \\
\textbf{Случай 2 \(i < j - 1\):} \\
Тогава
\begin{align*}
    n = f((i, j)) + 1 = \left(\displaystyle\frac{(j - 1)j}{2} + i\right) + 1 = \displaystyle\frac{(j - 1)j}{2} + (i + 1) = f((i + 1, j)).
\end{align*}
Очевидно \(i + 1 < j - 1 + 1 = j\) и значи \((i + 1, j) \in \LN\) и \(f((i + 1, j)) = n\). \\
Така и в двата случая доказахме \(\exists (l, k) \in \LN \; f((l, k)) = n = m + 1\). \\
\textbf{Заключение:}
\(\forall n \in \N \; \exists (i, j) \in \LN \; f((i, j)) = n\). \\
Което искахме да докажем. Тоест \(f\) е сюрекция. \\
Следователно \(f\) е биекция от където \(\LN\) е изброимо безкрайно. \(\qed\)
\end{document}
