\documentclass[a4paper, 12pt, oneside]{article}
\usepackage[left=3cm,right=3cm,top=1cm,bottom=2cm]{geometry}
\usepackage{amsmath,amsthm}
\usepackage{amssymb}
\usepackage{lipsum}
\usepackage{stmaryrd}
\usepackage[T1,T2A]{fontenc}
\usepackage[utf8]{inputenc}
\usepackage[bulgarian]{babel}
\usepackage[normalem]{ulem}

\newcommand{\N}{\mathbb{N}}
\newcommand{\Z}{\mathbb{Z}}
\newcommand{\Q}{\mathbb{Q}}
\newcommand{\T}{\mathbb{T}}
\newcommand{\LN}{\N_<}

\setlength{\parindent}{0mm}

\title{Някои изброимо безкрайни множества}
\author{Иво Стратев}

\begin{document}
\maketitle
\section*{Увод}
С този документ целя да докажа, че няколко множества са изброимо безкрайни.
Идеята се зароди от желанието ми да докажа, че \(\N\times\N\) и \(\Q\) са изброимо безкрайни.
Ще се опитам да подам идеите как се достига до съответните биекции,
но преди това ще формулирам и докажа няколко помощни твърдения,
които ще използвам в доказателствато, че \(\N\times\N\) e изброимо безкрайнo.
\section*{Лема 1.}
Нека разгледаме следното множество \(\LN = \{(i, j) \in \N\times\N \; | \; i < j\}\),
това същност е релацията "по-малко" над множеството на естествените числа.
Няма да се интересуваме от този факт. Ще разглеждаме \(\LN\) като множество и ще докажем,
че е изброимо безкрайно.
\subsection*{Твърдение:}
Множеството \(\LN\) е изброимо безкрайно.
\subsection*{Идея за биекция:}
Очевидно е вярно \(\LN = \{(i, j) \; | \; j \in \N, \; j > 0, \; i \in \{0, 1, \dots, j - 1\}\}\).
Нека фиксираме \(n \in \N\) и нека \(n > 0\), тогава наредените двойки, които участват в \(\LN\)
и втората им компонента е \(n\) са \((0, n), (1, n), \dots (n - 1, n)\).
Те очевидно са \(n\) на брой и очевидно броят на тези с втора компонента строго по-малка от \(n\)
e \(1 + 2 + \dots + n - 1\). Тоест \(\displaystyle\frac{(n - 1)n}{2}\) на брой.
Тогава на наредената двойка \((i, j)\) от \(\LN\) ще съпоставим естественото число \(\displaystyle\frac{(j - 1)j}{2} + i\). \\
\subsection*{Доказателство:}
Дефинираме функцията \(f := \left\{\left((i, j), \displaystyle\frac{(j - 1)j}{2} + i\right) \; \Big| \; (i, j) \in \LN \right\}\)
Ще докажем, че \(f\) е биекция между \(\LN\) и \(\N\). Това, че \(f\) е функция от \(\LN\) в \(\N\) вземаме за очевидно.
Очевидно е също, че \(f((0, 1)) = 0, f((0, 2)) = 1, f((1, 2)) = 2\) и така нататък.
\subsubsection*{\(f\) е инекция:}
Нека \((i, j)\) и \((l, k)\) са различни елементи на \(\LN\).
Тогава имаме два случая: \\
\textbf{Случай 1 \(j \neq k\):} \\
Нека без ограничение на общноста \(j < k\). Тогава
\begin{align*}
    f((i, j)) = \displaystyle\frac{(j - 1)j}{2} + i = \displaystyle\sum_{m = 1}^{j - 1} m  + i < \displaystyle\sum_{m = 1}^{j - 1} m + j = \\
    \displaystyle\sum_{m = 1}^j m \leq \displaystyle\sum_{m = 1}^{k - 1} m = \displaystyle\frac{(k - 1)k}{2} \leq \displaystyle\frac{(k - 1)k}{2} + l = f((l, k)).
\end{align*}
Така \(f((i, j)) < f((l, k))\) тоест \(f((i, j)) \neq f((l, k))\). \\
\textbf{Случай 2 \(j = k \; \land \; i \neq l\):} \\
Нека \(n = j\), тогава
\begin{align*}
    f((i, j)) = f((i, n)) = \displaystyle\frac{(n - 1)n}{2} + i \neq \displaystyle\frac{(n - 1)n}{2} + l = f((l, n)) = f((l, k)).
\end{align*}
Тоест \(f((i, j)) \neq f((l, k))\). \\
Така в и двата случая доказахме, че е в сила импликацията
\begin{align*}
    (i, j) \neq (l, k) \longrightarrow f((i, j)) \neq f((l, k)).
\end{align*}
Понеже \((i, j)\) и \((l, k)\) бяха произволни е в сила
\begin{align*}
    \forall (i, j) \in \LN \; \forall (l, k) \in \LN \; (i, j) \neq (l, k) \longrightarrow f((i, j)) \neq f((l, k)).
\end{align*}
Тоест \(f\) е инекция.
\subsubsection*{\(f\) е сюрекция:}
Трябва да докажем \(\forall n \in \N \; \exists (i, j) \in \LN \; f((i, j)) = n\)
Доказателството ще извършим по индукция. \\
\textbf{База:} \\
\begin{align*}
    f((0, 1)) = \displaystyle\frac{(1 - 1)1}{2} + 0 = 0 + 0 = 0.
\end{align*}
\textbf{Индукционна хипотеза:} \\
Допускаме, че е вярно \(\exists m \in \N \; \exists (i, j) \in \LN \; f((i, j)) = m\). \\
\textbf{Индукционна стъпка:}
Ще докажем за \(n = m + 1\). От Индукционната хипотеза имаме \(\exists (i, j) \in \LN \; f((i, j)) = m\).
Нека тогава \((i, j) \in \LN \; f((i, j)) = m\). Следното \(f((i, j)) + 1 = m + 1 = n\).
Има две възможности: \\
\textbf{Случай 1 \(j = i + 1\):} \\
Тогава
\begin{align*}
    n = f((i, j)) + 1 = f((i, i + 1)) + 1 = \left(\displaystyle\frac{(i + 1 - 1)(i + 1)}{2} + i\right) + 1 = \\
    \displaystyle\frac{i(i + 1)}{2} + i + 1 = \displaystyle\frac{(i + 1)(i + 2)}{2} = \displaystyle\frac{j(j + 1)}{2} = \displaystyle\frac{(j + 1 - 1)(j + 1)}{2} + 0 = f((0, j + 1)).
\end{align*}
Очевидно \((0, j + 1) \in \LN\) и \(f((0, j + 1)) = n\). \\
\textbf{Случай 2 \(i < j - 1\):} \\
Тогава
\begin{align*}
    n = f((i, j)) + 1 = \left(\displaystyle\frac{(j - 1)j}{2} + i\right) + 1 = \displaystyle\frac{(j - 1)j}{2} + (i + 1) = f((i + 1, j)).
\end{align*}
Очевидно \(i + 1 < j - 1 + 1 = j\) и значи \((i + 1, j) \in \LN\) и \(f((i + 1, j)) = n\). \\
Така и в двата случая доказахме \(\exists (l, k) \in \LN \; f((l, k)) = n = m + 1\). \\
\textbf{Заключение:}
\(\forall n \in \N \; \exists (i, j) \in \LN \; f((i, j)) = n\). \\
Което искахме да докажем. Тоест \(f\) е сюрекция. \\
Следователно \(f\) е биекция между \(\LN\) и \(\N\) от където \(\LN\) е изброимо безкрайно. \(\qed\)
\section*{Дефиниция (Композиция).}
Нека \(A, B\) и \(C\) са множества
и нека \(f\) е функция от \(B\) в \(C\)
и нека \(g\) е функция от \(A\) в \(B\).
Тогава множеството \(f \circ g := \{(a, c) \in A \times C \; | \; \exists b \in B \; g(a) = b \; \land \; f(b) = c\}\)
наричаме композицията на \(f\) и \(g\).
\section*{Лема 2.}
Нека \(A, B\) и \(C\) са множества
и нека \(f\) е функция от \(B\) в \(C\)
и нека \(g\) е функция от \(A\) в \(B\).
Тогава \(f \circ g\) е функция от \(A\) в \(C\) и \(\forall a \in A \; (f \circ g)(a) = f(g(a))\).
\subsection*{Доказателство:}
Нека \(a \in A\) тогава понеже \(g\) е функция от \(A\) в \(B\)
то \(\exists! b \in B \; g(a) = b\). \\
Нека тогава \(b \in B \; g(a) = b\).
\(f\) е функция от \(B\) в \(C\) следователно \(\exists! c \in C \; f(b) = c\).
Тогава \((a, c) \in f \circ g\),
при това е вярно \(\exists! (a, c) \in A \times C \; (a, c) \in f \circ g\)
и значи \((f \circ g)(a) = c = f(b) = f(g(a))\).
Следователно \\
\(\forall a \in A \; \exists! c \in C \; (a, c) \in f \circ g \; \land \; (f \circ g)(a) = f(g(a))\)
и значи \(f \circ g\) е функция от \(A\) в \(C\) и \(\forall a \in A \; (f \circ g)(a) = f(g(a))\). \(\qed\)
\section*{Лема 3.}
Нека \(A, B\) и \(C\) са множества
и нека \(f\) е биекция от \(B\) в \(C\)
и нека \(g\) е биекция от \(A\) в \(B\).
Тогава \(f \circ g\) е биекция от \(A\) в \(C\).
\subsection*{Доказателство:}
\subsubsection*{Инекция:}
Нека \(a_1, a_2 \in A\) и \(a_1 \neq a_2\).
\(g\) е биекция от \(A\) в \(B\), в частност \(g\) е инекция
следователно \(g(a_1) \neq g(a_2)\), но \(f\) е биекция от \(B\) в \(C\),
в частност \(f\) е инекция и значи \(f(g(a_1)) \neq f(g(a_2))\).
От \textbf{Лема 2.} \((f \circ g)(a_1) = f(g(a_1)) \neq f(g(a_2)) = (f \circ g)(a_2)\).
Следователно \(\forall a_1 \in A \; \forall a_2 \in A \; a_1 \neq a_2 \longrightarrow (f \circ g)(a_1) \neq (f \circ g)(a_2)\).
Значи \(f \circ g\) е инекция.
\subsubsection*{Сюрекция:}
Нека \(c \in C\) \(f\) е биекция, в частност \(f\) е сюрекция
следователно \(\exists b \in B \; f(b) = c\).
Нека тогава \(b \in B\) и \(f(b) = c\).
\(g\) е биекция, в частност \(g\) е сюрекция
следователно \(\exists a \in A \; g(a) = b\).
Нека тогава \(a \in A\) и \(g(a) = b\).
Тогава \(f(g(a)) = f(b) = c\).
От \textbf{Лема 2.} \((f \circ g)(a) = f(g(a)) = c\).
Следователно \(\forall c \in C \; \exists a \in A \; (f \circ g)(a) = c\).
Значи \(f \circ g\) е сюрекция.
\subsubsection*{Заключение:}
\(f \circ g\) е инекция и \(f \circ g\) е сюрекция следователно \(f \circ g\) е биекции. \(\qed\)
\section*{Лема 4.}
\(\N\times\N\) е равномощно с \(\LN\).
\subsection*{Идея за биекция:}
Нека фиксираме произволно естествено число \(n\).
Нека \((i, j) \in \N\times\N\) е такава, че \(i + j = n\).
Тогава е вярно, че \(i \in \{0, 1, \dots, n\}\) и \(j = n - i\).
Следователно има \(n + 1\) на брой наредени двойки \((i, j) \in \N\times\N\),
за които е вярно \(i + j = n\). От друга страна е вярно, че
има \(n + 1\) на брой наредени двойки \((k, n + 1) \in \LN\).
Тогава на наредената двойка \((i, j)\) от \(\N\times\N\) ще съпоставим наредената двойка \((i, i + j + 1)\).
\subsection*{Доказателство:}
Дефинираме функцията \(f := \{((i, j), (i, i + j + 1)) \; | \; (i, j) \in \N\times\N\}\)
Ще докажем, че \(f\) е биекция между \(\N\times\N\) и \(\LN\). Това, че \(f\) е функция от \(\N\times\N\) в \(\LN\) вземаме за очевидно.
Очевидно е също, че \(f((0, 0)) = (0, 1), f((0, 1)) = (0, 2), f((1, 1)) = (1, 2)\) и така нататък.
\subsubsection*{\(f\) е инекция:}
Нека \((i, j) \in \N\times\N\) и \((l, k) \in \N\times\N\) и \((i, j) \neq (l, k)\).
Тогава имаме два случая: \\
\textbf{Случай 1 \(i \neq l\):} \\
Тогава \(f((i, j)) = (i, i + j + 1) \neq (l, l + k + 1) = f((l, k))\). \\
\textbf{Случай 2 \(i = l \; \land \; j \neq k\):} \\
Тогава e вярно \(i + j \neq i + k = l + k\) и значи \(i + j + 1 \neq l + k + 1\).
Следователно \(f((i, j)) = (i, i + j + 1) \neq (i, i + k + 1) = (l, l + k + 1) = f((l, k))\). \\
\textbf{Заключение:}
Тогава е в сила \((i, j) \neq (l, k) \longrightarrow f((i, j)) \neq f((l, k))\). \\
Тогава \(\forall (i, j) \in \N\times\N \; \forall (l, k) \in \N\times\N \; (i, j) \neq (l, k) \longrightarrow f((i, j)) \neq f((l, k))\) \\
следователно \(f\) е инекция.
\subsubsection*{\(f\) е сюрекция:}
Нека \((i, k) \in \LN\) и нека \(j = k - i - 1\).
\(k > i\) тогава \(k - i > 0\) и значи \(k - i \geq 1\)
тогава \(j = k - i - 1 \geq 0\) и \((i, k) \in \N\times\N\)
следователно \(j \in \N\).
\(f((i, j)) = (i, i + j + 1) = (i, i + (k - i - 1) + 1) = (i, k)\).
Тоест \((i, j) \in \N\times\N\) и \(f((i, j)) = (i, k)\).
Тогава \(\forall (i, k) \in \LN \; \exists (l, j) \in \N\times\N \; f((l, j)) = (i, k)\)
следователно \(f\) е сюрекция.
\subsubsection*{Заключение:}
\(f\) е инекция и \(f\) е сюрекция следователно \(f\) е биекции. \(\qed\)
\section*{Теорема 1.}
\(\N\times\N\) e изброимо безкрайнo.
\section*{Доказателство:}
От \textbf{Лема 4.} \(\N\times\N\) е равномощно с \(\LN\).
Следователно съществува биекция между \(\N\times\N\) и \(\LN\).
Нека тогава \(g\) е една биекция между \(\N\times\N\) и \(\LN\).
От \textbf{Лема 1.} \(\LN\) е изброимо безкрайнo, тоест съществува биекция между \(\LN\) и \(\N\).
Нека тогава \(f\) е една биекция между \(\LN\) и \(\N\).
Според \textbf{Лема 3.} \(f \circ g\) е биекция между \(\N\times\N\) и \(\N\).
Следователно \(\N\times\N\) e изброимо безкрайнo. \(\qed\)
\section*{Теорема 2.}
\(\Z\) e изброимо безкрайнo.
\subsection*{Идея за биекция:}
Множеството на целите числа се разбива на две множества - отрицателни и неотрицателни.
От друга страна множеството на естествените числа се разбива на нечетни и на четни.
Тогава ако \(z \in \Z\) и \(z < 0\) на цялото отрицателно число \(z\)
ще съпоставим нечетно естественото число \(2(-z) - 1\).
Ако \(z \in \Z\) и \(z \geq 0\) на цялото неотрицателно число \(z\)
ще съпоставим четно естественото число \(2z\).
\subsection*{Доказателство:}
Дефинираме функцията \(f \; : \; \Z \to \N\)
\begin{align*}
    f(z) = \begin{cases}
        2z & , \; z \geq 0 \\
        2(-z) - 1 & , \; z < 0
    \end{cases}
\end{align*}
Очевидно \(f(0) = 0\), \(f(-1) = 1\), \(f(1) = 2\), \(f(-2) = 3\) и така нататък.
\subsubsection*{\(f\) е инекция:}
Нека \(z_1, z_2 \in \Z\) и \(z_1 \neq z_2\).
Тогава без ограничение на общността има три случая. \\
\textbf{Случай 1 \(z_1 \geq 0 \; \land \; z_2 \geq 0\):}
\begin{align*}
    z_1 \neq z_2 \longrightarrow 2z_1 \neq 2z_2 \longrightarrow f(z_1) \neq f(z_2)
\end{align*}
\textbf{Случай 2 \(z_1 \geq 0 \; \land \; z_2 < 0\):} \\
В такъв случай \(f(z_1) = 2z_1\) и \(f(z_2) =  2(-z_2) - 1\).
Това са две естествени числа с различна четност и значи не са равни.
Тоест \(f(z_1) \neq f(z_2)\). \\
\textbf{Случай 3 \(z_1 < 0 \; \land \; z_2 < 0\):}
\begin{align*}
    z_1 \neq z_2 \longrightarrow 2(-z_1) \neq 2(-z_2) \longrightarrow 2(-z_1) - 1 \neq 2(-z_2) - 1 \longrightarrow f(z_1) \neq f(z_2)
\end{align*}
\textbf{Заключение:} \\
В сила е \(\forall z_1 \in \Z \; \forall z_2 \in \Z \; z_1 \neq z_2 \longrightarrow f(z_1) \neq f(z_2)\). \\
Тоест \(f\) е инекция.
\subsubsection*{\(f\) е сюрекция:}
Нека \(n \in \N\) тогава има два случая. \\
\textbf{Случай 1 \(n\) е четно:}
Тогава \(\exists k \in \N \; n = 2k\).
Нека \(k \in \N\) и \(n = 2k\). \(k \in \N\) значи \(k \in \Z\) и \(n = 2k = f(k)\). \\
\textbf{Случай 2 \(n\) е нечетно:}
Тогава \(\exists k \in \N \; n = 2k + 1\).
Нека \(k \in \N\) и \(n = 2k + 1\). 
Разглеждаме уравнението \(f(z) = n\) и търсим целите решения.
\begin{align*}
    f(z) = 2(-z) - 1 = 2k + 1 = n  \longrightarrow \\
    2(-z) = 2k + 2 \longrightarrow \\
    -z = k + 1 \longrightarrow \\
    z = -(k + 1)
\end{align*}
Ако \(k \in \N\), то очевидно \(-(k + 1) \in \Z\) и при това
\begin{align*}
    k \geq 0 \longrightarrow k + 1 > 0 \longrightarrow -(k + 1) < 0.
\end{align*}
Значи ако \(z = -(k + 1)\), то \(z \in \Z \; \land \; z < 0 \; \land \; f(z) = n\). \\
\textbf{Заключение:} \\
Вярно е \(\forall n \in \N \; \exists z \in \Z \; f(z) = n\).
Тоест \(f\) е сюрекция.
\subsubsection*{Заключение:}
\(f\) е функция от \(\Z\)  в \(\N\),  \(f\) е инекция и \(f\) е сюрекция
значи \(f\) е биекция от \(\Z\)  в \(\N\). Следното \(\Z\) e изброимо безкрайнo. \(\qed\)
\section*{\(\Q\) е изброимо безкрайно}
Множеството на рационалните числа от една страна се разбива на следните три множества
\begin{align*}
    \Q_{> 0} := \{q \in \Q \; | \; q > 0\} \\
    \Q_{= 0} := \{0\} \\
    \Q_{< 0} := \{q \in \Q \; | \; q < 0\}.
\end{align*}
От друга страна на цели и нецели рационални числа.
\begin{align*}
    \Q_{\in \Z} := \Z \\
    \Q_{\notin \Z} := \left\{\frac{p}{q} \; \Big| \; p \in \Z, \; q \in \N, \; p \neq 0, \; q > 1, \; \gcd(p, q) = 1\right\}
\end{align*}
Където \(\gcd(p, q)\) е най-големият общ делител на \(p\) и \(q\). \\
Ще докажем няколко твърдения за да докажем, че следното множество
\begin{align*}
    \T := \Q_{> 0} \cap \Q_{\notin \Z} = \left\{\frac{p}{q} \; \Big| \; p \in \N, \; q \in \N, \; p \geq 1, \; q \geq 2, \; \gcd(p, q) = 1\right\}.
\end{align*}
е изброимо. 
\end{document}
