\documentclass[12pt]{article}
\usepackage[utf8]{inputenc}

\usepackage[T2A]{fontenc}
\usepackage[english,bulgarian]{babel}
\usepackage{amsmath}
\usepackage{amssymb}
\usepackage{amsthm}
\usepackage{relsize}
\usepackage{tensor}

\newcommand{\injection}[0]{\,\mathlarger{\mathlarger{\rightarrowtail}}\,}
\newcommand{\surjection}[0]{\,\mathlarger{\mathlarger{\twoheadrightarrow}}\,}
\newcommand{\bijection}[0]{\mathbin{\mathlarger{\mathlarger{\rightarrowtail}} \hspace{-14pt} \mathlarger{\mathlarger{\twoheadrightarrow}}}}

\title{Задачи по ТМ}
\author{Иво Стратев}

\begin{document}
\maketitle

Нека \(RelEq(R, A)\) е съкращение за "\(R\) е релация на еквивалетност в \(A\)".
Разбира се това е изразимо в езика на теория на множествата.

\section*{Задача 1.}

Нека \(R\) и \(S\) са релации на еквивалетност в \(A\).
Тогава
\begin{align*}
RelEq(R \cup S, A) \iff (\forall a \in A)([a]_R \subseteq [a]_S \; \lor \; [a]_S \subseteq [a]_R).
\end{align*}
\subsection*{Решение:}
Ако \(A = \emptyset\), то \(R = S = R \cup S = \emptyset\).
В такъв случай твърдението е вярно.
\\
\vspace{5mm}
\\
Нека тогава \(A \neq \emptyset\).
От \(RelEq(R, A)\), \(RelEq(S, A)\) и \(A \neq \emptyset\),
следва че \(Id_A \subseteq R\) и \(Id_A \subseteq S\) и тогава
\((\exists a \in A) \; \& \; (\forall a \in A)(a \in [a]_R \; \& \; a \in [a]_S)\).
Тоест \((\forall a \in A)([a]_R \neq \emptyset \; \& \; [a]_S \neq \emptyset)\).

\subsubsection*{(\(\implies\))}
Нека \(RelEq(R \cup S, A)\).
Допускаме, че не е вярно
\\
\((\forall a \in A)([a]_R \subseteq [a]_S \; \lor \; [a]_S \subseteq [a]_R)\).
\\
Тогава е вярно\((\exists a \in A)([a]_R \not\subseteq [a]_S \; \& \; [a]_S \not\subseteq [a]_R)\).
\\
Значи е вярно
\((\exists a \in A)([a]_R \setminus [a]_S \neq \emptyset \; \& \; [a]_S \setminus [a]_R \neq \emptyset)\).
\\
\vspace{1mm}
\\
Нека тогава \(a \in A\) и \([a]_R \setminus [a]_S \neq \emptyset \; \& \; [a]_S \setminus [a]_R \neq \emptyset\).
\\
Нека тогава \(r \in [a]_R \setminus [a]_S\) и \(s \in [a]_S \setminus [a]_R\).
\\
Тогава \(<a, r> \; \in R \setminus S\) и \(<a, s> \; \in S \setminus R\).
\\
Тогава \(<a, r> \; \in R \cup S\) и \(<a, s> \; \in R \cup S\).
\\
Но понеже \(RelEq(R \cup S, A)\), то \(<r, a> \; \in R \cup S\) и \(<r, s> \; \in R \cup S\).
\\
Възможни са два случая.
\\
\vspace{1mm}
\\
Случай 1: \(<r, s> \; \in R\).
\\
\vspace{1mm}
\\
Тогава \(<a, r> \; \in R\) и \(<r, s> \; \in R\), но \(RelEq(R, A)\)
и значи \(<a, s> \; \in R\), но това е Абсурд, защото \(<a, s> \; \notin R\).
\\
\vspace{1mm}
\\
Случай 2: \(<r, s> \; \notin R\).
\\
\vspace{1mm}
\\
Тогава \(<r, s> \; \in S\). Но тогава \(<s, a> \; \in S\)
и \(<r, s> \; \in S\), защото \(RelEq(S, A)\).
Но тогава \(<r, a> \; \in S\) и \(<a, r> \; \in S\), но това е Абсурд!
\\
\vspace{1mm}
\\
Така и в двата възможни случая достигнахме до Абсурд,
който е следствие на допускането, че не е вярно
\((\forall a \in A)([a]_R \subseteq [a]_S \; \lor \; [a]_S \subseteq [a]_R)\).
\\
Тогава е вярно
\((\forall a \in A)([a]_R \subseteq [a]_S \; \lor \; [a]_S \subseteq [a]_R)\).

\subsubsection*{(\(\impliedby\))}
Нека \((\forall a \in A)([a]_R \subseteq [a]_S \; \lor \; [a]_S \subseteq [a]_R)\).
Ще покажем, че е в сила \(RelEq(R \cup S, A)\).
\\
\vspace{1mm}
\\
Рефлексивност:
\\
\vspace{1mm}
\\
Понеже \(RelEq(R, A)\), то \(Id_A \subseteq R\) и значи \(Id_A \subseteq R \cup S\).
Значи \(R \cup S\) е рефлексивна в \(A\).
\\
\vspace{1mm}
\\
Симетричност:
\\
\vspace{1mm}
\\
Нека \(<a, b> \; \in R \cup S\).
Тогава ако \(<a, b> \; \in R\), то понеже \(RelEq(R, A)\),
следва че \(<b, a> \; \in R\), а значи и \(<b, a> \; \in R \cup S\).
Ако пък \(<a, b> \; \in S\), то понеже \(RelEq(S, A)\),
следва че \(<b, a> \; \in S\), а значи и \(<b, a> \; \in R \cup S\).
И е в сила \(<a, b> \; \in R \; \lor <a, b> \; \in S\).
Значи \(<b, a> \; \in R \cup S\).
Следователно \(R \cup S\) е симетрична.
\\
\vspace{1mm}
\\
Транзитивност:
\\
\vspace{1mm}
\\
Нека \(<a, b> \; \in R \cup S\) и \(<b, c> \; \in R \cup S\).
Тогава са възможни два случая.
\\
\vspace{1mm}
\\
Случай 1 \((<a, b> \; \in R \; \& <b, c> \; \in R) \; \lor (<a, b> \; \in S \; \& <b, c> \; \in S)\):
\\
\vspace{1mm}
\\
Възможни са два случая (под слуачая).
\\
\vspace{1mm}
\\
Случай 1.1 \(<a, b> \; \in R \; \& <b, c> \; \in R\):
\\
\vspace{1mm}
\\
Тогава понеже \(RelEq(R, A)\), то \(<a, c> \; \in R\), а значи и \(<a, c> \; \in R \cup S\).
\\
\vspace{1mm}
\\
Случай 1.2 \(<a, b> \; \in S \; \& <b, c> \; \in S\):
\\
\vspace{1mm}
\\
Тогава понеже \(RelEq(S, A)\), то \(<a, c> \; \in S\), а значи и \(<a, c> \; \in R \cup S\).
\\
\vspace{1mm}
\\
Случай 2 \((<a, b> \; \in R \setminus S \; \& <b, c> \; \in S \setminus R) \; \lor (<a, b> \; \in S \setminus R \; \& <b, c> \; \in S \setminus R)\):
\\
\vspace{1mm}
\\
Ще покажем, че това не е възможно.
Нека \(<u, v> \; \in R \setminus S\)
\\
и \(<v, w> \; \in S \setminus R\).
Понеже \([v]_R \subseteq [v]_S \; \lor \; [v]_S \subseteq [v]_R\), то са възможни два случая.
\\
Ако \([v]_R \subseteq [v]_S\), тогава \(u \in [v]_S\) и значи \(<u, v> \; \in S\), но това e Абсурд!
\\
Ако \([v]_S \subseteq [v]_R\), тогава \(w \in [v]_R\) и значи \(<v, w> \; \in R\), но това е Абсурд!
\\
\vspace{1mm}
\\
Излезе, че този логически случай не е възможен, поради връзката между двете реалции.
Тогава тривиално следва, че \(<a, c> \; \in R \cup S\).
\\
\vspace{1mm}
\\
Така и в двата случая излезе, че \(<a, c> \; \in R \cup S\) (във втория предпоставката, че има такива двойки просто не е вярна и тогава следствието е тривиално).
Следователно \(R \cup S\) е транзитивна.
\\
\vspace{1mm}
\\
Следователно \(RelEq(R \cup S, A)\). \(\qed\)

\section*{Задача 2.}
Нека \(A\), \(B\) и \(C\) са такива, че
\(\overline{\overline{A \cup B}} = \overline{\overline{C \times C}}\).
Тогава
\begin{align*}
    \exists g (g : A \surjection C) \; \lor \; \exists h (h : C \injection B)
\end{align*}

\subsection*{Решение:}
След като \(\overline{\overline{A \cup B}} = \overline{\overline{C \times C}}\).
Нека \(f : A \cup B \bijection C \times C\).
\\
\vspace{1mm}
\\
Възможни са два случая.
\\
\vspace{1mm}
\\
Случай 1 \((\forall c \in C)(\exists a \in A)(\exists d \in C)(f(a) = <c, d>)\):
\\
\vspace{1mm}
\\
Понеже \(f : A \cup B \bijection C \times C\), то \(f_{\restriction A} : A \to C \times C\).\\
Нека \(left : C \times C \to C\) и \((\forall x \in C)(\forall y \in C)(left(<x, y>) = x)\).
\\
Понеже \((\forall c \in C)(left(<c, c>) = c)\), то \(left : C \times C \surjection C\).
\\
Ще покажем, че \(f_{\restriction A} \circ left : A \surjection C\).
За сега имаме \(f_{\restriction A} \circ left : A \to C\),
\\
понеже \(Dom(left) = C \times C\),
\(Range(left) = C\), \(Dom(f_{\restriction A}) = A\)
\\
и \(Range(f_{\restriction A}) \subseteq C \times C\).
Нека \(c \in C\). Тогава
\\
\((\exists a \in A)(\exists d \in C)(f(a) = <c, d>)\).
Нека тогава \(a \in A\) и \(d \in C\) и са такива, че \(f(a) = <c, d>\).
Тогава \((f_{\restriction A} \circ left)(a) = left(f_{\restriction A}(a)) = left(<c, d>) = c\).
\\
Следователно \((\forall y \in C)(\exists x \in A)((f_{\restriction A} \circ left)(x) = y)\).
\\
Тоест \(f_{\restriction A} \circ left : A \surjection C\).
\\
Следователно \(\exists g (g : A \surjection C) \; \lor \; \exists h (h : C \injection B)\). 
\\
\vspace{1mm}
\\
Случай 2 \((\exists c \in C)(\forall a \in A)(\forall d \in C)(f(a) \neq <c, d>)\):
\\
\vspace{1mm}
\\
Нека тогава \(c \in C\) е такова, че
\((\forall a \in A)(\forall d \in C)(f(a) \neq <c, d>)\).
\\
\(f : A \cup B \bijection C \times C\) следователно \(f^{-1} : C \times C \bijection  A \cup B\).
\\
Ще докажем, че \((\forall x \in C)(f^{-1}(<c, x>) \in B)\) е истина.
\\
Тоест \(Range(f^{-1}_{\restriction \{c\} \times C}) \subseteq B\).
Нека \(d \in C\). Нека \(z = f^{-1}(<c, x>)\).
\\
Да допуснем, че \(z \notin B\).
Тогава \(z \in A\) понеже \(z \in A \cup B\).
\\
Но тогава \(f(z) = <c, d>\) и \(d \in C\) и \(z \in A\).
Това е Абсурд!
\\
Следователно \(z \in B\).
Следователно \((\forall x \in C)(f^{-1}(<c, x>) \in B)\).
\\
Това значи, че \(Range(f^{-1}_{\restriction \{c\} \times C}) \subseteq B\).
\\
Но \(f^{-1} : C \times C \injection  A \cup B\), следователно \(f^{-1}_{\restriction \{c\} \times C} : \{c\} \times C \injection B\).
\\
Тогава нека \(t : C \to B\) и \((\forall x \in C)(t(x) = f^{-1}(<c, x>)\).
\\
Тогава е очевидно \(t : C \injection B\).
\\
Следователно \(\exists g (g : A \surjection C) \; \lor \; \exists h (h : C \injection B)\). 
\\
\vspace{1mm}
\\
И в двата възможни случая доказахме
\\
\(\exists g (g : A \surjection C) \; \lor \; \exists h (h : C \injection B)\). \(\qed\)

\section*{Задача 3.}
Нека \(S\) е такова, че
\begin{align*}
    \{A\} \subseteq S \subseteq \mathcal{P}(A) \\
    \& \\
    \forall X(\emptyset \neq X \subseteq S \implies \cap X \in S).
\end{align*}
Нека \(f : S \to S\) е монотонна функция.
Тогава \(f\) има най-малка неподвижна точка.

\subsection*{Решение:}
Нека \(B = \{X \; | \; X \in S \; \& \; f(X) \subseteq X\}\).
Очевидно \(B \subseteq S\). Понеже \(S \subseteq \mathcal{P}(A)\),
то \((\forall T \in S)(T \subseteq A)\). Тогава \(A \in S\)
и \(f(A) \subseteq A\), понеже \(f(A) \in S\). Следователно
\(A \in B\) и значи \(B \neq \emptyset\).
Така \(\emptyset \neq B \subseteq S\) и значи \(\cap B \in S\).
Нека тогава \(X_0 = \cap B\). Нека \(X \in B\).
Тогава \(X_0 = \cap B \subseteq X\), но понеже \(f\) е монотонна,
то \(f(X_0) \subseteq f(X)\). Но \(X \in B\) следователно
\(f(X) \subseteq X\) и значи \(f(X_0) \subseteq X\).
Така \((\forall T \in B)(f(X_0) \subseteq T)\) и значи
\(f(X_0) \subseteq X_0\). Но \(X_0 = \cap B \in S\)
и така \(X_0 \in B\). От \(f(X_0) \subseteq X_0\) и \(f\)
е монотнонна, следва че \(f(f(X_0)) \subseteq f(X_0)\).
Обаче \(Range(f) \subseteq S\) и значи \(f(X_0) \in S\).
Така \(f(X_0) \in B\). Следователно \(X_0 \subseteq f(X_0)\).
И значи получихме \(f(X_0) \subseteq X_0\) и \(X_0 \subseteq f(X_0)\).
Следователно \(X_0 \in S\) и \(X_0 = f(X_0)\), тоест \(X_0\) е неподвижна точка за \(f\).
Нека \(Z \in S\) и \(f(Z) = Z\). Тогава \(f(Z) \subseteq Z\). \\
Следователно \(Z \in B\). Тогава \(X_0 = \cap B \subseteq Z\). \\
Следователно \((\forall Z \in S)(f(Z) = Z \implies X_0 \subseteq Z)\). \\
Следователно \(X_0\) е най-малката неподвижна точка на \(f\). \(\qed\)

\section*{Лема за разделените инекции}
Нека \(f : A \injection B\) и \(g : C \injection D\)
и \(A \cap C = \emptyset\) и \(B \cap D = \emptyset\).
\\
Тогава \(f \cup g : A \cup C \injection B \cup D\).

\subsection*{Доказателство:}
Понеже \(Dom(f) \cap Dom(g) = \emptyset\), то \(f\) и \(g\) са съвместими
\\
и \(f \cup g \; : \; A \cup C \to B \cup D\).
Нека \(x \in A \cup C\) и \(y \in A \cup C\) и \(x \neq y\).
Възможни са три случая:

\subsubsection*{Случай 1. \(x \in A \; \& \; y \in A\)}
Тогава \((f \cup g)(x) = f(x) \neq f(y) = (f \cup g)(y)\),
понеже \(f\) е инекция.

\subsubsection*{Случай 2. \(x \in C \; \& \; y \in C\)}
Тогава \((f \cup g)(x) = g(x) \neq g(y) = (f \cup g)(y)\),
понеже \(g\) е инекция.

\subsubsection*{Случай 3. \(x \in A \; \& \; y \in C\)}
Тогава \((f \cup g)(x) = f(x) \in B\) и \((f \cup g)(y) = g(y) \in D\)
и \(B \cap D = \emptyset\) следователно \((f \cup g)(x) \neq (f \cup g)(y)\).

\subsubsection*{Заключение:}
Следователно понеже \(x\) и \(y\) бяха произволни,
то \(f \cup g : A \cup C \injection B \cup D\). \(\qed\)

\section*{Задача 4.}
Нека \(<A, \leq_A>\) и \(<B, \leq_B>\) са линейно наредени множества
и \(<A, \leq_A>\) е изоморфно с начален отрез на \(<B, \leq_B>\),
а \(<B, \leq_B>\) е изоморфно с финален отрез на \(<A, \leq_A>\).
Тогава \(<A, \leq_A> \; \cong \; <B, \leq_B>\).

\subsection*{Съкращения:}
Нека \(<L, \leq>\) е линейно наредено множество. Тогава
\begin{align*}
StartingCut(S, L, \leq) \leftrightharpoons S \subseteq L \; \& \; (\forall y \in S)(\forall x \in L)(x \leq y \implies x \in S) \\
FinalCut(F, L, \leq) \leftrightharpoons  F \subseteq L \; \& \; (\forall x \in F)(\forall y \in L)(x \leq y \implies y \in F)
\end{align*}

\subsection*{Лема 1:}
Нека \(<L, \leq>\) е линейно наредено множество.
\\
Тогава \(StartingCut(S, L, \leq) \implies FinalCut(L \setminus S, L, \leq)\).

\subsubsection*{Доказателство:}
Нека \(S\) е такова, че \(StartingCut(S, L, \leq)\).
Тогава \(S \subseteq L\) и значи \(L \setminus S \subseteq L\).
Нека \(x \in L \setminus S\). Нека \(y \in L\) и нека \(x \leq y\).
Тогава да допуснем, че \(y \notin L \setminus S\),
Но тогава \(y \in S\) и тогава \(x \in S\), защото \(StartingCut(S, L, \leq)\).
Това е Абсурд! Следователно \(y \in L \setminus S\).
Следователно \(FinalCut(L \setminus S, L, \leq)\)). \(\qed\)

\subsection*{Лема 2:}
Нека \(<L, \leq>\) е линейно наредено множество.
\\
Тогава \(FinalCut(F, L, \leq) \implies StartingCut(L \setminus F, L, \leq)\).

\subsubsection*{Доказателство:}
Нека \(F\) е такова, че \(FinalCut(F, L, \leq)\).
Тогава \(F \subseteq L\) и значи \(L \setminus F \subseteq L\).
Нека \(y \in L \setminus F\). Нека \(x \in L\) и нека \(x \leq y\).
Тогава да допуснем, че \(x \notin L \setminus F\),
Но тогава \(x \in F\) и тогава \(y \in F\), защото \(FinalCut(F, L, \leq)\).
Това е Абсурд! Следователно \(x \in L \setminus F\).
Следователно \(StartingCut(L \setminus F, L, \leq)\)). \(\qed\)

\subsection*{Решение:}
\(<A, \leq_A>\) е изоморфно с начален отрез на \(<B, \leq_B>\)
нека тогава \(S\) е такова, че \(StartingCut(S, B, \leq_B)\)
и \(<A, \leq_A> \; \cong \; <S, \leq_B^S>\).
Нека тогава \(f : A \bijection S\) е изоморфзъм на \(<A, \leq_A>\) върху \(<S, \leq_B^S>\).
В частност \(f : A \injection B\).
\(<B, \leq_B>\) е изоморфно с финален отрез на \(<A, \leq_A>\)
нека тогава \(F\) е такова, че \(FinalCut(F, A, \leq_A)\)
и \(<B, \leq_B> \; \cong \; <F, \leq_A^F>\).
Нека тогава \(g : B \bijection F\) е изоморфзъм на \(<B, \leq_B>\) върху \(<F, \leq_A^F>\).
В частност \(g : B \injection A\).
\\
\vspace{1mm}
\\
Ще построим изоморфизъм на \(<A, \leq_A>\) върху \(<B, \leq_B>\).
Идеята да е разделим множеството \(A\) на две части \(X_0\) и \(A \setminus X_0\).
Като ще искаме \(X_0 \subseteq Range(g)\), което да е финален отрез и елементите на \(X_0\) ще изпращаме с \(g^{-1}\),
а на \(A \setminus X_0\), което ще е начален отрез с \(f\).
За тези цел ще поискаме да разделим и елементите на \(B\) на две части.
Ще искаме
\\
\(g^{-1}[X_0] = B \setminus f[A \setminus X_0]\) или \(X_0 = g[B \setminus f[A \setminus X_0]]\).
\\
\vspace{1mm}
\\
Нека \(X \in \mathcal{P}(A)\), тогава \(X \subseteq A\) и значи \(A \setminus X \subseteq A\).
Тогава \(f[A \setminus X] \subseteq S \subseteq B\) и значи \(B \setminus f[A \setminus X] \subseteq B\).
Следователно \(g[B \setminus f[A \setminus X]] \subseteq F \subseteq A\) и значи
\(g[B \setminus f[A \setminus X]] \in \mathcal{P}(A)\).
Тоест \((\forall T \in \mathcal{P}(A))(g[B \setminus f[A \setminus T]] \in \mathcal{P}(A))\).
Тогава разглеждаме фукцията \(h : \mathcal{P}(A) \to \mathcal{P}(A)\), за която
\\
\((\forall T \in \mathcal{P}(A))(h(T) = g[B \setminus f[A \setminus T]] \in \mathcal{P}(A))\).
\\
\vspace{1mm}
\\
Ще покажем, че \(h\) е монотонна.
Нека \(X_1 \subseteq X_2 \subseteq A\) тогава \(A \setminus X_2 \subseteq A \setminus X_1\)
и значи \(f[A \setminus X_2] \subseteq f[A \setminus X_1]\), но тогава
\(B \setminus f[A \setminus X_1] \subseteq B \setminus f[A \setminus X_2]\)
и значи \(g[B \setminus f[A \setminus X_1]] \subseteq g[B \setminus f[A \setminus X_2]]\).
Тоест \(h(X_1) \subseteq h(X_2)\). Следователно \(h : \mathcal{P}(A) \to \mathcal{P}(A)\) е монотоннa.
\\
\vspace{1mm}
\\
Ще построим неподвижна точка на \(h\), която да е финален отрез на
\\
\(<A, \leq_A>\). За целта ще докажем две леми.

\subsubsection*{Лема 3: \(\forall M (StartingCut(M, A, \leq_A) \implies StartingCut(f[M], B, \leq_B))\)}
Нека \(M\) е такова, че \(StartingCut(M, A, \leq_A)\).
Нека \(y \in f[M]\). Нека \(x \in B\) и \(x \leq_B y\).
От \(y \in f[M]\) следва, че \(y \in f[A] = S\), защото \(M \subseteq A\).
Но от \(StartingCut(S, B, \leq_B)\), следва че \(x \in S = Range(f)\).
Понеже \(f : A \bijection S\) и \(f\) е изоморфизъм, то \(f^{-1}(x) \leq_A f^{-1}(y)\).
Но \(y \in f[M]\) следователно \(f^{-1}(y) \in M\).
Но \(StartingCut(M, A, \leq_A)\) и \(f^{-1}(x) \leq_A f^{-1}(y)\)
следователно \(f^{-1}(x) \in M\) и значи \(x = f(f^{-1}(x)) \in f[M]\).
\\
Заключение: \(StartingCut(f[M], B, \leq_B)\). \(\qed\)

\subsubsection*{Лема 4: \(\forall M (FinalCut(M, B, \leq_B) \implies FinalCut(g[M], A, \leq_A))\)}
Нека \(M\) е такова, че \(FinalCut(M, B, \leq_B)\).
Нека \(x \in g[M]\). Нека \(y \in A\) и \(x \leq_A y\).
От \(x \in g[M]\) следва, че \(x \in g[B] = F\), защото \(M \subseteq B\).
Но от \(FinalCut(F, A, \leq_A)\), следва че \(y \in F = Range(g)\).
Понеже \(g : B \bijection F\) и \(g\) е изоморфизъм, то \(g^{-1}(x) \leq_B g^{-1}(y)\).
Но \(x \in g[M]\) следователно \(g^{-1}(x) \in M\).
Но \(FinalCut(M, B, \leq_B)\) и \(g^{-1}(x) \leq_B g^{-1}(y)\)
следователно \(g^{-1}(y) \in M\) и значи \(y = g(g^{-1}(y)) \in g[M]\).
\\
Заключение: \(FinalCut(g[M], A, \leq_A)\). \(\qed\)
\\
\vspace{1mm}
\\
Нека \(P = \{X \; | \; FinalCut(X, A, \leq_A) \; \& \; X \subseteq h(X) \}\).
\(P\) е множество, защото отделяме от \(\mathcal{P}(A)\) със свойството
\begin{align*}
(\forall z \in X)(\forall y \in A)(z \leq_A y \implies y \in X) \; \& \; X \subseteq h(X).
\end{align*}
Ще се нуждаем от следната Лема:
\subsubsection*{Лема 5: \((\forall X \in P)(h(X) \in P)\)}
Нека \(X \in P\). Тогава \(X \subseteq h(X)\) следователно \(h(X) \subseteq h(h(X))\), защото \(h\) е монотонна.
Но от \(X \in P\) следва и \(FinalCut(X, A, \leq_A)\). Тогава
\\
по Лема 1 \(StartingCut(A \setminus X, A, \leq_A)\).
Тогава от Лема 3
\\
\(StartingCut(f[A \setminus X], B, \leq_B)\).
Тогава от Лема 2
\\
\(FinalCut(B \setminus f[A \setminus X], B, \leq B)\).
Тогава от Лема 4
\\
\(FinalCut(g[B \setminus f[A \setminus X]], A \leq_A)\).
Следователно \(FinalCut(h(X), A, \leq_A)\).
Така \(FinalCut(h(X), A, \leq_A)\) и \(h(X) \subseteq h(h(X))\)
следователно \(h(X) \in P\).
Заключение: \((\forall u \in P)(h(u) \in P)\). \(\qed\)
\\
\vspace{1mm}
\\
Ще покажем, че \(\cup P \in P\).
\\
\vspace{1mm}
\\
Първо ще покажем, че \(FinalCut(\cup P, A, \leq_A)\).
В сила е \((\forall u \in P)(u \subseteq A)\) следователно \(\cup P \subseteq A\).
Нека \(x \in \cup P\). Нека \(y \in A\) и нека \(x \leq_A y\).
От \(x \in \cup P\), следва \((\exists u \in P)(x \in u)\).
Нека тогава \(u \in P\) е такова, че \(x \in u\).
\(u \in P\) следователно \(FinalCut(u, A, \leq_A)\)
но \(x \in u\) и \(x \leq_A y\).
Следователно \(y \in u\).
Следователно \(y \in \cup P\).
Следователно \(FinalCut(\cup P, A, \leq_A)\).
\\
\vspace{1mm}
\\
Остава да покажем, че \(\cup P \subseteq h(\cup P)\).
Нека \(a \in \cup P\). Тогавва
\\
\((\exists M \in P)(a \in M)\).
Нека тогава \(M \in P\) е такова, че \(a \in M\).
От \(M \in P\) следва \(M \subseteq \cup P\).
Но \(h\) е монотонна, следователно \(h(M) \subseteq h(\cup P)\).
\\
Но \(M \in P\) значи \(M \subseteq h(M)\).
Така \(M \subseteq h(\cup P)\) и значи \(a \in h(\cup P)\).
Следователно \(\cup P \subseteq h(\cup P)\).
Така \(\cup P \in P\).
\\
\vspace{1mm}
\\
Тогава от \(\cup P \in P\) и Лема 5 следва \(h(\cup P) \in P\) и значи \(h(\cup P) \subseteq \cup P\).
Следователно \(\cup P = h(\cup P)\) и \(FinalCut(\cup P, A,leqA_A)\).
\\
\vspace{1mm}
\\
Нека тогава \(X_0 = \cup P\).
В сила са:
\begin{align*}
    X_0 = g[B \setminus f[A \setminus X_0]] \\
    FinalCut(X_0, A, \leq_A) \\
    StartingCut(A \setminus X_0, A, \leq_A) \; \text{(по Лема 2)} \\
    StartingCut(f[A \setminus X_0], B, \leq_B) \; \text{(по Лема 3)} \\
    FinalCut(B \setminus f[A \setminus X_0], B, \leq_B) \; \text{(по Лема 1)}
\end{align*}
Така \(X_0 \subseteq Range(g)\) и значи \(X_0 \subseteq Dom(g^{-1})\).
Нека \(u = (g^{-1})_{\restriction X_0}\) и \(v = f_{\restriction A \setminus X_0}\).
Тогава \(Dom(u) = X_0\) и \(Dom(v) = A \setminus X_0\) са непресичащи 
и значи \(u\) и \(v\) са съвместими. Нека тогава \(t = u \cup v\).
Тогава \(Dom(t) = Dom(u) \cup Dom(v) = X_0 \cup (A \setminus X_0) = A\)
и \(Range(t) = Range(u) \cup Range(v)
= Range(f_{\restriction A \setminus X_0}) \cup Range(g^{-1}_{\restriction X_0})
= f[A \setminus X_0] \cup g^{-1}[X_0]
\\
= f[A \setminus X_0] \cup g^{-1}[g[B \setminus f[A \setminus X_0]]]
= f[A \setminus X_0] \cup B \setminus f[A \setminus X_0] = B\).
Следователно \(t : A \surjection B\).
Понеже \(f\) и \(g^{-1}\) са инекции, то и \(u\) и \(v\) са инекции,
но и \(Dom(u) \cap Dom(v) = \emptyset\) и \(Range(u) \cap Range(v) = \emptyset\)
и от Лемата за разделените инекции, следва че \(t : A \injection B\).
Така \(t : A \bijection B\). Ще покажем, че \(t\) е силен хомоморфизъм на линейни наредби.
Нека \(a_1 \in A\) и \(a_2 \in A\) и нека \(a_1 <_A a_2\).
Възможни са три случая.
\\
\vspace{1mm}
\\
Случай 1 \(a_1 \in A \setminus X_0 \; \& \; a_2 \in A \setminus X_0\):
\\
\vspace{1mm}
\\
Тогава \(t(a_1) = v(a_1) = f(a_1) <_B f(a_2) = v(a_2) = t(a_1)\).
\\
\vspace{1mm}
\\
Случай 2 \(a_1 \in X_0 \; \& \; a_2 \in X_0\):
\\
\vspace{1mm}
\\
Тогава \(t(a_1) = u(a_1) = g^{-1}(a_1) <_B g^{-1}(a_2) = u(a_2) = t(a_1)\).
\\
\vspace{1mm}
\\
Случай 3 \(a_1 \in A \setminus X_0 \; \& \; a_2 \in X_0\):
\\
\vspace{1mm}
\\
Понеже \(t\) е инекция да допуснем, че
\(t(a_2) <_B t(a_1)\).
Тогава \(u(a_2) <_B v(a_1)\),
обаче \(Range(v) = f[A \setminus X_0]\) и \(StartingCut(f[A \setminus X_0], A, \leq_A)\).
Тогава
\\
\(u(a_2) \in f[A \setminus X_0]\).
Но това е Абсурд, защото \(Range(u) = B \setminus f[A \setminus X_0]\).
Следователно \(t(a_1) <_B t(a_2)\), защото \(t\) е инекция, a \(a_1 \in A \setminus X_0 \; \& \; a_2 \in X_0\).
\\
\vspace{1mm}
\\
Не е възможно \(a_1 \in X_0 \; \& \; a_2 \in A \setminus X_0 \; \& \; a_1 <_A a_2\),
защото
\\
\(FinalCut(X_0, A, \leq_A)\) и значи \(a_2 \in X_0\).
\\
\vspace{1mm}
\\
Така получаваме \((\forall a_1 \in A)(\forall a_2 \in A)(a_1 \leq_A a_2 \implies t(a_1) \leq_B t(a_2)\)).
\\
\vspace{1mm}
\\
Нека сега \(b_1 \in B\), \(b_2 \in B\) и \(b_1 \leq_B b_2\).
\(<A, \leq_A>\) е л.н.м. В частност всеки два елемента на \(A\) са \(\leq_A\) сравними.
Да допуснем, че
\\
\(t^{-1}(b_2) <_A t^{-1}(b_1)\).
Тогава \(t(t^{-1}(b_2)) <_B t(t^{-1}(b_1))\) и значи \(b_2 <_B b_1\).
\\
Но това е Абсурд!
Следователно \(t^{-1}(b_1) <_A t^{-1}(b_2)\).
Следователно \\
\((\forall b_1 \in B)(\forall b_2 \in B)(b_1 \leq_B b_2 \implies t^{-1}(b_1) \leq_A t^{-1}(b_2)\)).
Следователно
\((\forall a_1 \in A)(\forall a_2 \in A)(t(a_1) \leq_B t(a_2) \implies a_1 \leq_A a_2\)).
Така \(t\) е силен хомоморфизъм на \(<A, \leq_A>\) върху \(<B, \leq_B>\)
и значи \(t\) е изоморфизъм.
Следователно \(<A \leq_A> \; \cong  \; <B \leq_B>\). \(\qed\)

\section*{Задача 5.}
Нека \(\mathcal{W}_1 = \; <W_2, \leq_1>\) и \(\mathcal{W}_2 = \; <W_2, \leq_2>\) са добре наредени множества.
Тогава е в сила точно едно от трите:
\begin{itemize}
    \item \(\mathcal{W}_1\) и \(\mathcal{W}_2\) са изоморфни;
    \item \(\mathcal{W}_1\) е изоморфно на начален сегмент на \(\mathcal{W}_2\);
    \item \(\mathcal{W}_2\) е изоморфно на начален сегмент на \(\mathcal{W}_1\).
\end{itemize}
\vspace{1mm}
Преди да преминем към решението ще формулираме и докажем две леми.
\subsection*{Лема 1:}
Нека \(\mathcal{A} = \; <A, \leq_A>\) и \(\mathcal{B} = \; <B, \leq_B>\) са добре наредени множества.
Нека \(a \in A\), \(b \in B\), \(d \in B\) и
\(<seg_A(a), \leq_A^a> \; \cong \; <seg_B(b), \leq_B^b>\)
и \(<seg_A(a), \leq_A^a> \; \cong \; <seg_B(d), \leq_B^d>\).
Тогава \(b = d\).
\subsubsection*{Доказателство:}
Да допуснем, че \(b \neq d\).
Тогава нека Б.О.О. \(b \leq_B d\).
Тогава
\\
\(b \in seg_B(d) \setminus seg_B(b)\) и \(seg_B(b) \subseteq seg_B(d)\).
Но очевидно тогава \(seg_B(b) = seg_{seg_B(d)}(b) \subseteq seg_B(d)\).
Понеже
\(<seg_A(a), \leq_A^a> \; \cong \; <seg_B(b), \leq_B^b>\), то
нека \(f : seg_A(a) \bijection seg_B(b)\) е изоморфизъм.
Понеже \(<seg_A(a), \leq_A^a> \; \cong \; <seg_B(d), \leq_B^d>\),
то нека
\(g : seg_A(a) \bijection seg_B(d)\) е изоморфизъм.
Тогава очевидно
\(f^{-1} \circ g : seg_B(b) \bijection seg_B(d)\)
е изоморфизъм. Но тогава
\\
\(seg_B(b) = seg_{seg_B(d)}(b) \subset seg_B(d)\)
и \(<seg_B(b), \leq_B^b> \; \cong \; <seg_B(d), \leq_B^d>\).
\\
Но това знаем, че е невъзможно, защото никое добре наредено множество
не е изоморфно на свой собствен начален сегмет.
Следователно \(b = d\). \(\qed\)
\subsection*{Лема 2:}
Нека \(\mathcal{A} = \; <A, \leq_A>\) и \(\mathcal{B} = \; <B, \leq_B>\) са добре наредени множества.
Нека \(a \in A\), \(b \in B\), \(c \in B\) и нека \(a <_A c\).
Нека \(f : seg_A(c) \bijection seg_B(b)\) е изоморфизъм на
\(<seg_A(c), \leq_A^c>\) върху \(<seg_B(b), \leq_B^b>\).
Тогава \(f_{\restriction seg_A(a)}\) е изоморфизъм на
\(<seg_A(a), \leq_A^c>\) върху \(<seg_B(f(a)), \leq_B^{f(a)}>\).
\subsection*{Доказателство:}
Понеже \(a <_A c\), то \(seg_A(a) \subseteq seg_A(c)\).
Тогава нека \(g = f_{\restriction seg_A(a)}\).
\\
Понеже \(f : seg_A(c) \bijection seg_B(b)\), то \(g : seg_A(a) \injection seg_B(b)\).
\\
\(Range(g) = Range(f_{\restriction seg_A(a)}) = f[seg_A(a)] = f[\{x \in A \; | \; x <_A a\}]
\\
= \{f(x) \; | \; x \in A \; \& \; x <_A a\}
= \{f(x) \; | \; x \in seg_A(a) \; \& \; f(x) <_B f(a)\}
\\
= \{f(f^{-1}(y)) \; | \; y \in seg_B(c) \; \& \; f(f^{-1}(y)) <_B f(a)\}
\\
= \{y \; | \; y \in seg_B(c) \; \& \; y <_B f(a)\}
= \{y \; | \; y \in B \; \& \; y <_B f(a)\}
\\ = seg_B(f(a))\). Следователно \(g : seg_A(a) \bijection seg_B(f(a))\).
Понеже
\\
\((\forall x \in seg_A(a))(\forall y \in seg_A(a))
(x \leq_A^a y \iff x \leq_A y \\
\iff f(x) \leq_B f(y) \iff g(x) \leq_B g(y) \iff g(x) \leq_B^{f(a)} g(y))\),
\\
то следва че
\(g : seg_A(a) \bijection seg_B(f(a))\) е изоморфизъм на
\\
\(<seg_A(a), \leq_A^a>\) върху \(<seg_B(f(a)), \leq_B^{f(a)}>\). \(\qed\)

\subsection*{Решение:}
Нека \(f \subseteq W_1 \times W_2\) е следната релация:
\begin{align*}
f = \{<x, y> \in W_1 \times W_2 \; | \; <seg_{W_1}(x), \leq_1^x> \; \cong \; <seg_{W_2}(y), \leq_2^y> \}.
\end{align*}

\subsubsection*{\(f\) е функция:}
Нека \(x \in W_1\), \(y \in W_2\) и \(z \in W_2\)
са такива, че \(<x, y> \; \in f\) и \(<x, z> \; \in f\).
Тогава \(<seg_{W_1}(x), \leq_1^x> \; \cong \; <seg_{W_2}(y), \leq_2^y>\)
и \(<seg_{W_1}(x), \leq_1^x> \; \cong \; <seg_{W_2}(z), \leq_2^z>\).
Но тогава по Лема 1 \(y = z\).
Следователно \(Funct(f)\).

\subsubsection*{\(f\) е инекция:}
Нека \(x \in W_1\), \(y \in W_2\) и \(s \in W_1\)
са такива, че \(<x, y> \; \in f\) и \(<s, y> \; \in f\).
Тогава \(<seg_{W_1}(x), \leq_1^x> \; \cong \; <seg_{W_2}(y), \leq_2^y>\)
и \(<seg_{W_1}(s), \leq_1^s> \; \cong \; <seg_{W_2}(y), \leq_2^y>\).
Следователно \(<seg_{W_2}(y), \leq_2^y> \; \cong \; <seg_{W_1}(x), \leq_1^x>\)
и \(<seg_{W_2}(y), \leq_2^y> \; \cong \; <seg_{W_1}(s), \leq_1^s>\).
Но тогава по Лема 1 \(x = s\).
Следователно \(f\) е инекция.

\subsubsection*{\(f\) запазва наредбата:}
Нека \(x \in Dom(f)\), \(s \in Dom(f)\) и \(x <_1 s\).
След като \(s \in Dom(f)\), то
\((\exists y \in W_2)(<seg_{W_1}(s), \leq_1^s> \; \cong \; <seg_{W_2}(y), \leq_2^y>)\).
Нека тогава \(y \in W_2\) и
\(<seg_{W_1}(s), \leq_1^s> \; \cong \; <seg_{W_2}(y), \leq_2^y>\).
Тогава нека \(h : seg_{W_1}(s) \bijection seg_{W_2}(y)\) е изоморфизъм на
\(<seg_{W_1}(s), \leq_1^s>\) върху \(<seg_{W_2}(y), \leq_2^y>\).
Тогава според Лема 2
\(<seg_{W_1}(x), \leq_1^x> \; \cong \; <seg_{W_2}(g(x)), \leq_2^{g(x)}>\).
Но \(g(x) \in seg_{W_2}(y)\) и значи \(g(x) <_2 y\).
Но тогава \(f(x) = g(x) <_2 y = f(s)\).
Следователно
\((\forall a \in Dom(f))(\forall b \in Dom(f))(a <_1 b \implies f(a) <_2 f(b))\).

\subsubsection*{\(Dom(f)\) е начален сегмент:}
\(Dom(f) \subseteq W_1\).
Нека \(v \in Dom(f)\). Нека \(u \in W_1\) и \(u <_1 v\).
Нека \(t = f(v)\).
От \(v \in Dom(f)\), следва че
\(<seg_{W_1}(v), \leq_1^v> \; \cong \; <seg_{W_2}(t), \leq_2^{t}>\).
Тогава нека \(h : seg_{W_1}(v) \bijection seg_{W_2}(t)\) е изоморфизъм.
Тогава според Лема 2
\\
\(<seg_{W_1}(u), \leq_1^u> \; \cong \; <seg_{W_2}(h(u)), \leq_2^{h(u)}>\).
Следователно \(<u, h(u)> \; \in f\).
Тогава \(u \in Dom(f)\).
Следователно \(Dom(f)\) е начален сегмент на \(W_1\).

\subsubsection*{\(Range(f)\) е начален сегмент:}
\(Range(f) \subseteq W_2\).
Нека \(v \in Range(f)\). Нека \(u \in W_2\) и \(u <_2 v\).
Нека \(z \in W_1\) и \(v = f(z)\).
От \(z \in Dom(f)\), следва че
\(<seg_{W_1}(z), \leq_1^z> \; \cong \; <seg_{W_2}(v), \leq_2^{v}>\).
Тогава нека \(h : seg_{W_2}(v) \bijection seg_{W_1}(z)\) е изоморфизъм.
Тогава според \\
Лема 2
\(<seg_{W_2}(u), \leq_2^u> \; \cong \; <seg_{W_1}(h(u)), \leq_1^{h(u)}>\).
Следователно
\\
\(<h(u), u> \; \in f\).
Тогава \(u \in Range(f)\).
Следователно \(Range(f)\) е начален сегмент на \(W_2\).
\\
\vspace{1mm}
\\
Така понеже \(f\) запазва наредбата и е инекция, то \(f\) е изоморфизъм на
\\
\(<Dom(f), \leq_1 \cap (Dom(f) \times Dom(f))>\)
върху
\\
\(<Range(f), \leq_2 \cap (Range(f) \times Range(f))>\).

\subsubsection*{\(Dom(f) = W_1 \; \lor \; Range(f) = W_2\):}

Допускаме, че \(Dom(f) \neq W_1 \; \& \; Range(f) \neq W_2\). Но понеже
\\
\(Dom(f) \subseteq W_1\) и \(Range(f) \subseteq W_2\),
следва че \(W_1 \setminus Dom(f) \neq \emptyset\)
\\
и \(W_2 \setminus Range(f) \neq \emptyset\).
\\
Нека тогава \(x = \underset{\leq_1}{min}(W_1 \setminus Dom(f))\)
и \(y = \underset{\leq_2}{min}(W_2 \setminus Range(f))\).
\\
Понеже \((\forall s \in W_1) (s \in W_1 \setminus Dom(f) \implies x \leq_1 s)\), то
\\
\((\forall s \in W_1) (s <_1 x \implies s \in Dom(f))\).
Следователно \(seg_{W_1}(x) \subseteq Dom(f)\).
Нека \(s \in Dom(f)\). Допускаме, че \(s \notin seg_{W_1}(x)\).
Тогава \(x \leq_1 s\). Но понеже \(x \notin Dom(f)\), то \(x <_1 s\).
Понеже \(s \in Dom(f)\), то
\\
\(<seg_{W_1}(s), \leq_1^s> \; \cong \; <seg_{W_2}(f(s)), \leq_2^{f(s)}>\).
Нека тогава
\\
\(h : seg_{W_1}(s) \bijection seg_{W_2}(f(s))\)
е изоморфизъм на
\(<seg_{W_1}(s), \leq_1^s>\) върху \(<seg_{W_2}(f(s)), \leq_2^{f(s)}>\).
Тогава според Лема 2
\\
\(<seg_{W_1}(x), \leq_1^x> \; \cong \; <seg_{W_2}(h(x)), \leq_2^{h(x)}>\).
Но тогава \(x \in Dom(f)\). Това е Абсурд!
Следователно \(s \in seg_{W_1}(x)\).
Следователно \(Dom(f) \subseteq seg_{W_1}(x)\).
Така \(Dom(f) = seg_{W_1}(x)\).
\\
С аналогични разсъждения получаваме \(Range(f) = seg_{W_2}(y)\).
\\
Понеже \(f\) е инекция, то \(f : seg_{W_1}(x) \bijection seg_{W_2}(y)\).
Но \(f\) запазва и наредбата, следователно \(f\) е изоморфизъм на
\(<seg_{W_1}(x), \leq_1^x>\) върху
\\
\(<seg_{W_2}(y), \leq_2^{y}>\).
Така \(<x, y> \; \in f\).
Следователно
\\
\(x \in Dom(f)\) и \(y \in Range(f)\).
Това е Абсурд!
\\
Следователно \(Dom(f) = W_1 \; \lor \; Range(f) = W_2\).
\\
\vspace{1mm}
\\
Възможни са три случая.
\\
\vspace{1mm}
\\
Случай 1 \(Dom(f) = W_1 \; \& \; Range(f) = W_2\):
\\
\vspace{1mm}
\\
Тогава понеже \(f\) е изоморфизъм, то \(\mathcal{W}_1 \cong \mathcal{W}_2\).
\\
\vspace{1mm}
\\
Случай 2 \(Dom(f) = W_1 \; \& \; Range(f) \neq W_2\):
\\
\vspace{1mm}
\\
Тогава понеже \(Rage(f)\) е начален сегмент на \(W_2\) относно \(\leq_2\)
и \(f\) е изоморфизъм, то \(\mathcal{W}_1\) е изоморфно на \(Range(f)\),
което е собствен начален сегмент на \(\mathcal{W}_2\). 
\\
\vspace{1mm}
\\
Случай 3 \(Dom(f) \neq W_1 \; \& \; Range(f) = W_2\):
\\
\vspace{1mm}
\\
Тогава понеже \(Dom(f)\) е начален сегмент на \(W_1\) относно \(\leq_1\)
и \(f\) е изоморфизъм, то \(\mathcal{W}_2\) е изоморфно на \(Dom(f)\),
което е собствен начален сегмент на \(\mathcal{W}_1\).

\section*{Лема за най-големия елемент.}
Нека \(<L, \leq_L>\) е линейно наредено множество.
Нека \(\emptyset \neq M \subseteq L\) и \(M\) е крайно.
Тогава в \(M\) има най-голям елемент относно \(\leq_L\).

\subsection*{Доказателство:}
Нека \(<L, \leq_L>\) е линейно наредено множество.
С индукция в множеството на естествените числа ще докажем
следното твърдение:
\begin{align*}
\forall n \forall M (\overline{\overline{M}} = \overline{\overline{s(n)}} \; \& \; M \subseteq L \implies (\exists m \in M)(\forall x \in M)(x \leq_L m)).
\end{align*}

\subsubsection*{База: \(n = 0\)}
Нека \(M \subseteq L\) и \(\overline{\overline{M}} = \overline{\overline{1}}\).
Тогава \(M = \{m\}\) и понеже \(\leq_L\) е рефлексивна в \(L\), то
\(m \leq m\) и значи \((\forall x \in M)(x \leq_L m)\).

\subsubsection*{Индукционно предположение:}
Нека \(n \in \omega\) и 
\(\forall M (\overline{\overline{M}} = \overline{\overline{s(n)}} \; \& \; M \subseteq L \implies (\exists m \in M)(\forall x \in M)(x \leq_L m))\).

\subsubsection*{Индукционна стъпка:}
Нека \(M \subseteq L\) и \(\overline{\overline{M}} = \overline{\overline{s(s(n))}}\).
Тогава \(M \neq \emptyset\). Нека тогава \(a \in M\).
Тогава очевидно \(\overline{\overline{M \setminus \{a\}}} = \overline{\overline{s(n)}}\)
и \(M \setminus \{a\} \subseteq L\). Тогава по И.П. за \(M \setminus \{a\}\),
следва че \((\exists b \in M \setminus \{a\})(\forall x \in M \setminus \{a\})(x \leq_L b)\). Нека тогава \(b \in M\) и \((\forall x \in M \setminus \{a\})(x \leq_L b)\).
Понеже \(<L, \leq_L>\) е линейно наредено множество, то са възможни два случая.
\\
\vspace{1mm}
\\
Случай 1 \(a \leq_L b\):
\\
\vspace{1mm}
\\
Нека \(x \in M\). Тогава, ако \(x \in M \setminus \{a\}\), то \(x \leq_L b\) понеже
\(b\) е най-голямия относно \(\leq_L\) в \(M \setminus \{a\}\),
ако пък \(x = a\), то директно \(x \leq_L b\).
Следователно \(b \in M\) и \((\forall x \in M)(x \leq_L b)\).
\\
\vspace{1mm}
\\
Случай 2 \(b \leq_L a\):
\\
\vspace{1mm}
\\
Нека \(x \in M\). Тогава, ако \(x \in M \setminus \{a\}\), то \(x \leq_L b\) понеже
\(b\) е най-голямия относно \(\leq_L\) в \(M \setminus \{a\}\) и от \(b \leq_L a\) и транзитивността на \(\leq_L\), следва че \(x \leq_L a\),
ако пък \(x = a\), то от рефлексивността на \(\leq_L\) следва \(x \leq_L a\).
Следователно \(a \in M\) и \((\forall x \in M)(x \leq_L a)\).
\\
\vspace{1mm}
\\
Така и в двата случая е в сила \((\exists m \in M)(\forall x \in M)(x \leq_L m))\).
Но \(M\) е произволно, следователно след обобщение получаваме
\\
\(\forall M (\overline{\overline{M}} = \overline{\overline{s(s(n))}} \; \& \; M \subseteq L \implies (\exists m \in M)(\forall x \in M)(x \leq_L m))\).

\subsubsection*{Заключение:}
\begin{align*}
\forall n \forall M (\overline{\overline{M}} = \overline{\overline{s(n)}} \; \& \; M \subseteq L \implies (\exists m \in M)(\forall x \in M)(x \leq_L m)).
\end{align*} \(\qed\)

\section*{Лема за добрата наредба и ординала.}
Нека \(<W, \leq_W>\) е добре наредено множество.
Тогава има при това единствен ординал \(\alpha\),
такъв че \(<W, \leq_W> \; \cong \; <\alpha, \leq_\alpha>\).

\subsection*{Доказателство:}
Да допуснем, че за всяко \(\alpha\), \(<\alpha, \leq_\alpha>\) е изоморфно със собствен начален сегмент на \(<W, \leq_W>\). Тогава по аксиомната схема за замяната има множество
\(A\), такова че \\
\((\forall w \in W)(\exists a \in A)(ord(a) \; \& <seg(w), \leq_W^w> \; \cong \; <a, \leq_a>)\).
Нега тогава \(A\) е такова множество и нека \(B = \{a \; | \; a \in A \; \& \; ord(a)\}\). Тогава по аксиомната схема за отделянето \(B\) е множество.
Нека тогава \(\alpha\) е произволен ординал, тогава според допускането
\((\exists w \in W)(<seg(w), \leq_W^w> \; \cong \; <a, \leq_a>)\).
Тогава \(a \in A\), но в сила е и \(ord(a)\).
Следователно \(a \in B\).
Следователно \(\forall \alpha (\alpha \in B)\). Но това е Абсурд, защото няма множество,
което да съдържа всички ординали!
Но от Задача 5. следва, че наредбата на всеки ординал е сравнима с \(\leq_W\).
Тогава нека \\
\(\alpha = \mu\beta[(\exists \gamma \leq \beta)(<W, \leq_W> \; \cong \; <\gamma, \leq_\gamma>)]\). Да допуснем, че \(<W, \leq_W>\) не е изоморфно с \(<\alpha, \leq_\alpha>\). Тогава от свойството, което \(\alpha\) минимизира има ординал
\(\gamma \leq \alpha\), такъв че \(<W, \leq_W> \; \cong \; <\gamma, \leq_\gamma>\).
Нека тогава \(\gamma \leq \alpha\) и \(<W, \leq_W> \; \cong \; <\gamma, \leq_\gamma>\).
Понеже \(<W, \leq_W>\) не е изоморфно с \(<\alpha, \leq_\alpha>\), то \(\gamma < \alpha\)
и \(<W, \leq_W> \; \cong \; <\gamma, \leq_\gamma>\). Но това противоречи на факта,
че \(\alpha\) е най-малкият ординал със свойството \((\exists \delta \leq \alpha)(<W, \leq_W> \; \cong \; <\delta, \leq_\delta>)\).
Следователно \(<W, \leq_W> \; \cong \; <\alpha, \leq_\alpha>\).
От фактите, че има единствен изоморфизъм между две добри наредби и изоморфизма на добри наредби е рефлексивно, симетрично и транзитивно свойство е ясно, че \(\alpha\) е единствения ординал, за който стандартната ординална наредба е изоморфна с \(\leq_W\). \(\qed\)

\section*{Лема за образа на функция с краен домейн.}
Нека \(f : A \to B\) и нека \(A\) е крайно.
Тогава \(Range(f)\) е крайно и \(\overline{\overline{Range(f)}} \leq \overline{\overline{A}}\).

\subsection*{Доказателство:}
Нека \(g = \{<b, f^{-1}[\{b\}]> | \; b \in Range(f)\}\).
От аксиомните схеми за замяна и отделяне е ясно, че \(g\) е множество.
Очевидно \(Rel(g)\).
Очевидно и \(Funct(g)\).
Нека \(b_1 \in Range(f)\) и \(b_2 \in Range(f)\)
и \(g(b_1) = g(b_2)\).
Тогава \(f^{-1}[\{b_1\}] = f^{-1}[\{b_2\}]\).
Но тогава \(\{b_1\} = \{b_2\}\) и значи \(b_1 = b_2\).
Следователно \(g\) е инекция.
Следователно \(g : Range(f) \bijection Range(g)\).
Нека \(b \in Range(f)\). Тогава \((\exists a \in A)(f(a) = b)\).
Нека тогава \(a \in A\) и \(f(a) = b\).
Тогава \(a \in f^{-1}[\{b\}]\) и значи \(g(b) \neq \emptyset\).
Тогава \(Range(g)\) е множество от непразни множества.
Но очевидно \(Range(g) \subseteq \mathcal{P}(A)\).
Понеже \(A\) е крайно, то знаем и че \(\mathcal{P}(A)\) е крайно.
Следователно \(Range(g)\) е крайно множество от непразни множества.
Тогава \(Range(g)\) има функция на избора.
Нека тогава \(h\) е функция на избора за \(Range(g)\) (Правим креан избор).
Тогава \((\forall b \in Range(f))(h(g(b)) \in g(b) \subseteq A)\).
Следователно \(Range(g \circ h) \subseteq A\).
Но \(A\) е крайно и значи \(Range(g \circ h)\) също е крайно.
Но тогава \(g \circ h : Range(f) \bijection Range(g \circ h)\). \\
Следователно \(\overline{\overline{Range(f)}} = \overline{\overline{Range(g \circ h)}} \leq \overline{\overline{A}}\). \(\qed\)

\section*{Задача 6.}
Нека \(A\) е множество. Тогава \(A\) е крайно тогава и само тогава когато
има бинарна релация \(R\), такава че \(<A, R>\) и \(<A, R^{-1}>\) са добре наредени.

\subsection*{Решение:}

\subsubsection*{\((\implies)\) Нека \(A\) е крайно.}
Тогава нека \(n\) е такова, че \(\overline{\overline{A}} = \overline{\overline{n}}\).
Нека тогава \(f : A \bijection n\). Дефинираме релация \(\leq_f\) в \(A\)
така
\begin{align*}
    \leq_f \; = \{u \; | \; u \in A \times A \; \& \; \exists a \exists b (u = <a, b> \; \& \; f(a) \leq_n f(b)) \}.
\end{align*}
По аксиомната схема за отделянето следва, че \(\leq_f\) е множество.
От факта, че отделяме от \(A \times A\) следва \(Rel(\leq_f)\).
\\
\vspace{1mm}
\\
Нека \(a \in A\). Тогава \(<a, a> \in A \times A\), но също така \(f(a) \leq_n f(a)\).
Следователно \(a \leq_f a\). Следователно \(\leq_f\) е рефлексивна.
\\
\vspace{1mm}
\\
Нека \(b \in A\) и \(c \in A\) и \(b \leq_f c\) и \(c \leq_f b\).
Тогава \(f(b) \leq_n f(c)\) и \(f(c) \leq_n f(b)\).
Но \(\leq_n\) е антисиметрична, следователно \(f(b) = f(c)\).
Но \(f\) е бикеция следователно \(b = c\).
Следователно \(\leq_f\) е антисиметрична.
\\
\vspace{1mm}
\\
Нека \(x \in A\), \(y \in A\), \(z \in A\)
и \(x \leq_f y\) и \(y \leq_f z\).
Тогава \(f(x) \leq_n f(y)\) и \(f(y) \leq_n f(z)\).
Но \(\leq_n\) е транзитивна, следователно \(f(x) \leq_n f(z)\).
Следователно \(x \leq_f z\).
Следователно \(\leq_f\) е транзитивна.
\\
\vspace{1mm}
\\
Следователно \(\leq_f\) задава частична наредба в \(A\).
\\
\vspace{1mm}
\\
Ще докажем, че всяко непразно подмножество на \(A\) има най-голям и най-малък елемент.
\\
\vspace{1mm}
\\
Нека \(B \subseteq A\) и \(B \neq \emptyset\).
Нека \(C = f[B]\). Тогава \(C \subseteq n\) и \(C \neq \emptyset\).
Тогава \(C\) има най-малък елемент спрямо \(\leq_n\). Нека това е \(m\).
Също така понеже \(n\) е крайно и добре наредено,
то \(C\) е крайно и \(n\) е линейно наредено.
Тогава според Лемата за най-големия елемент \(C\) има най-голям елемент спрямо \(\leq_n\).
Нека това е \(k\).
Тогава в сила е \((\forall l \in C)(m \leq_n l \; \& \; l \leq_n k)\).
Нека \(u \in B\). Тогава \(m \leq_n f(u)\) и \(f(u) \leq_n k\).
Следователно \(f^{-1}(m) \leq_f u\) и \(u \leq_f f^{-1}(k)\).
Понеже \(m \in C\), \(k \in C\), \(C = f[B]\) и \(f\) е биекция,
то \(f^{-1}(m) \in B\) и \(f^{-1}(k) \in B\).
Така \(f^{-1}(m)\) е най-малкия за \(B\) спрямо \(\leq_f\)
и \(f^{-1}(k)\) е най-големия за \(B\) спрямо \(\leq_f\).
Но тогава \(f^{-1}(m)\) е най-малкия за \(B\) спрямо \(\leq_f\)
и \(f^{-1}(k)\) е най-малкия за \(B\) спрямо \(\geq_f\).
Понеже \(B\) беше произволно, то следва че
\(<A, \leq_f>\) и \(<A, \geq_f>\) са добре наредени множества.

\subsubsection*{\((\impliedby)\) Нека има бинарна релация \(R\), таква че \\ \(<A, R>\) и \(<A, R^{-1}>\) са добре наредени.}
Нека тогава \(R\) е такава бинарна релация, че \(<A, R>\) и \(<A, R^{-1}>\) са добре наредени. \(<A, R>\) е добре наредено тогава по Лемата за добрата наредба и ординала
има ординал \(\alpha\), такъв че \(<A, R> \; \cong \; <\alpha, \leq_\alpha>\).
Нека тогава \(\alpha\) е такъв че \(<A, R> \; \cong \; <\alpha, \leq_\alpha>\).
Нека тогава \(g\) е единствения изоморфизъм на \(<A, R>\) върху \(<\alpha, \leq_\alpha>\).
Допускаме, че \(\omega \leq \alpha\).
Тогава \(\omega \subseteq \alpha\) по свойствата на ординалите.
Понеже \(0 \in \omega\), то \(\emptyset \neq \omega \subseteq \alpha\).
Нека \(D = g^{-1}[\omega]\). Тогава \(\emptyset \neq D \subseteq A\) и значи
\(D\) има най-малък елемент относно \(R^{-1}\).
Тогава нека \(y \in D\) е такъв, че \((\forall d \in D)(<y, d> \in R^{-1})\).
Следователно \((\forall d \in D)(<d, y> \in R)\).
Следователно \((\forall d \in D)(g(d) \leq_\alpha g(y))\).
Но \(g(y) \in \omega\) и \(s(g(y)) \in \omega\) и \(g(y) < s(g(y))\) и \(\lnot(s(g(y)) \leq g(y))\).
Следователно допускането \(\omega \leq \alpha\) доведе до Абсурд,
но от свойството трихотомия на сравнението на ординали следва \(\alpha < \omega\).
Следователно \(\alpha \in \omega\) и значи \(Nat(\alpha)\).
Но понеже \(g : A \bijection \alpha\), то \(\overline{\overline{A}} = \overline{\overline{\alpha}}\) и значи \(A\) е крайно. \(\qed\)

\section*{Задача 7.}
Нека \(A\) е множество. Тогава \(A\) е крайно тогава и само тогава когато
За всяко непразно подмножество \(B\) на \(A\) \(<B, \subseteq>\) има максимален елемент.

\subsection*{Решение:}

\subsubsection*{\((\implies)\) Нека \(A\) е крайно множество.}
Нека \(n\) е такова, че \(\overline{\overline{A}} = \overline{\overline{n}}\).
Тогава както знаем \(\mathcal{P}(A)\) е крайно множество от крайни множества.
Нека тогава \(B \in \mathcal{P}(A) \setminus \{\emptyset\}\).
Тогава \(B\) е крайно множество от крайни множества.
Нека тогава \\
\(C = \{k \; | \; k \in \omega \; \& \; (\exists M \in B)(\overline{\overline{k}} = \overline{\overline{M}})\}\).
От аксиомната схема за отделянето \(C\) е множество.
Ако разгледаме \(id_C\), то очевидно \(id_C : C \injection s(n)\),
понеже \((\forall S \in \mathcal{P}(A))(Fin(S) \; \& \; \overline{\overline{S}} \leq \overline{\overline{A}})\) и \(\emptyset \subset B \subseteq \mathcal{P}(A)\) и следователно \\ \(Fin(C) \; \& \; \emptyset \neq C \subseteq s(n)\). Но \(<s(n), \leq_{s(n)}>\) е добре наредено и крайно и значи линейно наредено и крайно и тогава по Лемата за най-големия елемент \(C\) има най-голям елемент. Нека това е \(m\). 
Но \(m \in C\) следователно \((\exists S \in B)(\overline{\overline{S}} = \overline{\overline{m}})\). Нека тогава \(S \in B\) и \(\overline{\overline{S}} = \overline{\overline{m}}\). Да допуснем, че \(S\) не е максимален за \(B\) спрямо \(\subseteq\). Тогава \((\exists Y \in B)(S \subset Y)\).
Нека тогава \(Y \in B\) и \(S \subset Y\).
Тогава понеже \(Y \in B\), то \(Fin(Y)\) и тогава \(\overline{\overline{S}} < \overline{\overline{Y}}\). Нека \(l\) е такова, че \(\overline{\overline{l}} = \overline{\overline{Y}}\). Тогава е ясно, че \(m < l\).
От \(Y \in B\) следва, че \(l \in C\). Но тогава \(l \leq m\), което е Абсурд!
Следователно \(S\) е максимален за \(B\) спрямо \(\subseteq\).
Но понеже \(B\) е произволно, то след обобщение получаваме,
че всяко непразно подмножество на \(A\) има максимален елемент спрямо \(\subseteq\).

\subsubsection*{\((\impliedby)\) Нека всяко непразно подмножество на \(A\) има максимален елемент спрямо \(\subseteq\).}

Нека тогава разгледаме \(P = \{T \; | \; T \in \mathcal{P}(A) \; \& \; Fin(T)\}\).
По аксиомната схема за отделянето \(P\) е множество.
При това очевидно \(P \subseteq \mathcal{P}(A)\).
Също така понеже \(\emptyset \in \mathcal{P}(A)\) и \(Fin(\emptyset)\),
то \(\emptyset \in P\) и значи \(P \neq \emptyset\).
Но тогава \(P\) има максимален елемент относно \(\subseteq\).
Нека тогава \(D \in P\) и \(\lnot(\exists M \in P)(D \subset M)\).
Понеже \(D \in P\), то \(D \subseteq A\) и \(Fin(D)\).
Да допуснем, че \(D \neq A\). Тогава \(A \setminus D \neq \emptyset\).
Нека тогава \(x \in A \setminus D\).
Тогава \(D \cup \{x\} \subseteq A\) и както знаем \(Fin(D \cup \{x\})\).
Следователно \(D \cup \{x\} \in P\) и \(D \subset D \cup \{x\}\).
Но това е Абсурд, защото \(D\) е максимален относно \(\subseteq\).
Следователно \(A = D\) и така \(Fin(A)\). \(\qed\)

\section*{Задача 8.}

\subsection*{Част 1 (ZF). \(\forall A (\exists B (B \subset A \; \& \; \overline{\overline{A}} = \overline{\overline{B}}) \implies \lnot Fin(A))\).}

Твърдението е логически еквивалентно с: \\
\(\forall A (Fin(A) \implies (\forall B \in \mathcal{P}(A) \setminus \{A\})(\overline{\overline{A}} \neq \overline{\overline{B}}))\), което и ще докажем. \\

Нека \(A\) е крайно множество. Нека \(B \in \mathcal{P}(A) \setminus \{A\}\).
Тогава понеже \(A\) е крайно, то както знаем и \(B\) е крайно.
Обаче понеже \(B \subset A\), то \(\overline{\overline{B}} < \overline{\overline{A}}\)
и следователно \(\overline{\overline{A}} \neq \overline{\overline{B}}\). \(\qed\)

\subsection*{Част 2 (ZF + ACC).}
\begin{align*}
    \forall A (\lnot Fin(A) \implies (\exists B \in \mathcal{P}(A) \setminus A)(\overline{\overline{B}} = \overline{\overline{A}}))
\end{align*}

Нека \(A\) е безкрайно множество, тоест не е крайно.
Първо с индукция ще покажем, че \(\forall n (\exists S \in \mathcal{P}(A))(\overline{\overline{S}} = \overline{\overline{n}})\).
\\
\vspace{1mm}
\\
База: \(\emptyset \in \mathcal{P}(A)\) и \(\emptyset = 0\).
\\
\vspace{1mm}
\\
Индукционна хипотеза: Нека \(n\) е такова, че \((\exists S \in \mathcal{P}(A))(\overline{\overline{S}} = \overline{\overline{n}})\).
\\
\vspace{1mm}
\\
Индукционна стъпка: От И.Х. знаем, че \((\exists S \in \mathcal{P}(A))(\overline{\overline{A}} = \overline{\overline{n}})\).
Нека тогава \(X \in \mathcal{P}(A)\) и \(\overline{\overline{X}} = \overline{\overline{n}}\). Ако допуснем, че \(A \setminus X = \emptyset\),
то \(X = A\) и значи \(Fin(A)\), което е Абсурд!
Тогава \(A \setminus X \neq \emptyset\).
Нека тогава \(a \in A \setminus X\).
Тогава както знаем \(\overline{\overline{X \cup \{a\}}} = \overline{\overline{s(n)}}\)
и \(X \cup \{a\} \in \mathcal{P}(A)\).
\\
\vspace{1mm}
\\
Заключение: \(\forall n (\exists S \in \mathcal{P}(A))(\overline{\overline{S}} = \overline{\overline{n}})\).
\\
\vspace{1mm}
\\
Както знаем \(\forall n \; Fin(\mathcal{P}(n))\).
Нека тогава \\
\(U = \{P \; | \; P \in \mathcal{P}(\mathcal{P}(A)) \; \& \; \exists n (\forall S \in P)(\overline{\overline{S}} = \overline{\overline{\mathcal{P}(n)}}) \}\).
Тогава по аксиомната схема за отделянето \(U\) е множество.
От \(\forall n (\exists S \in \mathcal{P}(A))(\overline{\overline{S}} = \overline{\overline{n}})\) и \(\forall n \; Fin(\mathcal{P}(n))\) 
следва \((\forall P \in U)(P \neq \emptyset)\).
Нека \\
\(f = \{<n, P> \; | \; n \in \omega \; \& \; P \in U \; \& \; (\forall S \in P)(\overline{\overline{S}} = \overline{\overline{\mathcal{P}(n)}})\}\).
По аксиомните схеми за замяна и отделяне следва, че \(f\) е множество.
От \(f \subseteq \omega \times U\) следва \(Rel(f)\).
От факта \(\forall n \forall m (n \neq m \iff \overline{\overline{P(n)}} \neq \overline{\overline{P(m)}})\) следва, че \(f : \omega \injection U\).
Очевидно \(Range(f) = U\) и значи \(f : \omega \bijection U\).
Следователно \(U\) е изброимо. Но също така \(U\) е множество от непразни множества.
Тогава от (ACC) \(U\) има функция на избора.
Нека тогава \(g\) е функция на избора за \(U\).
Тогава очевидно \(\forall n ((f \circ g)(n) \in f(n) \; \& \; \overline{\overline{(f \circ g)(n)}} = \overline{\overline{\mathcal{P}(n)}} \; \& \; (f \circ g)(n) \subseteq A)\).
Нека
\begin{align*}
    G(x, y) \leftrightharpoons \\
        (Funct(x) \; \& \; (\exists n \in \omega)(Dom(x) = n \; \& \; y = (f \circ g)(n) \setminus (\cup Range(x))) \\
        \lor \\
        (\lnot (Funct(x) \; \& \; (\exists n \in \omega)(Dom(x) = n)) \; \& \; y = \emptyset)
\end{align*}

Очевидно \(G\) задава формулна операция.
Тогава според Теоремата за трансфинитна рекурсия има при това единствена формулна операция \(F\) такава, че \(\forall \alpha (F(\alpha) = G(F_{\restriction\alpha}))\).
Тогава нека \(h = F_{\restriction\omega}\).
Тогава очевидно \(\forall n (h(n) = (f \circ g)(n) \setminus (\cup Range(h_{\restriction n})))\).
\\
\vspace{1mm}
\\
По индукция ще докажем
\(\forall n (\overline{\overline{\cup Range(h_{\restriction n}) }} \leq \overline{\overline{\mathcal{P}(n) \setminus \{\emptyset\}}} \; \& \; h(n) \neq \emptyset)\).
\\
\vspace{1mm}
\\
База: \(0 = \emptyset\) и значи \(Range(h_{\restriction 0}) = Range(\emptyset) = \emptyset\). Както видяхме \\
\(\mathcal{P}(0) \setminus \{\emptyset\} = \emptyset\).
Следователно \(\overline{\overline{\cup Range(h_{\restriction 0}) }} \leq \overline{\overline{\mathcal{P}(0) \setminus \{\emptyset\}}}\). \\
Имаме и \(h(0) = (f \circ g)(0) \setminus (\cup Range(h_{\restriction 0}))
= (f \circ g)(0) \setminus \emptyset = (f \circ g)(0)\). \\
Но \(\overline{\overline{(f \circ g)(0)}} = \overline{\overline{\mathcal{P}(0)}}
= \overline{\overline{\{\emptyset\}}} = \overline{\overline{\{0\}}}
= \overline{\overline{1}}\) и така \(\exists e(h(0) = \{e\})\). \\
Следователно \(h(0) \neq \emptyset\).
\\
\vspace{1mm}
\\
Индукционна хипотеза: Нека \(n \in \omega\) е такова, че \\
\(\overline{\overline{\cup Range(h_{\restriction n}) }} \leq \overline{\overline{\mathcal{P}(n) \setminus \{\emptyset\}}} \; \& \; h(n) \neq \emptyset\).
\\
\vspace{1mm}
\\
Индукционна стъпка:
\(Range(h_{\restriction s(n)}) = Range(h_{\restriction n}) \cup \{h(n)\}\). \\
\(h(n) = (f \circ g)(n) \setminus (\cup Range(h_{\restriction n}))\).
Също така \\
\(\mathcal{P}(s(n)) = \mathcal{P}(n \cup \{n\}) = \mathcal{P}(n) \cup \{n \cup p \; | \; p \in  \mathcal{P}(n)\}\).
И тогава  \\
\(\mathcal{P}(s(n)) \setminus \{\emptyset\}
= (\mathcal{P}(n) \setminus \emptyset) \cup \{n \cup p \; | \; p \in  \mathcal{P}(n)\}\). \\
От И.Х. \(\overline{\overline{\cup Range(h_{\restriction n}) }} \leq \overline{\overline{\mathcal{P}(n) \setminus \{\emptyset\}}}\). \\
Нека тогава
\(u : \cup Range(h_{\restriction n}) \injection \mathcal{P}(n) \setminus \{\emptyset\}\).
Имаме още \(\overline{\overline{(f \circ g)(n)}} = \overline{\overline{\mathcal{P}(n)}}\).
Нека тогава \(v : (f \circ g)(n) \bijection \mathcal{P}(n)\).
Но \(h(n) \subseteq (f \circ g)(n)\)
и значи \\
\(v_{\restriction h(n)} : h(n) \injection \mathcal{P}(n)\).
Очевидно \(r : \mathcal{P}(n) \to \{n \cup p \; | \; p \in  \mathcal{P}(n)\}\),
за която \(r(p) = n \cup p\) е биекция.
Следователно \(v_{\restriction h(n)} \circ r : h(n) \injection \{n \cup p \; | \; p \in  \mathcal{P}(n)\}\).
В сила е \(\cup Range(h_{\restriction s(n)}) = (\cup Range(h_{\restriction n})) \cup h(n)\). Но \\
\(h(n) = (f \circ g)(n) \setminus (\cup Range(h_{\restriction n}))\).
Следователно \((\cup Range(h_{\restriction n})) \cap h(n) = \emptyset\).
Очевидно и \(P(n) \cap \{n \cup p \; | \; p \in  \mathcal{P}(n)\} = \emptyset\).
Тогава е приложима Лемата за разделените инекции.
От нея следва \\
\(u \cup (v_{\restriction h(n)} \circ r) : \cup Range(h_{\restriction s(n)}) \injection \mathcal{P}(s(n)) \setminus \{\emptyset\}\). \\
Следователно \(\overline{\overline{\cup Range(h_{\restriction s(n)})}} \leq \overline{\overline{\mathcal{P}(s(n)) \setminus \{\emptyset\}}}\). \\
В сила са:
\begin{align*}
    h(s(n)) = (f \circ g)(s(n)) \setminus (\cup Range(h_{\restriction s(n)}) \\
    \overline{\overline{(f \circ g)(s(n))}} = \overline{\overline{\mathcal{P}(s(n))}} \; \& \; Fin((f \circ g)(s(n))) \\
    \overline{\overline{\cup Range(h_{\restriction s(n)})}} \leq \overline{\overline{\mathcal{P}(s(n)) \setminus \{\emptyset\}}}
\end{align*}
Да допуснем, че \(h(s(n)) = \emptyset\).
Тогава получаваме \(\overline{\overline{\cup Range(h_{\restriction s(n)})}} = \overline{\overline{\mathcal{P}(s(n))}}\), което е Абсурд!
Следователно \(h(s(n)) \neq \emptyset\).
\\
\vspace{1mm}
\\
Заключение:
\(\forall n (\overline{\overline{\cup Range(h_{\restriction n}) }} \leq \overline{\overline{\mathcal{P}(n) \setminus \{\emptyset\}}} \; \& \; h(n) \neq \emptyset)\).
\\
\vspace{1mm}
\\
Да допуснем, че \(\exists n \exists m (n \neq m \; \& \; h(n) \cap h(m) \neq \emptyset)\).
Нека тогава \(n\) и \(m\) са такива, че
\(n \neq m \; \& \; h(n) \cap h(m) \neq \emptyset\).
Тогава Б.О.О. \(m < n\).
Но \(h(n) = (f \circ g)(n) \setminus (\cup Range(h_{\restriction n}))\)
и \(h(m) \subseteq \cup Range(h_{\restriction n})\),
следователно \(h(n) \cap h(m) = \emptyset\).
Така \(h(n) \cap h(m) \neq \emptyset\) и \(h(n) \cap h(m) = \emptyset\),
което е Абсурд! Следователно
\(\forall n \forall m (n \neq m \iff h(n) \cap h(m) \neq \emptyset)\).
Така директно следва \(h : \omega \bijection Range(h)\).
Понеже \(\forall n ((f \circ g)(n) \subseteq A)\) и от до тук доказаното получаваме
\(\forall n (h(n) \subseteq A \; \& \; h(n) \neq \emptyset)\).
Следователно \(Range(h)\) е изброимо множество от непразни множества.
Следователно от ACC \(Range(h)\) има функция на избора.
Нека тогава \(t\) е функция на избора за \(Range(h)\).
Така \(\forall n ((h \circ t)(n) \in h(n) \; \& \; (h \circ t)(n) \in A)\).
Обаче \(Range(h)\) е множество от непресичащи се множества,
от където \((h \circ t) : \omega \bijection Range(h \circ t)\)
и \(Range(h \circ t) \subseteq A\).
Нека \(M = Range(h \circ t)\).
Нека \(y = h \circ t\).
Нека \(z = (y^{-1} \circ s_{\restriction \omega}) \circ y\).
Очевидно \(z : M \bijection M \setminus \{y(0)\}\).
Тогава прилагаме Лемата за разделените инекции
и получаваме \(z \cup id_{\restriction A \setminus M} : A \bijection A \setminus \{y(0)\}\).
Така от \(A \setminus \{y(0)\} \subseteq A\) и \(y(0) \in M \subseteq A\)
следва \(A \setminus \{y(0)\} \subset A\) и \(\overline{\overline{A}} = \overline{\overline{A \setminus \{y(0)\}}}\).
\(\qed\)

\section*{Задача 9. Да се докаже}
\begin{align*}
\forall A \forall a \forall b (a \notin A \; \& \; b \notin \mathcal{P}(A) \; \& \; \overline{\overline{A}} = \overline{\overline{A \cup \{a\}}}
\implies \overline{\overline{\mathcal{P}(A)}} = \overline{\overline{\mathcal{P}(A) \cup \{b\}}})
\end{align*}

\subsection*{Решение:}
Нека \(A\), \(a\) и \(b\) са такива, че
\(a \notin A \; \& \; b \notin \mathcal{P}(A) \; \& \; \overline{\overline{A}} = \overline{\overline{A \cup \{a\}}}\).
Понеже \(id_{\mathcal{P}(A)} : \mathcal{P}(A) \bijection \mathcal{P}(A)\),
то \(id_{\mathcal{P}(A)} : \mathcal{P}(A) \injection \mathcal{P}(A) \cup \{b\}\).
Следователно \(\overline{\overline{\mathcal{P}(A)}} \leq \overline{\overline{\mathcal{P}(A) \cup \{b\}}}\).
Имаме \(\overline{\overline{A}} = \overline{\overline{A \cup \{a\}}}\)
тогава нека \(f : A \cup \{a\} \bijection A\).
Нека \(g = f_{\restriction A}\) тогава \(g : A \bijection A \setminus \{f(a)\}\).
Нека тогава \(h \; : \; \mathcal{P}(A) \to \mathcal{P}(A)\) и 
\((\forall S \in \mathcal{P}(A))(h(S) = g[S])\).
Ще покажем, че \(h\) е инекция.
\\
Нека \(X \in \mathcal{P}(A)\) и \(Y \in \mathcal{P}(A)\)
и нека са такива, че \(h(X) = h(Y)\).
Тогава \(g[X] = g[Y]\) и тогава понеже \(g : A \bijection A \setminus \{f(a)\}\),
то \(X = g^{-1}[g[X]] = g^{-1}[g[Y]] = Y\). Следователно 
\((\forall X \in \mathcal{P}(A))(\forall Y \in \mathcal{P}(A))(h(X) = h(Y) \implies X = Y)\). Следователно \(h : \mathcal{P}(A) \injection \mathcal{P}(A)\).
Понеже \(f : A \cup \{a\} \bijection A\), то \(f(a) \in A\) и значи \(\{f(a)\} \in \mathcal{P}(A)\). От \(g : A \bijection A \setminus \{f(a)\}\),
следва че \((\forall S \in \mathcal{P}(A))(h(S) = g[S] \subseteq A \setminus \{f(a)\})\).
Следователно \((\forall S \in \mathcal{P}(A))(f(a) \notin h(S))\).
Следователно \(\{f(a)\} \notin Range(h)\).
Тогава \(Funct(\{<b, \{f(a)\}>\})\) и \(b \notin Dom(h)\) и \(\{f(a)\} \notin Range(h)\)
и така по Лемата за разделените инекции \(h \cup \{<b, \{f(a)\}>\} : \mathcal{P}(A) \cup \{b\} \injection Range(h) \cup \{\{f(a)\}\}\). Също така имаме
\((\forall S \in \mathcal{P}(A))(h(S) = g[S] \subseteq A \setminus \{f(a)\})\)
и значи \(Range(h) \subseteq \mathcal{P}(A)\). Но \(f(a) \in A\) и така
\(Range(h) \cup \{\{f(a)\}\} \subseteq \mathcal{P}(A) \cup \mathcal{P}(A) = \mathcal{P}(A)\).
Следователно \(h \cup \{<b, \{f(a)\}>\} : \mathcal{P}(A) \cup \{b\} \injection \mathcal{P}(A)\). Така \(\overline{\overline{\mathcal{P}(A) \cup \{b\}}} \leq \overline{\overline{\mathcal{P}(A)}}\) и от Теоремата на Кантор-Шрьодер-Бернщайн следва, че \(\overline{\overline{\mathcal{P}(A)}} = \overline{\overline{\mathcal{P}(A) \cup \{b\}}}\). \(\qed\)

\section*{Задача 12. В (ZFC) всяко множество от непразни крайни множества има минимална трансверзала}
\(\forall A ((\forall x \in A)(\exists n \in \omega)(\overline{\overline{x}} = \overline{\overline{s(n)}}) \implies \exists Y ((\forall x \in A)(Y \cap x \neq \emptyset) \; \& \; \forall Z ((Z \subseteq Y \; \& \; (\forall x \in A)(Z \cap x \neq \emptyset)) \implies Z = Y)))\)

\subsection*{Решение:}
Нека \(A\) е такова, че \((\forall x \in A)(\exists n \in \omega)(\overline{\overline{x}} = \overline{\overline{s(n)}})\).
Нека \\
\(T = \{Y \; | \; Y \in \mathcal{P}(\cup A) \; \& \; (\forall x \in A)(Y \cap x \neq \emptyset)\}\).
\(T\) е множество, защото отделяме от \(\mathcal{P}(\cup A)\) със свойството "е трансверзала за \(A\)". Също така \(<T, \supseteq_T>\) е частично наредено множество.
В (ZFC) е сила Лемата на Цорн, така че ще я приложим.
Първо в сила е следното \((\forall x \in A)(x \cap (\cup A) = x \neq \emptyset)\).
Следователно \(\cup A \in T\) и значи \(T \neq \emptyset\).
Нека \(C\) е произволна непразна верига в \(<T, \supseteq_T>\).
Тогава \(<C, \supseteq_T\cap \; (C \times C)>\) е линейно наредено множество.
Нека \(Y = \cap C\). Тогава \((\forall X \in C)(C \supseteq Y)\).
Понеже \(C \subseteq T \subseteq \mathcal{P}(\cup A)\), то \(Y \in \mathcal{P}(\cup A)\).
Да допуснем, че \(Y \notin T\). Тогава \\
\((\exists x \in A)(Y \cap x = \emptyset)\).
Нека тогава \(x \in A\) и \(Y \cap x = \emptyset\).
Тогава \\
\((\forall s \in x)(x \notin Y)\).
Тогава \((\forall s \in x)(\exists t \in C)(x \notin t)\).
Нека \\
\(\varphi(s, U, x, C) \leftrightharpoons (s \in x \; \& \; U = \{t \; | \; t \in C \; \& \; s \notin t \}) \; \lor \; (s \notin x \; \& \; U = \emptyset)\).
Очевидно \(\varphi\) е функционално свойство относно \(s\) при фиксирани параметри \(x\) и \(C\).
Тогава по аксиомната схема за замяната относно \(\varphi\) и \(x\) с параметри \(x\) и \(C\) има множество \(B\), такова че
\((\forall s \in x)(\exists U \in B)(\varphi(s, U, x, C))\).
Нека тогава \(D = \{U \; | \; U \in B \; \& \; (\exists s \in x)(\varphi(s, U, x, C))\}\). \(D\) е множество според аксиомната схема за отделянето.
Също така очевидно \(D = \{U \; | \; (\exists s \in x)(U = \{t \; | \; t \in C \; \& \; s \notin t \}) \}\) и \(\overline{\overline{D}} \leq \overline{\overline{x}}\).
Но \(x\) е крайно и непразно и значи \(D\) също. От \((\forall s \in x)(\exists t \in C)(x \notin t)\) следва, че \((\forall U \in D)(U \neq \emptyset)\).
Тогава нека \(f\) е функция на избора за \(D\) (правим краен избор).
Тогава очевидно \((\forall U \in D)(f(U) \in U \; \& \; (\exists s \in x)(f(U) \in C \; \& \; s \notin f(U)))\). Така \(\emptyset \neq Range(f) \subseteq C\) и от Лемата за образа на фунцкия с креан домейн имаме \(\overline{\overline{Range(f)}} \leq \overline{\overline{D}}\) значи \(Range(f)\) е крайно и е подмножество на линейно наредено множество. 
Тогава по Лемата за най-големия елемент \(Range(f)\) има най-голям елемент относно \(\supseteq_T \cap \; (C \times C)\).
Нека тогава \(t \in Range(f)\) и \((\forall v \in Range(f))(t \subseteq v)\).
Очевидно \(\cap Range(f) \subseteq t\).
Нека \(w \in t\). Нека \(r \in Range(f)\).
Тогава понеже \(t \subseteq r\), то \(w \in r\).
Така \((\forall v \in Range(f))(w \in v)\).
Следователно \(w \in \cap Range(f)\).
Следователно \(t \subseteq \cap Range(f)\).
Понеже \(t \in Range(f) \subseteq C\), то \(t \in C\).
Нека \(s \in x\), ако допуснем, че \(s \in t\),
то \((\forall v \in Range(f))(s \in v)\).
Но тогава \((\forall U \in D)(s \in f(U))\).
Нека \(U \in D\) и \(\varphi(s, U, x, C)\).
Тогава \((\forall v \in U)(s \notin v)\).
В частност \(s \notin f(U)\). Но това е Абсурд!
Следователно \((\forall s \in x)(s \notin t)\).
Следователно \(t \cap x = \emptyset\).
Но тогава \(t \notin C\), което е Абсурд!
Следователно \((\forall c \in C)(Y \supseteq_T c) \; \& \; Y \in T\).
Така всяка непразна верига в \(<T, \supseteq_T>\) има горна граница в \(T\).
Следователно по Лемата на Цорн в \(T\) има максимален елемент относно \(\supseteq_T>\).
Нека тогава \(V \in T\) и \(V\) е максимален относно \(\supseteq_T>\).
Тогава \(V\) е минимален относно \(\subseteq_T\).
Нека \(Z \subseteq V\) и \((\forall a \in A)(Z \cap a \neq \emptyset)\).
Тогава понеже \(V \in \mathcal{P}(\cup A)\), то \(Z \in \mathcal{P}(\cup A)\).
Но тогава \(Z \in T\). Понеже \(V\) е минимален относно \(\subseteq_T\),
то \(Z \not\subset V\) и значи \(Z = V\). \(\qed\)

\section*{Задача 13.}
Нека \(\varphi(A, B, C) \leftrightharpoons \\
A \subseteq B \; \& \; \forall f((f : C \to B) \implies (\cap Range(f) \in B \; \& \; \cup Range(f) \in B))\). \\
Тогава
\(\forall C \forall A ((A \neq \emptyset \; \& \; C \neq \emptyset) \implies \exists! B (\varphi(A, B, C) \; \& \; \forall D (\varphi(A, D, C) \implies B \subseteq D)))\).

\subsection*{Решение:}
Нека \(A\) и \(C\) са непразни множества.

\subsubsection*{Наблюдение 1:}
\((\forall a \in A)(a \subseteq \cup A) \implies (\forall a \in A)(a \in \mathcal{P}(\cup A)) \implies A \subseteq \mathcal{P}(\cup A) \)
следователно \(A \in \mathcal{P}(\mathcal{P}(\cup A))\).

\subsubsection*{Наблюдение 2:}
Нека \(X \in \mathcal{P}(\mathcal{P}(\cup A))\).
Нека \(f : C \to X\). Тогава
\(Range(f) \subseteq X \subseteq \mathcal{P}(\cup A)\).
Следователно \(Range(f) \in \mathcal{P}(\mathcal{P}(\cup A))\).
Но тогава очевидно \\
\(\cap Range(f) \in \mathcal{P}(\cup A)\) и
\(\cup Range(f) \in \mathcal{P}(\cup A)\).
Тогава по аксиомната схема за отделянето следните две са множества:
\begin{align*}
\{U \; | \; U \in \mathcal{P}(\cup A) \; \& \; \exists g ((g : C \to X) \; \& \; U = \cap Range(g))\} \\
\{U \; | \; U \in \mathcal{P}(\cup A) \; \& \; \exists g ((g : C \to X) \; \& \; U = \cup Range(g))\}
\end{align*}
Нека тогава \(N = \{U \; | \; U \in \mathcal{P}(\cup A) \; \& \; \exists g ((g : C \to X) \; \& \; U = \cup Range(g))\}\) и
\(S = \{U \; | \; U \in \mathcal{P}(\cup A) \; \& \; \exists g ((g : C \to X) \; \& \; U = \cap Range(g))\}\).
Тогава очевидно \(N \subseteq \mathcal{P}(\cup A)\) и \(S \subseteq \mathcal{P}(\cup A)\).
Следователно \(N \in \mathcal{P}(\mathcal{P}(\cup A))\) и
\(S \in \mathcal{P}(\mathcal{P}(\cup A))\).
Следователно \(\cup\{X, N, S\} \in \mathcal{P}(\mathcal{P}(\cup A))\).

\subsubsection*{Наблюдение 3:}
По аксиомната схема за отделянето \\
\(\{T \; | \; T \in \mathcal{P}(\mathcal{P}(\cup A)) \; \& \; A \subseteq T \}\) е множество. \\
Нека  \(S = \{T \; | \; T \in \mathcal{P}(\mathcal{P}(\cup A)) \; \& \; A \subseteq T \}\). Очевидно \(S \subseteq \mathcal{P}(\mathcal{P}(\cup A))\). \\
От Наблюдение 1 
\(A \in \mathcal{P}(\mathcal{P}(\cup A))\), следователно \\
\(A \subseteq \mathcal{P}(\cup A)\) и \(\mathcal{P}(\cup A) \in \mathcal{P}(\mathcal{P}(\cup A))\).
Следователно \(\mathcal{P}(\cup A) \in S\). \\
Нека \(Y \subseteq S\) е произволно и \(Y \neq \emptyset\).
Понеже \((\forall Z \in Y)(Z \in S)\), \\
то \((\forall Z \in Y)(A \subseteq Z \; \& \; Z \in \mathcal{P}(\mathcal{P}(\cup A)))\).
Следователно \\
\(A \subseteq \cap Y \; \& \; \cap Y \in \mathcal{P}(\mathcal{P}(\cup A))\).
Следователно \(\cap Y \in S\). \\
Следователно \((\forall Y \in \mathcal{P}(S) \setminus \{\emptyset\})(\cap Y \in S)\).

\subsubsection*{Същинско решение:}
Нека \(P = \mathcal{P}(\mathcal{P}(\cup A)))\)
и нека \(S = \{T \; | \; T \in P \; \& \; A \subseteq T \}\). \\
Дефинираме три оператора.
Нека \(h_\cap : P \to P\), \(h_\cup : P \to P\) и \(h : S \to S\) и
\begin{align*}
    (\forall X \in P)(h_\cap(X) = \{U \; | \; U \in \mathcal{P}(\cup A) \; \& \; \exists g ((g : C \to X) \; \& \; U = \cap Range(g))\}) \\
    \& \\
    (\forall X \in P)(h_\cup(X) = \{U \; | \; U \in \mathcal{P}(\cup A) \; \& \; \exists g ((g : C \to X) \; \& \; U = \cup Range(g))\}) \\
    \& \\
    (\forall X \in S)(h(X) = \cup \{X,  h_\cap(X), h_\cup(X)\}).
\end{align*}

От Наблюдение 2 е ясно, че \(h_\cap\) и \(h_\cup\) са коректно дефинирани. \\
Нека \(X \in S\). Тогава \(X \subseteq \cup \{X,  h_\cap(X), h_\cup(X)\}\).
Понеже \(X \in S\), то \(X \in P\) и \(A \subseteq X\).
Следователно \(\cup \{X,  h_\cap(X), h_\cup(X)\} \in P\) и \(A \subseteq \cup \{X,  h_\cap(X), h_\cup(X)\}\). Следователно \(\cup \{X,  h_\cap(X), h_\cup(X)\} \in S\).
Тогава очевидно \(h : S \to S\). \\
Нека \(Y \in S\) и \(Z \in S\) и \(Y \subseteq Z\).
От \(Y \subseteq Z\) следва \(\tensor[^C]{Y}{} \subseteq \tensor[^C]{Z}{}\).
Тогава очевидно \(h_\cap(Y) \subseteq h_\cap(Z)\) и \(h_\cup(Y) \subseteq h_\cup(Z)\).
Следователно очевидно \(h(Y) \subseteq h(Z)\). Следователно \(h\) е монотонен в \(S\).
Но тогава от Наблюдение 3 следва, че е приложима Задача 3.
Нека тогава \(B\) е най-малката неподвижна точка за \(h\).
Ще покажем, че \(\varphi(A, B, C)\) е в сила.
Понеже \(B\) е неподвижна точка за \(h\), то \(B \in S\) и значи \(A \subseteq B\).
Нека \(f\) е произволно, такова че \(f : C \to B\).
Нека \(I = \cap Range(f)\) и \(U = \cup Range(f)\).
От Наблюдение 2 следва, че \(I \in \mathcal{P}(\cup A)\) и \(U \in \mathcal{P}(\cup A)\).
Така \(I \in h_\cap(B)\) и \(U \in h_\cup(B)\).
Но \(B = h(B) = \cup\{B, h_\cap(B), h_\cup(B)\}\),
следователно \(h_\cap(B) \subseteq B\) и \(h_\cup(B) \subseteq B\).
Значи \(I \in B\) и \(U \in B\).
Но \(f\) беше произволно, следователно \(\varphi(A, B, C)\).
Нека \(D\) е произволно и такова, че \(\varphi(A, D, C)\).
Тогава \(A \subseteq D\). Но \(A \subseteq \mathcal{P}(\cup A)\)
и значи \(\emptyset \neq A \subseteq (D \cap \mathcal{P}(\cup A))\).
Нека \(T = D \cap \mathcal{P}(\cup A)\). Тогава \(A \subseteq T\)
и \(T \subseteq D\) и \(T \subseteq \mathcal{P}(\cup A)\).
Следователно \(T \in P\) и \(A \subseteq T\).
Следователно \(T \in S\). Понеже \(T \subseteq T\), то \(T \subseteq h(T)\).
Нека \(R \in h_\cap(T)\) и \(R\) е произволно.
Тогава \(\exists g ((g : C \to T) \; \& \; R = \cap Range(g))\).
Нека тогава \(g\) е такова, че \(g : C \to T\) и \(R = \cap Range(g)\).
Понеже \(g : C \to T\) и \(T \subseteq D\), то \(g : C \to D\).
Но тогава от \(\varphi(A, D, C)\), следва че \(R \in D\).
Но \(R \in h_\cap(T)\) и значи \(R \in \mathcal{P}(\cup A)\).
Така \(R \in D\) и \(R \in \mathcal{P}(\cup A)\), 
следователно \(R \in T\). Но понеже \(R\) беше произволно,
то \(h_\cap(T) \subseteq T\).
По аналогични разсъждения \(h_\cup(T) \subseteq T\).
Но също така \(T \subseteq T\) и така \(h(T) \subseteq T\).
Следователно \(h(T) = T\) и \(T \in S\).
Но \(B\) е най-малката неподвижна точка за \(h\),
следователно \(B \subseteq T\).
Но \(T \subseteq D\), следователно \(B \subseteq D\).
Понеже \(D\) е произволно, то следва че
\(\forall D (\varphi(A, D, C) \implies B \subseteq D)\).
Нека \(V\) е такова, че \(\varphi(A, V, C)\) и
\(\forall D (\varphi(A, D, C) \implies V \subseteq D)\).
Тогава понеже \(\varphi(A, B, C)\), то \(V \subseteq B\).
Но от \(\varphi(A, V, C)\) и доказаното следва, че \(B \subseteq V\).
Следователно \(V = B\). Следователно \(B\) е единствено със свойството
\(\varphi(A, B, C)\) и \(\forall D (\varphi(A, D, C) \implies B \subseteq D)\). \(\qed\)

\section*{Задача 14.}
Нека \(<A, \leq_1>\) е добре наредено множество и \(\forall n (\overline{\overline{A}} \neq \overline{\overline{n}})\). Тогава има добра наредба \(\leq_2\) в \(A\),
такава че \(<A, \leq_1>\) и \(<A, \leq_2>\) не са изоморфни.

\subsection*{Решение:}
От Лемата за добрата наредба и ординала има единствен ординал \(\alpha\),
такъв че \\
\(<A, \leq_1> \; \cong \; <\alpha, \leq_\alpha>\).
Нека тогава \(\alpha\) е този ординал.
Нека тогава \\
\(g : A \bijection \alpha\) е единствения изоморфизъм между \(<A, \leq_1>\) и \(<\alpha, \leq_\alpha>\).
Ако допуснем, че \(\alpha < \omega\), то \(\alpha \in \omega\)
и \(\overline{\overline{A}} \neq \overline{\overline{\alpha}}\),
което е Абсурд. Следователно \(\omega \leq \alpha\).
Тогава \(f(\beta) = \begin{cases}
 0 & , \; \beta = \alpha \\
 \beta & , \; \omega \leq \beta < \alpha \\
 s(\beta) & , \; \beta < \omega
\end{cases}\)
\quad е биекция от \(\alpha\) към \(s(\alpha)\).
Но \(\alpha\) е собствен начален сегмент на \(s(\alpha)\)
относно наредбата на \(s(\alpha)\) по свойствата на ординалите.
Следователно \(f\) не запазва наредбата, защото ако я запазваше
\(f\) щеше да е изоморфизъм между \(<\alpha, \leq_\alpha>\) и \(<s(\alpha), \leq_{s(\alpha)}>\), което е невъзможно. 
Очевидно \(g \circ f : A \bijection s(\alpha)\).
В \(A\) въвеждаме релацията \(\leq_2\) по следния начин
\(a_1 \leq_2 a_2 \iff <a_1, a_2> \; \in A \times A \; \& \; (g \circ f)(a_1) \leq (g \circ f)(a_2)\).
Очевидно \(\leq_2\) е добра наредба в \(A\) пренесена от \(<s(\alpha), \leq_{s(\alpha)}>\).
Тоест \(<A, \leq_2> \; \cong \; <s(\alpha), \leq_{s(\alpha)}>\).
Ако допуснем, че \(<A, \leq_1> \; \cong \; <A, \leq_2>\),
то ще получим, че \(<\alpha, \leq_\alpha> \; \cong \; <s(\alpha), \leq_{s(\alpha)}>\),
защото свойството изоморфизъм на добри наредби е симетрично и транзитивно,
което ще доведе до противоречие.
Следователно \(<A, \leq_1>\) и \(<A, \leq_2>\) не са изоморфни. \(\qed\)



\section*{Задача 15. \(\forall A (\overline{\overline{A}} < \overline{\overline{A}} + \overline{\overline{\mathcal{H}(A)}})\)}

\subsection*{Решение:}
Нека \(B = (\{0\} \times A) \cup (\{1\} \times \mathcal{H}(A)) \).
Тогава \(\overline{\overline{B}} = \overline{\overline{A}} + \overline{\overline{\mathcal{H}(A)}}\).
\\
Нека \(zero \; : \; A \to \{0\} \times A\) е такава, че
\((\forall a \in A)(zero(a) = <0, a>)\).
\\
Тогава очевидно \(zero : A \injection B\)
и значи \(\overline{\overline{A}} \leq \overline{\overline{B}}\).
Да допуснем, че \(\overline{\overline{A}} = \overline{\overline{B}}\).
\\
Нека тогава \(f : B \bijection A\).
Тогава \(f_{\restriction \{1\} \times \mathcal{H}(A)} :
\{1\} \times \mathcal{H}(A) \injection A\).
\\
Нека \(one \; : \; \mathcal{H}(A) \to \{1\} \times \mathcal{H}(A)\) е такава, че
\\
\((\forall x \in \mathcal{H}(A))(one(x) = <1, x>)\).
Очевидно \(one : \mathcal{H}(A) \bijection \{1\} \times \mathcal{H}(A)\).
Следователно \(one \circ f : \mathcal{H}(A) \injection A\).
Така \(\overline{\overline{\mathcal{H}(A)}} \leq \overline{\overline{A}}\), което е Абсурд!
Следователно \(\forall A (\overline{\overline{A}} < \overline{\overline{A}} + \overline{\overline{\mathcal{H}(A)}})\). \(\qed\)


\end{document}
