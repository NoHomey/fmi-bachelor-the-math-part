\documentclass[a4paper, 12pt, oneside]{article}
\usepackage[left=3cm,right=3cm,top=1cm,bottom=2cm]{geometry}
\usepackage{amsmath,amsthm}
\usepackage{amssymb}
\usepackage{lipsum}
\usepackage{stmaryrd}
\usepackage[T1,T2A]{fontenc}
\usepackage[utf8]{inputenc}
\usepackage[bulgarian]{babel}
\usepackage[normalem]{ulem}

\setlength{\parindent}{0mm}

\title{Лексикографското произведение на фундирани множества е фундирано множество}
\author{Иво Стратев}

\begin{document}
\maketitle
\section*{Твърдение:}
Нека \((A, <_A)\) е фундирано множество и нека \((B, <_B)\) е фундирано множество.
Тогава \((A \times B, <_{lex A, B})\) е фундирано множество, където \\
\((\forall (a, b) \in A \times B)(\forall (a', b') \in A \times B)[(a, b) <_{lex A, B} (a', b') \iff \\
a <_A a' \lor (a = a' \land b <_B b')]\)
\section*{Доказателство:} 
Нека \((A, <_A)\) е фундирано множество и нека \((B, <_B)\) е фундирано множество.
Нека \((\forall (a, b) \in A \times B)(\forall (a', b') \in A \times B)[(a, b) <_{lex A, B} (a', b') \iff \\
a <_A a' \lor (a = a' \land b <_B b')]\) \\
Ще докажем, че в \((A \times B, <_{lex A, B})\) няма безкрайни спускания,
което е еквивалентно на това множеството \((A \times B, <_{lex A, B})\) да е фундирано. \\
Допускаме, че в \((A \times B, <_{lex A, B})\) има безкрайно спускане. Тогава е вярно,
че съществува редица \(p_0, p_1, \dots\) с елементи от \(A \times B\), таква че \(p_0 >_{lex A, B} p_1  >_{lex A, B} \dots\).
Нека тогава \((a_0, b_0), (a_1, b_1), \dots\) с елементи от \(A \times B\) е такава, че \((a_0, b_0) >_{lex A, B} (a_1, b_1) >_{lex A, B} \dots\).
Тогава са възможни няколко случаи:
\subsection*{Случай 1:}
\((\forall i \in \mathbb{N})[a_i >_A a_{i + 1}]\). Tогава \(a_0 >_A a_1 >_A \dots\) е безкрайно спускане в \((A, <_A)\),
но това е абсурд, понеже \((A, <_A)\) е фундирано.
\subsection*{Случай 2:}
\((\exists I \subseteq \mathbb{N})[\overline{\overline{I}} = \overline{\overline{\omega}} \land (\forall i_1 \in I)(\forall i_2 \in I)[i_1 <_{\mathbb{N}} i_2 \implies a_{i_1} = a_{i_2} \land b_{i_1} >_B b_{i_2}]]\) \\
Нека тогава \(I \subseteq \mathbb{N}\) и \(\overline{\overline{I}} = \overline{\overline{\omega}} \land (\forall i_1 \in I)(\forall i_2 \in I)[i_1 <_{\mathbb{N}} i_2 \implies a_{i_1} = a_{i_2} \land b_{i_1} >_B b_{i_2}]\)
\subsection*{Случай 2.1:}
\((\exists J \subseteq I)(\exists a \in A)[\overline{\overline{J}} = \overline{\overline{\omega}} \land (\forall j \in J)[a_j = a] \land (\forall j_1 \in J)(\forall j_2 \in J)[j_1 <_{\mathbb{N}} j_2 \implies b_{j_1} >_B b_{j_2}]]\) \\
Нека тогава \(J \subseteq I\), нека \(a \in A\) и нека \\
\(\overline{\overline{J}} = \overline{\overline{\omega}} \land (\forall j \in J)[a_j = a] \land (\forall j_1 \in J)(\forall j_2 \in J)[j_1 <_{\mathbb{N}} j_2 \implies b_{j_1} >_B b_{j_2}]]\).
Тогава е вярно \((\forall j_1 \in J)(\forall j_2 \in J)[j_1 <_{\mathbb{N}} j_2 \implies b_{j_1} >_B b_{j_2}]\).
Понеже \(J \subseteq I \subseteq \mathbb{N}\) и \(\overline{\overline{J}} = \overline{\overline{\omega}}\),
то елементите на \(J\) могат да бъдат наредени в строго растяща редица \(j_0 <_{\mathbb{N}} j_1  <_{\mathbb{N}} \dots\).
Но тогава \(b_{j_0} >_B b_{j_1} >_B \dots\) е безкрайно спускане, което е абсурд, защото \((B, <_B)\) е фундирано множество.
\subsection*{Случай 2.2:}
\((\forall J \subseteq I)[(\exists a \in A)[(\forall j \in J)[a_j = a] \land (\forall j_1 \in J)(\forall j_2 \in J)[j_1 <_{\mathbb{N}} j_2 \implies b_{j_1} >_B b_{j_2}]] \implies (\exists n \in \mathbb{N})[\overline{\overline{J}} = \overline{\overline{\{0, 1, \dots, n - 1\}}}]]\) \\
Тогава въвеждаме следната релация \(R := \{(i_1, i_2) \in I \times I \; | \; a_{i_1} = a_{i_2}\}\). \\
\subsubsection*{\(R\) е рефлексивна}
В сила е \((\forall i \in I)[a_i = a_i] \implies (\forall i \in I)[(i, i) \in R]\), тоест \(R\) е рефлексивна.
\subsubsection*{\(R\) е симетрична}
Нека \((i_1, i_2) \in R\) тогава \(a_{i_1} = a_{i_2}\). Но понеже равенството е симетрично, то \(a_{i_2} = a_{i_1}\) и значи \((i_2, i_1) \in R\).
Следователно \((\forall (i_1, i_2) \in R)[(i_2, i_1) \in R]\), следователно \(R\) е симетрична.
\subsubsection*{\(R\) е транзитивна}
Нека \((i_1, i_2) \in R\) и нека \((i_2, i_3) \in R\).
Тогава \(a_{i_1} = a_{i_2} \land a_{i_2} = a_{i_3}\), но понеже равенството е транзитивна релация, то \(a_{i_1} = a_{i_3}\).
Следователно \((i_1, i_3) \in R\) и тогава е в сила \((\forall i_1 \in R)(\forall i_2 \in R)(\forall i_3 \in R)[(i_1, i_2) \in R \land (i_2, i_3) \in R \implies (i_1, i_3) \in R]\).
Тоест \(R\) е транзитивна. 
\subsubsection*{Заключение \(R\) е релация на еквивалентност.}
Тогава нека \(K := \{[i]_R \; | \; i \in I\}\).
\subsubsection*{Лема: \((\forall J \in K)(\exists n \in \mathbb{N})[\overline{\overline{J}} = \overline{\overline{\{0, 1, \dots, n - 1\}}}]\)}
Нека \(J \in K\) и нека \(j \in J\) тогава е в сила \\
\((\forall j' \in J)[(j, j') \in R] \implies (\forall j' \in J)[a_{j'} = a_j]\).
Понеже \(J \subseteq I\), то е в сила \((\forall j_1 \in J)(\forall j_2 \in J)[j_1 <_{\mathbb{N}} j_2 \implies b_{j_1} >_B b_{j_2}]\). Тогава е в сила \\
\((\exists n \in \mathbb{N})[\overline{\overline{J}} = \overline{\overline{\{0, 1, \dots, n - 1\}}}]\).
Нека тогава \(n \in \mathbb{N}\) и \(\overline{\overline{J}} = \overline{\overline{\{0, 1, \dots, n - 1\}}}\). \\
Тогава \((\forall J \in K)(n \in \mathbb{N})[\overline{\overline{J}} = \overline{\overline{\{0, 1, \dots, n - 1\}}}]\). \(\qed\) \\
Както знаем \(K\) е разбиване на \(I\) и доказахме, че всеки елемент на \(K\) е крайно множество.
Тогава \(K\) изброимо безкрайно иначе ще се окаже, че \(I\), което е изброимо безкрайно е обединение на краен брой крайни множества, тоест е крайно, което е абсурд.
Прилагаме аксиомата за избора за множеството \(I\) и получаваме функция \(f : \mathcal{P}(I)\setminus\emptyset \to I\),
за която \((\forall S \in \mathcal{P}(I)\setminus\emptyset)[f(S) \in S]\).
\(K\) е разбиване, тогава е в сила \((\forall J_1 \in K)(\forall J_2 \in K)[a_{f(J_1)} = a_{f(J_2)} \iff J_1 = J_2]\).
Нека тогава \(J := \{f(k) \; | \; k \in K\}\).
Така \((\forall j_1 \in J)(\forall j_2 \in J)[j_1 <_{\mathbb{N}} j_2 \implies (a_{j_1}, b_{j_1}) >_{lex A, B} (a_{j_2}, b_{j_2}) \land a_{j_1} \neq a_{j_2}] \implies
(\forall j_1 \in J)(\forall j_2 \in J)[j_1 <_{\mathbb{N}} j_2 \implies a_{j_1} >_A a_{j_2}]\).
Очевидно \(\overline{\overline{J}} = \overline{\overline{K}} = \overline{\overline{\omega}}\).
Тогава елементите на \(J\) могат да бъдат наредени в строго растяща редица \(j_0 <_{\mathbb{N}} j_1  <_{\mathbb{N}} \dots\).
Но тогава \(a_{j_0} >_A a_{j_1} >_A \dots\) е безкрайно спускане, което е абсурд, защото \((A, <_A)\) е фундирано множество.
\subsection*{Случай 3:}
Нека \(I := \{n \in \mathbb{N} \; | \; (\exists k \in \mathbb{N})[k <_{\mathbb{N}} n \land a_k = a_n]\}\) е крайно.
Понеже множеството \(I\) е крайно множество от естествени числа то има максимален елемент относно релацията \(<_{\mathbb{N}}\), която е линейна наредба.
Нека тогава \(i\) е този максимален елемент.
Да допуснем, че \((\exists i' \in \mathbb{N})[i' >_{\mathbb{N}} i \land a_{i' - 1} = a_{i'} \land b_{i' - 1} >_B b_{i'}]\).
Нека \(i' \in \mathbb{N}\) и \(i' >_{\mathbb{N}} i \land a_{i' - 1} = a_{i'} \land b_{i' - 1} >_B b_{i'}\).
Тогава \(i' \in I\) и \(i <_{\mathbb{N}} i'\) значи \(i\) не е максимален, което е противоречие. Тогава е в сила 
\((\forall n_1 \in \mathbb{N})(\forall n_2 \in \mathbb{N})[i \leq_{\mathbb{N}} n_1 \land n_1 <_{\mathbb{N}} n_2 \implies a_{n_1} >_A a_{n_2}]\).
Тогава редицата \((a_i, b_i), (a_{i + 1}, b_{i + 1}), \dots\) попада в \textbf{Случай 1} понеже сме премахнали само краен брой членове от оригиналната. \\
Разгледахме всички възможни случаи: когато нямаме повтарящи се първи елементи,
когато имаме изброимо много с два подслучая (изброимо дълга поредица и изборимо много крайни поредици)
и когато имаме само краен брой повторения и при всички достигнахме до противоречие. Няма други възможни случай.
Тогава не е вярно, че в \((A \times B, <_{lex A, B})\) има безкрайно спускане. 
Следователно в \((A \times B, <_{lex A, B})\) няма безкрайно спускане.
Следователно \((A \times B, <_{lex A, B})\) е фундирано.
Твърдението е доказано. \(\qed\)
\end{document}