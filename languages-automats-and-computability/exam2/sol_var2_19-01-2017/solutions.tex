\documentclass[12pt]{article}
    
\usepackage[left=3cm,right=3cm,top=1cm,bottom=2cm]{geometry}
\usepackage{amsmath,amsthm}
\usepackage{amssymb}
\usepackage{lipsum}
\usepackage{qtree}
\usepackage{tikz}
\usetikzlibrary{automata,positioning}
\usepackage[T1,T2A]{fontenc}
\usepackage[utf8]{inputenc}
\usepackage[T1]{fontenc}
\usepackage[bulgarian]{babel}
\usepackage[normalem]{ulem}
\newcommand{\Lang}{\mathcal{L}}
\newcommand{\N}{\mathbb{N}}
    
\setlength{\parindent}{0mm}
            
\title{Езици и автомати. Решения на теоретично контролно вариант 1 от 19.01.2017г.}
\author{Иво Стратев}

\begin{document}
\maketitle
\section{Задача 1.}

\subsection{Дайте дефиниция за контекстно-свободна граматика.}

Нека $G = (V, \; \Sigma, \; P, \; S)$, където: \\

\begin{itemize}
    \item $V$ е крайно множество от променливи (нетерминали)
    \item $\Sigma$ е азбука (крайно множество от терминали) и $\Sigma \cap V = \emptyset$
    \item $P \subseteq (V \cup \Sigma)^+ \times (V \cup \Sigma)^*$ е крайно множество от правила
    \item $S \in V$ е началната променлива
\end{itemize}

$G$ е контекстно-свободна граматика, ако $P \subseteq V \times (V \cup \Sigma)^*$

\subsection{Дефинирайте кога една дума $u \in (V \cup \Sigma)^*$ е изводима от думата $v \in (V \cup \Sigma)^*$ с граматиката $G \quad (v \overset{*}{\implies} u)$.}

Нека $\Lambda = (V \cup \Sigma)^*$ \\

Нека $R_{\underset{G}{\to}} \; = \; \left\{(w, \; w') \in \Lambda^2 \; | \; \exists \alpha, \; \beta, \; \beta', \; \gamma \in  \; \Lambda \; : \;
w = \alpha\beta\gamma \land \; w' = \alpha\beta'\gamma \land \exists \; \beta \to \beta' \in P\right\}$ \\\\

и нека $R_{\overset{n}{\underset{G}{\to}}} = \begin{cases}
    R_{\underset{G}{\to}} & , n = 0 \\
    \left\{(\alpha, \; \gamma) \in \Lambda^2\; | \; \exists \beta \in \Lambda, \; \exists m \in \N \; : \; m < n \land (\alpha, \; \beta), \; (\beta, \; \gamma) \in R_{\overset{m}{\underset{G}{\to}}}\right\} & , n > 0
\end{cases}$ \\\\

Тогава $R_{\overset{*}{\implies}} \; = \; \{(w, \; w) \; | \; w \in \Lambda\} \; \cup \; \displaystyle\bigcup_{n \in \N} R_{\overset{n}{\underset{G}{\to}}}$ \\\\

тоест $R_{\overset{*}{\implies}}$ е рефлексивното и транзитивното затваряне на $R_{\underset{G}{\to}}$.\\\\

$v \overset{*}{\implies} u \iff (v, \; u) \in R_{\overset{*}{\implies}}$

\subsection{$G$ е контекстно-свободна граматика определете езика $\Lang(G)$}

$\Lang(G) = \{w \in \Sigma^* \; | \; S \overset{*}{\implies} w\}$

\subsection{}

Нека $G = (\{S, \; A, \; B\}, \; \{a, \; b\}, \; \{S \to ABA \; | \; ab, \; A \to aA \; | \; a, \; B \to bBb \; | \; b\}, \; S)$

\subsubsection{Покажете, че думите $aaba$ и $abbba$ са изводими от $G$ и покажете синтактично дърво за тях.}

За $aaba$: \\

$S \implies (S \to ABA) \\\\
ABBA \implies (A \to aA) \\\\
aABA \implies (A \to a) \\\\
aaBA \implies (B \to b) \\\\
aabA \implies (A \to a) \\\\
aaba$  \\

\Tree [.S [.A [a [.A [ a ] ] ] ] [.B [ b ] ] [.A [ a ] ] ] \\\\

За $abbba$: \\

$S \implies (S \to ABA) \\\\
ABA \implies (A \to a) \\\\
aBA \implies (B \to bBb) \\\\
abBbA \implies (B \to b) \\\\
abbbA \implies (A \to a) \\\\
abbba$ \\

\Tree [.S [.A [ a ] ] [.B [ b [.B [ b ] ] b ] ] [.A [ a ] ] ] \\\\

\subsubsection{Вярно ли е, че езикът $\Lang(G) \cap \{a^{2n}b^{2k + 1}a \; | \; n, \; k \in \N\}$ е контекстно-свободен?}

Лесно се вижда, че $\Lang(G) = \{a^{n}b^{2k - 1}a^{m} \; | \; n, \; k, \; m \in \N\backslash\{0\}\}$ следователно\\

$\Lang(G) \cap \{a^{2n}b^{2k + 1}a \; | \; n, \; k \in \N\} = \{a^{2n}b^{2k - 1}a \; | \; n, \; k \in \N\backslash\{0\}\}$ \\

Една граматика, която описва $\{a^{2n}b^{2k - 1}a \; | \; n, \; k \in \N\backslash\{0\}\}$ е: \\

$(\{S, \; A, \; B\}, \; \{a, \; b\}, \; \{S \to AABa, \; A \to aAa \; | \; a, \; B \to bBb \; | \; b\}, \; S)$ \\

Следователно $\Lang(G) \cap \{a^{2n}b^{2k}a \; | \; n, \; k \in \N\}$ е контекстно-свободен.

\subsubsection{Вярно ли е, че езикът $\{a, \; b\}^* \backslash \Lang(G)$ е контекстно-свободен?}

Лесно се вижда, че $\Lang(G) = \Lang(a^+(bbb \; + \; b)^+a^+)$, тоест че $\Lang(G)$ всъщност е регулярен език. \\

Ние знаем, че допълнението на всеки регулярен език е регулярен език, тоест $\{a, \; b\}^* \backslash \Lang(G)$ е регулярен. \\

Също така знаем, че всеки регулярен език е контекстно-свободен език, тогава $\{a, \; b\}^* \backslash \Lang(G)$ е контекстно-свободен.

\section{Задача 2.}

Нека $G_1 = (V_1, \; \Sigma, \; P_1, \; S_1)$ и $G_2 = (V_2, \; \Sigma, \; P_2, \; S_2)$
са контекстно-свободни граматики, за които $V_1 \cap V_2 = \emptyset$. Опишете
контрукция за построяването на контекстно-свободна граматика $G = (V, \; \Sigma, \; P, \; S)$ с език

\subsection{$\Lang(G) = \Lang(G_1) \cup \Lang(G_2)$}

$S \notin V_1 \cup V_2, \; S \notin \Sigma, \; G = (\{S\} \cup V_1 \cup V_2, \; \Sigma, \{S \to S_1 \; | \; S_2\} \cup P_1 \cup P_2, \; S)$

\subsection{$\Lang(G) = \Lang(G_1)^*$}

$S \notin V_1, \; S \notin \Sigma, \; G = (\{S\} \cup V_1, \; \Sigma, \{S \to S_1S \: | \; \varepsilon\} \cup P_1, \; S)$

\section{Задача 3.}

Постройте граматика $G$, за която $\Lang(G) = \Lang(A)$ \\

за крайният автомат $A = (\{s, \; p, \; q\}, \; \{a, \; b\}, \; \delta, \; s, \; \{q\})$ за \\

$\delta(s, \; a) = p \\\\
\delta(s, \; b) = s \\\\
\delta(p, \; a) = p \\\\
\delta(p, \; b) = q \\\\
\delta(q, \; a) = p \\\\
\delta(q, \; b) = s$ \\

\begin{tikzpicture}[shorten >=1pt,node distance=2cm,on grid,auto]
    \node[state,initial] (s) {s};
    \node[state] (p) [right=of s] {p};
    \node[state, accepting] (q) [right=of p] {q};
    \path[->]
        (s) edge[bend left=30] node {$a$} (p)
        (s) edge[loop above] node {$b$} ()
        (p) edge[loop above] node {$a$} ()
        (p) edge[bend left=30] node {$b$} (q)
        (q) edge[bend left=30] node {$a$} (p)
        (q) edge[bend left=60] node {$b$} (s);
\end{tikzpicture}

\subsection{Общ вид на конструкцията}

Нека $A = (Q, \; \Sigma, \; \delta, \; s, \; F)$ е краен автомат. Тогава \\

$G = (Q, \; \Sigma, \; P, \; s)$ е регулярна граматика. \\

$P = \{q \to ap \; | \; a \in \Sigma, \; \delta(q, \; a) = p\} \cup \{q \to a \; | \; a \in \Sigma, \; \delta(q, \; a) = p \in F\} \cup \\
\{f \to \varepsilon \; | \; f \in F\} $ \\\\

Обратно на задачата:

$G = (\{s, \; p, \; q\}, \; \{a, \; b\}, \; \{s \to bs \; | \; ap, \; p \to ap \; | \; bq \; | \; b, \; q \to ap \; | \;  bs \; | \; \varepsilon \}, \; s)$ 

\section{Задача 4.}

Нека $G = (V, \; \Sigma, \; P, \; S)$ е контекстно-свободна граматика.
Дефинирайте стеков автомат $M$, завърващ с празен стек, за който $\Lang(M) = \Lang(G)$. \\

$M = (\{q\}, \; \Sigma, \; \Sigma \cup V, \; S, \; q, \; \Delta, \; \emptyset) \\\\
\forall A \in V \; \Delta(q, \; \varepsilon, \; A) = \{(q, \; \alpha) \; | \; A \to \alpha \in P\} \\\\
\forall a \in \Sigma \; \Delta(q, \; a, \; a) = \{(q, \; \varepsilon)\}$


\subsection{Дефинирайте стеков автомат $M$ с горното свойство за $G$ от Задача 1.}

$M = (\{q\}, \; \{a, \; b\}, \; \{a, \; b, \; S, \; A, \; B\}, \; S, \; q, \; \Delta, \; \emptyset) \\\\
\Delta(q, \; \varepsilon, \; S) = \{(q, \; ABA), \; (q, \; ab)\} \\\\
\Delta(q, \; \varepsilon, \; A) = \{(q, \; aA), \; (q, \; a)\} \\\\
\Delta(q, \; \varepsilon, \; B) = \{(q, \; bBb), \; (q, \; b)\} \\\\
\Delta(q, \; a, \; a) = \{(q, \; \varepsilon)\} \\\\
\Delta(q, \; b, \; b) = \{(q, \; \varepsilon)\}$

\section{Задача 5.}

Формулирайте Лемата за покачването (The Pumping Lemma) за контекстно-свободни езици. \\

Ако $L$ е контекстно-свободен език то \\

$\exists p \in \N \backslash \{0\} \; : \; \forall \alpha \in L \; : \; |\alpha| \geq p \; \exists x, \; y, \; u, \; v, \; z \; : \; \alpha = xyuvz \land |yv| \geq 1 \land |yuv| \leq p \; \land \\
\forall i \in \N \; xy^iuv^iz \in L$. \\\\

Контра позиция на лемата: Ако \\

$\forall p \in \N \backslash \{0\} \; : \; \exists \alpha \in L \; : \; |\alpha| \geq p \; \forall x, \; y, \; u, \; v, \; z \; : \; \alpha = xyuvz \land |yv| \geq 1 \land |yuv| \leq p \; \land \\
\exists i \in \N \; xy^iuv^iz \notin L$ то $L$ не е контекстно-свободен език.

\end{document}