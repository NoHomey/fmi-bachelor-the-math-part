\documentclass[12pt]{article}
    
\usepackage[left=3cm,right=3cm,top=1cm,bottom=2cm]{geometry}
\usepackage{amsmath,amsthm}
\usepackage{amssymb}
\usepackage{lipsum}
\usepackage[T1,T2A]{fontenc}
\usepackage[utf8]{inputenc}
\usepackage[T1]{fontenc}
\usepackage[bulgarian]{babel}
\usepackage[normalem]{ulem}
\usepackage{hyperref}
\hypersetup{
    colorlinks=true,
    linkcolor=blue,
    filecolor=magenta,      
    urlcolor=cyan,
}
 
\urlstyle{same}

\newcommand{\Lang}{\mathcal{L}}
\newcommand{\N}{\mathbb{N}}

\setlength{\parindent}{0mm}
        
\title{Езици и автомати}
\author{Иво Стратев}
        
\begin{document}
\maketitle

\section*{Регулярни езици}

\subsection*{Основни}

Нека $\Sigma$ е азбука. Тогава $\emptyset, \; \{\varepsilon\}, \; \forall \; a \in \Sigma \; \{a\}$ - са регурлярни езици

\subsection*{Операции}

\subsubsection*{Обединение}

Ако $L_1, \; L_2$ са регулярни, то и $L_1 \cup L_2 = \{\omega \; | \; \omega \in L_1 \; \lor \; \omega \in L_2 \}$ е регулярен.

\subsubsection*{Конкатенация}

Ако $L_1, \; L_2$ са регулярни, то и $L_1 \; . \; L_2 = \{\omega_1.\omega_2 \; | \; \omega_1 \in L_1 \; \land \; \omega_2 \in L_2\}$ е регулярен. \\\\

$L^0 = \{\varepsilon\}$ \\

$\forall n \in \N \; L^{n + 1} = L^n\;.\;L $

\subsubsection*{Звезда на Клини}

Ако $L$ е регулярен, то и $L^* = \displaystyle\bigcup_{n \in \N} L^n$ е регулярен.

\subsection*{Дефиниция}

Един език е регулярен, ако се получава от основните с помощта на краен брой прилагания на операциите: обединение, конкатенация и звезда на Клини. 

\section*{Регулярни изрази}

\subsection*{Базови}

Символите $\emptyset, \; \varepsilon, \; a$ са регулярни изрази $(\forall \; a \in \Sigma)$ \\

\subsection*{"Производни"}

Ако $r_1$ и $r_2$ са регулярни изрази, то и $r_1 + r_2, \; r_1 \; . \; r_2$ и $r_1^*$ също са регулярни изрази.

\subsection*{Език на регулярен израз}

$\Lang(\emptyset) = \emptyset$ \\

$\Lang(\varepsilon) = \{\varepsilon\}$ \\

$\forall a \in \Sigma \; \Lang(a) = \{a\}$ \\

Нека $r_1$ и $r_2$ са регулярни изрази и $L_1 = \Lang(r_1), \; L_2 = \Lang(r_2)$. Тогава: \\

$L_1 \cup L_2 = \Lang(r_1) \cup \Lang(r_2) = \Lang(r_1 + r_2)$ \\

$L_1 \; . \; L_2 = \Lang(r_1) \; . \; \Lang(r_2) = \Lang(r_1 \; . \; r_2)$ \\

$L_1^* = \Lang(r_1)^* = \Lang(r_1^*)$

\section*{Недетерминирани крайни автомати}

\subsection*{Определение}

Недермининар краен автомат представлява: \\

$N = (\Sigma, \; Q, \; s, \; \Delta, \; F)$, където: \\

$\Sigma$ - крайно множество от букви (азбука) \\

$Q$ - крайно множество от състояния \\

$s \in Q$ - начално състояние \\

$\Delta \; : \; Q \times \Sigma \to 2^Q $ - функция на преходите. Тя е тотална. Ако за някоя двойка $(q, \; a)$ няма преход в автомата, то $\Delta(q, \; a) = \emptyset$; \\

$F \subseteq Q$ - множество от финални състояния \\

\subsubsection*{Дефиниция за $\Delta^*$}

$\Delta^* \; : \; Q\times\Sigma^* \to 2^Q \; : \; \\\\
\Delta^*(q, \; \varepsilon) = {q} \\\\
\Delta^*(q, \; a\alpha) = \displaystyle\bigcup_{p \in \Delta(q, \; a)} \Delta^*(p, \; \alpha) $

\subsubsection*{Език на недетерминирания автомат $\Delta^*$, $\Lang(N)$}

$\Lang(N) = \{\omega \in \Sigma^* \; | \; \Delta^*(s, \; \omega) \in F\}$

\subsection*{"Действия" с недетерминирани автомати}

Нека $\Sigma$ e азбука. \\

Нека $N_1 = (\Sigma, \; Q_1, \; s_1, \; \Delta_1, \; F_1)$ и $N_2 = (\Sigma, \; Q_2, \; s_2, \; \Delta_2, \; F_2)$  - К.Н.А и $Q_1 \cap Q_2 = \emptyset$ \\

Нека $N = (\Sigma, \; Q, \; s, \; \Delta, \; F)$

\subsubsection*{Конкатенация, $\Lang(N) = \Lang(N_1) \; . \; \Lang(N_2)$}

$Q = Q_1 \cup Q_2$ \\

$s = s_1$ \\

$\Delta(q, \; a) = \begin{cases}
    \Delta_1(q, \; a), & q \in Q_1 \backslash F_1 \\
    \Delta_2(q, \; a), & q \in Q_2 \\
    \Delta_1(q, \; a) \cup \Delta_2(s_2, \; a), & q \in F_1
\end{cases}$ \\\\

$F = \begin{cases}
  F_1 \cup F_2, & s_2 \in F_2\\
  F_2, & s_2 \notin F_2  
\end{cases}$

\subsubsection*{Обединение, $\Lang(N) = \Lang(N_1) \cup \Lang(N_2)$}

$s \not \in Q_1 \cup Q_2$ \\\

$Q = Q_1 \cup Q_2 \cup \{s\}$ \\

$\Delta(q, \; a) = \begin{cases}
    \Delta_1(q, \; a), & q \in Q_1 \\
    \Delta_2(q, \; a), & q \in Q_2 \\
    \Delta_1(s_1, \; a) \cup \Delta_2(s_2, \; a), & q = s
\end{cases}$ \\\\

$F = \begin{cases}
  F_1 \cup F_2 \cup \{s\}, & s_1 \in F_1 \; \lor \; s_2 \in F_2 \\
  F_1 \cup F_2, & s_1 \notin F_1 \; \land \; s_2 \notin F_2 \\
\end{cases}$

\subsubsection*{Позитивна обвивка, $\Lang(N) = \Lang(N_1)^+$}

$Q = Q_1$ \\

$s = s_1$ \\

$\Delta(q, \; a) = \begin{cases}
    \Delta_1(q, \; a), & q \in Q_1 \backslash F_1 \\
    \Delta_1(q, \; a) \cup \Delta_1(s, \; a), & q \in F_1
\end{cases}$ \\\\

$F = F_1$

\subsubsection*{Звезда на Клини, $\Lang(N) = \Lang(N_1)^* = \{\varepsilon\} \cup \Lang(N_1)^+$}

$E = (\Sigma, \; \{s_\varepsilon\}, \; s_\varepsilon, \; \Delta_\varepsilon, \; \{s_\varepsilon\})$ \\

$s_\varepsilon \notin Q_1$ \\

$\Delta_\varepsilon(s_\varepsilon, \; \varepsilon) = \{s_\varepsilon\} \\\\
\forall a \in \Sigma \quad \Delta_\varepsilon(s_\varepsilon, \; a) = \emptyset $ \\

$Q = Q_1 \cup \{s_\varepsilon\}$ \\

$s = s_\varepsilon$ \\

$\Delta(q, \; a) = \begin{cases}
    \Delta_1(q, \; a), & q \in Q_1 \backslash F_1 \\
    \Delta_1(q, \; a) \cup \Delta_1(s, \; a), & q \in F_1 \\ 
    \Delta_1(s, \; a), & q = s_\varepsilon
\end{cases}$ \\\\

$F = F_1 \cup \{s_\varepsilon\}$

\section*{Лема за покачването (Pumping Lemma)}

Нека $L \subseteq \Sigma^*$ е безкраен регулярен език. \\

$\exists \; p \in \N^+ \; : \; \forall \; \omega \in L \; : \; |\omega| \geq p, \; \exists \; x, \; y, \; z \in \Sigma^* \; : \\\\
\omega = xyz \; \land \; |xy| \leq p \; \land \; |y| \geq 1 \; \land \; (\forall \; i \in \N \; xy^iz \in L) $

\subsection*{Следствие (Контрапозиция на лемата за покачването)}

Нека $L \subseteq \Sigma^*$ е безкраен език. \\

Ако е изпълнено, че: \\
 
$\forall \; p \in \N^+ \; \exists \; \omega \in L \; : \; |\omega| \geq p, \; \forall x, \; y, \; z \in \Sigma^* \; : \\\\
 \omega = xyz \; \land \; |xy| \leq p \; \land \; |y| \geq 1 \; \land \; (\exists \; i \in \N \; xy^iz \notin L) $. \\

Тогава $L$ не е регулярен.

\section*{Релация на Майхил-Нероуд и минимален автомат}

\subsection*{Релация на Майхил-Нероуд}

Нека $\Sigma$ e азбука и $L \subseteq \Sigma^*$ \\

$R_L = \{(x, \; y) \in \Sigma^*\times\Sigma^* \; | \; \forall z \in \Sigma^* \; xz \in L \iff yz \in L\} \\\\
\Sigma^*/R_L = \{[\alpha]_{R_L} \; | \; \alpha \in \Sigma^*\}$

\subsection*{Теорема за съществуване на минимален автомат}

Ако съществува $n \in \N \; : \; n = |\Sigma^*/R_L|$. Тогава $L$ е регулярен и съществува минимален Д.К.А
$M = (\Sigma, \; Q_M, \; s_M, \; \delta_M, \; F_M)$, такъв че $\Lang(M) = L$. \\

$Q_M = \Sigma^*/R_L$ \\

$s_M = [\varepsilon]_L$ \\

$\delta_M([\alpha]_{R_L}, \; a) = [\alpha.a]_L$ \\

$F_M = \{[\alpha]_{R_L} \; | \; \alpha \in L\} = \{q \in Q_M \; | \; q \subseteq L\}$

\subsubsection*{Лема $\delta_M$ е коректно дефинирана ($\delta_M$ задава функция)}
\subsubsection*{Лема $\delta_M^*([\varepsilon]_{R_L}, \; \omega) = [\omega]_{R_L}$}
\subsubsection*{Лема $\Lang(M) = L$}

\section*{Еквивалетност на състояния на автомат и минимален автомат}

Нека $\Sigma$ e азбука и $A = (\Sigma, \; Q, \; s, \; \delta, \; F)$ е свързан, тотален Д.К.А

\subsection*{Еквивалетност на състояния на автомат}

$R_\equiv = \{(p, \; q) \in Q \; | \; \forall \omega \in \Sigma^* \; \delta^*(p, \; \omega) \in F \iff \delta^*(q, \; \omega) \in F\}$

\subsection*{Минимален Д.К.А построен по състояния на класовете на еквивалентност на $R_\equiv$}

Нека $A_\equiv = (\Sigma, \; Q_\equiv, \; s_\equiv, \; \delta_\equiv, \; F_\equiv)$ \\

$Q_\equiv = Q/R_\equiv = \{[q]_{R_\equiv} \; | \; q \in Q\}$ \\

$s_\equiv = [s]_{R_\equiv}$ \\

$\delta_\equiv([q]_{R_\equiv}, \; a) = [\delta(q, \; a)]_{R_\equiv}$ \\

$F_\equiv = \{[q]_{R_\equiv} \; | \; q \in Q, \; [q]_{R_\equiv} \subseteq F\} = \{[q]_{R_\equiv} \; | \; q \in Q, \; [q]_{R_\equiv} \cap F \neq \emptyset\}$

\subsubsection*{Лема $\delta_\equiv$ е коректно дефинирана ($\delta_\equiv$ задава функция)}
\subsubsection*{Лема $\delta_\equiv^*([s]_{R_\equiv}, \; \omega) = [\delta^*(s, \; \omega)]_{R_\equiv} \implies \Lang(A_\equiv) = L$}
\subsubsection*{Лема $A_\equiv$ е с минимален брой състояния}

\section*{Задачи:}

Ако $L \subseteq \{a, \; b\}^*$ е регулярен език.

То езикът $\{(ab)^n(ba)^k \; | \; n, \; k \in \N\}$ е регулярен, защото очевидно се описва от регулярния израз $(ab)^*(ba)^*$. \\

Ако $D$ е краен език над $\{a, \; b\}^*$, то $L \cup \overline{D}$ винаги е регулярен, защото щом $D$ е регулярен, то и $\overline{D}$ е регулярен, следователно и $L \cup \overline{D}$ е регулярен. \\

Ако $K \subseteq \{a, \; b\}^*$ не е регулярен. То $L \cap K$ може да е не регулярен, но може и да е регулярен, например ако $L \cap K = \emptyset$ то той ще е регулярен. \\

\section*{Използвана литература}

\href{https://learn.fmi.uni-sofia.bg/pluginfile.php/148269/mod_resource/content/1/eai-stefan-2017-10-03.pdf}{Записки по "Езици, автомати, изчислимост" на главен асистент д-р Стефан Вътев от Факултет по Математика и Информатика на Софийски университет "Св. Климент Охридски"} \\

\href{https://learn.fmi.uni-sofia.bg/pluginfile.php/151077/mod_resource/content/1/intro2017_2018inf1.pdf}{Уводни записки от курса по Езици, автомати и изчислимост четен на спец. Информатика през зимния семстър на 2017г във ФМИ на СУ "Св. Климент Охридски" от доц. д-р Александра Соскова} \\

\href{https://learn.fmi.uni-sofia.bg/pluginfile.php/148277/mod_resource/content/1/1automata.pdf}{Записки по темата "Регулярни езици" от курса по Езици, автомати и изчислимост четен на спец. Информатика през зимния семстър на 2017г във ФМИ на СУ "Св. Климент Охридски" от доц. д-р Александра Соскова}
\end{document}