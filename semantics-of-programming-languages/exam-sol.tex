\documentclass[a4paper,9pt]{extarticle}
\usepackage[T2A]{fontenc}
\usepackage[utf8]{inputenc}
\usepackage[bulgarian]{babel}
\usepackage{amsmath}
\usepackage{amsthm}
\usepackage{amssymb}
\usepackage{amsfonts}                
\usepackage{enumerate}
\usepackage{multirow}
\usepackage{alltt}
\usepackage{euler}

\theoremstyle{definition}
\newtheorem{problem}{Зад.}
\newtheorem{solution}{Решение на Зад.}

\setlength{\textwidth}{210mm} 
\setlength{\textheight}{297mm} 
\parindent 0.5cm
\oddsidemargin 0cm
\evensidemargin 0cm

\newcommand{\Nat}{\mathbb{N}}
\newcommand{\Int}{\mathbb{Z}}
\newcommand{\Real}{\mathbb{R}}
\newcommand{\F}{\mathcal{F}}
\def\dotminus{\mathbin{\ooalign{\hss\raise1ex\hbox{.}\hss\cr\mathsurround=0pt$-$}}}

\begin{document}

\begin{problem} Нека  $\Gamma\!: \F_1 \rightarrow \F_1$ и  $\Delta\!: \F_1 \rightarrow \F_1$ са следните оператори:
   $$\Gamma(f)(x) \simeq
  \begin{cases}
   0, & \text{ако } x=0\\
    x.f(x-1), & \text{иначе. }
  \end{cases}
 {}\quad  \text{и}\quad \Delta(f) = f\circ f.
$$ \\
а) Определете операторите $\Gamma\circ \Delta$ и $\Delta\circ \Delta$. \\
б) Намерете  явния вид на  функциите $\Gamma(f)$, $(\Gamma\circ \Delta)(f)$ и $(\Delta\circ \Delta)(f)$, 
\vskip1pt    

където   $f(x)= x+1 $ за всяко $x\in \Nat.$
\end{problem}

\begin{solution}
а) \((\Gamma\circ \Delta)(f)(x) \simeq \Gamma(\Delta(f))(x) \simeq \Gamma(f \circ f)(x) \simeq \begin{cases}
    0 &, \; x = 0 \\
    x.f(f(x - 1)) &, \; x > 0
\end{cases}\).

\vskip7pt

\((\Delta\circ \Delta)(f) = \Delta(\Delta(f)) = \Delta(f \circ f) = (f \circ f) \circ (f \circ f) = f^4\).

\vskip7pt

б) \(\Gamma(f)(x) \simeq \begin{cases}
    0 &, \; x = 0 \\
    x.((x - 1) + 1) &, \; x > 0
\end{cases} \simeq \begin{cases}
    0 &, \; x = 0 \\
    x^2 &, \; x > 0
\end{cases}\).

\vskip7pt

\((\Gamma \circ \Delta)(f)(x) \simeq \begin{cases}
    0 &, \; x = 0 \\
    x.(((x - 1) + 1) + 1) &, \; x > 0
\end{cases} \simeq \begin{cases}
    0 &, \; x = 0 \\
    x(x +1) &, \; x > 0
\end{cases}\).

\vskip7pt

\((\Delta \circ \Delta)(f)(x) \simeq f^4(x) \simeq ((((x + 1) + 1) + 1) + 1) = x + 4\).


\end{solution}

\newpage

\begin{problem} а) Определете дали следната функция $f\!:\ \Nat_\bot^2 \rightarrow \Nat_\bot$: 
    $$f(x,y) =
  \begin{cases}
   0, & \text{ако } x=0\\
    x.y, & \text{ако } x>0\ \&\ y\geq 0\\
   \bot, & \text{в останалите случаи} 
  \end{cases}
$$ \\
е точна, монотонна или непрекъсната. Обосновете се.
\vskip3pt
б) Нека  $f, g_1$ и $ g_2$ са  функции от $\F_2^\bot$. Вярно ли е, че:
\vskip2pt
--   ако $f, g_1$ и $ g_2$ са точни функции, то и $f(g_1, g_2)$   е точна;
\vskip2pt
--   ако $f, g_1$ и $g_2$ са монотонни, то и $f(g_1, g_2)$  е монотонна?
\vskip2pt
Обосновете отговорите си.
\end{problem}

\begin{solution}
а) От лекции знаем, че ако една функция е точна, то тя е монотонна и \\
една функция е монотонна ТСТК е непрекъсната. \\
Тоест ако покажем, че $f$ е точна, то тя ще е и монотонна и непрекъсната. \\
Обаче $f$ не е точна, защото $f(0, \bot) = 0 \neq \bot$.
Нека проверим дали е монотонна, \\
като се възползваме от следното твърдение от лекции: \\
\textit{Една функция е монотонна ТСТК е монотонна по всеки аргумент}.
\begin{itemize}
    \item (по първия аргумент): Нека $a \in \Nat$ и $b \in \Nat_\bot$. Тогава $f(\bot, b) = \bot \sqsubseteq f(a, b)$ \\
    и значи $f$ е монотнна по първия си аргумент.
    \item (по втория): Нека $a \in \Nat_\bot$ и $b \in \Nat$. Тогава са възможни два случая.
    \begin{enumerate}
        \item $a = \bot$. Тогава $f(a, \bot) = \bot = f(a, b)$.
        \item $a \in \Nat$. Тогава $f(a, \bot) = \bot \sqsubseteq a.b = f(a, b)$.
    \end{enumerate}
    Следователно $f$ е монотнна по втория си аргумент.
\end{itemize}
Следователно $f$ е монотнна и непрекъсната.

\vskip9pt

б) \begin{itemize}
    \item (\textit{точност}): Нека $a, b \in \Nat_\bot$ са такива, че $a = \bot \lor b = \bot$. \\
    Тогава $g_1(a, b) = \bot$ и значи $(f(g_1, g_2))(a, b) = f(g_1(a, b), g_2(a, b)) = f(\bot, g_2(a, b)) = \bot$. \\
    Следователно $f(g_1, g_2)$ е точна, защото $g_1, g_2, f$ са точни.
    \item (\textit{монотонност}): Нека $a, b, c, d \in \Nat_\bot$ са такива, че $(a, b) \sqsubseteq (c, d)$. \\
    Тогава $g_1(a, b) \sqsubseteq g_1(c, d)$ и  $g_2(a, b) \sqsubseteq g_2(c, d)$. \\
    Следователно $(g_1(a, b), g_2(a, b)) \sqsubseteq (g_1(c, d), g_2(c, d))$. \\
    Следователно $(f(g_1, g_2))(a, b) \sqsubseteq (f(g_1, g_2))(c, d)$, понеже $f$ е монотонна. \\
    Следователно $f(g_1, g_2)$ е монотонна.
\end{itemize}
\end{solution}

\newpage

\begin{problem} Нека $(A, \leqslant, a_0)$ и $(B, \leqslant', b_0)$  са ОС, а $f\!: A \rightarrow B$ е сюрективно изображение, такова че за всички $a_1, a_2\in A$:
$$a_1\leqslant a_2\ \Longrightarrow\ f(a_1)\leqslant'f(a_2).$$
Докажете, че:

\vskip1pt

a) $f(a_0) = b_0$.

\vskip2pt


б) Ако редицата $\{a_n\}_{n \in \Nat}$  е монотонно растяща в $A$, то редицата $\{f(a_n)\}_{n \in \Nat}$ е монотонно растяща в $B$.

\vskip2pt

в) За всяка монотонно растяща  редица $\{a_n\}_{n \in \Nat}$ в $A$  е изпълнено:
$$\underset{n \in \Nat}{lub'}\hspace{1pt} f(a_n)\ \leqslant'f( \underset{n \in \Nat}{lub}\ a_n),$$
където с  $lub$ и $lub'$ са означени т.г.гр. в $A$ и $B$, съответно.

\vskip2pt

г) Дайте пример за ОС, за които $\underset{n \in \Nat}{lub'}\hspace{1pt} f(a_n)\  \neq \ f( \underset{n \in \Nat}{lub}\ a_n)$.

\end{problem} 

\vskip7pt 

\begin{solution}
а) Понеже \(f(a_0) \in B\), то \(b_0 \leqslant' f(a_0)\).
Но \(f\) е сюрекция. Нека тогава \(a \in A\) е такова, че \(b_0 = f(a)\). \\
Тогава \(a_0 \leqslant a\) и значи \(f(a_0) \leqslant' f(a) = b_0\).
Така \(b_0 \leqslant' f(a_0)\) и \(f(a_0) \leqslant' b_0\) и понеже $\leqslant'$ е антисиметрична, то \(f(a_0) = b_0\).

\vskip7pt 

б) Нека \(n \in \Nat\). Тогава \(a_n \leqslant a_{n + 1}\) и значи \(f(a_n) \leqslant' f(a_{n + 1})\).
Следователно \(\{f(a_n)\}_{n \in \Nat}\) е монотонно растяща в $(B, \leqslant', b_0)$.

\vskip7pt 

в) Нека $\{a_n\}_{n \in \Nat}$ е монотонно растяща редица в $(A, \leqslant, a_0)$.
Тогава от б) следва, че \(\{f(a_n)\}_{n \in \Nat}\) е монотонно \\
растяща в $(B, \leqslant', b_0)$ и значи $\underset{n \in \Nat}{lub'}\hspace{1pt} f(a_n)$ съществува.
Нека $n \in \Nat$ тогава $a_n \leqslant \underset{n \in \Nat}{lub}\ a_n$ и значи
$f(a_n) \leqslant' f\left(\underset{n \in \Nat}{lub}\ a_n\right)$. \\
Но тогава $f\left(\underset{n \in \Nat}{lub}\ a_n\right)$ е горна граница за $\{f(a_n)\}_{n \in \Nat}$.
Следователно $\underset{n \in \Nat}{lub'}\hspace{1pt} f(a_n)\ \leqslant'f( \underset{n \in \Nat}{lub}\ a_n)$.

\vskip7pt

г) Нека \(K = \Nat \cup \{\Nat\}\) и нека \(\leqslant\) е следната наредба $a \leqslant b \overset{def}{\iff} (a \in \Nat \;\&\; b \in \Nat \;\&\; a \leq_\Nat b) \;\lor\; b = \Nat$. \\
Ясно е, че $(K, \leqslant, 0)$ е ОС. 
Нека \(T = \{\emptyset, \; \{\emptyset\}\}\) и нека \(\prec = \{\{(\emptyset, \{\emptyset\})\}\}\).
Нека \(\preceq\) е рефлексивното затваряне на $\prec$ в $T$. \\
Тогава $(T, \preceq, \emptyset)$ е ОС. Нека \(q = \{n\}_{n \in \Nat}\). Тогава $q$ е монотонно растяща редица в $(K, \leqslant, 0)$ и т.г.гр. е $\Nat$. \\
Нека \(f : K \to T\) е таква, че \(f(x) = \begin{cases}
    \emptyset &, \; x \in \Nat \\
    \{\emptyset\} &, \; x = \Nat
\end{cases}\). Тогава $\underset{n \in \Nat}{lub_\preceq}\hspace{1pt} f(q_n) = \underset{n \in \Nat}{lub_\preceq}\hspace{1pt} f(n) = \underset{n \in \Nat}{lub_\preceq}\hspace{1pt} \emptyset = \emptyset$. \\
От друга страна $f\left(\underset{n \in \Nat}{lub_\leqslant}\hspace{1pt} q_n\right) = f(\Nat) = \{\emptyset\}$ и значи $\underset{n \in \Nat}{lub_\preceq}\hspace{1pt} f(q_n) = \emptyset \prec \{\emptyset\} = f\left(\underset{n \in \Nat}{lub_\leqslant}\hspace{1pt} q_n\right)$. \\
Следователно $\underset{n \in \Nat}{lub_\preceq}\hspace{1pt} f(q_n) \neq f\left(\underset{n \in \Nat}{lub_\leqslant}\hspace{1pt} q_n\right)$.

\vskip9pt

\textbf{Колегата Данчо} (\textit{Йордан Петров}) се сети за аналогичен на този пример. \\
\textbf{Може би неговият беше по-ясен}, но този е малко по-\textit{интересен}.  

\end{solution}

\newpage

\begin{problem} Нека $\Gamma\!: \F_1 \rightarrow \F_1$  е непрекъснат оператор. Докажете, че за всяко $n\geq 1$ са еквивалентни условията:

\vskip2pt

\begin{itemize}
    \item $f\in \F_1$ е най-малката функция със свойството $\Gamma(f)\subseteq f$;
    \item $f\in \F_1$ е най-малката функция със свойството $\Gamma^n(f)\subseteq f$.
\end{itemize}

\end{problem}

\begin{solution}
Както знаем от лекции понеже $\Gamma$ е непрекъснат, \\
то най-малката функция със свойството $\Gamma(f)\subseteq f$ 
съвпада с най-малката неподвижна точка на $\Gamma$. \\
Лесно се доказва по индукция, че за всяко $n \in \Nat_+$ операторът $\Gamma^n$ е непрекъснат. \\
Нека \(n \in \Nat\).
Ще докажем, че е в сила $lfp(\Gamma^{n + 1}) = lfp(\Gamma)$. От където ще следва условието на задачата.

\vskip9pt

От теорията знаем, че $lpf(\Gamma^{n + 1}) = \displaystyle\bigcup_{k \in \Nat} (\Gamma^{n + 1})^k(\emptyset) = \displaystyle\bigcup_{k \in \Nat} \Gamma^{kn + k}(\emptyset)$ и $lfp(\Gamma) = \displaystyle\bigcup_{k \in \Nat} \Gamma^k(\emptyset)$ \\
и редиците $\{\Gamma^{kn + k}(\emptyset)\}_{k \in \Nat}$ и $\{\Gamma^{k}(\emptyset)\}_{k \in \Nat}$ са монотонно растащи и сходящи. \\

Сега ще докажем частен случай на добре познато твърдение от Анализа: \\
\textit{Границата на подредица на сходяща редица съвпада с границата на самата редицата}.

\vskip9pt

Нека $k \in \Nat$. Тогава $\Gamma^k \subseteq \Gamma^{kn + k} \subseteq \displaystyle\bigcup_{l \in \Nat} \Gamma^{ln + l}(\emptyset) = lfp(\Gamma^{n + 1})$ и значи $lfp(\Gamma^{n + 1})$ е горна граница за редицата $\{\Gamma^{l}(\emptyset)\}_{l \in \Nat}$. \\
Следователно $lfp(\Gamma) = \displaystyle\bigcup_{l \in \Nat} \Gamma^{l}(\emptyset) \subseteq lfp(\Gamma^{n + 1})$.

\vskip9pt

Нека $k \in \Nat$. Тогава $\Gamma^{kn + k} \subseteq \displaystyle\bigcup_{l \in \Nat} \Gamma^l(\emptyset) = lfp(\Gamma)$,
защото $\displaystyle\bigcup_{l \in \Nat} \Gamma^l(\emptyset)$ е т.г.гр. на редицата $\{\Gamma^{l}(\emptyset)\}_{l \in \Nat}$, a $kn + k \in \Nat$. \\ Следователно $lfp(\Gamma)$ е горна граница за редицата $\{\Gamma^{ln + l}(\emptyset)\}_{l \in \Nat}$.
Следователно $lfp(\Gamma^{n + 1}) = \displaystyle\bigcup_{l \in \Nat} \Gamma^{ln + l}(\emptyset) \subseteq lfp(\Gamma)$.

\vskip9pt

Така $lfp(\Gamma^{n + 1}) = lfp(\Gamma)$ и от това следва условието на задачата.


\end{solution}

\end{document}
