\documentclass{article}
\usepackage{amsmath,amsthm}
\usepackage{amssymb}
\usepackage{lipsum}
\usepackage{stmaryrd}
\usepackage[T1,T2A]{fontenc}
\usepackage[utf8]{inputenc}
\usepackage[bulgarian]{babel}
\usepackage[normalem]{ulem}

\setlength{\parindent}{0mm}

\title{Решени задачи по СЕП}
\author{Иво Стратев}

\begin{document}
\maketitle
\section*{Зад. 1. от второ контролно по СЕП (10/05/2019)}
Да се докаже, че операторът \(\Gamma \; : \; \mathcal{F}_2 \to \mathcal{F}_2\),
дефиниран с условието
\begin{align*}
\Gamma(f)(x, y) \simeq \begin{cases}
    y, & f(x, y) \simeq 0 \\
    f(x, y + 1), & f(x, y) > 0 \\
    \lnot !, & \lnot !f(x, y)
\end{cases}
\end{align*}
има безброй много неподвижни точки.
\subsection*{Решение:}
Забелязваме, че когато \(f(x, y) > 0\), то \(\Gamma(f)(x, y)\) зависи само от \(y\).
Тоест свободно можем да се възползваме от факта, че стойността \(x\) остава фиксирана!.

Понеже \(fix(\Gamma) = \{f \in \mathcal{F}_2 \; | \; \Gamma(f) = f\} \subseteq \mathcal{F}_2\) и 
\(\mathcal{F}_2\) е неизброймо безкрайно множество, то е напълно достатъчно да покажем,
че за някое \(S \subseteq \mathcal{F}_2\), което е неизброймо безкрайно ще е изпълнено \(S \subseteq fix(\Gamma)\).

Нека \(F_1^+ \rightleftharpoons \{f \; : \; \mathbb{N} \to \mathbb{N}^+\}\) и нека \(F_2^+ \rightleftharpoons \{f \; : \; \mathbb{N}^2 \to \mathbb{N}^+\}\).

Нека \(\Delta : F_1^+ \to F_2^+\) и \(\Delta(f)(x, y) = f(x)\).

Нека \(S \rightleftharpoons \{\Delta(f) \; | \; f \in F_1^+\}\).

Очевидно \(S\) е неизброймо безкрайно понеже \(F_1^+\) е и \(S \subseteq F_2^+ \subseteq \mathcal{F}_2\).

Ще докажем, че \(S \subseteq fix(\Gamma)\).

Нека \(f \in S\) тогава \((\exists h \in F_1^+)[f = \Delta(h)]\).
Нека тогава \(h \in F_1^+\) и \(f = \Delta(h)\).
Имаме \((\forall (x, y) \in \mathbb{N}^2)[f(x, y) = \Delta(h)(x, y) = h(x) > 0]\).

Нека \((x, y) \in \mathbb{N}^2\) тогава \(\Gamma(f)(x, y) \simeq f(x, y + 1) = h(x) = f(x, y)\).

Следователно \((\forall (x, y) \in \mathbb{N}^2)[\Gamma(f)(x, y) = f(x, y)]\).
Тоест \(f \in fix(\Gamma)\). 

Следователно \((\forall f \in S)[f \in fix(\Gamma)]\). Следователно \(S \subseteq fix(\Gamma)\).

Извод: \(fix(\Gamma)\) е неизброймо безкрайно и значи има безброй много неподвижни точки.

\section*{Зад. 2. от второ контролно по СЕП (10/05/2019)}
Нека операторът \(\Gamma \; : \; \mathcal{F}_1 \to \mathcal{F}_1\)
е дефиниран по следния начин:
\begin{align*}
\Gamma(f)(x) \simeq \begin{cases}
    1, & x \leq 1 \\
    \displaystyle\frac{x}{2}, & x > 1 \; \& \; x \equiv 0 \pmod{2}\\
    \displaystyle{f\left(f\left(\frac{3x + 1}{2}\right)\right)}, & x > 1 \; \& \; x \equiv 1 \pmod{2}
\end{cases}    
\end{align*}
Докажете, че:

а) \(\Gamma\) е компактен (непрекъснат) оператор.

б) \((\forall x \in \mathbb{N})\left[!f_\Gamma(x) \; \& \; x > 1 \implies f_\Gamma(x) \leq \displaystyle\frac{x}{2}\right]\), където \(f_\Gamma = lfp(\Gamma)\).
\subsection*{Решение:}
\subsubsection*{а)}
\(\Gamma\) е компактен (непрекъснат) оператор, когато \(\Gamma\) е монотоннен и краен, което ще докажем.

Нека \(f, \; g \in \mathcal{F}_1\) и \(f \subseteq g\). Нека \(x \in Dom(\Gamma(f))\).
\begin{itemize}
\item \(x \leq 1\)

Имаме \(\Gamma(f)(x) \simeq 1 \simeq \Gamma(g)(x)\).
\item \(x > 1 \; \& \; x \equiv 0 \pmod{2}\)

Имаме \(\Gamma(f)(x) \simeq \displaystyle\frac{x}{2} \simeq \Gamma(g)(x)\).
\item \(x > 1 \; \& \; x \equiv 1 \pmod{2}\)

Имаме \(!\Gamma(f)(x)\) значи \(!\displaystyle{f\left(\frac{3x + 1}{2}\right)} \; \& \; !\displaystyle{f\left(f\left(\frac{3x + 1}{2}\right)\right)}\).

Следователно \(!\displaystyle{g\left(\frac{3x + 1}{2}\right)} \; \& \; !\displaystyle{g\left(g\left(\frac{3x + 1}{2}\right)\right)}\) и

\(\displaystyle{g\left(\frac{3x + 1}{2}\right)} = \displaystyle{f\left(\frac{3x + 1}{2}\right)}\) и \(\displaystyle{g\left(g\left(\frac{3x + 1}{2}\right)\right)} = \displaystyle{f\left(f\left(\frac{3x + 1}{2}\right)\right)}\).

Понеже \(f \subseteq g\).

Така \(\Gamma(f)(x) \simeq \displaystyle{f\left(f\left(\frac{3x + 1}{2}\right)\right)} = \displaystyle{f\left(g\left(\frac{3x + 1}{2}\right)\right)} = \\
\displaystyle{g\left(g\left(\frac{3x + 1}{2}\right)\right)} \simeq \Gamma(g)(x)\).
\end{itemize}
Така \(!\Gamma(f)(x) \implies \Gamma(f)(x) \simeq \Gamma(g)(x)\).

Следователно \((\forall x \in \mathbb{N})[!\Gamma(f)(x) \implies \Gamma(f)(x) \simeq \Gamma(g)(x)]\).

Така \(\Gamma(f) \subseteq \Gamma(g)\). От тук \((\forall (f, g) \in \mathcal{F}_1^2)[f \subseteq g \implies \Gamma(f) \subseteq \Gamma(g)]\).

Тоест \(\Gamma\) е монотоннен. (1)

Нека \(h \in \mathcal{F}_1\) и нека \(x \in Dom(\Gamma(h))\).
\begin{itemize}
\item \(x \leq 1\)

Имаме \(\Gamma(h)(x) \simeq 1 \simeq \Gamma(\emptyset)(x)\) и очевидно \(\emptyset \; \underset{fin}{\subseteq} \;  h\). 
\item \(x > 1 \; \& \; x \equiv 0 \pmod{2}\)

Имаме \(\Gamma(h)(x) \simeq \displaystyle\frac{x}{2} \simeq \Gamma(\emptyset)(x)\) и очевидно \(\emptyset \; \underset{fin}{\subseteq} \;  h\). 
\item \(x > 1 \; \& \; x \equiv 1 \pmod{2}\)

Имаме

\(\Gamma(h)(x) \simeq \displaystyle{h\left(h\left(\frac{3x + 1}{2}\right)\right)} \simeq \Gamma\left(h_{\displaystyle{\restriction{\left\{\displaystyle\frac{3x + 1}{2}, h\left(\displaystyle\frac{3x + 1}{2}\right)\right\}}}}\right)(x)\)

и очевидно \(h_{\displaystyle{\restriction{\left\{\displaystyle\frac{3x + 1}{2}, h\left(\displaystyle\frac{3x + 1}{2}\right)\right\}}}} \; \underset{fin}{\subseteq} \;  h\). 
\end{itemize}
Следователно \((\forall x \in Dom(\Gamma(h)))(\exists \theta \in \mathcal{F}_1)[\theta \; \underset{fin}{\subseteq} \;  h \; \& \; \Gamma(h)(x) \simeq \Gamma(\theta)(x)]\).

Следователно

\((\forall h \in \mathcal{F}_1)(\forall x \in \mathbb{N})[!\Gamma(h)(x) \implies (\exists \theta \in \mathcal{F}_1)[\theta \; \underset{fin}{\subseteq} \;  h \; \& \; \Gamma(h)(x) \simeq \Gamma(\theta)(x)]]\).

Тоест \(\Gamma\) е креан. (2).

От (1) и (2) следва, че \(\Gamma\) е компактен (непрекъснат) оператор.

\subsubsection*{б)}
Нека \(Q, R\) са свойства в \(\mathcal{F}_1\)
дефинирани по следния начин:
\begin{align*}
Q(f) \rightleftharpoons (\forall x \in \mathbb{N})[!f(x) \; \& \; x \leq 1 \implies f(x) = 1]\\
R(f) \rightleftharpoons (\forall x \in \mathbb{N})\left[!f(x) \; \& \; x > 1 \implies f(x) \leq \displaystyle\frac{x}{2}\right]
\end{align*}
\(Q\) и \(R\) са свойства от тип частична коректност и следователно са непрекъснати.
Нека \(P(f) \rightleftharpoons Q(f) \; \& \; R(f)\).
\(P\) е конюнкцията на \(Q\) и \(R\), които са непрекъснати, следователно \(P\) е непрекъснато. (3)

Имаме \((\forall x \in \mathbb{N})[\lnot!\emptyset(x)]\) следователно

\((\forall x \in \mathbb{N})[!\emptyset(x) \; \& \; x \leq 1 \implies \emptyset(x) = 1] \; \& \; (\forall x \in \mathbb{N})\left[!\emptyset(x) \; \& \; x > 1 \implies \emptyset(x) \leq \displaystyle\frac{x}{2}\right]\)

тоест \(Q(\emptyset) \; \& \; R(\emptyset)\). Следователно \(P(\emptyset)\). (4)

Нека \(f \in \mathcal{F}_1\) и \(P(f)\). Нека \(x \in Dom(\Gamma(f))\).
\begin{itemize}
\item \(x \leq 1\)

Имаме \(\Gamma(f)(x) = 1\).

\item \(x > 1\)

Възможни са два подслучая:

\begin{itemize}
\item \(x \equiv 0 \pmod{2}\)

Тогава \(\Gamma(f)(x) = \displaystyle\frac{x}{2} \leq \displaystyle\frac{x}{2}\).

\item \(x \equiv 1 \pmod{2}\)

Тогава \(\Gamma(f)(x) = \displaystyle{f\left(f\left(\frac{3x + 1}{2}\right)\right)}\)

Понеже \(x \equiv 1 \pmod{2}\), то \(\displaystyle\frac{3x + 1}{2} > 1\). Но тогава от \(P(f)\),

в частност \(R(f)\). Получаваме \(\displaystyle{f\left(\frac{3x + 1}{2}\right)} \leq \frac{3x + 1}{4}\).

Нека \(z := \displaystyle{f\left(\frac{3x + 1}{2}\right)}\). Тогава за \(z\) са възможни два случая.

\begin{itemize}
\item \(z \leq 1\)

Тогава \(\Gamma(f)(x) = f(z) = 1\).

Понеже \(x > 1\), то \(1 \leq \displaystyle\frac{x}{2}\).

Следователно \(\Gamma(f)(x) = 1 \leq \displaystyle\frac{x}{2}\).

\item \(z > 1\)

Тогава от \(R(f)\) получаваме

\(\Gamma(f)(x) = f(z) \leq \displaystyle\frac{z}{2} \leq \displaystyle\frac{3x  + 1}{8} \leq \displaystyle\frac{4x}{8} = \displaystyle\frac{x}{2}\).
\end{itemize}
Следователно получаваме \(\Gamma(f)(x) \leq \displaystyle\frac{x}{2}\)
\end{itemize}
\end{itemize}
Получихме, че \(!\Gamma(f)(x) \; \& \; x \leq 1 \implies \Gamma(f)(x) = 1\) и 

\(!\Gamma(f)(x) \; \& \; x > 1 \implies \Gamma(f)(x) \leq \displaystyle\frac{x}{2}\).

Следователно \(Q(\Gamma(f)) \; \& \; R(\Gamma(f))\) е истина,
тоест \(P(\Gamma(f))\).
Следователно имаме \(P(f) \implies P(\Gamma(f))\).
Така \((\forall f \in \mathcal{F}_1)[P(f) \implies P(\Gamma(f))]\). (5)

Тогава от (3), (4), (5) и правилото на Скот получаваме \(P(f_\Gamma)\).
В частност \(R(f_\Gamma)\). Тоест \((\forall x \in \mathbb{N})\left[!f_\Gamma(x) \; \& \; x > 1 \implies f_\Gamma(x) \leq \displaystyle\frac{x}{2}\right]\).

\section*{Трето контролно по СЕП (31/05/2019)}
Нека P е следната рекурсивна програма:
\begin{align*}
h(x, y) = f(x, y, 26, 5, 2) + 995 \; where \\
f(x, y, 0, 0, w) = g(w, 9) \\
f(x, y, 0, t, w) = g(x, y) + g(y, x) + f(x, y, 0, t - 1, w) \\
f(x, y, z, t, w) = g(x, y) + f(x, y, z - 1, t, w) \\
g(x, 0) = x \\
g(x, y) = x.g(x, y - 1)
\end{align*}
Да се докаже, че:

а) \((\forall (x, y) \in \mathbb{N}^2)[!D_V[P](x, y) \implies D_V[P](x, y) \simeq 31x^{y + 1} + 5y^{x + 1} + 2019]\)

б) \(D_V[P]\) е тотална функция.

\subsection*{Решение:}

\subsubsection*{а)}
Дефинираме следните оператори: \(\Gamma \; : \; \mathcal{F}_5 \times \mathcal{F}_2 \to \mathcal{F}_5\) и \(\Delta \; : \; \mathcal{F}_2 \to \mathcal{F}_2\).
\begin{align*}
\Gamma(f, g)(x, y, z, t, w) \simeq \begin{cases}
g(w, 9), & z = 0 \; \& \; t = 0 \\
g(x, y) + g(y, x) + f(x, y, 0, t - 1, w), & z = 0 \; \& \; t > 0 \\
g(x, y) + f(x, y, z, t, w), & z > 0
\end{cases}
\end{align*}
\begin{align*}
\Delta(g)(x, y) \simeq \begin{cases}
x, & y = 0 \\
x.g(x, y - 1), & y > 0
\end{cases}
\end{align*}

\(\Gamma\) и \(\Delta\) са термални оператори, следователно са непрекъснати.

Тогава системата
\begin{align*}
\Gamma(f, g) = f \\
\Delta(g) = g
\end{align*}

Има единствено най-малко решение.

Нека \(p \in \mathcal{F}_2\) и \(p(x, y) = x^{y + 1}\).
Ще доажем, че \(fix(\Delta) = \{p\}\).

Проверяваме, че \(\{p\} \subseteq fix(\Delta)\).

Нека \((x, y) \in \mathbb{N}^2\). Тогава са възможни два случая.
\begin{itemize}
\item \(y = 0\)

Тогава \(\Delta(p)(x, y) \simeq \Delta(p)(x, 0) \simeq x \simeq x^{0 + 1} \simeq x^{y + 1} \simeq p(x, y)\).
\item \(y > 0\)

Тогава \(\Delta(p)(x, y) \simeq x.p(x, y - 1) \simeq x.x^{y - 1  + 1} \simeq x.x^y \simeq x^{y + 1} \simeq p(x, y)\).
\end{itemize}
Следователно \((\forall (x, y) \in \mathbb{N}^2)[\Delta(p)(x, y) \simeq p(x, y)]\).
Следователно \(p = \Delta(p)\).
Следователно \(p \in fix(\Delta)\). Така \(\{p\} \subseteq fix(\Delta)\). (1)

Проверяваме, че \(fix(\Delta) \subseteq \{p\}\).

Нека \(f \in fix(\Delta)\).
Ще докажем чрез индукция следното твърдение

\((\forall y \in \mathbb{N})(\forall x \in \mathbb{N})[f(x, y) \simeq p(x, y)]\).
\begin{itemize}
\item База \(y = 0\)

Нека \(x \in \mathbb{N}\). Тогава

\(f(x, y) \simeq \Delta(f)(x, y) \simeq \Delta(f)(x, 0) \simeq x \simeq x^{0 + 1} \simeq x^{y + 1} \simeq p(x, y) \simeq p(x, 0)\).

Следователно \((\forall x \in \mathbb{N})[f(x, 0) \simeq p(x, 0)]\).
\item Индукционна хипотеза: Допускаме \((\exists n \in \mathbb{N})(\forall x \in \mathbb{N})[f(x, n) \simeq p(x, n)]\).
Нека тогава \(n \in \mathbb{N}\) и \((\forall x \in \mathbb{N})[f(x, n) \simeq p(x, n)]\).
\item Индукционна стъпка \(y = n + 1\)

Нека \(x \in \mathbb{N}\). Тогава \(f(x, y) \simeq f(x, n + 1) \simeq \Delta(f)(x, n + 1) \simeq \\
x.f(x, n + 1 - 1) \simeq x.f(x, n) \simeq x.p(x, n) \simeq x.x^{n + 1} \simeq \\
x^{n + 1 + 1} \simeq x^{y + 1} \simeq p(x, y)\).

Следователно \((\forall x \in \mathbb{N})[f(x, y) \simeq p(x, y)]\).
\item Заключение: \((\forall y \in \mathbb{N})(\forall x \in \mathbb{N})[f(x, y) \simeq p(x, y)]\).
\end{itemize}
Следователно \(f = p\), от тук \(f \in \{p\}\).
Следователно \((\forall f \in fix(\Delta))[f \in \{p\}]\).
Следователно \(fix(\Delta) \subseteq \{p\}\). (2)

От (1) и (2), следва, че \(fix(\Delta) = \{p\}\).

Тогава понеже \(p\) е тотална, можем свободно да заместим в първото функционално уравнение.

Разглеждаме оператор \(\Psi \; : \; \mathcal{F}_5 \to \mathcal{F}_5\). Дефиниран така \(\Psi(f) = \Gamma(f, p)\).

Тоест: 
\begin{align*}
\Psi(f)(x, y, z, t, w) \simeq \begin{cases}
w^{10}, & z = 0 \; \& \; t = 0 \\
x^{y + 1} + y^{x + 1} + f(x, y, 0, t - 1, w), & z = 0 \; \& \; t > 0 \\
x^{y + 1} + f(x, y, z - 1, t, w), & z > 0
\end{cases}
\end{align*}

Разглеждаме следното свойство в \(\mathcal{F}_5\):

\(Q(f) \rightleftharpoons (\forall (x, y, z, t, w) \in \mathbb{N}^5)[!f(x, y, z, t, w) \\
\implies f(x, y, z, t, w) \simeq (z + t)x^{y + 1} + ty^{x + 1} + w^{10}]\)

Това свойство очевидно е непрекъснато понеже е от тип частична коректност.
Очевидно е и, че \(Q(\emptyset^{(5)})\) понеже \((\forall (x, y) \in \mathbb{N}^2)[\lnot!\emptyset^{(5)}(x, y)]\).

Ще докажем, че \(Q\) е монотонно.

Нека \(h \in \mathcal{F}_5\) и нека \(Q(h)\) е истина.
Нека \((x, y, z, t, w) \in Dom(\Psi(h))\).

Възможни са три случая:
\begin{itemize}
\item \((z, t) = (0, 0)\)

Тогава \(\Psi(h)(x, y, z, t, w) \simeq w^{10} \simeq 0 + 0 + w^{10} \simeq 0.x^{y + 1} + 0.y^{x + 1} + w^{10} \simeq (z + t)x^{y + 1} + ty^{x + 1} + w^{10}\).
\item \(z = 0 \; \& \; t > 0\)

Тогава \(\Psi(h)(x, y, z, t, w) \simeq x^{y + 1} + y^{x + 1} + h(x, y, 0, t - 1, w) \simeq x^{y + 1} + y^{x + 1} + (t - 1)x^{y + 1} + (t - 1)y^{x + 1} + w^{10} \simeq tx^{y + 1} + ty^{x + 1} + w^{10} \simeq (z + t)x^{y + 1} + ty^{x + 1} + w^{10}\).
\item \(z > 0\)

Тогава \(\Psi(h)(x, y, z, t, w) \simeq x^{y + 1} + h(x, y, z - 1, t, w) \simeq (z - 1 + t)x^{y + 1} + ty^{x + 1} + w^{10} \simeq (z + t)x^{y + 1} + ty^{x + 1} + w^{10}\).
\end{itemize}
Следователно \(Q(\Psi(h))\).
Тогава \((\forall h \in \mathcal{F}_5)[Q(h) \implies Q(\Psi(h))]\).

Така \(Q\) е монотонно и от правилото на Скот получаваме \(Q(f_\Psi)\).
В частност \((\forall (x, y) \in \mathbb{N}^2)[!f_\Psi(x, y, 26, 5, 2) \implies f_\Psi(x, y, 26, 5, 2) \simeq (26 + 5)x^{y + 1} + 5y^{x + 1} + 2^{10} \simeq 31x^{y + 1} + 5y^{x + 1} + (30 + 2)^2 \simeq 31x^{y + 1} + 5y^{x + 1} + 1024]\).

Следователно \((\forall (x, y) \in \mathbb{N}^2)[!D_V[P](x, y) \implies \\
D_V[P](x, y) \simeq f_\Psi(x, y, 26, 5, 2) + 995 \simeq 31x^{y + 1} + 5y^{x + 1} + 2019]\).

\subsubsection*{б)}
Ясно е, че \(D_V[P]\) е тотална функция, точно когато и \(f_\Psi\) е.

За това ще докажем, че \(f_\Psi\) е тотална функция.

Да допуснем противното, тогава \(\mathbb{N}^5 \setminus Dom(f_\Psi) \neq \emptyset\).

Имаме, че \((\mathbb{N}^5, <_{lex \mathbb{N}^5})\) е фундирано множество.

Тогава \(\mathbb{N}^5 \setminus Dom(f_\Psi)\) има минален елемент относно \(<_{lex \mathbb{N}^5}\).

Нека тогава \((x^*, y^*, z^*, t^*, w^*)\) е минимален елемент на \(\mathbb{N}^5 \setminus Dom(f_\Psi)\).

Тогава \(\lnot!f_\Psi(x^*, y^*, z^*, t^*, w^*)\).

Очевидно \((z^*, t^*) \neq (0, 0)\) понеже \(f_\Psi(x^*, y^*, 0, 0, w) = w^{*10}\).

Тогава са възможни два случая:
\begin{itemize}
\item \(z^* = 0 \; \& \; t^* > 0\)

Но това значи, че \(\lnot ! f_\Psi(x^*, y^*, 0, t^* - 1, w^*)\),
но също така \((x^*, y^*, 0, t^* - 1, w^*) <_{lex \mathbb{N}^5} (x^*, y^*, z^*, t^*, w^*)\).
Тогава \((x^*, y^*, z^*, t^*, w^*)\), не е минимален за \(\mathbb{N}^5 \setminus Dom(f_\Psi)\) ... Абсурд!

\item \(z^* > 0\)

Но това значи, че \(\lnot ! f_\Psi(x^*, y^*, z^* - 1, t^*, w^*)\),
но също така \((x^*, y^*, z^* - 1, t^*, w^*) <_{lex \mathbb{N}^5} (x^*, y^*, z^*, t^*, w^*)\).
Тогава \((x^*, y^*, z^*, t^*, w^*)\), не е минимален за \(\mathbb{N}^5 \setminus Dom(f_\Psi)\) ... Абсурд!
\end{itemize}

Достигнахме до противоречие и в двата случая.
Противоречието идва от допускането, което направихме, че \(f_\Psi\) не е тотална.
Но тогава \(f_\Psi\) е тотална. Следователно и \(D_V[P]\) също е тотална!

\section*{Зад. 1. Писмен изпит по СЕП 07/02/2019}
Дайте пример за оператор от тип \(\Gamma \; : \; \mathcal{F}_1 \to \mathcal{F}_1\), който:

a) няма неподвижни точки;

б) има неподвижни точки, но няма най-малка неподвижна точка;

в) има най-малка неподвижна точка.

Обосновете се!

\subsection*{Решение:}

\subsubsection*{а)}
\begin{align*}
\Gamma(f)(x) \simeq \begin{cases}
9, & \lnot !f(x) \\
\lnot!, & !f(x)
\end{cases}
\end{align*}

Да допуснем, че така дефиниран \(\Gamma\) има неподвижна точка.

Нека \(f \in fix(\Gamma)\). Нека \(x \in \mathbb{N}\).

Възможни са два случая:
\begin{itemize}
\item \(!f(x)\)

Тогава \(\lnot!\Gamma(f)(x)\), но тогава \(\lnot[f(x) \simeq \Gamma(f)(x)]\).

Следователно \(f \notin fix(\Gamma)\). Абсурд!

\item \(\lnot!f(x)\)

Тогава \(\Gamma(f)(x) = 9\) и значи \(!\Gamma(f)(x)\), но тогава \(\lnot[f(x) \simeq \Gamma(f)(x)]\).

Следователно \(f \notin fix(\Gamma)\). Абсурд!
\end{itemize}

Достигнахме до противоречие. Следователно \(fix(\Gamma) = \emptyset\).

\subsubsection*{б)}
\begin{align*}
\Gamma(f)(x) \simeq \begin{cases}
9, & \lnot !f(x) \\
f(x), & !f(x)
\end{cases}
\end{align*}

Нека \(f \in \mathcal{F}_1\) и \(Dom(f) = \mathbb{N}\).
Нека \(x \in \mathbb{N}\).
Тогава имаме \(\Gamma(f)(x) \simeq f(x)\).
Следователно \((\forall x \in \mathbb{N})[\Gamma(f)(x) \simeq f(x)]\).
Тогава \(f = \Gamma(f)\) и значи \(f \in fix(\Gamma)\).

Така получаваме, че всяка тотална функция е неподвижна точка за \(\Gamma\).

Да допуснем, че има ненотална функция \(h\), която е неподвижна за \(\Gamma\).
\(h\) не е тотална, следователно \(\mathbb{N} \setminus Dom(h) \neq \emptyset\).
Нека тогава \(x \in \mathbb{N} \setminus Dom(h)\).
Тогава имаме \(\lnot!h(x)\) и \(\Gamma(h)(x) = 9\).
Очевидно тогава \(h \neq \Gamma(h)\). Това е Абсурд!

Следователно \(fix(\Gamma) = \{g \in \mathcal{F}_1 \; | \; Dom(g) = \mathbb{N}\} \neq \emptyset\).

Понеже всяка неподвижна точка на \(\Gamma\) е тотална,
а всеки две различни тотални функции са несравними помеждуси чрез релацията "подфункция".
И за всяка тотална функция няма частична функция, която да я разширява.
Следователно всеки елемент на \(fix(\Gamma)\) е минимален.
И понеже \(fix(\Gamma)\) е неизброймо безкрайно,
то значи \(fix(\Gamma)\) няма най-малък елемент.
Следователно \(\Gamma\) има неподвижна точка,
но няма най-малка неподвижна точка.

\subsubsection*{в)}
Нека \(\Gamma = Id(\mathcal{F}_1)\).
Тогава \(fix(\Gamma) = \mathcal{F}_1\).

И понеже \((\mathcal{F}_1, \subset, \emptyset)\) е фундирано,
то \(\emptyset = f_\Gamma = lfp(\Gamma)\). 

\section*{Зад. 2. Писмен изпит по СЕП 07/02/2019}
Да разгледаме следния непрекъснат оператор \(\Gamma \; : \; \mathcal{F}_3 \to \mathcal{F}_3\), където:
\begin{align*}
\Gamma(f)(x, y, z) \simeq \begin{cases}
0, & y > 0 \; \& \; (z + 1)y > x \\
f(x, y, z + 1) + 1, & y > 0 \; \& \; (z + 1)y \leq x \\
f(x, 0, z + 1) + 1, & y = 0
\end{cases}
\end{align*}
Докажете, че най-малката неподвижна точка на \(\Gamma\) е следната частична функция:
\begin{align*}
h(x, y, z) \simeq \begin{cases}
\displaystyle\left\lfloor\frac{x}{y}\right\rfloor \ominus z, & y > 0\\
\lnot!, & y = 0
\end{cases}
\end{align*}
където
\begin{align*}
a \ominus b = \begin{cases}
a - b, & a \geq b \\
0. & a < b
\end{cases}
\end{align*}

\subsection*{Решение:}
Ще докажем, че \(f_\Gamma\) има еквивалетно представяне:
\begin{align*}
f_\Gamma(x, y, z) \simeq \begin{cases}
    \displaystyle\left\lfloor\frac{x}{y}\right\rfloor - z, & y > 0 \; \& \; \displaystyle\left\lfloor\frac{x}{y}\right\rfloor \geq z \\
    0, & y > 0 \; \& \; \displaystyle\left\lfloor\frac{x}{y}\right\rfloor < z \\
    \lnot!, & y = 0
\end{cases}
\end{align*}

Нека \(Q, R\) са свойства в \(\mathcal{F}_3\)
дефинирани по следния начин:
\begin{align*}
Q(f) \rightleftharpoons (\forall (x, y, z) \in \mathbb{N}^3)[!f(x, y, z) \; \& \; y > 0 \; \& \; \displaystyle\left\lfloor\frac{x}{y}\right\rfloor \geq z \implies f(x, y, z) \simeq \displaystyle\left\lfloor\frac{x}{y}\right\rfloor - z] \\
R(f) \rightleftharpoons (\forall (x, y, z) \in \mathbb{N}^3)[!f(x, y, z) \; \& \; y > 0 \; \& \; \displaystyle\left\lfloor\frac{x}{y}\right\rfloor < z \implies f(x, y, z) \simeq 0]
\end{align*}
\(Q\) и \(R\) са свойства от тип частична коректност и следователно са непрекъснати.
Нека \(P(f) \rightleftharpoons Q(f) \; \& \; R(f)\).
\(P\) е конюнкцията на \(Q\) и \(R\), които са непрекъснати, следователно \(P\) е непрекъснато. (3)

Имаме \((\forall x \in \mathbb{N})[\lnot!\emptyset^{(3)}(x)]\) следователно \(P(emptyset^{(3)})\) е изпълнено.

Ще докажем, че \(P\) е монотонно свойство.

Нека \(f \in \mathcal{F}_3\) и \(P(f)\). Нека \((x, y, z) \in Dom(\Gamma(f))\) и нека \(y > 0\).
Нека \(k = \displaystyle\left\lfloor\frac{x}{y}\right\rfloor\).
Тогава \(k \leq \displaystyle\frac{x}{y} < k + 1\) и значи
\(ky \leq x \; \& \; x < (k + 1)y\).
Ако
\begin{itemize}
\item \(k < z\)
Тогава имаме \(x < (k + 1)y \; \& \; k < z\).
Следователно \(x < (z + 1)y\).
Но тогава \(\Gamma(f)(x, y, z) \simeq 0\).

\item \(k \geq z\)
Възможни са два подслучая:

\begin{itemize}
\item \(k = z\)

Тогава имаме \(x < (z + 1)y\) и значи

\(\Gamma(f)(x, y, z) \simeq 0 \simeq k - z = \displaystyle\left\lfloor\frac{x}{y}\right\rfloor - z\).
\item \(z < k\)

Тогава имаме \(zy < (z + 1)y \leq ky \leq x\). Следователно

\(\Gamma(f)(x, y, z) \simeq f(x, y, z + 1) + 1 \simeq \displaystyle\left\lfloor\frac{x}{y}\right\rfloor - (z + 1) + 1 \simeq \displaystyle\left\lfloor\frac{x}{y}\right\rfloor - z\).
\end{itemize}
\end{itemize}
Следователно очевидно \(P(\Psi(f))\) е истина.
Тогава очевидно \(P\) е монотонно свойство.

Тогава от правилото на Скот получаваме

\begin{align*}
(\forall (x, y, z) \in \mathbb{N}^3)\\
\left[!f_\Psi(x, y, z) \; \& \; y > 0 \implies f_\Psi(x, y, z) \simeq \begin{cases}
    \displaystyle\left\lfloor\frac{x}{y}\right\rfloor - z, &  \displaystyle\left\lfloor\frac{x}{y}\right\rfloor \geq z \\
    0, & \displaystyle\left\lfloor\frac{x}{y}\right\rfloor < z
\end{cases}\right]
\end{align*}

Остава да покаже, че ако \(y > 0\), то \(f_\Psi\) е дефинирана и ако \(y = 0\) не е.

Първо ще докажем, че \((\forall (x, y, z) \in \mathbb{N}^3)[y > 0 \implies !f_\Psi(x, y, z)]\).

Нека допуснем противното. Тогава нека

\((x^*, y^*, z^*) \in \mathbb{N}^3\) и \(y^* > 0\) и \(\lnot !f_\Psi(x^*, y^*, z^*)]\).

Очевидно ако \((z^* + 1)y^* > x^*\), то \(\Psi(f_\Psi)(x^*, y^*, z^*) = 0\) и значи \(!f_\Psi(x^*, y^*, z^*)\).

Тогава за следната редица от точки \(\{(x^*, y^*, z^* + n)\}_{n \in \mathbb{N}}\)

имаме \((\forall n \in \mathbb{N})[(x^*, y^*, z^* + n) \in \mathbb{N}^3 \setminus Dom(f_\Psi)]\),
което значи, че

\((\forall n \in \mathbb{N})[(z^* + n + 1)y^* \leq x^*]\).
Но тогава \((z^* + x^* + 1)y^* \leq x^*\) при положение, че \(y^* > 0\).
Това е Абсурд! Следователно е в сила

\((\forall (x, y, z) \in \mathbb{N}^3)[y > 0 \implies !f_\Psi(x, y, z)]\).

За второ твърдение нека допуснем,
че за някоя двойка \((a, b) \in \mathbb{N}^2\) е в сила
\(!f_\Psi(a, 0, b)\).
Нека тогава \(n = f_\Psi(a, 0, b)\).
Тогава получаваме \(n = f_\Psi(a, 0, b) = \Psi(f_\Psi)(a, 0, b) = f_\Psi(a, 0, b + 1) + 1 = \Psi(f_\Psi)(a, 0, b + 1) + 1 = f_\Psi(a, 0, b + 2) + 2 = \dots = f_\Psi(a, 0, b + n + 1) + n + 1\).
Но тогава \(n = f_\Psi(a, 0, b + n + 1) + n + 1\) и \(f_\Psi(a, 0, b + n + 1) \in \mathbb{N}\).
Това е пълен Абсурд! Следователно

\((\forall (x, z) \in \mathbb{N}^2)[\lnot!f_\Psi(x, 0, z)]\).

Тогава излиза, че
\begin{align*}
f_\Gamma(x, y, z) \simeq \begin{cases}
    \displaystyle\left\lfloor\frac{x}{y}\right\rfloor - z, & y > 0 \; \& \; \displaystyle\left\lfloor\frac{x}{y}\right\rfloor \geq z \\
    0, & y > 0 \; \& \; \displaystyle\left\lfloor\frac{x}{y}\right\rfloor < z \\
    \lnot!, & y = 0
\end{cases}
\end{align*}
\section*{Зад. 3. Писмен изпит по СЕП 07/02/2019}
Нека \(\Gamma, \; \Delta \; : \; \mathcal{F}_2 \times \mathcal{F}_2 \to \mathcal{F}_2\) са непрекъснати, където:
\begin{align*}
\Gamma(f, g)(x, y) \simeq \begin{cases}
    1, & x = 0 \\
    f(x - 1, y).g(x, y), & x > 0
\end{cases}    
\end{align*}
\begin{align*}
\Delta(f, g)(x, y) \simeq \begin{cases}
    1, & x = 0 \; \& \; y = 0 \\
    3.g(0, y - 1), & x = 0 \; \& \; y > 0 \\
    2.g(x - 1, y + 1), & x > 0
\end{cases}    
\end{align*}
Нека \((f_0, g_0) = lfp(\Gamma \times \Delta)\). Докажете, чe

\((\forall x \in \mathbb{N})[!f_0(x, 0) \implies f_0(x, 0) \simeq 6^{\displaystyle\frac{x(x + 1)}{2}}]\)
\subsection*{Решение:}
Разглеждаме \(\Phi \; : \; \mathcal{F}_2 \to \mathcal{F}_2\), за който:
\begin{align*}
\Phi(g)(x, y) \simeq \begin{cases}
    1, & x = 0 \; \& \; y = 0 \\
    3.g(0, y - 1), & x = 0 \; \& \; y > 0 \\
    2.g(x - 1, y + 1), & x > 0
\end{cases}    
\end{align*}
Дефинираме следното свойство в \(\mathcal{F}_2\):

\(P(f) \rightleftharpoons (\forall (x, y) \in \mathbb{N}^2)[!f(x, y) \implies f(x, y) \simeq 6^x.3^y]\)

Очевидно то е непрекъснато понеже е от тип частична коректност.

Очевидно е и, че \(P(\emptyset)\) понеже \((\forall (x, y) \in \mathbb{N}^2)[\lnot!\emptyset(x, y)]\).

Ще докаже, че \((\forall f \in \mathcal{F}_2)[P(f) \implies P(\Phi(f))]\).

Нека \(f \in \mathcal{F}_2\) и \(P(f)\). Нека \((x, y) \in Dom(\Phi(f))\).
\begin{itemize}
\item \((x, y) = (0, 0)\)

Тогава \(\Phi(f)(x, y) \simeq 1 \simeq 6^0.3^0 \simeq 6^x.3^y\).
\item \(x = 0 \; \& \; y > 0\)

Тогава \(\Phi(f)(x, y) \simeq 3.f(0, y - 1) \simeq 3.(6^0.3^{y - 1}) \simeq 6^x.3^y\).

\item \(x > 0\)

Тогава \(\Phi(f)(x, y) \simeq 2.f(x - 1, y + 1) \simeq 2.(6^{x - 1}.3^{y + 1}) \simeq 6.6^{x - 1}.3^y \simeq 6^x.3^y\).
\end{itemize}
Следователно \(!\Phi(f)(x, y) \implies \Phi(f)(x, y) \simeq 6^x.3^y\).
Следователно \(P(\Phi(f))\).

Следователно \((\forall f \in \mathcal{F}_2)[P(f) \implies P(\Phi(f))]\).

Така от до тук доказаното и правилото на Скот получаваме, че

\((\forall (x, y) \in \mathbb{N}^2)[!f_\Phi(x, y) \implies f_\Phi(x, y) \simeq 6^x.3^y]\).

Сега ще докажем, че \(f_\Phi\) е тотална.

Нека допуснем противното. Тогава \(\mathbb{N}^2 \setminus Dom(f_\Phi) \neq \emptyset\).

\((\mathbb{N}^2, <_{lex \mathbb{N}^2})\) е фундирано множество.

Тогава \(\mathbb{N}^2 \setminus Dom(f_\Phi)\) има минален елемент относно \(<_{lex \mathbb{N}^2}\).

Нека тогава \((x^*, y^*)\) е минимален елемент на \(\mathbb{N}^2 \setminus Dom(f_\Phi)\).

Тогава \(\lnot!f_\Phi(x^*, y^*)\). Очевидно \((x^*, y^*) \neq (0, 0)\) понеже \(f_\Phi(0, 0) = 1\).

\begin{itemize}
\item \(x^* = 0 \; \& \; y^* > 0\)

Тогава \(\lnot!f_\Phi(0, y^* - 1)\), но \((0, y^* - 1) \in \mathbb{N}^2\)

следователно \((x^*, y^* - 1) \in \mathbb{N}^2 \setminus Dom(f_\Phi)\) и \((x^*, y^* - 1) <_{lex \mathbb{N^2}} (x^*, y^*)\).

Но тогава \((x^*, y^*)\) не е минимален ... Абсурд!
\item \(x^* > 0\)

Тогава \(\lnot!f_\Phi(x^* - 1, y^* + 1)\), но \((x^* - 1, y^* + 1) \in \mathbb{N}^2\)

следователно \((x^* - 1, y^* + 1) \in \mathbb{N}^2 \setminus Dom(f_\Phi)\) и \((x^* - 1, y^* + 1) <_{lex \mathbb{N^2}} (x^*, y^*)\).

Но тогава \((x^*, y^*)\) не е минимален ... Абсурд!
\end{itemize}

Получихме противоречие с допускането, че \(\mathbb{N}^2 \setminus Dom(f_\Phi) \neq \emptyset\).

Следователно \(Dom(f_\Phi) = \mathbb{N}^2\).

От тук и от \(P(f_\Phi)\) получаваме \((\forall (x, y) \in \mathbb{N}^2)[f_\Phi(x, y) = 6^x.3^y]\).

Тогава разглеждаме \(\Psi \; : \; \mathcal{F}_2 \to \mathcal{F}_2\).

\(\Psi(f)(x, y) \simeq \Gamma(f, g_0)(x, y) \simeq \Gamma(f, f_\Phi)(x, y)\). Следователно
\begin{align*}
\Psi(f)(x, y) \simeq \begin{cases}
    1, & x = 0 \\
    6^x.3^y.f(x - 1, y), & x > 0
\end{cases}    
\end{align*}

Дефинираме следното свойство в \(\mathcal{F}_2\):

\(Q(f) \rightleftharpoons (\forall (x, y) \in \mathbb{N}^2)[!f(x, y) \implies f(x, y) \simeq 6^{\displaystyle\frac{x(x + 1)}{2}}.3^{xy}]\)

Очевидно то е непрекъснато понеже е от тип частична коректност.

Очевидно е и, че \(Q(\emptyset)\) понеже \((\forall (x, y) \in \mathbb{N}^2)[\lnot!\emptyset(x, y)]\).

Ще докаже, че \((\forall f \in \mathcal{F}_2)[Q(f) \implies Q(\Phi(f))]\).

Нека \(f \in \mathcal{F}_2\) и \(Q(f)\). Нека \((x, y) \in Dom(\Psi(f))\).
\begin{itemize}
\item \(x = 0\)

Тогава \(\Psi(f)(x, y) \simeq \Psi(f)(0, y) \simeq 1 \simeq 6^{0}.3^{0.y} \simeq 6^{\displaystyle\frac{x(x + 1)}{2}}.3^{xy}\).

\item \(x > 0\)

Тогава

\(\Psi(f)(x, y) \simeq 6^x.3^y.f(x - 1, y) \simeq 6^x.3^y.6^{\displaystyle\frac{x(x - 1)}{2}}.3^{(x - 1)y} \simeq 6^{\displaystyle\frac{x(x + 1)}{2}}.3^{xy}\).
\end{itemize}
Следователно \(Q(\Psi(f))\).
Следователно \((\forall f \in \mathcal{F}_2)[Q(f) \implies Q(\Phi(f))]\).

Тогава използвайки правилото на Скот получаваме \(Q(f_0)\).

В частност \((\forall x \in \mathbb{N})[!f_0(x, 0) \implies f_0(x, 0) \simeq 6^{\displaystyle\frac{x(x + 1)}{2}}]\).
\end{document}
