\documentclass[14pt]{extarticle}
\usepackage{amsmath}
\usepackage{amssymb}
\usepackage{amsthm}
\usepackage{mathdots}
\usepackage{mathtools}
\usepackage{stmaryrd}
\usepackage{tikz}
\usepackage[T1,T2A]{fontenc}
\usepackage[utf8]{inputenc}
\usepackage[bulgarian]{babel}
\usepackage[normalem]{ulem}
\usepackage[margin=0.5in, top=1cm, left=1in]{geometry}

\newcommand{\N}{\mathbb{N}}
\newcommand{\Z}{\mathbb{Z}}
\newcommand{\R}{\mathbb{R}}

\title{Домашна работа 1, № 45342, Група 3}
\author{Иво Стратев}

\DeclareMathSizes{14}{22}{18}{15}

\begin{document}
\maketitle
\section*{Задача 1. Да се решат рекурентните уравнения:	}
\subsection*{а)}
\(T(0) = 44, \; \forall n \in \N: \; n \geq 1 \quad T(n) = 2017^{T(n - 1)} \\\\
T(1) = 2017^{T(0)} = 2017^{44}, \quad T(2) = 2017^{T(1)} = 2017^{2017^{44}} \\\\
T(3) = 2017^{T(2)} = 2017^{2017^{2017^{44}}}, \quad T(4) = 2017^{T(3)} = 2017^{2017^{2017^{2017^{44}}}}\)
\subsubsection*{Дефиниция: \(\forall r \in \R \quad \lfloor r \rfloor = \max\{z \in \Z \; | \; z \leq r\}\)}
\subsubsection*{Твърдение: \(\forall n \in \N \quad T(n) = 44^{\left\lfloor \frac{1}{n + 1} \right\rfloor}\;{\underbrace{2017^{2017^{\Large{\iddots}^{2017}}}}_n}^{\;44}\)}
\subsubsection*{Доказателство:}
\subsubsection*{База:}
Има смисъл \({\underbrace{2017^{2017^{\Large{\iddots}^{2017}}}}_0}\) да се дефинира, като неутралния елемент относно операцията умножение, който е \(1\), защото повдигането на степен цяло число, каквито са \(2017\) и \(44\) се свежда до подходящ брой умножения. \\\\
Тогава \(44^{\left\lfloor \frac{1}{0 + 1} \right\rfloor}\;{\underbrace{2017^{2017^{\Large{\iddots}^{2017}}}}_0}^{\;44} = 
44^{\left\lfloor 1 \right\rfloor}\;1^{\;44} = 44^1 = 44 = T(0)\)
\subsubsection*{Индукционна хипотеза: \(\exists m \in \N \quad T(m) = 44^{\left\lfloor \frac{1}{m + 1} \right\rfloor}\;{\underbrace{2017^{2017^{\Large{\iddots}^{2017}}}}_m}^{\;44}\)}
\subsubsection*{Индукционна стъпка:}
\(\frac{1}{m + 2} < \frac{1}{m + 1} \leq 1 \implies \left\lfloor \frac{1}{m + 2} \right\rfloor = 0 \implies \\\\
44^{\left\lfloor \frac{1}{m + 2} \right\rfloor}\;{\underbrace{2017^{2017^{\Large{\iddots}^{2017}}}}_{m + 1}}^{\;44} = \quad {\underbrace{2017^{2017^{\Large{\iddots}^{2017}}}}_{m + 1}}^{\;44} \geq \; 2017^{44}\) \\\\
По условие: \(T(m + 1) = 2017^{T(m)}\) \\\\
От индукционната хипотеза за \(T(m\)) имаме \(T(m) = 44^{\left\lfloor \frac{1}{m + 1} \right\rfloor}\;{\underbrace{2017^{2017^{\Large{\iddots}^{2017}}}}_m}^{\;44}\) \\\\
Ако \(m = 0 \implies 44^{\left\lfloor \frac{1}{m + 1} \right\rfloor}\;{\underbrace{2017^{2017^{\Large{\iddots}^{2017}}}}_m}^{\;44} = \; 44 \) \\\\
Ако \(m \geq 1 \; \frac{1}{m + 1} < 1 \implies \left\lfloor \frac{1}{m + 1} \right\rfloor = 0 \implies \\\\
44^{\left\lfloor \frac{1}{m + 1} \right\rfloor}\;{\underbrace{2017^{2017^{\Large{\iddots}^{2017}}}}_m}^{\;44} = \quad {\underbrace{2017^{2017^{\Large{\iddots}^{2017}}}}_m}^{\;44} \geq \; 2017^{44} \) \\\\
Следователно за \(T(m + 1)\) получаваме: \(T(m + 1) = {\underbrace{2017^{2017^{\Large{\iddots}^{2017}}}}_{m + 1}}^{\;44} \geq \; 2017^{44}\) \\\\
Тоест: \(T(m + 1) = 44^{\left\lfloor \frac{1}{m + 2} \right\rfloor}\;{\underbrace{2017^{2017^{\Large{\iddots}^{2017}}}}_{m + 1}}^{\;44}\)
\subsubsection*{Заключение: \(\forall n \in \N \quad T(n) = 44^{\left\lfloor \frac{1}{n + 1} \right\rfloor}\;{\underbrace{2017^{2017^{\Large{\iddots}^{2017}}}}_n}^{\;44} \qed \)}
\subsubsection*{Отговор: \(\forall n \in \N \quad T(n) = 44^{\left\lfloor \frac{1}{n + 1} \right\rfloor}\;{\underbrace{2017^{2017^{\Large{\iddots}^{2017}}}}_n}^{\;44}\)}
\subsection*{б)}
\(T(1) = 1, \; \forall n \in \N: \; n \geq 2 \quad T(n) = \frac{1}{\frac{1}{T(n - 1)} + n^2} \implies \\\\
T(n)\left(\frac{1}{T(n - 1)} + n^2\right) = 1 \implies \frac{1}{T(n - 1)} + n^2 = \frac{1}{T(n)} \) \\\\
Нека \(\forall n \in \N: \; n \geq 1 \quad b_n = \frac{1}{T(n)} \implies b_n = n^2 + b_{n - 1} \implies \\\\
b_n = n^2 + (n - 1)^2 + b_{n - 2} \implies b_n = n^2 + (n - 1)^2 + (n - 2)^2 + b_{n - 3} \implies \\\\
b_n = n^2 + \displaystyle\sum_{k = 1}^{n - 2}(n -k)^2 + b_1 \implies b_n = \displaystyle\sum_{k = 1}^{n} k^2 \)
\subsubsection*{Твърдение: \(\forall n \in \N: \; n \geq 1 \quad b_n = \frac{n(n + 1)(2n + 1)}{6}\)}
\subsubsection*{Доказателство:}
\subsubsection*{База:}
\(b_1 = \displaystyle\sum_{k = 1}^{1} k^2 = 1 = \frac{6}{6} = \frac{1(2)(3)}{6}\)
\subsubsection*{Индукционна хипотеза: \(\exists m \in \N, \; m \geq 1 : \quad b_m = \frac{m(m + 1)(2m + 1)}{6}\)}
\subsubsection*{Индукционна стъпка:}
\(b_{m + 1} = \displaystyle\sum_{k = 1}^{m + 1} k^2 = \displaystyle\sum_{k = 1}^{m} k^2 + (m + 1)^2 = b_m + (m + 1)^2 = \frac{m(m + 1)(2m + 1)}{6} + (m + 1)^2 = \\ \\\
= \frac{2m^3 + 3m^2 + m + 6m^2 + 12m + 6}{6} = \frac{2m^3 + 9m^2 + 13m + 6}{6} \\\\\\
\frac{(m + 1)(m + 2)(2m + 3)}{6} = \frac{(m^2 + 3m + 2)(2m + 3)}{6} = \frac{2m^3 + 9m^2 + 13m + 6}{6} \implies \\\\
b_{m + 1} = \frac{(m + 1)(m + 2)(2(m + 1) + 1)}{6} \)
\subsubsection*{Заключение: \(\forall n \in \N : \; n \geq 1 \quad b_n = \frac{n(n + 1)(2n + 1)}{6} \qed \implies \)}
\(\forall n \in \N : \; n \geq 1 \quad \frac{1}{T(n)} = \frac{n(n + 1)(2n + 1)}{6} \implies T(n) = \frac{6}{n(n + 1)(2n + 1)} \)
\subsubsection*{Отговор: \(\forall n \in \N : \; n \geq 1 \quad T(n) = \frac{6}{n(n + 1)(2n + 1)} \)}
\subsection*{в)}
\(T(n) = 13T(n - 1) - 42T(n - 2) + 9 + 15n.8^n + 5n^2 .19^n + 16n^3 + 8^{n + 1} + 18n^4 \implies \\\\
T(n) = 13T(n - 1) - 42T(n - 2) + (15n + 8)8^n + 5n^2 .19^n + (18n^4 + 16n^3 + 9)1^n \\\\\\
T(n) - 13T(n - 1) + 42T(n - 2) = 0 \implies r^n -13r^{n - 1} + 42r^{n - 2} = 0 \implies \\\\
r^2 - 13r + 42 = 0 \implies (r - 6)(r - 7) = 0 \implies \\\\
T(n) = \left(\displaystyle\sum_{k = 0}^4 A_kn^k \right)1^n + B6^n + C7^n + (D_1n + D_0)8^n + (Q_2n^2 + Q_1n + Q_0)19^n \)
\subsubsection*{Отговор: \(T(n) = \left(\displaystyle\sum_{k = 0}^4 A_kn^k \right)1^n + B6^n + C7^n + (D_1n + D_0)8^n + \left(\displaystyle\sum_{m = 0}^2 Q_mn^m \right)19^n \)}
\section*{Задача 2.}
Да се пресметне броят на непразните алгебрични сборове, които могат да бъдат образувани от числата \(1, \; 2, \; 3, \; \dots, \; n\) при спазване на следните правила: \\
— Всяко число участва в алгебричния сбор най-много веднъж. (1)\\
— Всяко число може да участва или със знак плюс, или със знак минус. (2)\\
— По абсолютна стойност всяко събираемо е с поне две единици по-голямо от предходното (3)
\subsection*{Решение:}
Без ограничение на общността ще разглеждаме алгебрични сборове, в които събираемите участват в нарастващ ред спрямо абсолютната им стойност. Имаме право да разглеждаме търсените сборове налагай това допълниелно ограничение, защото операцията събиране в множеството на целите числа е комутативна и с ново въведеното ограничение не променяме техният брой. \\\\
Допълнително на всеки от търсените алгебрични сборове можем да съпоставим множество от числата, които участват в сбора и нека означим множеството от всички такива сборове с \(S\). Нека \(S_n = \{z \in \Z \; | \; 1 \leq |z| \leq n\} \) е множеството от всички числа, които могат да участват в търсените сборове.
\section*{Задача 3.}
Даден е граф с 10 върха номерирани с числата от 1 до 10 със следния списък на съседства:\\\\\\
1: 4, 6, 9, 10 \\
2: 4, 6 \\
3: 4, 5 , 8 \\
4: 1, 2, 3, 5, 9 \\
5: 3, 4 \\
6: 1, 2, 10 \\
7: 8, 9 \\
8: 3, 7, 10 \\
9: 1, 4, 7 \\
10: 1, 6, 8 \\\\
\begin{tikzpicture}
  [auto=left,every node/.style={circle,fill=blue!20}]

  \node (n4) at (0, 0) {4};
  \node (n2) at (-2, 0) {2};
  \node (n3) at (0, -2) {3};
  \node (n5) at (-2, -2) {5};
  \node (n9) at (2, 0) {9};
  \node (n7) at (4, 0) {7};
  \node (n8) at (6, 0) {8};
  \node (n1) at (0, 2) {1};
  \node (n6) at (-2, 4) {6};
  \node (n10) at (6,4) {10};

  \foreach \from/\to in {n4/n2,n4/n3,n4/n5,n4/n9,n5/n3,n9/n7,n7/n8,n3/n8,n4/n1,n9/n1,n2/n6,n6/n1,n6/n10,n10/n8,n1/n10}
    \draw (\from) -- (\to);

\end{tikzpicture}
\subsection*{а) Има ли хамилтонов цикъл?}
Отоговор: Да има. Хамилтонов цикъл е цикъла: 7, 9, 1, 10, 6, 2, 4, 5, 3, 8, 7
\subsection*{б) Има ли хамилтонов път?}
Отоговор: Да има. Хамилтонов път е пътя: 2, 6, 10, 8, 3, 5, 4, 1, 9, 7
\subsection*{в)  Има ли ойлеров цикъл? }
Отговор: не няма, защото: върхове: 8, 9 и 10 са от степен 3, която е нечетна следователно НДУ за съществуване на Ойлеров цикъл изискващо всички върхове да са от четна степен е нарушено.
\subsection*{г)  Има ли ойлеров път? }
Отговор: не няма, защото: върхове: 8, 9 и 10 са от степен 3, която е нечетна следователно НДУ за съществуване на Ойлеров път изискващо да има точно два върха от нечетна степен е нарушено.
\subsection*{д) Колко е хроматичното число?}
Върховете 3, 4, 5 образуват клика с дължина 3 и трябва да бъдат оцветени в различни цветове, защото всеки връх е съсед с другите два.
Същото нещо важи и за върховете: 1, 4, 9, които също образуват клика с дължина 3. Тогава избираме 3 цвята - син, зелен, червен. Избираме един от трите цвята - червения и оцветяваме връх 4. С останалите два цвята оцветяваме другите два върха от всеки от двата цикъла.
1 и 3 оцветихме в синьо, а 5 и 9 в зелено. Връх 7 е съседен на връх 9 и връх 8 е едновременно съсед на 7 и на 3. За да избегнем еднаквото оцветяване на съседни върхове - оцветяваме връх 7 в червено, а връх 8 в зелено. Остава ни да оцветим върхове: 2, 6 и 10.
Върхове 1, 6, 10 образуват клика с дължина 3 и трябва да бъдат оцветени в различни цветове. Връх 10 е свързан с връх 1 и 8, които са в различни цветове - син и зелен, за това него оцветяваме в червено. Тогава връх 6 трябва да бъде оцветен в зелено.
Така остава да оцветим само връх 2, но той е свързан с връх 4, който е червен и връх 6, който оцветихме в зелено тогава имаме единствена възможност и тя е да го оцветим в синьо.
Успяхме да оцветим всички десет върха в различни цветове и никой два съседни върха не са оцветени в един цвят, за което използвахме 3 цвята следователно хроматичното число на графа е 3.\\\\
Отговор: Хроматичното число на графа е 3. \\\\
\begin{tikzpicture}
  [auto=left,every node/.style={circle}]

  \node[fill=red!70] (n4) at (0, 0) {4};
  \node[fill=blue!30] (n2) at (-2, 0) {2};
  \node[fill=blue!30] (n3) at (0, -2) {3};
  \node[fill=green!50] (n5) at (-2, -2) {5};
  \node[fill=green!50] (n9) at (2, 0) {9};
  \node[fill=red!70] (n7) at (4, 0) {7};
  \node[fill=green!50] (n8) at (6, 0) {8};
  \node[fill=blue!30] (n1) at (0, 2) {1};
  \node[fill=green!50] (n6) at (-2, 4) {6};
  \node[fill=red!70] (n10) at (6,4) {10};

  \foreach \from/\to in {n4/n2,n4/n3,n4/n5,n4/n9,n5/n3,n9/n7,n7/n8,n3/n8,n4/n1,n9/n1,n2/n6,n6/n1,n6/n10,n10/n8,n1/n10}
    \draw (\from) -- (\to);

\end{tikzpicture}
\subsection*{е) Колко е хроматичният индекс?}
Връх 4 е от степен 5 и неговата степен е най-голяма в дадения граф. Ребрата (1, 4), (2, 4), (3, 4), (5, 4), (9, 4) са съседни следователно ще са ни необходими поне пет различни цвята за да оцветим ребрата на графа, така че да няма съседни ребра в един цвят.
Оцветяваме ги по следния начин: (1, 4) - синьо, (2, 4) - червено, (3, 4) - зелено, (5, 4) - оранжево и (9, 4) - лилаво.
След това (2, 6) оцветяваме в зелено, (1, 6) - оранжево, (1, 10) - червено, (6, 10) - синьо, (8, 10) - зелено, 
(3, 8) - синьо, (7, 8) - червено, (9, 7) - оранжево и (1, 9) - зелено. Така оцветихме всички ребра на графа в различни цветове и няма съседни ребра в един и същ свят. Бяха ни необходими 5 цвята следователно хроматичният индекс на графа е 5.\\\\
Отговор: Хроматичният индекс на графа е 5. \\\\
\begin{tikzpicture}
  [auto=left,every node/.style={circle,fill=yellow!40}]

  \node (n4) at (0, 0) {4};
  \node (n2) at (-2, 0) {2};
  \node (n3) at (0, -2) {3};
  \node (n5) at (-2, -2) {5};
  \node (n9) at (2, 0) {9};
  \node (n7) at (4, 0) {7};
  \node (n8) at (6, 0) {8};
  \node (n1) at (0, 2) {1};
  \node (n6) at (-2, 4) {6};
  \node (n10) at (6,4) {10};

  \draw[blue!30, line width=2mm] (n1) -- (n4);
  \draw[red!90, line width=2mm] (n2) -- (n4);
  \draw[green!60, line width=2mm] (n3) -- (n4);
  \draw[orange!70, line width=2mm] (n5) -- (n4);
  \draw[purple!70, line width=2mm] (n9) -- (n4);
  \draw[green!60, line width=2mm] (n2) -- (n6);
  \draw[orange!70, line width=2mm] (n1) -- (n6);
  \draw[red!90, line width=2mm] (n1) -- (n10);
  \draw[blue!30, line width=2mm] (n6) -- (n10);
  \draw[green!60, line width=2mm] (n8) -- (n10);
  \draw[blue!30, line width=2mm] (n3) -- (n8);
  \draw[red!90, line width=2mm] (n7) -- (n8);
  \draw[orange!70, line width=2mm] (n9) -- (n7);
  \draw[green!60, line width=2mm] (n1) -- (n9);

\end{tikzpicture}
\section*{Задача 4.}
Докажете, че ако в ориентиран тегловен граф има контур с отрицателно тегло, то има и прост контур с отрицателно тегло. \\\\
Доказателство:
Без ограничение на общността ще счиаме, че в графа има само един контур с отрицателно тегло. Тогава ще разгледаме подграфа, който е този контур. \\\\
Ако контура е прост то съществува прост контур с отрицателно тегло. \\\\
Ако контура не е прост то отново без ограничение на общността можем да смятаме, че същестува само един вътрешен контур и то той е прост. Всеки друг случай можем да сведем до този, като изрежем всички вътрешни контури с изключение на един, който е с най-малко тегло и е прост контур.
Ако изрежем този вътршен контур и и външният вече не е контур то очевидно, че изрязаният контур е имал отрицателно тегло.
Ако изрязвайки вътрешния контур външния остане контур то да допуснем, че външния контур без вътрешния има положително тегло.
Тогава ако вътрешния контур е бил с отрицателно тегло отново показахме, че същестува прост контур с отрицателно тегло.
Ако не е бил с отрицателно тегло то очевидно част от него трябва да бъде с отрицателно тегло. И тъйкато той е прост контур можем да го разглеждаме като две части избирайки частта с по-малко тегло и изрязвайки другата получаваме единствен прост контур.
Ако допуснем, че той е с неотрицателно тегло стигаме до противоречие понеже останахме без контур с отрицателно телго а ние знаем, че имаме такъв. Следователно получения контур е с отрицателно тегло и е прост следователно намерихме прост контур с отрицателно тегло.
\end{document} 
