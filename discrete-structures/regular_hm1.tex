\documentclass[10pt]{article}
\usepackage{amsmath}
\usepackage{amssymb}
\usepackage{amsthm}
\usepackage{stmaryrd}
\usepackage[T1,T2A]{fontenc}
\usepackage[utf8]{inputenc}
\usepackage[bulgarian]{babel}
\usepackage[normalem]{ulem}
\usepackage[margin=0.5in, top=1cm, left=1in]{geometry}

\newcommand{\N}{\mathbb{N}}
\newcommand{\Z}{\mathbb{Z}}
\newcommand{\R}{\mathbb{R}}
\newcommand{\Q}{\mathbb{Q}}
\newcommand{\Prime}{\mathbb{P}}

\title{Домашна работа 1, № 45342, Група 3}
\author{Иво Стратев}

\DeclareMathSizes{10}{14}{10}{7}

\begin{document}
    \maketitle
    \section*{Задача 1.}
    Седем сандъка с различни материали трябва да бъдат разпределени на десететажен строеж, като сандъците не бива да
    се отварят и на десетия етаж трябва да бъдат оставени не по-малко от два вида материали. \\\\
    Сандъците не бива да се отварят това значи, че всеки различен материал ще бъде точно на един етаж. \\\\
    Сандъците са с различни материали това ги прави различими и всеки етаж си има номер, който го различава от останалите. \\\\
    На десетия етаж трябва да има поне два вида материали, което значи че предварително избираме колко от седемте сандъка ще бъдат
    на десетия етаж, като техния брой може да е: 2, 3, 4, 5, 6, или 7. Този избор можем да направим по
    \(\binom{7}{i} \; i = 2, \dots,  7\) различни начина. \\\\
    След като изберем колко и кои от седемте сандъка да отидат на десетия етаж остава да разпределим останалите сандъци.
    Всеки от които може да бъде поставен на останалите етажи, които са 9. Сандъците и етажите са раличими, което значи,
    че на всеки сандък гледаме като позиция във вектор с елемнти от множеството на останлите сандъци. Броят на всички такива вектори спрямо броя на сандъците поставени на десетия етаж е: \(9^{7 - i} \; i = 2, \dots,  7\) където \(i\) е броя на сандъците поставени на десетия етаж. \\\\
    Всеки избор на сандъците, които да са на десетия етаж ни води до следната формула за разпределянето на седемте сандъка:
    \(\binom{7}{i}9^{7 - i} \; i = 2, \dots,  7\) \\\\
    Разпределянето на сандъците зависи от избора ни на тези, които поставяме на десетия етаж и всеки такъв избор води до 
    решение за разпределянето на всички, което значи, че броят на всички решения се задава с формулата: 
    \(\displaystyle\sum_{i = 2}^7\binom{7}{i}9^{7 - i} \\\\\\
    i = 2, \quad \frac{7.6}{2}9^5 = 21.59049 = 1240029, \quad i = 3, \quad \frac{7.6.5}{3!}9^4 = 35.9^4 = 35.6561 = 229635 \\\\
    i = 4, \quad \frac{7.6.5.4}{4!}9^3 = 35.9^3 = 35.729 = 25515, \quad  i = 5, \quad \frac{7.6.5.4.3}{5!}9^2 = 21.81 = 1701\\\\
    i = 6, \quad \frac{7.6.5.4.3.2}{6!}9 = 7.9 = 63, \quad i = 7, \quad \frac{7!}{7!}9^0 = 1.1 = 1 \\\\
    1 + 63 + 1701 + 25515 + 229635 + 1240029 = 1496944 \)
    Отговор: \(1496944\)
%%%%%%%%%%%%%%%%%%%%%%%%%%%%%%%%%%%%%%%%%%%%%%%%%%%%%%%%%%%%%%%%%%%%%%%%%%%%%%%%%%%%%%%%%%%%%%%%%%%%%%%%%%%%%%%%%%%%%%%%%%%%%%%%%%%%%%%%%%%%%%%%%%%%%%%%%%%%%%%%%
	\section*{Задача 2.}
	\(P(n): n\) прави са прекарани през равнината. Секторите, на които тези прави разделят равнината,
	могат да се оцветят в два цвята, така че всички сектори с обща граница - линия, да са разноцветни.,
	\(\forall n \in \N\)
	\subsubsection*{База:}
	\(P(0): 0\)  прави са прекарани през равнината. Секторите, на които тези прави разделят равнината,
	могат да се оцветят в два цвята, така че всички сектори с обща граница - линия, да са разноцветни. \\\\
	\(0\) прави са прекарани през равнината. Следва в равнината има само един сектор - самата равнина.
	Следва няма и сектори с обща граница, която да е линия и следователно можем да оцветим равнината в един цвят,
	така че всички сектори с обща граница - линия, да са разноцветни. 
	\subsection*{Индукционно предположение:}
	\(P(k): k\) прави са прекарани през равнината. Секторите, на които тези прави разделят равнината,
	могат да се оцветят в два цвята, така че всички сектори с обща граница - линия, да са разноцветни. за някое \(k \in \N\)
	\subsection*{Индукционно стъпка:}
	\(P(k + 1): k + 1\) прави са прекарани през равнината. Секторите, на които тези прави разделят равнината,
	могат да се оцветят в два цвята, така че всички сектори с обща граница - линия, да са разноцветни. \\\\
	От индукционното предположение знаем, че секторите, получени при прекарването на \(k\) прави в равнина,
	могат да се оцветят в два цвята, така че всички сектори с обща граница - линия, да са разноцветни.
	Тогава нека имаме равнина, през която са прекарани \(k\) прави, всички сектори са оцветени в два цвята, такаче да
	няма сектори с обща граница, която да е линия и да са оцветени в един цвят. \\\\
	Сега ако всички сектори в равнината си разменят цветовете отново е вярно, че
	в равнината са прекарани \(k\) прави, оцветена е двуцветно и всички сектори с обща граница,
	която е линия са оцветени в различен цвят. (1) \\\\
	Прекарваме нова права в равнината. \\\\
	Ако новата права съвпада с някоя от вече прекараните прави е същото като тя да не е прекарана,
	защото същата права вече е прекарана и е вярно, че всички сектори с обща граница - линия,
	са оцветни разнозветно в равнина, през която са прекарани \(k\) прави и е оцветена в два цвята. \\\\
	Ако новата права не съвпада с никоя от вече прекараните прави, получаваме нови сектори,
	от които поне два ще имат обща граница линия и ще са оцветени в един цвят.
	Новата права разделя равнината на две полуравнини и от всички сектори през,
	които е минала прекараната права получаваме по два нови сектора,
	които имат граница линия и са оцветени в един цвят, нека ги наречем - граничните сектори. \\\\
	За всяка от двете полуравнини поотделно е вярно, че всички сектори в полуравнината,
	които имат обща граница - линия са оцветени в различен цвят, а полуравнина е оцветена в два цвята. (2) \\\\
	Избираме едната полуравнина и разменяме цветовете само на секторите в нея.
	Тогава за граничните сектори е в сила, че секторите с обща граница - линия, са оцветени в различен цвят. \\\\
	От (2) следва, че секторите, които са в неизбраната полуравнина и имат обща граница,
	която е линия са оцветени в различен цвят и тази полуравнина е оцветена в два цвята. \\\\
	Но от (1) и (2) следва, че всички секторите, които са в избраната полуравнина и имат обща граница,
	която е линия са оцветени в различни цветове, а полуравнината е оцветена в два цвята. \\\\
	Така всички сектори в равнината, които имат граница линия са оцветени разноцветно, a равнината е оцветена в два цвята. \\\\
	Седоватлно за равнина, в която са прекарани \(k + 1\) прави е вярно, че секторите,
	на които тези прави разделят равнината, могат да се оцветят в два цвята, така че всички сектори с обща граница - линия, да са разноцветни.
	\subsection*{Заключение:}
	\(\forall n \in \N \; P(n)\)
%%%%%%%%%%%%%%%%%%%%%%%%%%%%%%%%%%%%%%%%%%%%%%%%%%%%%%%%%%%%%%%%%%%%%%%%%%%%%%%%%%%%%%%%%%%%%%%%%%%%%%%%%%%%%%%%%%%%%%%%%%%%%%%%%%%%%%%%%%%%%%%%%%%%%%%%%%%%%%%%%%%%%%%%%%%%%%%%%%%%%%
    \section*{Задача 3.}
    \subsection*{a)}
    \(P = \N \backslash \{0\} \\\\
    R = \{(a, b) \in P^2 \: | \; \exists q \in \Q : \; \frac{a}{b} = q^2\} \\\\
    Q = \Q \backslash \{0\} \\\\
    \forall (a, b) \in P^2 \; a, b \geq 1 \implies \not\exists (c, d) \in P^2 : \; \frac{c}{d} = 0 \implies \\\\
    R = \{(a, b) \in P^2 \: | \; \exists q \in Q : \; \frac{a}{b} = q^2\} \)
    \subsubsection*{рефлексивност}
    \(a \in P \\\\
    \frac{a}{a} = 1 = 1^2 \overset{1 \in Q}{\implies} (a, a) \in R \implies \\\\
    \forall a \in P \; (a, a) \in R \implies R \; \text{ е рефлексивна}\)
    \subsubsection*{симетричност}
    \(a, b \in P : \; a \neq b, \; (a, b) \in R \implies \\\\
    \exists q \in Q : \; \frac{a}{b} = q^2 \\\\
    \frac{b}{a} = \frac{1}{q^2} = \left(\frac{1}{q}\right)^2 \overset{\frac{1}{q} \in Q}{\implies} (b, a) \in R \implies \\\\
    \forall a, b \in P : \; a \neq b, \; (a, b) \in R \implies (b, a) \in R \implies \\\\
    R \; \text{ е симетрична}\)
    \subsubsection*{транзитивност}
    \(a, b, c \in P : \; (a, b) \in R, \; (b, c) \in R \implies \\\\
    \exists q_1, q_2 \in Q : \; \frac{a}{b} = q_1^2, \; \frac{b}{c} = q_2^2 \\\\
    (q_1q_2)^2 = q_1^2q_2^2 = \frac{a}{b}\frac{b}{c} = \frac{ab}{bc} = \frac{a}{c} \overset{q_1q_2 \in Q}{\implies} (a, c) \in R \implies \\\\
    \forall a, b, c \in P : \; (a, b) \in R, \; (b, c) \in R \implies (a, c) \in R \implies \\\\
    R \; \text{ е транзитивна} \\\\
    \implies R \; \text{ е релация на еквивалентност} \implies \\\\
    \forall a \in P \; [a] = \{b \; | \; b \in P, \;  aRb\} = \{b \; | \; b \in P , \; \exists q \in Q : \; \frac{a}{b} = q^2\} = \\\\
    = \{\frac{a}{q^2} \; | \; \exists q \in Q : \; \frac{a}{q^2} \in P\} \)
%%%%%%%%%%%%%%%%%%%%%%%%%%%%%%%%%%%%%%%%%%%%%%%%%%%%%%%%%%%%%%%%%%%%%%%%%%%%%%%%%%%%%%%%%%%%%%%%%%%%%%%%%%%%%%%%%%%%%%%%%%%%%%%%%%%%%%%%%%%%%%%%%%%%%%%%%%%%%%%%%%%%%%%%%%%%%%%%%%%%%%%%%%%5
    \subsection*{б)}
    \(\forall n \in \N \; I_n = \begin{cases}
    	\{1, \dots, n\} \subset \N &, n \geq 1 \\
    	\emptyset &, n = 0
    \end{cases} \\\\ \)
    Нека \(\Prime\) е множеството от всички прости числа и нека\\\\ 
    \(M = \left\{\displaystyle\prod_{h \in I_n} p_h \; \Big| \; n \in P, \; \forall i \in I_n \; p_i \in \Prime, \; \forall j, k \in I_n \; p_j = p_k \iff j = k \right\} \cup \{1\} \implies \\\
    \forall p \in \Prime \; p \in M \overset{1 \in P}{\implies} \Prime \subset M \; (6 \notin \Prime) \\\\
    \forall p \in \Prime \; p \in P \implies \Prime \subset P \; (6 \notin \Prime) \implies \\\\
    M \subset P  \; (4 \notin M) \; (\text{P е затворено относно операцията умножение}) \\\\
    S = \Z \backslash \{0\} \implies \Prime \subset M \subset P \subset S \subset Q \subset \Q \; (\Prime \subset \N \subset \Z \subset \Q) \)
    \subsubsection*{Тв.}
    \(\forall m \in M \; [m] = \{m.s^2 \; | \; s \in S\} \)
    \subsubsection*{Док-во:}
    \(m \in M \\\\
    \lbrack m \rbrack = \{\frac{m}{q^2} \; | \; \exists q \in Q : \; \frac{m}{q^2} \in P\} \\\\
    u \in S \implies u^2 \in P \implies m.u^2 \in P \implies \\\\
    \frac{m}{m.u^2} = \frac{1}{u^2} = \left(\frac{1}{u}\right)^2 \overset{\frac{1}{u} \in Q}{\implies} m R m.u^2 \implies m.u^2 \in [m] \implies \\\\
    \forall u \in S \; m.u^2 \in [m] \implies \{m.s^2 \; | \; s \in S\} \subseteq [m] \\\\
    \frac{m}{q^2} \in [m] \implies q \in Q : \; \frac{m}{q^2} \in P \implies \exists t \in P : \; \frac{m}{q^2} = t \\\\
    q \in Q \implies \exists c,d \in S : \; c \nmid d, \; q = \frac{c}{d} \implies \frac{m}{(\frac{c}{d})^2} = t \implies \\\\
    m.d^2 = t.c^2 \overset{m,d,t,c \in S}{\implies} c^2 |m.d^2 \overset{c^2 \nmid d^2}{\implies} c^2 | m \implies \\\\
    c^2 = 1 \; \left(\begin{cases}
    c^2 | m \iff c^2 \in M, \; \exists k \in M : \; m = k.c^2  \\
    \forall z \in S, \; z \neq \pm 1 \implies z^2 \notin M 
    \end{cases}\right) \implies \\\\\\
    \frac{m}{q^2} = \frac{m}{\frac{1}{d^2}} = m.d^2 \implies \frac{m}{q^2} \in \{m.s	^2 \; | \; s \in S\} \implies \\\\
    \forall \frac{m}{q^2} \in [m] \; \frac{m}{q^2} \in \{m.s^2 \; | \; s \in S\} \implies [m] \subseteq \{m.s^2 \; | \; s \in S\} \implies \\\\
    \lbrack m \rbrack = \{m.s^2 \; | \; s \in S\} \implies \forall m \in M \; [m] = \{m.s^2 \; | \; s \in S\} \subseteq P \qed \)
    \subsubsection*{Тв.}
    \(\forall m_1, m_2 \in M, \; m_1 \neq m_2 \implies [m_1] \neq [m_2]\)
    \subsubsection*{Док-во:}
    \(m_1, m_2 \in M, \; m_1 \neq m_2 \implies m_1 \neq m_2.1 = m_2.1^2 = m_2.(-1)^2 \\\\
    \lbrack m_2 \rbrack = \{m_2.s^2 \; | \; s \in S\} \\\\
    \forall z \in S, \; z \neq \pm 1 \implies z^2 \notin M \implies \\\\
    \forall z \in S \; m_1 \neq m_2.z^2 \implies m_1 \notin [m_2] \implies [m_1] \neq [m_2] \implies \\\\
    \forall m_1, m_2 \in M, \; m_1 \neq m_2 \implies [m_1] \neq [m_2] \qed \\\\\\
    X \; \text{ е множеството от класовете на еквивалентност на } R \implies \\\\
    \begin{cases}
    \forall x \in X \; x \neq \emptyset \\\\
    \forall x, h \in X, \; x \neq h \implies x \cap h = \emptyset \\\\
    \underset{x \in X}{\bigcup} x = P
    \end{cases}\)
    \subsubsection*{Тв.}
    \(X = \{[m] \; | \; m \in M\}\)
    \subsubsection*{Док-во:}
    \(\forall m \in M \; [m] = \{m.s^2 \; | \; s \in S\} \neq \emptyset \\\\\\
    a, b \in M, \; a \neq b \implies [a] \neq [b] \\\\
    \text{Доп. } \; [a] \cap [b] \neq \emptyset \implies \exists c \in [a] \cap [b] \implies \\\\
    \begin{cases}
    	\exists s_1 \in S : c = a.s_1^2 \\\\
    	\exists s_2 \in S : c = b.s_2^2
    \end{cases} \implies a.s_1^2 = b.s_2^2 \implies \\\\\\
    s_1^2 = \frac{b.s_2^2}{a} = s_2^2.\frac{b}{a} \implies s_1 = \pm \sqrt{s_2^2.\frac{b}{a}} = \pm |s_2| \sqrt{\frac{b}{a}} \\\\
    s_1 \in S \overset{|s_2| \in P}{\iff} \sqrt{\frac{b}{a}} \in P \\\\
    \begin{cases}
    	a \neq b \implies \frac{b}{a} \neq 1 \\
    	b < a \implies \frac{b}{a} \in Q \backslash S \\
    	a \nmid b \implies \frac{b}{a} \in Q \backslash S \\
    	a | b \implies \frac{b}{a} \in M
    \end{cases} \implies \sqrt{\frac{b}{a}} \notin P \implies \\\\
    s_1 \notin S \implies \lightning \implies [a] \cap [b] = \emptyset \implies \\\\
    \forall a, b \in M, \; a \neq b \implies [a] \neq [b] \implies [a] \cap [b] = \emptyset \\\\\\
    1 \in M \implies [1] = \{1.s^2 \; | \; s \in S \} = \{s^2 \; | \; s \in S \} \ni 1 \; (\pm 1 \in S) \\\\
    \text{От основната теорема на аритметиката:} \\\\
    \forall p \in P, \; p \neq 1 \; \exists! k \in P, \; \exists! p_1, \dots, p_k \in \Prime, \; \exists! \alpha_1, \dots, \alpha_k \in P : \\\\
    p = \displaystyle\prod_{i \in I_k} p_i^{\alpha_i}, \; \forall i, j \in I_k \; p_i = p_j \iff i = j \implies \\\\
    \forall L \in 2^{\{p_1, \dots, p_k\}} \backslash \{\emptyset\} \; \displaystyle\prod_{l \in L} l \in M \\\\
    \text{Делим всяко } \; \alpha_i \text{ с частно и остатък на две и получаваме } \\\\
    \forall i \in I_k \; \exists q_i \in \N, \; r_i \in \{0, 1\} : \; \forall i \in I_k \; \alpha_i = 2.q_i + r_i \implies \\\\
    p = \displaystyle\prod_{i \in I_k} p_i^{2.q_i + r_i} \\\\
    A_0 = \{i \; | \; i \in I_k, \; \alpha_i \equiv 0 \pmod{2}\} \\\\
    A_1 = \{i \; | \; i \in I_k, \; \alpha_i \equiv 1 \pmod{2}\} \implies \\\\
   \forall i \in I_k \; i \in A_0 \lor i \in A_1 \implies \\\\
   A_0 \cap A_1 = \emptyset, \; A_0 \cup A_1 = I_k \implies \\\\
   |A_0| + |A_1| = k \implies \\\\
   \text{Ако } A_1 = \emptyset \implies A_0 = I_k \implies \\\\
   p = \displaystyle\prod_{i \in I_k} p_i^{2.q_i} = 1.\left(\displaystyle\prod_{i \in I_k} p_i^{q_i} \right)^2 \implies p \in [1] \\\\\\
   \text{Ако } A_0 = \emptyset \implies A_1 = I_k \implies \\\\
   p = \displaystyle\prod_{i \in I_k} p_i^{2.q_i + 1} = \displaystyle\prod_{i \in I_k} p_i.p_i^{2.q_i} = \displaystyle\prod_{i \in I_k} p_i.\displaystyle\prod_{i \in I_k} p_i^{2.q_i} = \\\\
   = \displaystyle\prod_{i \in I_k} p_i.\left(\displaystyle\prod_{i \in I_k} p_i^{q_i} \right)^2 \implies p \in \left[\displaystyle\prod_{i \in I_k} p_i\right] \;
   \left(\displaystyle\prod_{i \in I_k} p_i \in M \right) \\\\\\
   \text{Нека } A_0 \neq \emptyset, \; A_1 \neq \emptyset \implies p = \displaystyle\prod_{j \in \{0, 1\}} \displaystyle\prod_{i \in A_j} p_i^{2.q_i + j} = \\\\
    = \displaystyle\prod_{i \in A_1} p_i^{2.q_i + 1}.\displaystyle\prod_{j \in A_0} p_j^{2.q_j} = \displaystyle\prod_{i \in A_1} p_i.p_i^{2.q_i}.\displaystyle\prod_{j \in A_0} p_j^{2.q_j} = \\\\ 
    = \displaystyle\prod_{i \in A_1} p_i.\displaystyle\prod_{j \in I_k} p_j^{2.q_j} = \displaystyle\prod_{i \in A_1} p_i.\displaystyle{\left(\prod_{j \in I_k} p_j^{q_j}\right)^2} \implies \\\\
    p \in \left[\displaystyle\prod_{i \in A_1} p_i\right] \; \left(\displaystyle\prod_{i \in A_1} p_i \in M \right) \implies \\\\
    \forall p \in P \; \exists m \in M : \; p \in [m] \implies \underset{u \in M}{\bigcup} [u] = P \implies \\\\
    X = \{[u] \; | \; u \in M\} \qed \\\\\\
    Y = \{\{p_1, p_2, \dots, p_n \} \; | \; n \in P, \; \forall i \in I_n \; p_i \in \Prime \} \cup \{\emptyset\} \\\\
    \psi : Y \to M \\\\
    \psi(y) = \begin{cases}
    	\min(y).\psi(y \backslash \{\min(y)\}) &, y \neq \emptyset \\\\
    	1 &, y = \emptyset
    \end{cases} = \begin{cases}
    	\displaystyle\prod_{i \in I_n} p_i &, y = \{p_1, p_2, \dots, p_n \} \\\\
    	1 &, y = \emptyset
    \end{cases} \implies \\\\\\
    \forall A, B \in Y, \; A \neq B \implies \psi(A) \neq \psi(B) \implies \psi \; \text{ е инекция} \\\\
    1 \in M, \; \emptyset \in Y, \; \psi(\emptyset) = 1 \\\\
    m \in M, \; m \neq 1 \implies \exists! l \in P, \; \exists! w_1, \dots, w_l \in \Prime : \\\\
    m = \displaystyle\prod_{i \in I_l} w_i, \; \forall i, j \in I_l \; w_i = w_j \iff i = j \\\\
    W = \{w_1, \dots, w_l\} \implies W \in Y \\\\
    \psi(W) = \psi(\{w_1, \dots, w_l\}) = \displaystyle\prod_{i \in I_l} w_i = m \\\\
    \implies \exists w \in Y : \; \psi(w) = m \implies \\\\
    \forall m \in M \; \exists w \in Y : \; \psi(w) = m \implies \\\\
    \psi \; \text{ е сюрекция} \implies \psi \; \text{ е биекция} \\\\\\
    \begin{array}{cccc}
    	\varphi : & Y  & \to & X \\
     ~ &  y & \mapsto & [\psi(y)]
    \end{array} \implies \varphi(y) = [\psi(y)] = \{\psi(y).s^2 \; | \; s \in S \} \\\\\\
    y_1, y_2 \in Y, \; y_1 \neq y_2 \implies \psi(y_1) \neq \psi(y_2) \implies \\\\
    \lbrack \psi(y_1) \rbrack \neq [\psi(y_2)] \implies \varphi(y_1) \neq \varphi(y_2) \implies \\\\
    \forall y_1, y_2 \in Y, \; y_1 \neq y_2 \implies \varphi(y_1) \neq \varphi(y_2) \implies \varphi \; \text{ е инекция} \\\\
    X = \{[z] \; | \; z \in M\} \implies \forall x \in X \; \exists z \in M : \; x = [z] \implies \\\\
    \forall x \in X \; \exists y \in Y : \; x = [\psi(y)] = \varphi(y) \implies \\\\
    \varphi \; \text{ е сюрекция} \implies \varphi \; \text{ е биекция} \implies \\\\
    X = \{[\psi(y)] \; | \; y \in Y\} = \{\varphi(y) \; | \; y \in Y\}\)
%%%%%%%%%%%%%%%%%%%%%%%%%%%%%%%%%%%%%%%%%%%%%%%%%%%%%%%%%%%%%%%%%%%%%%%%%%%%%%%%%%%%%%%%%%%%%%%%%%%%%%%%%%%%%%%%%%%%%%%%%%%%%%%%%%%%%%%%%%%%%%%%%%%%%%%%%%%%%%%%%%%%%%%%%%%%%%%%%%%%%%%%
    \section*{Задача 4.}
    \(K = \{(a, b) \in \R^2 \; | \; \frac{a}{b} = 2\}, \quad L = \{(a, b) \in \R^2 \; | \; \frac{b}{a} = 3\} \\\\
    \begin{array}{cccc}
    \varphi : & K  & \to & L \\
     ~ &  (a, b) & \mapsto & (\frac{a}{6}, b)
    \end{array} \)
    \subsection*{Проверка за коректност}
    \((a, b) \in K \implies \frac{a}{b} = 2 \implies \frac{b}{a} = \frac{1}{2}\\\\
    \frac{b}{\frac{a}{6}} = \frac{b}{a}.6 = \frac{1}{2}.6 = 3, \; \frac{a}{6}, b \in \R \implies \\\\
    \varphi((a,b)) = (\frac{a}{6}, b) \in L \implies \varphi : \; K \to L\)
    \subsection*{Инекция}
    \((a_1, b_1), (a_2, b_2) \in K, (a_1, b_1) \neq (a_2, b_2) \implies \\\\
    \begin{cases}
    	a_1 \neq a_2 \\
    	b_1 \neq b_2
    \end{cases} \implies \begin{cases}
    	\frac{a_1}{6} \neq \frac{a_2}{6} \\
    	b_1 \neq b_2
    \end{cases} \implies \\\\
    (\frac{a_1}{6}, b_1) \neq (\frac{a_2}{6}, b_2) \implies \varphi((a_1, b_1)) \neq \varphi((a_2, b_2)) \implies \\\\
	\forall k_1, k_2 \in K, \; k_1 \neq k_2 \implies \varphi(k_1) \neq \varphi(k_2) \implies\varphi \; \text{ е инкеция} \)
	\subsection*{Сюрекция}
	\((a, b) \in L \implies \frac{b}{a} = 3 \implies \frac{a}{b} = \frac{1}{3} \implies 6.\frac{a}{b} = 2 \implies \frac{6.a}{b} = 2 \\\\
	6.a, b \in \R \implies (6.a, b) \in K \\\\
	\varphi((6.a, b)) = (\frac{6.a}{6}, b) = (a, b) \implies \\\\
	\forall l \in L \; \exists k \in K : \; \varphi(k) = l \implies \\\\
	\varphi \; \text{ е сюрекция} \implies \varphi \; \text{ е биекция} \implies |K| = |L| \)
\end{document}
