\documentclass[17pt]{extarticle}
\usepackage{amsmath}
\usepackage{amssymb}
\usepackage{amsthm}
\usepackage{stmaryrd}
\usepackage[T1,T2A]{fontenc}
\usepackage[utf8]{inputenc}
\usepackage[bulgarian]{babel}
\usepackage[normalem]{ulem}
\usepackage[margin=0.5in, top=1cm, left=1in]{geometry}

\title{Домашна работа 4, № 45342, Група 3}
\author{Иво Стратев}

\begin{document}
    \maketitle
    \section*{Тв.}
    \(G(V, E)\) е свързан граф с поне два върха, но при отстраняване на произволно ребро се получава несвързан граф.
    Тогава съществува единствен прост път между всеки два върха.
    \section*{Док-во.}
    Нека \(G(V, E)\) е свързан граф с поне два върха, но при отстраняване на произволно ребро се получава несвързан граф. \\\\
    Нека \(|V| = 2\) от тук директно следва, че \(G = K_2\) понеже той е единствения свързан граф с два върха и от тук следва, че двата върха са свързани с единствен прост път. \\\\    
    Нека \(|V| \geq 3\) и нека \(v_i, v_j \in V : \; v_i \neq v_j\) са произволни тогава тъйкато \(G\) е свързан съществува някакъв път между \(v_i\) и \(v_j\). Нека допуснем,
    че съществува път, който не е прост и нека го означим с \(p\) и нека означим с \(V(p)\) върховете принадлежащи на \(p\). След като \(p\) не е прост път
    съществувa \(v_k \in V(p)\), който се повтаря поне два пъти в \(p\) тогава частта от \(p\) между първите две повторения на \(v_k\)
    образува цикъл, нека означим този цикъл с \(C\). След премахне на някое ребро от \(C\), \(G\) продължава да бъде свързан,
    това противоречи със "структурата" на \(G\) следователно съществуват само прости пътища между \(v_i\) и \(v_j\) (не съществуват цикли межди двата върха) \((1)\).
    Остава да докажем, че съществува единствен прост път между \(v_i\) и \(v_j\). Нека допуснем, че съществуват два различни
    прости пътя между \(v_i\) и \(v_j\). Тогава тръгваме от \(v_i\) и вървим докато върховете на двата пътя съвпадат. Нека означим
    края на съвпаденето с \(v_l\) ако \(v_l\) съвпада с \(v_j\) то двата пътя съвпадат и следва противоречие със допуснатото тяхно различие,
    тоест съществува единствен прост път между \(v_i\) и \(v_j\). Ако \(v_l\) е различен от \(v_j\) то продължавайки по единия от двата пътя към \(v_j\)
    непременно ще стигнем до връх принадлежащ и на другия път, нека означим първия такъв връх с \(v_m\). Tакъв връх със сигурност е \(v_j\) тогава \(v_l\) и \(v_m\) образуват
    цикъл, което противоречи с вече доказания факт \((1)\), че между \(v_i\) и \(v_j\) не съществуват цикли, което означава, че не същестуват различни
    прости пътища между \(v_i\) и \(v_j\). \\\\
    \(v_i\) и \(v_j\) са произволни следователно между всеки два върха принадлежащи на \(G\) съществува единствен прост път. \(\qed\)
 \end{document}
\grid
