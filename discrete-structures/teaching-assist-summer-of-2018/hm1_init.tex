\documentclass[a4paper, 12pt, oneside]{article}
    
\usepackage[left=3cm,right=3cm,top=1cm,bottom=2cm]{geometry}
\usepackage{amsmath,amsthm}
\usepackage{amssymb}
\usepackage{lipsum}
\usepackage{stmaryrd}
\usepackage[ampersand]{easylist}
\usepackage[T1,T2A]{fontenc}
\usepackage[utf8]{inputenc}
\usepackage[bulgarian]{babel}
\usepackage[normalem]{ulem}
            
\newcommand{\R}{\mathbb{R}}
\newcommand{\N}{\mathbb{N}}
            
\setlength{\parindent}{0mm}
                    
\title{Предложения за задачки}
\author{ДСТР}
                    
\begin{document}
\maketitle

\section*{Множества, релации и функции}

\subsection*{а)}

Нека $A$ е множество и нека $f$ е функция с домейни декартовия квадрат на множеството $A$
и ко-домейн множеството $A$, тоест $f : A \times A \to A$ имаща следните свойства:

\begin{itemize} 
    \item $\forall x \in A \; \forall y \in A \; \forall z \in A \; f(f(x, \; y), \; z) = f(x, \; f(y, \; z)) $
    \item $\exists x \in A \; \forall y \in A \; f(x, \; y) = y \; \land \; f(y, \; x) = y $
    \item $\exists x \in A \; \exists y \in A \; \lnot (f(x, \; y) = f(y, \; x))  $ 
\end{itemize}

Дайте добре обоснован пример за множеството $A$ и функцията $f$, в който множеството $A$ е крайно и пример, в който $A$ е изброимо безкрайно.

\subsection*{б)}

Нека $A$ е множество и нека $R$ е релация в множеството $A$, тоест $R \subseteq A \times A$ имаща следните свойства:

\begin{itemize} 
    \item $R$ е рефлексивна и не е транзитивна.
    \item $\forall x \in A \; \exists y \in A \; \left((y, \; x) \in R \; \land \; (x, \; y) \notin R \right) $
\end{itemize}

Дайте добре обоснован пример, в който $A$ и $R$ са крайни множества и пример, в който са изброимо безкрайни.

\section*{Индукция}

Нека $\forall n \in \N \quad F_n \; : \R \to \R$, като \\

$F_0(x) = \begin{cases}
    \sin x & , \; x < 0 \\
    \log_{1000}(x^2 + 3x + \cos x) & , \; x = 0 \\
    5^{\ln(x^3)} & , \; x > 0
\end{cases}$ \\\\

$F_1(x) = \mathrm{sign}(x)|x| + 1$ \\

$\forall n \in \N \quad F_{n + 2}(x) = F_n(nF_{n + 1}(x) + 3^n) - \displaystyle2^{F_n(2x - 1) + F_{n + 1}(\pi x + e)} $. \\

Докажете, че за всяко $n \in \N$ $F_n$ е интегруема функция в интервала $[-100, \; e^{1000}]$. \\

\textit{Забележка:} можете да използвате директно всяко знание от курсa по ДИС1.

\end{document}    