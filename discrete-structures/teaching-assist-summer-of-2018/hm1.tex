
\documentclass[a4paper, 12pt, oneside]{article}
    
\usepackage[left=3cm,right=3cm,top=1cm,bottom=2cm]{geometry}
\usepackage{amsmath,amsthm}
\usepackage{amssymb}
\usepackage{lipsum}
\usepackage{stmaryrd}
\usepackage[T1,T2A]{fontenc}
\usepackage[utf8]{inputenc}
\usepackage[bulgarian]{babel}
\usepackage[normalem]{ulem}
            
\newcommand{\R}{\mathbb{R}}
\newcommand{\N}{\mathbb{N}}
\newcommand{\Complex}{\mathbb{C}}
            
\setlength{\parindent}{0mm}
                    
\title{Домашно 1}
\author{Информатика 2017/18}
                    
\begin{document}
\maketitle
    
\section*{Релация}
Нека $D = \{0, \; 1, \; 2, \; 3, \; 4\}$ и нека $R \subseteq D^3 \times D^3$
е следната релация:
\begin{align*}
    (a, \; b, \; c) \: R \: (x, \; y, \; z) \iff ac \equiv xz \; (\mathrm{mod} \; 5).
\end{align*}

а) Докажете, че $R$ е релация на еквивалентност. \\

б) Намерете броят на класовете на еквивалентност относно $R$. \\

в) Намерете броят на елементите във всеки клас на еквивалентност относно $R$.

\subsection*{Решение:}
а) \\

- $R$ е рефлексивна, защото $\forall (a, \; b, \; c) \in D^3 \; ac \equiv ac \; (\mathrm{mod} \; 5)
\implies (a, \; b, \; c) \: R \: (a, \; b, \; c) $
(Рефлексивност на релацията $\equiv_{\mathrm{mod} \; 5}$). \\

- $R$ е симетрична, защото $\forall (a, \; b, \; c), \; (x, \; y, \; z) \in D^3 \; : \;
(a, \; b, \; c) \: R \: (x, \; y, \; z) \implies ac \equiv xz \; (\mathrm{mod} \; 5)
\implies ac \equiv xz \; (\mathrm{mod} \; 5) \implies (x, \; y, \; z) \: R \: (a, \; b, \; c)$ \\
(Симетричност на релацията $\equiv_{\mathrm{mod} \; 5}$). \\

- $R$ е транзитивна, защото \\
$\forall (a, \; b, \; c), \; (x, \; y, \; z), \; (m, \; n, \; k) \in D^3 \; : \;
(a, \; b, \; c) \: R \: (m, \; n, \; k) \; \land \; (m, \; n, \; k) \: R \: (x, \; y, \; z) \\
\implies ac \equiv mk \; (\mathrm{mod} \; 5) \; \land \; mk \equiv xz \; (\mathrm{mod} \; 5) \implies
ac \equiv xz \; (\mathrm{mod} \; 5) \implies (a, \; b, \; c) \: R \: (x, \; y, \; z)$
(Транзитивност на релацията $\equiv_{\mathrm{mod} \; 5}$). \\

$R$ е рефлексивна, симетрична и транзитивна следователно е релация на еквивалентност. \\

б) \\

Нека фиксираме един елемент на домейна $(a, \; b, \; c) \in D^3$ тогава класът на $(a, \; b, \; c)$ по дефиниция е:
\begin{align*}
    [(a, \; b, \; c)] = \{(x, \; y, \; z) \in D^3 \; | \; (a, \; b, \; c) \: R \: (x, \; y, \; z)\} = \\
    = \{(x, \; y, \; z) \in D^3 \; | \; ac \equiv xz \; (\mathrm{mod} \; 5) \},
\end{align*}
но понеже $(a, \; b, \; c)$ е фиксиран елемент на домейна на релацията, то и $ac$ е фиксирана константа,
която е с фиксиран остатък при делеление с частно и остатък на 5.
Ние знаем, че има точно пет различни остатъци "по модул" \: 5, това са елементите на множеството $D$.
Тогава множеството от класовете на еквивалентност относно $R$ е:
\begin{align*}
    \{\{(x, \; y, \; z) \in D^3 \; | \; xz \equiv k \; (\mathrm{mod} \; 5)\} \: | \; k \in D \}
\end{align*}
и броят на елементите му съвпада с броят на елементите на множеството $D$, който е $5$. \\implies

в) \\

От комбинаторния принцип на умножението следва, че броят на елементие на домейна е: $|D^3| = |D|^3 = 5^3 = 125$. \\

Нека разгледаме подробно всеки един клас на еквивалентност. \\

Започваме със следното наблюдение ако $m, \; n \in D$, то $0 \leq mn \leq 16$.\\

Нека означим с $M$ множеството от целите числата от 0 до 16, тоест $M = \{0, \; 1, \; 2, \; \dots, \; 16\}$. \\

Започвайки от онези наредени тройки с елементи от $D$, за които
\begin{align*}
    \{(x, \; y, \; z) \in D^3 \; | \; xz \equiv 0 \; (\mathrm{mod} \; 5)\}.
\end{align*}
Числата даващи остатък 0 при деление с частно и остатък на 5 са числата от множеството:
\begin{align*}
    \{m \in M \; | \; 5 \mid m \} = \{0, \; 5, \; 10, \; 15\}
\end{align*}
Понеже $5 \notin D$, то единственото число кратно на 5, което може да се получи като произведение на числа от $D$ е 0.
Тогава
$\{(x, \; y, \; z) \in D^3 \; | \; xz \equiv 0 \; (\mathrm{mod} \; 5)\}
= \{(x, \; y, \; z) \in D^3 \; | \; x = 0 \; \lor \; z = 0 \} = \{(0, \; y, \; z) \in D^3\} \cup \{(x, \; y, \; 0) \in D^3\}$.
Тогава от принципа на включването и изключването получаваме:
\begin{align*}
|\{(x, \; y, \; z) \in D^3 \; | \; xz \equiv 0 \; (\mathrm{mod} \; 5)\}| = |\{(0, \; y, \; z) \in D^3\} \cup \{(x, \; y, \; 0) \in D^3\}| = \\
= |\{(0, \; y, \; z) \in D^3\}| + |\{(x, \; y, \; 0) \in D^3\}| - |\{(0, \; y, \; z) \in D^3\} \cap \{(x, \; y, \; 0) \in D^3\}| = \\
= |D^2| + |D^2| - |\{(0, \; y, \; 0) \; | \; y \in D\}| = 2 |D|^2 - |D| = 50 - 5 = 45.
\end{align*}

Числата даващи остатък 1 при делеление с частно и остатък на 5 от $M$ са: $1, \; 6, \; 11, \; 16$.
11 е просто число строго по-голямо от 4, следователно никой две числа от $D$ не дават произведение 11.
За останалите числа съществува единствено (с точност до реда) разлагане като произведение на числа от множеството $D$,
по-кокретно: $1 = 1.1$, $6 = 2.3 = 3.2$ и $16 = 4.4$. Тогава:
\begin{align*}
    |\{(x, \; y, \; z) \in D^3 \; | \; xz \equiv 1 \; (\mathrm{mod} \; 5)\}| = \\
    = |\{(x, \; y, \; z) \; | \; y \in D, \; (x, \; z) \in \{(1, \; 1), \; (2, \; 3), \; (3, \; 2), \; (4, \; 4)\}\}| = \\
    = |D|.|\{(1, \; 1), \; (2, \; 3), \; (3, \; 2), \; (4, \; 4)\}| = 5.4 = 20.
\end{align*}

Числата даващи остатък 2 при делеление с частно и остатък на 5 от $M$ са: $2, \; 7, \; 12$.
7 е просто число строго по-голямо от 4, следователно никой две числа от $D$ не дават произведение 7.
За 2 и 12 съществува единствено (с точност до реда) разлагане като произведение на числа от множеството $D$,
по-кокретно: $2 = 1.2 = 2.1$ и $12 = 3.4 = 4.3$. Тогава:
\begin{align*}
    |\{(x, \; y, \; z) \in D^3 \; | \; xz \equiv 1 \; (\mathrm{mod} \; 5)\}| = \\
    = |\{(x, \; y, \; z) \; | \; y \in D, \; (x, \; z) \in \{(1, \; 2), \; (2, \; 1), \; (3, \; 4), \; (4, \; 3)\}\}| = \\
    = |D|.|\{(1, \; 2), \; (2, \; 1), \; (3, \; 4), \; (4, \; 3)\}| = 5.4 = 20.
\end{align*}

Числата даващи остатък 3 при делеление с частно и остатък на 5 от $M$ са: $3, \; 8, \; 13$.
13 е просто число строго по-голямо от 4, следователно никой две числа от $D$ не дават произведение 13.
За 3 и 8 съществува единствено (с точност до реда) разлагане като произведение на числа от множеството $D$,
по-кокретно: $3 = 1.3 = 3.1$ и $8 = 2.4 = 4.2$. Тогава:
\begin{align*}
    |\{(x, \; y, \; z) \in D^3 \; | \; xz \equiv 1 \; (\mathrm{mod} \; 5)\}| = \\
    = |\{(x, \; y, \; z) \; | \; y \in D, \; (x, \; z) \in \{(1, \; 3), \; (3, \; 1), \; (2, \; 4), \; (4, \; 2)\}\}| = \\
    = |D|.|\{(1, \; 3), \; (3, \; 1), \; (2, \; 4), \; (4, \; 2)\}| = 5.4 = 20.
\end{align*}

Получихме: $45 + 4.20 = 45 + 80 = 125$.

\section*{Биекция}
Постройте биекция между множествата $P = \{x^2 + ax + b \; | \; a, \; b \in \R\}$ и \\

$M = \left\{\begin{pmatrix}
    a.b & -a.c \\
    a.c & a.b
\end{pmatrix} \; \Big | \; a \in [0, \; \infty), \; b, \; c \in [-1, \; 1] \; : \; b^2 + c^2 = 1     
\right\}$ \\

Като използвате по точно една функция от функциите със сигнатури: \\
$\Complex \to [0, \; 2\pi), \; \Complex \to [0, \; \infty), \; [0, \; 2\pi) \times [0, \; \infty) \to M, \; P \to \Complex, \; \Complex \to [0, \; 2\pi) \times [0, \; \infty)$.

\subsection*{Решение:}
Първо ще разгледаме подробно на какво всъщност е равно всяко от множествата $P$ и $M$. \\

От училище ни е известно, че всеки разложим полином $ax^2 + bx + c$ с реални коефициенти от втора степен се представя по единствен начин
като произведение: $ax^2 + bx + c = a(x - r_1)(x - r_2)$, където $r_1, \; r_2$ са неговите реални корени. Ако водещия коефициент е равен на 1,
то разлагането е: $x^2 + bx + c = (x - r_1)(x - r_2)$. От часовете по Линейна алгебра и ДИС 1 е известно, че всеки полином от втора степен с реални коефициенти
(не непременно разложим над реалните числа) има точно два комплексни корена, които са комплексно спрегнати и се разлага по единствен начин.
Известна ни е и формула за намирането на двата корена на полинома с реални коефициенти $x^2 + bx + c$:
\begin{align*}
    r_{1, \; 2} = \frac{-b \pm \sqrt{b^2 - 4ac}}{2}.
\end{align*}
Ако водещия коефицинет е единица, то полинома с реални коефициенти: \\
$x^2 + bx + c$ се разлага по единствен начин като $(x - r)(x - \overline{r})$, където
$r = \frac{-b + \sqrt{b^2 - 4ac}}{2}$.
Тогава лесно можем да построим биекция между множеството $P$ и множеството на комплексните числа: $Root : P \to \Complex$
\begin{align*}
    Root(x^2 + bx + c) = \frac{-b + \sqrt{b^2 - 4ac}}{2}.
\end{align*}

От курса по Линейна алгебра ни е известно, че всяко комплексно число съществува представяне в тригонометричен вид.
Като за всяко комплексно число различно от нула това представяне е единствено. За да се подсигурим, че нулата се представя
по единствен начин, ще "разширим" дефинцията за функцията "главна стойност на аргумента" на едно комплексно число $Arg : \Complex \to [0, \; 2\pi)$
(приемаме, че главната стойност на едно ненулево комплексно число е онази стойност на аргумента в интервала $[0, \; 2\pi)$),
като дефинираме функцията $Angle : \Complex \to [0, \; 2\pi)$ по следния начин:
\begin{align*}
    Angle(z) = \begin{cases}
        0 & , \; z = 0 \\
        Arg(z) & , \; z \neq 0
    \end{cases}
\end{align*}
Функцията модул на комплексно число $|~| : \Complex \to [0, \; \infty)$ знаем, че се дефинира по следния начин:
\begin{align*}
    |a + bi| = \sqrt{a^2 + b^2}.
\end{align*}
Така за всяко комплексно число $c = a + bi$ съществува единствено представяне във вида:
\begin{align*}
    c = |c|(\cos(Angle(c)) + i\sin(Angle(c)))
\end{align*}
Сега можем да дефинираме биекция между множеството на комплексните числа и декартовото произведение
$[0, \; 2\pi) \times [0, \; \infty)$. $Trig : \Complex \to [0, \; 2\pi) \times [0, \; \infty)$
\begin{align*}
    Trig(z) = (Angle(z), \; |z|).
\end{align*}

Да разгледаме множеството $M$. От часовете по тригонометрия в училище или часовете по Линейна алгебра/ДИС 1
или от обща математичска култура ни е известно, че множеството от всички решения в реални числа на уравнението:
\begin{align*}
    x^2 + y^2 = 1
\end{align*}
е множеството $\{(\cos\varphi, \; \sin\varphi) \; | \; \varphi \in [0, \; 2\pi)\}$.
Тогава е ясно, че
\begin{align*}
    M = \left\{\begin{pmatrix}
    a.\cos\varphi & -a.\sin\varphi \\
    a.\sin\varphi & a.\cos\varphi
\end{pmatrix} \; \Big | \; a \in [0, \; \infty), \; \varphi \in [0, \; 2\pi) \right\}
\end{align*}
И значи можем да дефинираме биекция между множествата $[0, \; 2\pi) \times [0, \; \infty)$ и $M$.
$RotMatrix : [0, \; 2\pi) \times [0, \; \infty) \to M$
\begin{align*}
    RotMatrix((\varphi, \; a)) = \begin{pmatrix}
        a.\cos\varphi & -a.\sin\varphi \\
        a.\sin\varphi & a.\cos\varphi
    \end{pmatrix}.
\end{align*}
Ясно е, че композиция на биекции отново е биекция.Тогава търсената биекция между множествата $P$ и $M$,
която използва по точно от една от дадените фунцкии е: $f : P \to M$
\begin{align*}
    f = RotMatrix \circ Trig \circ Root
\end{align*}
\end{document}