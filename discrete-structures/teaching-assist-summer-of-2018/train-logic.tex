\documentclass[12pt]{article}
    
\usepackage[left=2cm,right=2cm,top=1cm,bottom=2cm]{geometry}
\usepackage{amsmath,amsthm}
\usepackage{amssymb}
\usepackage[T1,T2A]{fontenc}
\usepackage[utf8]{inputenc}
\usepackage[bulgarian]{babel}
\usepackage[normalem]{ulem}
            
\setlength{\parindent}{0mm}
                    
\title{Задачи за упражнение върху Логика}
\author{Иво Стратев}
        
\begin{document}
\maketitle

\section*{Задача 1.}

Определете вида (тавтология, противоречие или условност) на съждението
\begin{align*}
(p \; \oplus \; (p \; \land \; \neg q \; \lor \;  r)) \to r \; \leftrightarrow \; q
\end{align*}

\section*{Задача 2.}
Докажете по два начина (чрез табличен метод и чрез еквивалентни преобразувания):

\begin{align*}
p \; \oplus \; q \equiv \neg p \; \lor \; (\neg p \land q) \; \lor \; ((p \to p) \; \land \; (\neg s \to p)) \; \lor \; (p \; \land \; \neg q)
\end{align*}

\section*{Задача 3.}

Постройте еквивалентен израз на всеки от логическите съюзи, в който да участват само негация и импликация.
\end{document}
